%% LyX 2.4.1 created this file.  For more info, see https://www.lyx.org/.
%% Do not edit unless you really know what you are doing.
\documentclass[11pt,oneside,american]{book}
\usepackage[LGR,T1]{fontenc}
\usepackage{textcomp}
\usepackage[utf8]{inputenc}
\setcounter{secnumdepth}{3}
\setcounter{tocdepth}{3}
\usepackage{url}

\makeatletter

%%%%%%%%%%%%%%%%%%%%%%%%%%%%%% LyX specific LaTeX commands.
\DeclareRobustCommand{\greektext}{%
  \fontencoding{LGR}\selectfont\def\encodingdefault{LGR}}
\DeclareRobustCommand{\textgreek}[1]{\leavevmode{\greektext #1}}


\makeatother

\usepackage{babel}
\begin{document}
\title{Marcel Minnaert, astrophysicist 1893-1970}
\title{The cloth of the universe}
\author{Leo Molenaar}

\maketitle
Unauthorized translation of \url{https://www.dbnl.org/tekst/mole016marc01_01/}
by Mark van Rossum and Deepseek.

Original published in Dutch by Balans, Amsterdam / Van Halewyck, Leuven
2003

\pagebreak{}
\begin{verse}
In this time, what was always called Beauty,

beauty has burned her face

She no longer comforts people

She comforts the larvae, the reptiles, the rats

But she frightens humans

And strikes them with the realization

Of being a bread crumb on the cloth of the universe
\end{verse}
(From Lucebert, I try in a poetic way, Collected poems, Amsterdam
2002, page 52)

\pagebreak{}

\tableofcontents{}\pagebreak{}

\section*{Prologue\protect \\
Marcel Minnaert (1893-1970):\protect \\
Heir to the Great Dutch civilization}

There are many people for whom the name Minnaert still resonates dearly.
This certainly applies to science teachers who love observations in
the open field. My first acquaintance with him, therefore, took place
via the Minnaert: these are the three volumes of The Physics of the
Open Field that I acquired around 1970. The first edition of this
book appeared in the late 1930s. In the 1990s, new illustrated editions
of the first part of Minnaert's trilogy, Light and Color in the Landscape,
were published in English and German, alongside the reprint of Minnaert
in Dutch. It is therefore a unique book. Minnaert had observed natural
phenomena for twenty-five years, made notes, and conducted research
into physical explanations.

He had also sought descriptions in prose and poetry to establish connections
between natural science and literature. He had created a personal
compilation of hundreds of natural phenomena, where romance and rationality
come together. Especially in the United States, {*}Light and Color
in the Outdoors{*} has become a classic book. It inspired artists
such as the American James Turrell, who would design the landscape
project {*}The Hemels Gewelf{*} (1996) in Kijkduin based on an idea
from Minnaert.

However, {*}The Physics of the Open Field{*} was not the reason for
this biography. My fascination with the astronomer Marcel Minnaert
(1893-1970) stemmed from my dissertation on the Association of Scientific
Researchers. Minnaert was the chairman of this club of pioneers in
the field of Science and Society at the outbreak of the Cold War.
I wanted to go through ‘the century of Minnaert’ with him. How had
a Dutch representative of the left-wing, scientifically-oriented intelligentsia,
akin to J. Desmond Bernal, Leo Szilard, and Irène Curie, experienced
the issues of the 20th century? I knew that he was a celebrated astrophysicist
in the late 1930s. Why did he no longer want to leave the development
of science to politicians? At the time, I let the interviews with
the founders of this Association extend into conversations about Minnaert.
On the occasion of his 100th birthday, I wrote biographical sketches
for {*}Natuur \& Techniek{*} and {*}Zenit{*} (1993). When the Board
of Directors of Utrecht University decided to name the new building
for scientific education and research after Minnaert, I was commissioned
to write the brochure {*}Marcel Minnaert: een leven lang leraar{*}
(1998). I met his closest family members, interviewed his colleagues
and PhD students, and explored the terrain for a biography.

During this exploration, my motivation changed. This had several reasons.
Minnaert's youngest son, Boudewijn, presented me with a stack of his
grandparents' diaries in Sydney. Marcel's father had started a notebook
about him at his birth, and after his death, his mother had taken
over the writing: 1,300 handwritten pages testified to the attention
that two Flemish professional educators had devoted to their favorite
child. These Diaries describe the upbringing and education of young
Minnaert.

In 1909, he was Belgium's top student and graduated cum laude in biology
in the summer of 1914. Inspired by some teachers, he emerged as a
fanatic Flemish nationalist: an activist for whom, during World War
I, the independence of Flanders outweighed the German occupation of
Belgium. At the age of 23, he would teach physics at the Dutchified
University. It became clear to me that to understand his life and
work, I would have to pay much attention to his Flemish upbringing.
A closer examination of his actions during the occupation revealed
that Minnaert's political movement, Jong Vlaanderen (Young Flanders),
was the most notorious. In studies such as {*}Het Aktivistisch Avontuur{*}
(1991) by historian Daniel Vanacker, he plays a leading role. Minnaert
even wrote a defense against personal attacks: {*}De verdeeling van
den arbeid en het nationaliteitenbeginsel{*} (1916). (I have modernized
this archaic spelling for readability, consistently and drastically:
after this foreword, there will be no mention of 'Dietsch,' 'verdeeling,'
or 'sterrekunde,' but simply 'Diets,' 'verdeling,' and 'sterrenkunde.')

At the end of October 1918, Marcel fled to the Netherlands with his
mother to escape a long prison sentence or worse. By then, he had
already written numerous articles and lectures in addition to his
dissertation. Part I of the biography deals with Minnaert in Flanders
and requires some empathy from the Dutch reader for the issues surrounding
the Flemish Movement. At a meeting of Dutch and Flemish astronomers
in De Haan (1993), where Minnaert's centenary was a theme, Vanacker
presented a lecture on {*}De jonge Minnaert{*} ({*}The Young Minnaert{*}).
It was the first serious attempt to place him within his Flemish context.
Vanacker has provided commentary for my {*}Minnaert in Vlaanderen{*}
on two occasions, and I am deeply grateful to him. Minnaert's dedication
to Flemish nationalism would not diminish in the Netherlands.

His scientific career also held surprises for me. Minnaert immediately
received a position at the Heliophysical Institute of Utrecht University.
Four years later, he introduced the astronomical community to a unit
for the intensity of absorbed starlight: the equivalent width, which
indeed became the international standard a quarter-century later.
Before becoming a professor in Utrecht, he had already been appointed
in Chicago. He received the highest honors in astronomy: the British
Gold Medal and the American Bruce Medal. Minnaert's team produced,
following in the footsteps of illustrious predecessors such as the
German Kirchhoff and the American Rowland, an Atlas (1940) and a Table
(1966) of the Solar Spectrum that are still consulted today. He proved
to be a pioneer in the spectroscopic branch of modern astrophysics
and a prominent member of the International Astronomical Union.

Minnaert was not only prominently present in the Encyclopedia of the
Flemish Movement but also in the Encyclopaedia Britannica. His name
was given to an asteroid between Mars and Jupiter and a crater on
the far side of the moon. Therefore, Part II of the biography focuses
on the development of this scientist. The astrophysical chapter calls
upon the reader's diligence and capacity for comprehension. Minnaert
believed that any problem, no matter how difficult it seemed, could
be explained to an interested person in five minutes. I hope to have
worked in his spirit. The astrophysicist Kees de Jager, a student
of Minnaert and founder of space research in the Netherlands, read
two versions of this book and also acquainted me with Minnaert's work
and his astrophysical equipment.

During the Interbellum, Minnaert was fully engaged with Flanders but
not with the position of science in society. That issue only grasped
him during and after World War II. The war also formed a watershed
in other respects with the final part of his life, in which he tried
to connect issues of worldview, science, and society. After 1945,
he emphasized the socio-political context of science so strongly that
it had to become the guiding principle for Part III. At Utrecht University,
during the height of the Cold War, they were so wary of Minnaert,
a supposed 'fellow traveller' and 'friend of communists,' that they
imposed a professional ban on him. Minnaert's life spontaneously divided
into three equal parts of a quarter century, separated by the end
of a World War. In his childhood, the Boer War (1899) still played
a role, while death claimed him as coordinator of Books for Hanoi.
His life and thinking were therefore profoundly influenced by the
political and military events of the 20th century. Each part consists
of chapters. Each chapter begins with a characteristic quote from
Minnaert. In Chapter 1, the prophetic motto of his parents is cited:

\textquotedbl 'I am a lover of pure joy, of the beautiful and true,
art and virtue.'

For 35 years, I have taught natural sciences with great enthusiasm.
I lectured on 'dangerous substances' in adult vocational education,
physics at the Pedagogical Academy, and chemistry and General Natural
Sciences at Erasmus Gymnasium. I was surprised to discover that Minnaert,
more so than Ph. Kohnstamm, was the pioneer of physics didactics in
the Netherlands. As an adolescent, he already felt connected to the
work of pedagogues like Ellen Key, Jan Ligthart, and Tatiana Ehrenfest-Afanasjeva.
In 1917, he tried out a first practical experiment for children at
several Flemish primary schools, and in 1924, he wrote the first Dutch-language
publication on student experiments. Together with physicist Leonard
Ornstein, he designed a university teacher training program in the
1930s. After 1945, along with mathematician Hans Freudenthal, he advocated
for a reform of education in mathematics, natural sciences, and astronomy.
From its founding in 1950, he was a member of the Natural Science
Didactics Working Group, which laid the foundation for the Woudschoten
conferences for physics teachers. Throughout his life, he emphasized
the importance of astronomy in secondary education: he hoped in 1951
that this wonderful subject could be taught as part of a new subject,
General Natural Sciences. I could fully relate to this aspect of Minnaert's
life and gave it the space I believe it deserves.

The person of Minnaert was thus more multifaceted than I had imagined.
He didn't deserve a biography because he fit into my framework, but
turned out to be, in the wake of Simon Stevin, his fellow townsman
and kindred spirit, a heritage bearer of Dutch civilization. His son
Boudewijn and daughter-in-law Els Hondius provided me with material
that could give me a better picture of the flesh-and-blood man. They
wanted to review the processing beforehand. This led to friendly communication
that resulted in a refinement and further enlivenment of the texts.
I thank them both for this collaboration. The primary and secondary
sources are accounted for in the Notes section for each respective
part.

Writing this biography was a monumental task. In 1999, I applied for
a replacement grant from the Dutch Organization for Scientific Research
(NWO), which was supported by five professors: The historians Piet
de Rooy and Albert Kersten, my former promoters; the science historian
Anne Kox; and the astronomers Ed van den Heuvel and Max Kuperus. The
application stated that the interplay of scientific, didactic, and
societal issues would be the specific angle of the biography. Kox
indicated that if the grant were awarded, he wanted to secure a place
for me at the Center for History of Science of the Institute for Theoretical
Physics at the University of Amsterdam, so I could work in a 'prepared
environment.' The subsidy was indeed granted, allowing me to be exempt
from teaching duties during the 2000-2001 academic year to complete
this book. Additional funding from the Pieter Zeeman Fund, the Special
Journalism Projects Fund, and the Advisory Board of my school supplemented
my usual part-time salary during that year and the next. Without this
financial support, I would have been able to complete this book only
with great difficulty and much later.

Starting in May 2000, Kox and I discussed all draft texts, both from
the research work and the concepts for the later chapters and sections.
His comments and suggestions stimulated and motivated my work. He
helped me through depressions by trusting in the final result. In
the summer of 2001, an extensive version of the book was completed,
which I submitted to several readers. During the following academic
year, when I resumed my work, I created a second version, which was
reviewed by a wider circle of people. Here, I would like to thank
journalist Max van den Berg, columnist Dik de Boef, science journalist
Frank Biesboer, publisher Jan Geurt Gaarlandt, Flemish teacher Rudolf
Mahy, publisher and my neighbor Herman Masthoff, writer and science
journalist Marianne Offereins, Dutch studies scholar Hans Overheul,
my promoter Piet de Rooy, physics didactics expert Greet Smit-Miessen,
and my wife and discussion partner Rieme Wouters. The astrophysics
chapter was intensively discussed with my physics colleagues Leen
Bongers and Jan van den Koppel and also presented to the members of
the advisory committee. The interviews and consultations with dozens
of experts, as well as friends, colleagues, family members, and acquaintances
of Minnaert, are accounted for in the Notes section of part III. The
final version was discussed only with Anne Kox and Marianne Offereins.
The publisher provided the inspiring suggestion for the title, {*}De
Rok van het Universum{*} (The Robe of the Universe), referencing a
quote by Lucebert. In my view, this title aligns with the spirit of
Minnaert, who in 1949 translated and compiled the anthology {*}Dichters
over Sterren{*} ({*}Poets on Stars{*}).

Marcel Minnaert has no reason to complain about the extent to which
he continues to be remembered today in various circles. The remarkable
Minnaert Building (1998) has been mentioned. In recent years, no fewer
than two prizes have been named after him: a biennial Minnaert Prize
for the best contribution to physics education in secondary education
(since 1987) and a Minnaert Prize for young researchers and authors
who contribute to cultural and political cooperation between the Netherlands
and Flanders (1995-2002). Yet, he has remained relatively unknown.
Often, natural scientists remain anonymous greats for an audience
that seems to take more interest in poets, writers, historians, heads
of state, film stars, and politicians. Apparently, the work of scholars
is difficult to understand. Many biographers also shy away from this.
For example, the historian Vanacker ended his biographical sketch
at the point when Minnaert's astrophysical career began. I am glad
that my triple education and profession as a chemist, educator, and
historian do not present me with such a dilemma.

Moreover, my education failed on a crucial point. The biographer must
understand the main character in their development so that they can
provide an insight into the portrayed person. Minnaert's personality
turned out to be more complicated than I had thought possible. The
Diaries of his parents added a dimension but also required interpretation.
I occasionally sought the help of psychologist friends, presenting
them with characteristic events. Their comments made me realize that
I needed expert assistance. Ultimately, I systematically discussed
the draft texts in a series of evening-long sessions with clinical
psychologist and psychotherapist Gijs van der Zalm. He taught me to
read between the lines and provided me with comments on primary sources.
In addition to ongoing comments, at the end of each section (1919;
1945; 1970), I present an explicit psychological commentary in a snapshot.
The apparent influence of upbringing on Minnaert's life and work confronted
me with insights that were previously closed off to me. By delving
into Minnaert, I learned to know myself better.

Endnotes:

1 James Turrell, Kijkduin, Celestial Vault in the Dunes, Stroom; The
Hague Centre for Visual Arts, Den Haag 1996.\textquotedbl{}

\textquotedbl In this, among other things, Gerrit Willems: The Sensory
Splendor of Light: Turrell and Minnaert, 61-65, and from Minnaert's
{*}Natuurkunde van 't Vrije Veld{*}: The Apparent Flattening of the
Sky Dome, 67-102.

2 Leo Molenaar, We Can No Longer Leave It to the Politicians; The
History of the Union of Scientific Researchers (1946-1980), Rijswijk
1994.

\part{(1893-1919) Minnaert in Flanders\protect \\
A Youth in Obligatory Resistance}

The dawn drove away the nocturnal darkness, the swallows flew chattering
closely over the meadow, and the sun colored the horizon fiery red.
Klaas opened the window and spoke to Uilenspiegel:
\begin{quote}
Child with the helmet, behold, there is Mother Sun, who comes to greet
Flanders. Behold her when your eyes will be open; if you ever doubt
later, do not know what to do to act well, go then for advice to the
Sun; she is warm and bright; be as good as she is warm, as honest
as she is bright.
\end{quote}
Charles De Coster, {*}Tijl Uilenspiegel{*}, 1868

\chapter{The Fire Under the Ashes}
\begin{verse}
'I am Minnaer of pure joy, Of the beautiful and true, art and virtue.'
\end{verse}

\section*{Marcel's World}

In February of the year 1893, Marcel was born in Bruges, the son of
Jozef Minnaert and Jozefina van Overberge. He was the child of two
educators. Jozef was a teacher and Jozefina a teacher at a Normal
School. Jozef, who had to commute to Ledeganckstraat in Ghent, walked
around quarter past six through the Beenhouwersstraat to the station
on 't Zand. Jozefina, who only needed to leave home at eight o'clock,
walked in ten minutes to the rear entrance of the State Normal School
for Girls on Sint-Jorisstraat. Sometimes they only saw each other
again around half past ten in the 3 evening. People talked about the
coldest days of the 19th century. On Friday, February 10th, Jozefina
wanted to go to work when the contractions began. On Sunday, February
12th, at noon sharp, Marcel was born: 'You were very skinny and tall,
but your voice was strong.' He had large, dark eyes that amazed people.
This was written by Father Jozef in his Diary. He designed a family
crest for Marcel with the motto: 'I am a lover of pure joy, / Of beauty
and truth, art and virtue,' and wrote:
\begin{quote}
‘How we have longed for you,

dear, only, precious child!

How we have thought about you,

how we have always loved you.’
\end{quote}
Jozefina had knitted Marcel's clothes herself. The delivery had been
difficult, and Marcel would remain her only child. She had looked
forward to breastfeeding him when she came home at noon. But even
though Jozefina drank grain water until it disgusted her, she couldn’t
produce a single drop of milk. The baker and doctor forbade bottle
feeding. Only after a week, when Marcel was visibly deteriorating,
did the doctor allow them to seek a wet nurse. The child was baptized
on February 26th in the stately Sint Salvator. The godfather was Jozef's
brother Gillis Desideer Minnaert, and the godmother was Jozefina's
mother Catharina Dondt. Marcel was content with the carriage ride,
but ‘as soon as you arrived, you began to wail so loudly that neither
the pastor, sacristan, relatives, nor midwife could hear or see anything.
The sacristan’s Latin became completely confused.’ Jozef's Diary has
a twofold character. Besides observations of his child, it contains
many reflections. His child evoked feelings of regret and resentment
in him. Regret for marrying late at 44 years old and resentment over
missing out on medical studies as a working-class child. Everything
would be different with Marcel. Jozef would teach him about plants,
animals, and anatomy in a playful way. For Marcel was to become a
children's doctor. Not a physician for bread, but a selfless scholar:
'Comforting deeply grieving parents, poor mothers and fathers; uplifting
them; saving where possible, alleviating the suffering and pain of
the sick little ones, dedicating oneself entirely to the happiness
of others, to the salvation of mankind, creating progeny, By caring
for the child, healing it, and strengthening it.'

Father cherished the ideal of a nearly priestly dedication to science.
The child would sometimes feel downcast later on because it would
refrain from worldly pleasures, but what a field of work would open
up, 'what material for character formation, for self-denial or rather
self-abandonment, involving fear and courage simultaneously, relentless
patience, expressions of love, and compassion.' This is how Jozef
fantasized and tried to steer the upbringing: 'Our child is a book
in which we continuously read and study, but also write what we think
is best preserved.' In such sentences, the message follows the objection.
The child was there for the father, not the other way around.

Marcel's parents had rented the house on Guldenvlieslaan from Jozefina's
mother after their marriage in 1890. This avenue is part of the western
section of the belt of boulevards along the city walls. Gardens and
some workers' cottages were located on the boulevard. At that time,
a series of stately, sometimes kitschy buildings appeared. Number
22 was a simple house with identical buildings on either side. The
house was six meters wide and ten meters deep, had two floors above
ground level, and a small backyard. In front of the house lay a wide
sidewalk leading to a row of young trees, behind which was an unpaved
road used by horses, dogs, carriages, and carts. This road transitioned
into a second row of trees with benches for residents and passersby
behind them. The avenue was enclosed by a wire fence with a dense
hedge to screen off the embankment. In a lowered area, audible but
not visible from the house, the train to Blankenberge ran. The other
side of the railway was sparsely built in the direction of the Ezelspoort.
The sun set over open fields in front of the house. Playing on the
avenue was heavenly. The parents' fear of infection with fatal diseases
ensured that Marcel did not come into contact with neighborhood children
for the first three and a half years. Marcel's uncles, aunts, cousins,
and other relatives lived in Ghent or farther away. His environment
consisted of adults.

The child grew up prosperously. At nine months old, he was a kilogram
heavier and ten centimeters taller than the average child. Apart from
his parents, nanny Leonie Sippens and the maid Reinilde Jamees were
at his disposal. On the twelfth of each month, this community celebrated
the small child's remembrance day. For fifteen months, Joseph slept
in the guest room, while Josephine, the maid, and Marcel used the
bedroom. The housemaid had her own room. Marcel's daily routine was
strict. His bath water was at a temperature of 27.1 degrees. When
Joseph measured the child on September 29, 1893, it was 833.5 millimeters
long, and the distance from wrist to fingertips was 9.3 centimeters.

Whatever Marcel desired was immediately available. Both parents had
already led frugal and industrious lives, earned a decent salary as
teachers, while the rent must have been low. The domestic staff at
the time worked for board, lodging, and a modest allowance. Joseph
abstained from worldly pleasures. An occasional friend from Ghent
would visit. He didn’t make new acquaintances in Bruges. Joseph had
previously dedicated his life to the Flemish Movement and now lived
for his child.

\section*{Joseph: a devout, Flemish-minded liberal}

Joseph was 46 years old when Marcel was born. His father, Petrus Judocus
Minnaert (1796-1881), was a cotton printer, and his mother, Coleta
Joanna den Dooven (1799-1873), was a bleacher. They had twelve children,
two of whom went into education. The first was Gillis Desideer, born
in 1836 as the seventh child, followed by Josephus Ludovicus, born
in 1846 as the eleventh. The parents consistently referred to the
years between 1815 and 1830 as 'the good Dutch times.' After Belgium's
secession, the textile industry in Flanders declined: Judocus had
to seek work in Wallonia and northern France, and Coleta had to take
up additional trades. Joseph, like his revered mother Coleta, was
a hard worker; dedicated, self-sacrificing, and dutiful. Like his
father Judocus, he could be hot-tempered, impatient, gruff, and authoritarian.

He was liberal and devout. Every morning he attended the first mass.
In the respectable houses on the Guldenvlieslaan, the people of Bruges
spoke French. In the working-class neighborhood behind, they spoke
a Flemish dialect. Jozef Minnaert and his wife were part of a Gideon
faction of Flemish-minded individuals who spoke refined Dutch at home.
His life had been marked by the emancipation of Flanders. Jozef's
brother, Gillis, was one of the leaders of that movement. Their efforts
did not clash with their affection for the state: Flanders had to
rise again within Belgium.

The Habsburgs had maintained Flemish as an administrative language
in the Southern Netherlands for two centuries. During Napoleon's occupation,
the vernacular disappeared from public life for a quarter of a century.
In Vienna, the Kingdom of the United Netherlands (1815) was designed
as a power block against France: the Netherlands, Belgium, and Luxembourg
together under the Orange Prince William I. He had introduced Dutch
as the unifying language in Flanders. He founded the Dutch-speaking
State Normal School in Lier and the University of Ghent. The resistance
to this language policy, both from Walloon and Flemish-Catholic sides,
was one of the reasons for the 1830 uprising. Belgium was born and
became a modern, French-speaking state. The Flemings were denied the
right to continued and higher education in their native language.
The flemishization during the 'good Dutch times' had contributed to
the emergence of a Flemish Movement. The Flemish-minded adorned themselves,
just like the geuzen of old, with the French derogatory name flamingants:
they wanted to expel French from Flanders. This endeavor was romantic
because it appealed to a glorious past. But it was also a legacy of
the Enlightenment, as it promoted justice, equality, and the uplifting
of the people. The flamingants had to confront an economic elite of
frenchified Flemings: the franskiljons. Brussels, where the Belgian
state exerted its pull on career makers, and Ghent, where the dominant
groups used French as a class language, were their strongholds. In
Bruges as well, language expressed a sharp class division. The dominance
of French was also caused by the growth of the Walloon coal and steel
conglomerate producing material for the European railway network.
In contrast, the textile industry in Flanders faced declining economic
fortunes. Moreover, the Netherlands did not lift a finger to help
its 'ethnic kin.' There prevailed Protestant resentment toward the
Catholic South, which had plunged unification into a civil war. Dutch
songs celebrated the collective suicide of Commander Van Speyk, 'rather
die than surrender,' in the port of Antwerp.

The Flemings had to pull themselves out of the swamp by their own
hair. Marcel's godfather, Gillis Minnaert, managed this wonderfully.
He rose from being a working-class boy to inspector of municipal education
in Ghent, writer and compiler of anthologies, Knight in the Order
of Leopold (1880) and the Order of Orange-Nassau (1894), and chairman
of the liberal Willemsfonds. He introduced Pestalozzi's and Fröbel's
educational methods and brought Dutch books from the North into municipal
schools.

Jozef was fond of his brother, became a teacher, later a professor
at the Ghent Normal School, where he provided further French-language
education. Together, the brothers wrote story collections, and Jozef
also ventured independently into writing. They also did volunteer
work. In the 1880s, the social question began to play a role. Their
own social rise likely strengthened their belief in the progressive-liberal
philosophy of Ghent professor François Laurent, who taught that the
worker could take his fate into his own hands. Gillis wrote that for
twelve years, they had spent their weekends giving lectures in associations
for working girls and boys. They chose examples from life such as
'the benefits of saving' and alternated their lectures with songs,
music, lantern slides, theater performances, and excursions. The unmarried
Jozef married his ten-years-younger colleague, Jozefina, and moved
to Bruges for her sake. His child became the excuse to put an end
to his volunteer work. He kept a notebook of aphorisms on pedagogical
matters. One of them reads: 'If one knows the passions of a child,
one can control it, just as a fortress whose weak points have been
discovered is usually lost.' Marcel was the test case. Jozef's life
suddenly lay beyond his grave, 'like a sentence that is only ended
by a comma.'

\section*{Jozefina: A brave, progressive teacher}

Marcel's mother, Jozefina, was 36 years old at his birth. Her parents
were from Ghent.\textquotedbl{}

\textquotedbl Little is known about Joannes Benedictus Van Overberge
(1822-1885) and Catharina Dondt (1823-1904). They must have achieved
a certain prosperity through their own efforts, as they were able
to fund Jozefina's education and purchase a house for her in Bruges.
Jozefina's sisters, Emilie, Nathalie, and Clémence, spoke both French
and the Ghent dialect in Ghent and Wallonia. Only Jozefina spoke refined
Dutch. She was 34 years old when she married and was more liberal
and less devout than her husband.

Jozefina was one of the first Belgian girls to pursue teacher training
and become a regentes (headmistress). In 1873, when she began her
studies at the age of seventeen, there were no normal schools for
girls. Some training programs existed at municipal institutions. A
handful of boarding students followed a three-year French-language
course there, paying high tuition fees of 400 francs per year, and
took exams in reading, penmanship, Flemish, French, arithmetic, pedagogy,
methodology, geography, national history, singing, Christian doctrine,
and biblical history. The student had a weekly schedule of 70 hours:
six days of eleven hours each and Sunday morning for four hours. At
the time, liberal governments were trying to free Flanders from the
yoke of independent, thus Catholic, schools. Liberalism and militant
Catholicism clashed head-on. In the encyclical Quanta cura (1864),
condemned modern ideas such as liberalism and socialism, freedom of
religion and the press, and the separation of church and state. The
confrontation in Flanders focused on education policy. Assistant pastor
Guido Gezelle wrote in his West Flemish dialect in 1866: 'Rather than
entrusting the young daughters of Flanders to the government, which
is known as the most unchristian and disloyal among all governments
in Europe, one would sooner send mouse cubs to cats for lessons or
let lambs be devoured by wolves.'

Jozefina, as a liberal teacher and educator of 'she-wolves,' was the
target of such scorn. Around 1878, when she graduated, the final round
of this confrontation began. The liberals introduced a substantial
majority in parliament from the elections. Minister Van Humbeeck decided
in 1879 to establish a neutral Normal School in Bruges. The Catholic
press sounded the alarm: freethinkers, nihilists, Protestants, Jews,
and Freemasons would destroy the souls of children: 'The most dreadful
disaster for Belgium' was indeed the transformation of community schools
into 'gueux nurseries.' Catholic leaders called for a holy war 'against
the pernicious law' and used the same rallying cry as the Crusaders:
'God wills it!' Starting from September 1, 1879, in accordance with
the bishops' {*}Instructiones practicae pro confessariis{*}, absolution
was denied to all students, teachers, and parents who would send their
children to these Normal Schools.

Van Humbeeck instructed teachers to focus on moral education and science
instead of religion. The teacher had to guide the instructors toward
personal instruction and 'independent learning.' Instructors had to
be able to set up scientific collections, design demonstration equipment,
paint school charts, and create diagrams. The Frère-Van Humbeeck government
(1878-1884) unleashed a school struggle that tore the state apart.
As men's suffrage expanded, the electoral tide turned, while this
school struggle harmed the liberals. The Catholics defeated them in
1884 and condemned them to thirty years of opposition.

The Catholic authorities then halved the liberal decisions: in Bruges,
only a Normal School for girls was established, while the one in Ghent
would be limited to boys. The one in Bruges, completed in 1883 by
architect Delacenserie, who also designed the government building
on the Grand Market and the hospital near Minnewater, was in a style
that Karel Van de Woestijne called 'monastic gothicizing.' The building
still dominates the workers' neighborhood. The school looks like a
lavish monastery complex with a secular, high-roofed garden in the
center. This school struggle was fiercer in rural Bruges, where Guido
Gezelle's hate campaign still lingered, than in bourgeois Ghent, where
the church toned down its rhetoric. Jozefina had to defend herself
professionally against press accusations and the fear of parents and
students regarding excommunication. The language struggle also had
significant consequences for her work.

The 1883 Coremans-De Vigne language law was the result of a Flemish
lobby that united Catholics and liberals. That law imposed an obligation
on teachers at state schools in secondary and normal education in
Flanders to teach certain subjects in Dutch. Although there were no
penalties for non-compliance, the law gave Flemish-minded individuals
a sense of grounding. Jozefina, who was appointed to the Bruges Normal
School in 1882, could now teach her geography and Flemish subjects
in Dutch. After the transformation into a State Normal School for
Girls, an all-female teaching staff emerged. When she married Jozef
in 1891, she made it clear that she wanted to continue working. The
opportunity to work at the same school had been missed. They settled
in Bruges, and Jozef was condemned to his daily commute.

Jozefina envisioned a career for herself, even as a mother. Her lessons
received positive responses, and she joined excursions and workweeks
with her young female students. After receiving a pencil drawing of
the Minnewater from one of her students---a prized possession for
their elaborately decorated living room---her husband grumbled: 'It’s
lovely that young ladies are so kind to their teachers; we professors
never receive such gestures.' Alongside her trainee teachers, she
organized educational activities at the Brussels World's Fair and
the Congo Free State Exhibition in Tervuren. Four-year-old Marcel
and his father visited the exhibition, where real African people were
displayed to a gaping audience. Around that time, Jozefina’s young
colleague, who was the headmistress of a Normal School in Antwerp,
passed away. Jozefina applied for the director position, but Jozef’s
patient resistance was in vain. She went to Brussels to plead her
case at the Ministry and organized a lobbying effort involving acquaintances
and family members. Ultimately, another woman was appointed. Jozef
was delighted and even enjoyed the fresh air of Bruges’ boulevard,
which Marcel found very agreeable. Jozefina’s life did not revolve
around her child; she gave her husband the space to pursue his mission.

\section*{Jozef creates a researcher}

Jozef aimed to nurture a nature researcher in Marcel. Every child
has questions about nature and how things work at a young age. Jozef
noticed and encouraged this curiosity. He engaged in numerous little
conversations about it. Marcel, who was two years old at the time,
asked, 'Daddy, could the sun burn me here?' Our future nature researcher
wanted to know. His father replied, 'Oh, certainly not, it’s far too
far away from you.' Marcel pressed on, 'What if we get closer, Daddy?'
His father answered, 'Well, then yes.' The little boy concluded, 'But
we don’t have wings to fly to the sun.' Jozef interpreted this as
a sign that Marcel had an inquisitive mind and understood the limitations
of human capabilities. The following year, he noted that Marcel asked
thousands of questions about locks, steamboats, and lighthouses. At
three years old, the toddler confidently explained the cause of rain:
'When Marie does the laundry, you see the hot water rising, the steam
also flies out of the locomotive; later, the wind drives that vapor
together, it becomes heavy and falls, and that’s rain.' Jozef taught
the child the names of dozens of flowers and plants and marveled at
his memory. Marcel was restless, working 'tirelessly for hours,' switching
from one game to another. Whenever his father picked up a hammer or
pliers, the child couldn’t be held back. Jozef wrote: 'I must have
it, I will have it!' Marcel exclaimed excitedly, while Papa tried
to calm him with a warning. Marcel stood firm. Mama asked Papa in
French to let Marcel use the hammer to tap a nail into his toy horse,
but Marcel was so enthusiastic that he struck the wheel in two and
left the horse on its back for good. Then Marcel got scolded. Jozef
noted that when Marcel didn’t want to do something, he could become
furious, stamping his feet, rolling around, and throwing himself on
the ground. What is now seen as a phase of toddler stubbornness, Jozef
viewed as a questionable character trait.

On May 23, 1896, Jozef turned fifty. No one visited, but he reflected
on Marcel: 'A trait in your character already requires our attention:
your determination. Whatever you desire or want, it must be! Such
feelings can be for good or ill, depending on how they are guided.
Cowardice and laziness are disgusting, while excessive boldness and
stubbornness are equally harmful. Oh, if only I could guide you as
I wish.' One of his pedagogical aphorisms was: 'A small flaw is always
the beginning of something great.' He feared that this stubbornness
might eventually cause great harm.

Once, he showed Marcel 'views' of the Berner Oberland: 'Amazingly,
the boy sees infinitely better than his father and talks about glaciers,
snowfields, rocks, valleys, and many things his father didn’t understand
until he was eighteen. From time to time, I wonder if this isn't a
case similar to the young man Goethe describes, who no longer recognized
his father due to excessive book learning.' Joseph behaved in such
situations like a wronged man afraid of losing his child. Nevertheless,
he hoped Marcel would become his pride, a copy of his father---'his
image, his entire person, but free from flaws (a revised and corrected
edition!)'. The French addition seems both ironic and serious, succinctly
expressing that Marcel had become the indispensable object of his
father's affection.

Joseph observed that his wife interacted differently with Marcel.
While he demanded obedience, Jozefina would ask, 'Would you like to
make Mama sad?' Then Marcel would deny it and become obedient. Mother
and son exchanged nicknames, such as 'kokootje' for Marcel and 'poes'
for Jozefina. Their physical affection moved Joseph but felt foreign
to him.

Together with Marcel, Joseph designed 'moral' stories. One of the
tales, Koolzwartje, was meant to encourage cleanliness. This version
of Snow White dealt with wetting one's pants, being dismissed from
the army due to incontinence, and girls giggling while pinching their
noses. The story was effective, noted Marie van Pamel, the new maid
from Zeeuws-Vlaamse Retranchement. Marie also told Joseph that Marcel
refused to listen to Bluebeard and had no interest in stories about
cannibals. He couldn't bear to hear about the evil queen wanting to
kill Snow White. Joseph admonished her not to read such fairy tales.
The parents wanted to shield Marcel from everything ugly and bad.
They also found the fear of Sinterklaas and Zwarte Piet inappropriate,
so Joseph explained the origin of this tradition to the child. He
didn't realize that the Sinterklaas celebration plays a role in regulating
children's fears. Gradually, the child detaches from the magical world
and stops believing in Sinterklaas. When Marcel eventually faces real
obstacles, will he be able to cope?

From his fourth year onward, Marcel played outside, where he learned
to speak French with the neighborhood children. The child adored tools
and drew them everywhere. The Guldenvlieslaan was filled with trains
he had drawn in the sand with a stick. Joseph had to adjust his expectations
for the future. The neighbors predicted that Marcel would become an
engineer. In that case, he could go to a vocational school in Krefeld,
Germany, Jozef thought: 'If he doesn't understand something, he won't
let you go until you've given him a sufficient explanation, and that
explanation must be not only understandable for him but also very
clear and complete.' His playtime with Marcel took on a scientific
dimension. For example, when spinning tops under the same conditions,
he would vary the thickness or length of the string, and they would
both observe the result together. When Marcel hoopled with his right
hand, Jozef also had him use his left hand, so that he learned to
handle both hands smoothly. This didactic approach worked perfectly.
When the five-year-old saw fireworks for the first time, he asked:
'How do they get those different colors; how is fireworks made, and
how can that spinner rotate?' Jozef had no doubt that his boy had
an aptitude for natural science. Reflecting on the seven-year-old,
he thought: 'Which profession or business does Marcel have a inclination
toward? Lately, he keeps talking about being an electrical engineer;
words without meaning for him; I merely feel that the word electricity
holds a magical appeal. To know how it originates and works---that
is what attracts him now. Many people say: Marcel is born to be a
professor. Indeed, his methods of reasoning are clear; his enthusiasm
and movements are captivating; he takes examples where necessary,
draws comparisons or illustrates.' He found other qualities equally
important, such as devoutness, dedication, cooperation, and love for
peace. He taught the child to pray: 'May you be blessed, sweet evening
rest, as I solemnly place my hands on his little head and give him
a goodnight kiss, after which he reciprocally commends me to God's
care---it is sanctifying to have such innocent little hands placed
on you. And finally, that silvery voice innocently reciting the Lord’s
Prayer, wishing Christ and Mary a good night, and promising always
to be sweet; that is the most beautiful hour of the day.' Jozef found
his darling somewhat sickly: especially his airways gave problems.
As an infant, he had undergone a tonsillectomy. At the age of five,
that bloody procedure was repeated, which made the child deathly afraid
of white coats and medical instruments. Marcel used an inhalation
device for his daily dose of looizuur, had digestion issues, and displayed
a yellowish complexion. A lighter menu might have helped... Jozef
wrote that their 'daily bread' consisted of roast beef, veal, meatballs,
plaice, sturgeon, ray, fried tongue, sunfish, and plaice. Marcel swallowed
strengthening every day Fosfatinepap, a glass of red wine with sugar,
two beers, and two tablespoons of steel syrup. The rich Flemish life,
the land of Kokanje which most children could only dream of, was a
reality in the Minnaert household.

\section*{Going to school in beautiful Bruges}

Jozef’s starting point was that education about nature should take
place on-site, preferably in the open field: 'I draw his attention
to where the sun rises daily and teach him to see how it appears more
to the north every morning and moves much further west in the evening,
how its path grows longer and longer, making the day longer too. I
will therefore try to teach my boy to see well and to find out through
experience himself! No, it’s not enough to share knowledge with a
child; they must seek it themselves, learn to exert themselves. The
soul must work, suffer to become strong later, just like our body.'
He knew from experience that teachers sinned against this principle
of self-reliance. There was no compulsory education at the time. Home
schooling seemed a desirable alternative for a child who could work
independently.

The intensive interaction with adults encouraged Marcel to play with
language, as evidenced by this pun. 'Marcel asks for chocolate. 'No,
no,' says mama, 'that’s not allowed! It will make you sick!' 'I find
that very pleasant,' replies the scamp. 'What,' mama repeats, 'you
find that pleasant? Like this!' 'Yes, little one,' says the rascal
laughing, 'you say: to make music.' Jozef taught him how to play billiards
at the liberal cooperative De Eendracht. He found billiards a noble
and refined sport that exercises the body in all directions while
the mind thinks and calculates. Additionally, Jozef gave him carpentry
lessons every week. Marcel took up portrait drawing himself: the results
pleased his father.

From Marcel’s sixth birthday onward, the Minnaerts switched to systematic
home education. Jozef took care of music notation, his wife handled
reading and writing. Once he started reading, in Dutch and French,
there was no stopping him. Marcel read 'Cudlago, or The History of
the Young Eskimo' in two days, made harpoons, and went hunting with
them. The child wanted to turn his thoughts into actions! That was
in him. \textquotedbl Following this, Robinson Crusoe came along,
which he devoured. They immediately purchased a bookcase for Marcel
and bought the Dictionnaire universel par Larousse from Aunt Nathalie
for 400 francs, which would be 2000 euros in today's currency. If
something could promote Marcel's self-study, money played no role.
Jozef had to urge the book lover to go play marbles and hopscotch
with the neighborhood children.

At the start of the 1899-1900 school year, both parents coordinated
their schedules. They reduced their workload so they could personally
tutor Marcel for a few afternoons. The boy would start attending a
municipal fee-paying school at Easter 1900, aged seven. The parents
designed a preparatory curriculum: 'In the afternoon, our guest learns
division with two digits. He reads the book Te kust en te keur, draws
and colors, plays the 74th piano lesson, reads De geschiedenis van
België, and asks me about the attraction of the stars.' Singing and
making music played a prominent role at home. Jozef found it hard
to send Marcel to school. Wouldn't he develop a dislike for learning
if he were forced? It also meant that the boy would enter a new environment
and distance himself from his father. Jozef must have been apprehensive
about that as well.

For the first day of school, Marcel needed a full satchel: 'A complete
temple of knowledge for a 7-year-old toddler,' Jozef mocked. But Marcel
was full of enthusiasm, and Jozef could see how eager he was to go
to school. He didn't hesitate to help him with his homework. He urged
Marcel to write more neatly, but Marcel couldn't finish on time at
school if he wrote slower. Jozef was full of criticism: 'In the evenings,
I spend a considerable amount of time correcting mistakes and handwriting,
erasing stains. It's really unfortunate that our child learns so quickly.
They tire his mind by memorizing; fortunately, his memory is good.
We were told that the students wouldn't have any homework, yet...
Every day he has lessons to learn.' He disapprovingly went through
the weekly schedule: 'Our boy's subjects: Catechism, third question
and answer of the 11th lesson, History, A monastery in the 7th century,
French language, le verbe (radical and endings, 4 conjugations, present
tense endings, imperfect and future tense),' Calculating (divisions
of five-digit numbers by divisors 2 and 3), Homework: Replace the
underlined masculine nouns in sentences with given feminine substantives,
etc. Unripe fruit is not healthy.’ He found it nonsensical: ‘The Influence
of Monasteries was taught today, a subject for statesmen, philosophers,
and seven-year-old boys. Oh science! What fools are being made in
your name for future times.’

Marcel had his own world and thought it wonderful. The weekly proof
notes consistently mentioned ‘very good’. Jozef had told Marcel that
if he ever got into trouble, he would immediately and permanently
leave school. That was a bold threat from the jealous father towards
a well-behaved child with strict teachers. For Marcel told stories
like ‘about Landman, who was called to the front of the class because
he hadn’t worked, and Mr. Verkest, who mockingly called out: “Yes,
Ju!” and the class repeating it, making the boy turn bright red, and
about Piette, who had forgotten his notebook and whom the teacher
asked: “Piette, where is your notebook?” Every student had to repeat:
“Piette, where is your notebook?”’ Jozef forbade Marcel from laughing
at other children. The child had to carefully observe the differences
between the pedagogy of his teachers and that of his father and learn
in practice what he had to keep quiet about to stay on good terms
with his father.

Forty years later, he said: ‘School was good, but not fun. Temperamental
teachers who sometimes hit or scolded, little freedom, an old building.’
Marcel played and misbehaved heartily: ‘I remember how painters once
were at work, so their ladders stood in the courtyard; and how a ladder
was carefully placed against the classroom door, diagonally, so the
door could no longer be opened. And in the classroom sat a teacher
alone, correcting exams! And I also recall the carbide pellets we
put in the inkwells, which produced acetylene gas that smelled terribly.
Or how we stuck pens into the desks and strummed them to make music.’
That’s the only glimpse Marcel gave of himself, something he probably
didn’t grant his father at the time.

In December 1900, Marcel experienced his first prize-giving ceremony.
He was dressed in white with yellow leather gloves and boots and a
straw hat. He received the prize book: {*}La Basse-cour de Chaudine{*}.
‘Good for girls,’ his father sneered. From his aunt Nathalie in Ghent,
he also received an expensive French book, which his father went to
20 exchange for Bertha Von Suttner’s {*}The Weapons Laid Down!{*}
Jozefina's headmistress gave her {*}Pol De Mont en Vertellingen van
Andersen{*} on Saturday evening. For St. Nicholas, he had already
received {*}Het boek van Tom Tit{*}, full of beautiful physics experiments,
a soccer ball, a stereoscope with 50 cards, a book with 100 chemistry
experiences plus his own stamp with sticks of ink in different colors.

At the end of the school year, Marcel could reap a richer harvest
at the second prize-giving ceremony in the Boterhuis on the Sint Jacobstraat.
He achieved a shared first place for behavior and diligence; he was
the best in religion, and his average across all subjects was well
over seven. Only one student in his class had as many distinctions.
Jozef, who involved himself daily with the homework, added hypocritically:
'What makes us content, happy is not the number of books or rewards,
but that he has been well-behaved and did his best.'

The child wandered through those years in the beautiful city so frequently
sung about. There were no tourists yet; cars and bicycles were a sight
to behold. The Flemish 'primitives' Van Eyck, Van der Goes, and Memlinck,
of whom Bruges possesses an astonishing collection, still had to be
rediscovered. For the Flemish poet Karel Ledeganck, the city was beautiful
but dead. Around the turn of the century, Stijn Streuvels perceived
an atmosphere of boredom over that city of deserted streets and quays.
The fact that Bruges remained intact as a medieval city is due to
the lack of industrial projects for which, in the sister cities, walls
and monasteries were all too eagerly demolished. Karel Van de Woestijne
chose the metaphor of stagnant water: 'Against the walls of the weathered
houses, gray, foul-smelling sludge accumulates.' Georges Rodenbach
would portray the city in his morbid, French-language {*}Bruges-la-Morte{*}
as a metaphor for rigidity and death drive. Contemporary people could
only moderately appreciate Bruges' beauty.

For the children, the calm quays, romantic canals, slate, and cobblestones
held many adventures. Marcel had enjoyed it: 'We usually went to school
in small groups of friends, pumped water at all the pumps that still
stood here and there, leaned over the old stone parapets along the
canals, or slid down the rail of an inclined bridge, played marbles
between the cobblestones... Remarkable how dear a city becomes when
you've grown up in it in such a way!'

\section*{Jozef on education and pacifism.}

Jozef's diary reveals a gloomy and serious man. One of his aphorisms
goes: 'By their fruits you shall know the teacher.' The teacher had
to ensure that 'unconditional obedience' reigned in the classroom,
because without this outward discipline, nothing could be achieved.

He wrote quite charmingly about Marcel, his 'dear rascal' with 'that
roguish look, that smile of happiness spreading across his entire
face, those silken eyelashes and black velvet eyebrows, the dark blush
(tanned by the sun) on his cheeks, dressed in that new suit which
gives him a much smarter appearance: the wide sailor collar turned
up, the trousers pulled up above the knee, the straw hat on his short-cut
hair.' On another occasion, he described the child as a rose bud 'which
I watch opening, observing how the sepals recede one by one as the
crown swells, how it slowly unfolds, displays the liveliest and most
glorious colors, and allows me a glimpse into the heart of the flower
with its mysteries and wonders.' He struggled with conflicting views
on education: 'Now they say, \textquotedbl Help the child with his
studies\textquotedbl ; over there, \textquotedbl Let him acquire
independence\textquotedbl ; this one prescribes, \textquotedbl Keep
watch over your child\textquotedbl ; and that one preaches, \textquotedbl Do
not keep him in check\textquotedbl ; \textquotedbl Measure and boundary,
adapt to circumstances, reset the boundaries as the tide flows,\textquotedbl{}
another wisely suggests. True, people, but where is the boundary,
what is the measure, when does the tide flow? That's the knot, and
I'm confused.' He needed certainty, but raising Marcel shook his pedagogical
principles thoroughly.

His wife often tried to cheer up her despondent husband. She persuaded
him to accept his brother Gillis's proposal to participate together
in a theater competition. During the Christmas vacation of 1898, they
wrote {*}Siddhârtha or The Star of India{*}, which indeed won first
prize. The authors wrote about the origins of Buddhism, about the
metamorphosis of the powerful prince of Kapilawastu into the vulnerable
Buddha who tried to break through the Hindu caste system. They found
much appeal in the doctrine of renunciation and self-denial. It was
an impressive work that was not only published in Ghent but also,
as a golden opportunity, in Amsterdam. Jozefina persuaded her husband
to put on his gala suit and broke the rule barring women from attending
the celebration by simply joining the crowd. She wasn't even removed.

On January 1, 1900, Jozef looked back on the Century of Enlightenment.
England, he believed, was waging an unjust war against the kindred
'Boers. The British invented concentration camps for women and children.
Jozef and Jozefina spoke at solidarity meetings and took Marcel along,
who, already 28, had met his atheneum teacher Hippoliet Meert. The
child would have internalized two lines of thought: a pacifist attitude
of respect for life and the belief that in war there is always a 'good'
side that has the right to resist injustice and oppression with arms.
Jozef was an opponent of militarism. Many Flemish-minded people disliked
the army, which in Belgium was a bastion of French-speaking conservatism
and anti-Flemish sentiment. Jozef glorified Von Suttner's pacifism,
who advocated that war expenditures could better be spent on social
services. He gave Marcel her thick book, which would earn her the
Nobel Peace Prize. If he had to take stock of the century, progress
nonetheless tipped the scales. In the end, Jozef was optimistic.

\section*{Story book for Marcel}

Marcel's parents became seriously ill when Marcel was about six years
old. His mother recovered. At the end of January 1901, five out of
six doctors she consulted said she needed surgery. From February onwards,
she stayed home from school. On April 12, she was taken to La Sainte
Famille on Fabrieksplaats in Ghent. Jozef wrote in her prayer book:
'A better future is already dawning, let us wait and endure. Lord
God, we trust in Your Love and reverence your holy will. Yours, Jozef
forever.' The operation was successful, and there were no complications.
After two weeks, Marcel was allowed to see her briefly. Jozefina noted:
'Dag poes! is his sweet word, a name he often gave me laughing in
his childlike love.' She stayed with her mother in Ghent for a few
weeks to recover and prepare things.

Afterwards, Jozefina and Jozef made plans to move to Ghent and agreed
that Jozef would take early retirement. They felt Bruges did not offer
Marcel sufficient prospects: the Royal Atheneum and the University
of Ghent awaited him. That school year, Marcel had undeniably been
the best in his class. He would attend a fee-paying school in Gent
on Onderstraat. From their marriage onward, they had saved for this
move to Ghent and even speculated. In 1900, they invested 4,165 francs
to ensure Marcel would receive the maximum pension of 1,200 francs
from his fiftieth year onward. In 1902, they suddenly lost 40,000
francs in investments. The fact that they retained a similar amount
of capital was little consolation for Jozef, who directed bitter accusations
in his Diary toward the king, ministers, parliament members, newspaper
writers, and exchange agents, who even managed to deceive the educated
bourgeoisie with their escapades. Nevertheless, they were able to
convert a life of saving and investing into a construction project.
The decision by Jozefina's mother to sell her house in Bruges and
distribute the money among the four daughters was helpful. Marcel's
parents decided to build two large neighboring houses on Parklaan.

Meanwhile, Jozef suffered from chronic stomach pain, which had plagued
him since Marcel's early childhood and likely contributed to his melancholy.
Yet he had never missed a day of school. On August 14, 1902, at the
age of 56, Jozef retired. That same day marked Jozefina's last day
of school; she went on leave to care for her husband in Ghent. In
his beautiful new house, with an unobstructed view of the city park,
Jozef was tormented by ailments. A few days after his retirement,
he developed a sore, a 'seven-eye ulcer,' which he treated with phenol
fumes. Doctors had to remove a cyst from his eye. Hemorrhoids bothered
him as well. Jozef's last notes, dated July 7, 1902, concerned the
visit of the Chinese crown prince: 'Marcel now sees real Chinese people
in gala costumes, and our uncivilized public laughs openly at those
yellow people, shouting, \textquotedbl What ugly creatures, how ridiculous!\textquotedbl '
As they left the Vrijdagmarkt, they encountered Prince Albert's car.
A spontaneous ovation followed, which Jozef found embarrassing: 'How
the people admire those who have done nothing for their happiness
so far! Oh, those titles, the dazzlement brought by birth, that idolization
which seems to be an inheritance. Will humanity never learn to appreciate
true greatness? Wait until there is reason to praise!' Jozef believed
life had value only when a person had the courage to stake it on their
principles if necessary. The best part Of a people who had a sense
of freedom and noble ideas for which they were willing to make sacrifices,
but 'that hope' wanted only to be left in peace.

Jozef walked around dejectedly among some quacks and described their
perfidious practices. Finally, in response to Jozefina's questions,
a doctor stated that her husband had advanced stomach cancer and was
doomed to die. This was followed by six months of suffering. During
those months, Jozef, encouraged by his wife, edited old stories into
the collection {*}Reality and Ideal{*}. He dedicated it to Marcel.
A sentence from the introduction: 'To be happy, I needed a being to
whom I could fully dedicate my love, and that was the child.'

On Tuesday evening, January 27, 1903, at half past seven, Jozef Minnaert
passed away. On April 27, in the {*}Diary{*}, within mourning borders,
instead of Jozef's calligraphic script, appeared Jozefina's faint
and clumsy handwriting. With the help of Jozef's pious niece Elodie
Pepijn, she had the sacraments administered to him: 'Two days before
his death, he said, \textquotedbl My body consists of two parts:
a lower part that is cold and dead and hangs as if on threads, and
another that still lives and feels.\textquotedbl{} Indeed, that lower
part was cold as a corpse, and I tried in vain to warm it.'

Jozef had died in harness. That morning, he had begun dictating a
story about Gerard Duivelsteen of Ghent to Jozefina: 'He made the
sign of the cross with his emaciated hands... and blew out his last
breath... A deliverance for him! But for us! For me!... My wonderful
married life lay behind me, my good, brave man lay there like a corpse,
and I could say nothing more to him... You remain with me.' When Jozef
died, Jozefina was 46 years old.

\section*{Jozef's Farewell Letter}

Jozef had realized that the role of guide toward Marcel's adulthood
was not destined for him. He therefore wrote a farewell letter. On
the envelope it read: 'To my dear son Marcel. To be opened after my
death.' The letter was a will, full of instructions and commandments
that must have made a great impression on the child.

Jozef commanded the boy to love the open field and wide spaces: 'Love
nature; she knows to comfort, to move, and to educate towards something
great and sacred; what is petty, musty, or impure is purified by her
fresh breath of life; noble thoughts come to you on the breeze and
penetrate deeply within you, rejuvenating the entire soul. There is
something overwhelming in the sight of the infinite sky with its millions
of suns, something that makes trivial worldly thoughts vanish, dispelling
all worries. It would be right to say of life in nature: 'One feels
closer to God.' These words Marcel would certainly take to heart.
The main thing was Christian love; there was little elevation in materialism.
The soul's inclination towards perfection must be fulfilled: 'Remember
that religion does not consist of words but of actions, and that there
are only two commandments: to love God and your neighbor.' By fulfilling
his duty, Marcel would find peace. Among these duties were justice,
good faith, dedication, patience, self-denial, honesty, the expression
of love, simplicity, industriousness---all 'jewels in the links of
the chain that surrounds perfection, surrounding God.' Jozef warned
against the impulsive nature of the Minnaerts: 'You have also inherited
passions and tendencies. Keep your calm; on your father’s side, you
come from a hot-tempered family, although several relatives have tried
to restrain this passion. Remain pure and honorable; our lineage has
had a hard struggle against this ailment. Do not be blinded by the
vain assurances of others that you are one of the calmest people;
your mind will loudly teach you otherwise; in our family, the fire
smolders under the ashes for both aforementioned passions.' Jozef
also issued guidelines regarding chastity: 'Beware of the first step,
it is dangerous and leads to a path from which it is difficult to
return.' The most important thing was choosing a good partner: 'There
will come a time when you---most likely---will choose a life companion.
Do not do this too soon. Ideally, I think around 28 to 30 years old:
then the body is fully formed and in its prime; the childhood shoes
have been outgrown, and the time of maturity, of true seriousness,
approaches.' Jozef warned against sexual passion: 'With what care
one prepares for an exam, because life’s existence often depends on
it. But with what foolishness, genuine madness, people act when it
comes to marriage. One sees a dear person once or a few times. Her
charm delights (soon the senses and passions take over), reason and
judgment are no longer exercised, one becomes entangled and believes
oneself deeply in love, and... later the disasters come, irreversible
consequences of foolishness.' Hence his recipe: 'When you are around
27 years old, look around, look around without being distracted; \textquotedbl{}

\textquotedbl Show some plan; choose between respectable girls (the
proverb says, 'Once light, usually remains light'), take someone a
few years younger than yourself, almost of your social standing and
intellectual development (what could you say in your family where
you could have a pleasant conversation with a woman who is not very
intellectually formed and cannot grasp things), man and woman must
have respect for each other and truly find their better half in each
other, seeking to achieve perfection. Therefore, hold the woman in
high esteem, never degrade her; beware of scolding. Should a misunderstanding
ever arise between you two, never let the night pass with discord,
but reconcile as soon as possible. Every disagreement must be resolved
within the home and only between husband and wife; if you are still
upset, seek calmness in nature and know that the greatest of the two
is the one who first speaks the word of reconciliation or offers their
hand for it. Take such a spouse who sews well and knows useful crafts
(mending, darning, making new children's clothing). Pray for such
a woman and do not live with other people (in-laws or family) within.'
Jozef poured his lessons into commandments about which no communication
was possible. He had needed the child during his lifetime to stand
firm. Now he wanted to determine Marcel's fate from beyond his grave.
Marcel could rebel against his father's compulsion, but this would
hurt his mother. She was subsequently shielded by what Jozef called
the highest commandment: 'Be the support of your mother, for you can
never realize how much she has loved you, what she has suffered for
you, the fears and worries she has endured; how courageous she has
been for your sake. Many children love their mothers, many esteem
theirs as the best, but solemnly I can assure you, child, that yours
is among the very best, that she equals my own mother and that is
her highest praise. If there were a word powerful enough to express
all her noble qualities, her self-denial, diligence, warm devotion
to you and me, then I would speak it, and write this word as a testimony
of my veneration, my gratitude, as strong proof, as a solemn appeal
that you must honor your mother and carry her in the palm of your
hand.' Here Jozef made a confession. His own mother had been the best
of all women. His wife Jozefina had only equaled his mother Coleta.
Jozef, who could speak disdainfully about his father Judocus, had
never freed himself from his mother. He bound Marcel to a lifelong
bond with his mother as well. Yet he had sown abundantly: he had instilled
in the child the love for 'untiring' work, dedication to science,
and the study of nature.\textquotedbl{}

\textquotedbl Much seed had fallen on good soil and would bear manifold
fruit. Jozefina Van Overberge took up the pen in the Diary. She would
write until Marcel could take over: 'Thus, we three will have worked
on your little Book of Life... Here, among us, nothing was ever done
without all of us contributing our part.' Jozefina was clearly aware
of the Farewell Letter and had agreed to her role in the family triangle
and the life path laid out for Marcel. Until recently, she had nurtured
managerial ambitions. Her illness and Jozef's death had changed her
life's purpose. She retired early and decided, in turn, to focus all
her energy on Marcel.

\section*{Jozefina takes over the upbringing.}

For friends and acquaintances, Jozefina produced a brochure about
Jozef's life. Afterwards, she edited his last collection of stories,
which appeared in a print run of two thousand. She wrote: 'Often we
think of your good Papa; his portrait is in every room of my house,
and his image is never far from me. Is it true what he told me a few
hours before his death? 'My spirit will remain with you.' I trust
that.' Once a week, she went with Marcel to the cemetery: 'It seemed
to me that my heart stayed there, and I felt united with him.' Jozef
had asked her to have Marcel take a woodcarving course. She promptly
converted one of the rooms into a woodworking shop and arranged for
a teacher. Marcel practiced twice a week: 'You had a wooden workbench,
various saws, planes, chisels, and other tools.' Marcel carved a donkey
'to place Papa's medallion.' In this way, he got two right hands and
would greatly benefit from it.

Marcel's school performance declined that year in Ghent. At the 1903
prize-giving ceremony, he was fifth in his class. His father's death
must have affected him deeply. That summer, Jozefina went on vacation
with Marcel and his Walloon cousin Robert Niçaise. They took walks
in the Ardennes of more than 25 kilometers. She resolved to spend
a long vacation with Marcel every year and show him Europe. Just as
Jozef had shown him plants and flowers, Jozefina instilled in him
a respect for the vastness and grandeur of free nature. She decided
to keep Marcel at home for the next school year to optimally prepare
him for the atheneum. She resumed homeschooling, made herself available
to him, and devised stimuli to bring fresh content into their family
life. After a visit to the cemetery, they had seen an advertisement
for an Italian method in the tram shelter. Together, they set to work
on it. A study of the Scandinavian languages followed. Vacations in
Italy and Sweden would follow suit. Marcel was making good progress
in piano playing. He performed Haydn's {*}The Creation{*}, parts from
Wagner's {*}Lohengrin{*}, and Mozart's {*}Requiem{*}. They made the
house more cheerful and modern. Jozefina cut copies of famous works
from {*}The Modern Art{*} magazine and put them in new frames: 'This
is how your sense of beauty should develop.' At first glance, Jozefina's
pedagogy differed strongly from Jozef's. She tried to work with Marcel
seemingly as a friend, on an equal footing. Like her husband, she
aimed to enhance the child's achievements. However, she failed to
keep the boy at home for the entire year. School and his peers beckoned.
After Christmas, he moved to the seventh grade of the Athénée on the
Ottogracht. His father had recommended a gymnasium education, so he
had to catch up on Greek and Latin. His mother could teach him Latin,
and together they followed an introduction to Greek. He made rapid
progress. Their joint homework was likely not unrelated to this. Upon
transitioning to sixth grade, Marcel was second in his class and received
honorable mentions in eight subjects. He was the best in arithmetic,
writing, and drawing; second in Dutch and also received a mention
for gymnastics. Jozefina spoiled Marcel. When he wanted a parrot,
she wrote to the manager of the Antwerp Zoo. She got one from Congo
for 26.70 francs, Jacquot, a gray red-tailed parrot. The animal didn't
last a month: 'What a pity after the expenses.' That year, they saw
Wagner's {*}Tannhaüser{*} in Frankfurt. They visited Wiesbaden and
the Germania, the colossal unity monument of the German Empire. Many
Flemish sympathizers rejoiced at the time in the economic and political
successes of the Germanic brotherland. They took walks to the Patersberg
and the Drachenfels and cheerfully climbed the 525 steps to the tower
of the Cologne Cathedral. 

\section*{A teacher becomes a housewife}

Jozefina was an energetic woman. As a housewife, she often spent that
energy on trivial matters which now filled the diary. For instance,
she had hired a second piano teacher, Louis Lozin, in addition to
Marcel's nephew: 'Of course, we hadn't revealed anything to Louis;
but our married niece, Elisa Van de Wijnkel, who was married to Aimé
Pepijn, had accidentally come in as Miss Tilman was giving her lesson
and had quickly passed on the news to Lozin. His wife, Elodie, came
to inform us that Louis no longer had time to give piano lessons.
I understood the reason well enough, showed nothing, gave her the
small gift we had brought for her from our trip, and we parted as
seemingly good friends. Since then, we've heard very little from her
family. Therefore, I decided never again to receive two different
people in the same place, and for that reason, I will: 1) provide
the large room on the ground floor, where four spaces could actually
be created, with a lighter to turn on when someone arrives; 2) furnish
the second salon as a billiard room; 3) install doors so that the
single large hall is divided into four parts.' 

This peculiar account also provides an image of the increasing luxury
at Parklaan 72. Jozefina could hardly put herself in others' thoughts
and always sided with her own perspective. An eyewitness, Dé Fornier,
remembered Jozefina and Marcel's visits to Gillis Minnaert's wife
on 33 Gebroeders Van de Veldestraat: 'Marcel's mother only talked
about Marcel. I sat there quietly as an eight-year-old child. Marcel
was not allowed to look at girls, read novels, and had to study all
the time. He sat at a corner of the table, and she made sure he worked.
His mother was one of those self-assured, tasteless Flemish women,
typical for the emerging intellectual middle class: a fat woman with
one object in her life. His niece Marie, with whom I was intimately
befriended, thought that Marcel was under his mother's thumb. Marie's
parents were kind people who didn't think to interrupt her chatter
in front of her son.' 

Jozefina could afford a grand lifestyle thanks to Joseph's pension,
her allowance, and the rent from the neighboring property. She bought
land and financed the construction of two more houses on Muinklaan,
at the corner of Arendstraat. After the Dutch Marie, who married shortly
after Joseph's death, the maids changed rapidly. She wrote after hiring
another fifteen-year-old girl: 'I pay her 10 francs a month for now:
a trifle.' The better Jozefina's situation became, the less there
was for the maid. 

At Easter 1905, she decided to go to Paris. Twelve-year-old Marcel
was allowed to plan the trip. They traveled by carriage from the Gare
du Nord to Hotel Central on Rue du Louvre. They attended Wagner's
Tristan and Isolde; saw a fair at Olympia, Verdi's La Traviata and
Mascagni's Cavalleria Rusticana at the Opéra Comique, Buffalo Bill,
and the film Tom Pit, le roi des Pick-pockets at Châtelet. Additionally,
they experienced the famous L'Aiglon with Sarah Bernhardt in the lead
role. They visited the Sèvres museum with its porcelain, Versailles,
the Père Lachaise cemetery, the Conservatoire, the Observatory ('with
special permission, we were allowed to visit the halls, see the large
Equatorial in operation, and watch the dome turn'), Les Buttes de
Chaumont, and the workshops of the Louvre. Marcel drank greedily from
the cup of Parisian culture. 

In the 1904-1905 school year, Marcel became 'first in excellence':
the best in Dutch and French, second in arithmetic, history, and geography;
third in Latin and fourth in drawing and gymnastics. He received distinctions
in all subjects: 'A remarkable result that makes his mother very proud
of her child.' 

\section*{A golden child in a golden cage}

The projects with Marcel accelerated: his cage was gilded. Jozefina
stimulated his scientific interest by purchasing expensive equipment.
For St. Nicholas in 1905, he was allowed to buy electrical equipment,
including an electric motor, several bichromate batteries, a Ruhmkorff
inductor, an electromagnet, two Geissler tubes, and a Leyden jar.
They conducted spectacular experiments with this electromagnetic equipment
'together.' 

She bought a billiard table for 550 francs as a decoration for the
second salon, on which Marcel had to teach his mother: 'You start
by showing her a new carambole every week; you seem like a good professor,
I a good student, and we are best friends.' A sturdy gas lamp was
hung above the billiard table with a 'large fire' to heat the room.
Jozefina had doors with glass panels installed and set up a bathroom
with heating. 

A new distraction came in the spring of 1906 when Marcel received
'a clean bicycle,' first-class, with 'free two brakes and Englebert
tires.' He insisted that his mother also buy one for herself: 'I decided
to resign myself to it and learn how to do it. I won't mention the
fear and palpitations I initially felt, nor the sweat that dripped
from me at first.' They practiced at the Vélodrome and went on bike
rides along the Leie. Once, they biked back and forth to Bruges, over
90 kilometers: 'The worries do make me a bit nervous, but what compensation
I find in you---how loving, kind, and dutiful you are! You're a golden
child, full of wonderful promises.' 

They lived a reclusive life. Jozefina probably thought her boy could
find all the entertainment he needed at home. If Sunday visitors occasionally
broke the monotony, Marcel would come up with an entire program: 'What
you invent to make your people happy is unbelievable---various experiments
in physics and chemistry, electricity, card tricks, piano playing---I
truly enjoy being around you!' He didn't isolate himself but felt
obliged to entertain the guests. 

Jozefina thought it was time to start making plans for their joint
future: 'We decide to train you as a horticulturist. The reasons in
favor are: 1st, it's a healthy occupation; 2nd, an enjoyable one;
3rd, a highly respected one; 4th, you have the necessary aptitude
for it---love of travel, ease in learning languages, a desire for
research, and a stubborn determination to achieve what you want, love
for nature and flowers; 5th, you have a small capital that will allow
you to establish yourself and await the success of the venture. What
more could one want? You fully agree, you gladly see the plan through,
and we make our plan: at least until the age of 15, so until the end
of the third year of Greek-Latin studies in the atheneum, possibly
until the end of the second year, then three years at the Horticultural
School, affiliated with the university, while you continue with English
and German languages, maybe even start Spanish and take a chemistry
course as an external student at the University. One year working
with gardeners around Ghent, one in Germany, one in Versailles, one
in London. We settle there together, and while we work on your installation,
you take a solo trip, partly in America, partly in Southern Europe,
or wherever, so that you perfect yourself in your profession. Good
advice I once heard told me: 'There are many horticulturists, but
few are learned.' 

She presented this plan to Marcel, who agreed with it according to
her. She went a step further than her husband, who had nurtured his
future plans in the secrecy of his diary. Marcel wouldn't finish the
atheneum, even though he was the best student in his year! Nothing
more was ever heard of Jozefina's plan. Thirteen-year-old Marcel apparently
made no effort to argue with his mother. 

In 1906, she prepared a vacation trip to Germany, 'for the two of
us for about five weeks,' which would be beneficial for Marcel's speaking
skills. It turned out to be a journey with Aunt Nathalie and Cousin
Emiel Minnaert. They visited Kassel, Berlin, Potsdam, Dresden, walked
along the Elbe, saw the Schwedenlöcker, the Tyssaer rock walls which
'were so overwhelmingly grand and wild that nothing we ever saw on
our travels, not even in Switzerland, could compare,' Chemnitz, and
the Thuringian Forest. In Berlin, they attended Wagner's {*}Das Rheingold{*}:
'Needless to say, it made a very favorable impression on you.' Marcel
still had a month of vacation and decided to devote himself to chemistry.
Jozefina encouraged him: 'I gave you the balcony room, bought you
all sorts of products, and you had a new hobby.' That laboratory,
soon to become the study room, became Marcel's sanctuary: a free space
for the development of his personal life.\\

Endnotes:

1. The quote is the first sentence from De Coster in Richard Delbecq’s
brilliant Flemish translation (1914). The last sentence of Part I
connects to this.

2. In the Netherlands, a {*}Normaalschool{*} was called a training
school and is now referred to as a pedagogical academy or PABO. In
Flanders, a {*}regent{*} or {*}regentes{*} is a teacher at a pedagogical
academy who is also qualified to teach in the lower grades of secondary
education.

3. The source is Jozef and Jozefina’s Diary for Marcel. This chapter
quotes extensively from it. Jozef kept the diary from 1893 to 1902,
and Jozefina continued it from 1902 to 1916. Minnaert Archives.

4. The word 'Flemish' is used here for the language spoken in Flanders
at the time and now called 'Dutch.' The author uses the term 'vernederlandsing'
(Dutchification) for the former goal of the Flemish Movement, unless
he is quoting directly. This was the early objective of the liberal
branch of the Flemish Movement, which Minnaert initially identified
with.

5. Few people wonder who could have heard those final words...

6. The Flemish use 'professor' to refer to a teacher or educator in
Dutch. In the Netherlands, Jozef’s course was likely called a {*}hoofdacte{*}
(head act), at the time an extremely solid requirement for becoming
a primary school principal.

7. Laurent, 1880. Minnaert, G.D., 1919.

8. Portrait of Jozef Minnaert in the photo section. The left side
shows a strict aristocrat, the right side a serene, service-oriented
man: Judocus and Coleta combined.9 Citations from Joseph's notebook.
Minnaert Archive.

10 Portrait of Jozefina Van Overberge in the photo section.

11 This is the first time an amount in 19th-century francs is mentioned.
I follow Dedeurwaerder, 2002, in his conversion to contemporary currency.
For the guilder, this means a multiplication factor of around ten,
and for the euro, around five. The annual tuition fee amounted to
400 francs in 1880, or approximately 4,000 guilders in 2000. This
factor of ten for guilders is very convenient, even though it has
since become anachronistic.

12 De Clerck, 1979, pages 13-15, 21-22.

13 De Clerck, 1979, pages 46-49.

14 De Clerck, 1979, pages 54-62, 67-71.

15 De Clerck, 1979, page 38 and following.

16 Van de Woestijne, 1902, in De Bom, 1939. A beautiful image of Bruges
from a generation before Minnaert is provided by Michel van der Plas
in his biography of Guido Gezelle.

17 Van der Plas, 1990, portrays Gezelle as a political street fighter.
Especially chapters X and XI.

18 Municipal education included free schools and paid schools to which
parents had to contribute. The latter type of education was better
organized.

19 The only reflection on Minnaert's youth is found in letters from
October 8 and 12, 1943, to his youngest son Boudewijn, written from
the prisoner-of-war camp Sint Michielsgestel. This is also where the
passage about playing in Bruges later on comes from.

20 Bertha Von Suttner, {*}War and Peace{*}, Munich 1900, translated
into Dutch as {*}De Wapens neer!{*} ({*}Lay Down Your Arms!{*}).

21 Karel Lodewijk Ledeganck, {*}Aan Brugge{*} ({*}To Bruges{*}), 1883.

22 Stijn Streuvels, 1954.

23 Karel van de Woestijne in {*}De Bom{*}, 1902.

24 Georges Rodenbach, Paris 1893. Note that this is Marcel's birth
year!

25 Marcel in a sailor suit in the photo section.

26 Gillis Desideer Minnaert in the photo section.

27 Minnaert, Gillis D., and Jozef, 1901.

28 The monthly magazine of Meerts Algemeen Nederlands Verbond (ANV),
{*}Neerlandia{*}, was dedicated to the Boer cause. More than 20,000
children would die in these camps.

29 One of his aphorisms.

30 Minnaert, J., 1903b.

31 Minnaert, J., 1903a.

32 Information derived from Jozefina's {*}Levensboekje voor Marcel{*}
({*}Marcel's Life Book{*}), the continuation of the diary.

33 Interview with Dé Fornier. See the notes in part I.

34 The emphasis on 'we' is by the biographer.\textquotedbl{}

\chapter{Under the Spell of Wagner}
\begin{quote}
'It is characteristic of primitive peoples and especially of the Germans
to closely connect nature with the human spirit.'
\end{quote}

\section*{Marcel throws himself into social life}

At the Athénée Royal during the academic year 1906-1907, Marcel took
German classes from Hippoliet Meert, who was already known to him.
At the time, Meert was an advocate for farmers and a promoter of the
Dutchification of the University of Ghent. Marcel joined a students'
circle. In his first lecture, he discussed 'how industry utilizes
waste materials,' which was then an original theme. During a presentation
about Reinaert de Vos, he could rely on his mother's expertise. They
discussed the content together: 'Now it seems you have made up your
mind, and we will get to work. I can rightly say that we want mama
to stand by you faithfully and help you with your work many times.'

Marcel dutifully received his Holy Communion. Jozefina decided to
provide him with sexual education when he turned fourteen: 'Hair is
already appearing around your mouth; you are a head taller than any
boy of your age, and you are beginning to desire knowledge of certain
things that every young man of your development should know, no matter
how delicate. You have already had a few conversations with me about
this, to inform you of the most necessary things; mama insists that
she provide you with this instruction, as it will be done in the most
delicate way.' She could have involved Marcel's guardian, her brother-in-law
Gillis, but she did not find it necessary.

Jozefina meticulously noted Marcel's successes in her Diary. She wrote:
'At the beginning of 1907, you resigned from the students' circle
at the atheneum and preferred to become a member of the young men's
society De Heremans' Zonen (DHZ), Flemish and progressive. This aligns
more with your inclinations and upbringing, and this group is also
more serious and better organized. There you give your first lecture
on Wagner, accompanied by piano arrangements of the main motifs from
the Tetralogy. You receive great success, so much so that you are
almost pointed out as a future chairman. However, I do not strongly
support this, nor do you, so you would not accept it if asked.'

Marcel would rework that lecture into his debut in {*}De Goedendag{*},
the Flemish magazine for all Flemish-minded students. However, by
the fall of 1907, he did indeed become chairman of {*}De Heremans'
Zonen{*}. Marcel didn’t bother to consider Jozefina’s advice but simply
set it aside.

That school year brought a new joint project: 'We decide to start
a collection of musical instruments and begin around November 1907.
It is mainly during the fortnightly public auction at Mr. Mattijn's
that we can stock up; and if we add what we gather on trips, in other
auctions, our number of instruments soon increases quite rapidly:
we acquire, one after another, a square piano; a Chinese gong; a guitar;
an English horn; several flutes, violins, cellos, and a double bass.
All bought at a modest, sometimes ridiculous price.' Jozefina also
bought Marcel a new harmonium. Meanwhile, the boy had three music
teachers at home: for the piano, the organ, and solfège.

Shortly after, Jozefina bought a microscope at an auction for sixty
francs: 'And now an entirely new world has opened up to you. What
countless discoveries in the infinitely small, what joy in research!
The natural sciences become a favorite subject for you and seem to
captivate you to pursue your career in them.' With the microscope,
which his father had promised him at birth, Jozefina truly hit the
mark.

In Jozef’s time, Marcel had to play outside, even though he’d rather
bury his nose in books. Now, on Parklaan, Marcel had his study room,
a billiard table, a chemical and electrical laboratory, a piano, an
encyclopedia, a harmonium, a woodworking workshop, and the microscope
plus collections of books, musical instruments, shells, and stamps.
For outdoor activities, there was cycling, photography, walking, botanizing,
and shell collecting. Jozefina tried to anchor their time together
in every possible way. In Bruges, Marcel had been a good student,
but in Ghent, he became the pride of the school. According to Jozefina,
Marcel made strong progress across the board: 'You are so intensely
focused, even too much so, as you don’t have a single moment to lose.'
The highly gifted child not only needed many stimuli but also worked
‘unrelentingly,’ allowing himself no rest. He always had to study,
write, Meetings or developing a test. Jozefina's personal traits and
upbringing influenced each other and constructed the best Belgian
student of those years. He was also gripped by the Flemish Movement
and followed in the footsteps of predecessors who had fought for Dutch-language
education. In that context, Jozefina and Marcel traveled to the Netherlands
in the summer of 1908: they participated in the General Dutch Congress
in Leiden and visited numerous cities.

\section*{Following Heremans}

Marcel quickly oriented himself toward the worldview of natural researchers
like Darwin, Lorentz, and Van 't Hoff. His religiosity evaporated,
and in that regard, he distanced himself from Jozef's prescriptions.
Marcel wanted to belong to the elite of students who wrote articles,
composed poems, translated literature, understood physical subjects,
and expressed opinions on politics and philosophy. At his atheneum,
he found such a circle among De Heremans' Zonen. That’s where he became
a flamingant.

The Flemish atheneums at the time housed students from higher classes
and the petty bourgeoisie. Among the teachers were always leaders
who pointed the way to the Flemish Movement. Early pioneers at the
Ghent atheneum included, among others, the poet Julius Vuylsteke and
the Dutch studies scholar Jacob Heremans. The latter had helped establish
the Language-Loving Students' Society 't Zal Wel Gaan (1852). The
name referred to a poem by Vuylsteke, which glorified the determination
of youth:
\begin{verse}
'It will go well and it must go well,

for with us, as head heart and arm

are still warm.'
\end{verse}
The atheneums were then the breeding ground for Flemish-minded circles.
These also emerged at the 'free' colleges, where Catholic directors
even prohibited all Dutch conversations, outside of class hours. The
atheneums were the strongholds of liberalism and 'geuzendom'; the
free colleges, on the other hand, were the breeding grounds for the
Catholic cadre. Many Catholic flamingants consciously chose the regional
language for their poems and considerations. The liberals, on the
other hand, promoted civilized Dutch as the unifying language for
both Flanders and the Netherlands. Heremans, in particular, opposed
linguistic particularism. He once described the West Flemish dialect
poems of the priest-poet Gezelle 2 as 'mud splatters on a tailcoat.'
In 1885, after the death of the founder, student Willem De Vreese
renamed the society to De Heremans' Zonen (DHZ). The aforementioned
't Zal Wel Gaan became the national association for liberal-flamingant
students.

From 1890 onward, the students' circle had also become a publisher
of the monthly magazine De Goedendag. The reference to the plebeian
weapon symbolized both the combative spirit and nostalgia. During
Marcel's school years, the atheneum offered a selection of Flemish-minded
teachers such as Maurits Basse, Oscar Van Hauwaert, Victor Fris, Hippoliet
Meert, René De Clercq, his substitute Jan Oscar De Gruyter, and Leo
Van Puyvelde. In 1907, August Borms was a teacher for several months.

The 1906 edition added that this Monthly Magazine for North and South
was published by the General Union of Flemish Students in Secondary
Education, Voorwacht, thus the avant-garde in proper Dutch, of the
General Dutch Union. By 1908, the magazine was briefly called the
monthly for Jong-Vlaanderen. Jong-Vlaanderen was the collective name
for circles of students and pupils at atheneums, colleges, and higher
education institutions. In the 1906 edition, Marcel's atheneum passed
in review: out of twenty atheneums in Flanders, his school was considered
one of the three (!) where the language law of 1883 was actually enforced.
Upon closer analysis, however, this proved not to be the case. Out
of 350 students, ten were Walloons, while seven Walloon teachers provided
numerous French-language courses. Editor Achiel Petitjean noted that
physics classes should legally be taught in Dutch, but even the Flemish
teacher lectured in French. Neither mathematics nor physics laid a
Dutch-language foundation for studying natural sciences: 'What will
become of people formed in such an illogical way? What will their
interaction with the working class yield?' Petitjean called for a
boycott of the illegal, French-language classes.

The 1907 edition of {*}De Goedendag{*} was also dedicated to the student
J. De Hoedt, a pseudonym for Eugeen De Bock, who had been expelled
from the Antwerp Athénée for insisting on the implementation of the
language law. It was a tragic fact that young people like De Bock,
when it came down to it, found themselves faced with the entire teaching
staff, including the Flemish-minded ones. On April 20th, a demonstration
took place in Ghent advocating for Dutch education at the atheneum.
Petitjean delivered an impressive speech to a packed hall. The study
prefect and the teaching staff were notably absent. The speaker called
their education outdated: 'For too long have students been misled
by the old-fashioned ways. The moment has come for them to demand
a normal education, free from all influence and relying on themselves,
through their own language.' Ghent was in an uproar, {*}De Goedendag{*}
reported. Flemish national consciousness had awoken once and for all.
The youth were the hope of Flanders: 'The elders have traded freedom
for a political program.' During this turbulent time, Marcel had taken
over the torch from chairman Tijtgat and editor Petitjean. Initially,
it seemed his preference for culture and science would keep the policy
away from flamingant action. In his first session as chairman in November
1907, Marcel played Beethoven sonatas on the piano, accompanied by
his nephew Robert Niçaise on the violin.

The Sons of Heremans met every Saturday afternoon, always featuring
a lecture by a Flemish-minded teacher or student. The future poet
Richard Minne gave a lecture in 1908 on the American writer Sinclair.
That year, Marcel also delivered lectures on the Norwegian authors
Björnsen and Ibsen, as well as the Anglo-Saxon epic {*}Beowulf{*}.
Their teacher, the Germanist René De Clercq, sometimes presented his
own poems. If his muse prevented him from performing, Jan Oscar De
Gruyter, another accomplished Germanist, would often step in. Also,
Minnaert was a good reciter. A report mentions his interpretation
of {*}Krijgsrhapsodie 6 Linding{*}. Sometimes he ventured into his
own piano compositions and texts. The group singing of these young
people went straight to the heart. The societies performed at Flemish
rallies, where stirring dialogues between the reciters and the audience
put the listeners in a state of ecstasy. Albrecht Rodenbach’s famous
cry, 'Does the Bluefoot fly?' was answered with a cheering 'Storm
at Sea!'

At the Young Flanders congress in August 1908, Lodewijk De Raet, a
leader of the movement for the Dutchification of Ghent University,
encouraged them: 'In the atheneums and colleges lies the future of
the Flemish Movement; it is there that the people who truly lead the
Flemish Movement today received their baptism by fire, where their
love and dedication to Language and People were born, which have now
become one with their own lives.' The student youth had traditionally
provided the radical element that would push the Flemish Movement
into new paths. Marcel Minnaert had addressed the congress attendees
that same day on the topic of 16th-century Flemish Music! At that
time, he was mainly occupied with his written works, which were influenced
by his reverence for Wagner's music and poetry.

\section*{Marcel’s treatise on Wagner}

Marcel made his debut in the 1908-1909 edition of {*}De Goedendag{*}.
In an article about the Word-Tone Drama in Richard Wagner, he praised
the 'divine' German, who had created a genre that, with the exception
of Hoffmann’s {*}Undine{*}, had not existed before: all arts---dramaturgy,
painting, mime, and poetry---merged into a whole, 'their children
born from one mother, drama; united in sacred love, they live and
work only for her.' Marcel was clearly under the spell of Wagner's
pursuit of a Gesamtkunstwerk and explained how the individual arts
were incorporated into this synthesis. He mentioned, for example,
'the art of mime': 'Here the artist finds a wealth of expressive power:
the general posture that will be proud and powerful in {*}Siegfried{*},
hunched and fearful in {*}Mime{*}; movement, which can be quick or
slow; facial expression, where the eye, mouth, and head position all
play a role here. The speed and form of movement can be compared to
the rhythm of music: just as cheerful, exuberant music matches lively,
quick, and joyful gestures, harmony is also reflected in facial expressions.
The transition from joy to sorrow in facial features is a true modulation
from major to minor keys.

The fact that Wagner chose his themes from the world of Germanic gods
and Vikings struck a chord with Marcel. At the gymnasium, he was overwhelmed
with Greek and Roman history. The revelation came when he realized
there was a Germanic culture with stories rivaling those of Odysseus
and Penelope. Both the Germanic people and the Flemish had their roots!
Marcel wrote that the poetry of Richard Wagner, the content of his
dramas, was anchored in the human soul: 'Music must express the eternal.
That is why Wagner seeks his themes in universal laws, the deepest
feelings of the human soul, the inner drives, the purely human, which
can only be found in legends. It is there that one rediscovers the
national soul in its original purity.' The gods of the Edda and the
Iliad were humans with extraordinary virtues and vices. This was also
the material for the word-tone drama: Wagner had 'seen it with a stroke
of genius.' The German had shortened the verses, omitted the end rhymes,
and replaced them with Old German alliterative verse, thus 'returning
a part of their property to the people. Here too, what is one's own
crowns one richly!'

Marcel adored Siegfried and Brunhilde: the seriousness and pathos,
the absence of any relativization, the unconditional self-sacrifice---everything
resonated with the spirit of the refined teenager. Wagner had depicted
the interaction between culture and nature, which Marcel found characteristic
of the Germanic nature: 'Natural events play a special role, particularly
in the Nibelungenring, as they even intervene in the action, but they
are never accidental because they only occur as expressions of forces
that lead the drama; for example, Froh's rainbow, Loge's fire, and
Wotan's lightning. It is peculiar to primitive peoples, especially
Germans, to closely connect nature with the human spirit; thus, deeply
offended Achilles sinks into sorrowful reflection on the shore of
the thundering sea.' Marcel smuggled Achilles out of the Iliad, but
still...

Wagner's drama tapped into the deepest emotions of the teenager. He
was relatively young at fifteen to engage in such cultural-philosophical
reflections. He must have consulted articles and books. He could discuss
his ideas with his mother, three music teachers, and his godfather
Gillis. He analyzed freely and made significant connections on his
own. The music fueled his passion for piano playing. Perhaps he experienced
Wagner's Word-tone drama as a spiritual reality, just as real as the
world around him. The second part of his reflection may point to this.

Wagner had designed a worthy setting for the performance of his works.
Attending performances in the 'shimmering temple' of Bayreuth was,
according to Marcel, an unforgettable event: 'Everything we are about
to behold is arranged down to the smallest detail by the master, and
we cannot admire enough the spirit who knew how to complete his grand
creation so fully. It is four o'clock; a group of musicians with trumpets
and trombones gathers in front of the theater, and there blares and
resounds a brief fanfare, established by Wagner himself. It is the
main motif of the act that is about to begin. Everyone enters.' Marcel
acted as a guide: 'Suddenly all the lights go out; a solemn silence
reigns in the hall; one detaches from everything around, listens in
sacred anticipation; a minute passes; then the first note rises from
the orchestra. All the sounds merge wonderfully and it seems like
a giant organ resounding, never drowning out the singer, even in the
greatest climax of power.'

Marcel seemed to have indulged his eyes. However, Jozefina's Diary
reveals that Marcel had not been to Bayreuth. He had attended {*}Tannhäuser{*}
in Frankfurt, {*}Tristan and Isolde{*} in Paris, and {*}Das Rheingold{*}
in Berlin. He must have attended concert performances of other Wagner
operas. But he only knew Bayreuth by hearsay. Moreover, nowhere is
it literally stated that he himself visited that 'shimmering temple.'
Perhaps his imagination played a role. He didn’t need to be there
to vividly imagine everything and instead let himself be guided by
the reality in his mind. Be that as it may, Marcel would have solemnly
decided to visit Bayreuth. His mother managed to get tickets for the
performances three years later. Marcel could experience in 1911 what
he had previously described. They visited the Master’s grave and that
of his son-in-law Liszt. They saw the Tetralogy, visited Nuremberg
with its Germanic Museum, and also attended {*}Die Meistersinger{*}
and {*}Parzifal{*}. Jozefina wrote: 'The opulence of those art-filled
evenings amidst such a refined and luxurious audience; the quiet surroundings
of that temple dedicated to art; the presence of Siegfried Wagner,
Hans Richter---all things that remind one of the great master---the
air where his spirit still lingers, where his work continues to live.
Oh, what a pleasure it is to see all this, and yet it’s true that
possessing wealth isn’t a misfortune if one knows how to spend it
well.' If Marcel had not reservations about Wagner and his Germaniana,
his mother did not either. During those years, Marcel gave numerous
lectures on Wagner for Flemish-oriented circles, both for Jong-Vlaanderen
and the ANV. In a report from the Rodenbachs 11 Vrienden van Blankenberge,
it reads: 'Comrade Minnaert spoke about {*}Die Nürnberger Meistersinger{*},
one of Wagner’s most beautiful compositions. After briefly summarizing
the entire piece and showing us some photographs of the picturesque
city of Nuremberg, the talented speaker dissected the work for us
in its smallest details. With what ease he helped us understand the
different motifs of the piece---and this with piano accompaniment.'

His mother described how, as a student in 1913, he had to take over
the Wagner cycle from his old teacher Van Hauwaert: 'The grand hall
of the Notarissenhuis was always packed.' Gent’s singers and vocalists
performed themes under Marcel’s piano guidance until late at night.
Through Wagner, Marcel Minnaert presented himself as someone who wanted
to play a role in Flanders’ cultural and intellectual life.

\section*{The songs of Hildebrand and Volker}

Wagner inspired the adolescent to translate Frankish---hence Germanic---texts
and to write poems in staff rhyme. Part of his output between 1909
and 1911, Marcel published a Germanic saga from the 8th century, {*}Hildebrandslied{*},
translated from Frankish. In it, an old man and a young warrior fight
each other. Hildebrand suddenly recognizes his son Hadubrand, who
believes his father died in foreign military service. The son does
not recognize his father. Hildebrand gives him armlets, which the
son accepts while uttering insults in five-foot iambic verse:
\begin{verse}
'You want to strike me with your spear.

Your life, I believe, was a perpetual lie.'
\end{verse}
The battle is inevitable, the father understands:
\begin{verse}
'Now my own child will cut me down with a sword...

Strike me with his axe... Or I will kill him!'
\end{verse}
Was Marcel unconsciously moved by the theme? A father fighting a son
who does not recognize him? That was once his own father's fear fantasy.
In this Germanic Oedipus, both the outcome of the battle and Iocaste
are missing at the end. For the almanac {*}Zal Wel Gaan{*}, Marcel
wrote Volker's song with the Frankish subtitle {*}Kuener videlaere
diu sunne nie beschein{*} 13. Brave Burgundians under the leadership
of chieftain Hagen are massacred by a overwhelming force of Huns led
by the noble leader Etzel. The piece is full of alliterative rhyme:
\begin{verse}
'Then the wise Huns feared the heroes;

They decided to slink through the night like snakes,

And never again would an enemy hear the clear voice of the rooster.

They were shown their white, woolly beds

In a spacious hall with rich windows,

Greetings: \textquotedbl Goodnight!\textquotedbl , grimacing and
grinding

Their teeth, vanishing into the sultry, black darkness...'
\end{verse}
Marcel's production revolved around heroes and hordes, brave peoples
and cunning traitors, snow-white and pitch-black. This was part of
Flemish-national romanticism. Marcel remained stuck in this mindset.
Everything was either 'good' or 'evil,' just like the parties in the
Boer War, with no nuance whatsoever. His poems reveal his fascination
with Germanic sagas.

Yet he also translated Catullus’s {*}To Lesbia{*} and Heinrich Heine’s
{*}Zeegroet{*}. He wrote {*}Salamis{*}, an impression from the Greek
military camp on the morning of the decisive battle with the Persians,
and translated a battle song about the tyrant-slayers Hermodios and
Aristogeiton. In Sapphic verses reminiscent of his father's elevated
thoughts, he wrote about {*}The Ideal{*}:
\begin{verse}
'So the spirit longs for ideal beauty, Reaches for virtue, freedom,
and pure truth; So every noble soul feels a longing: A longing for
something higher!'
\end{verse}
With his lectures and articles, with his verbal skills and musical
talent, sixteen-year-old Marcel became a beloved Flemish propagandist.
Wherever he performed, he was the center of attention. His prose was
not literary, and his poems were not very poetic. His merits lay in
translation, description, analysis, and compiling essays. He could
explain things clearly, charm with his piano playing, and had a gift
for words.

\section*{A noteworthy outcome}

A striking fact about Marcel was that he surpassed most of his classmates
by far in erudition, knowledge, skills, and general development. There
is no known record of emotions or tantrums, of girlfriends or boyfriends,
of birthdays or parties, of a first kiss or love: likely, he lagged
behind emotionally. Midway through the 1908-1909 school year, his
average grade for all subjects was a nine. At the conservatory, he
won first prize in solfège. Marcel ended that school year as first
in seven subjects. In July 1909, he was invited to participate in
the national competition for students of atheneums and colleges in
the following subjects: Dutch, French, Greek, Latin, physics, and
geography. During the last school week, he completed the assignments
in the six subjects for which he had been selected; he did not participate
in French.

That year, the five-week vacation trip went to Central Europe. Jozefina:
'In this way, we almost completely traversed the Inn Valley on foot
and were enchanted by the crystal-clear lakes, white snowfields, and
wild glaciers.' Marcel found edelweiss among the rocks. They hiked
in the Dolomites, visited Innsbruck, Oberammergau, and Munich, toured
Vienna, took a boat to Budapest, and traveled back via Regensburg,
Nuremberg, Rothenburg, Mainz, and Cologne. In South Tyrol, he encountered
Rhaeto-Romanic. He bought a grammar book and translated a nationalist
poem. These Central European travel experiences would benefit his
work on Wagner.

The results of the national competition left no doubt. Marcel won
first prize in physics (99 out of 100) and Dutch, second prize for
Greek, geography, and history, and third prize for Latin. Jozefina
rejoiced: 'So, a distinction in every subject you could compete in,
which rarely happens since you excelled equally in literary and scientific
subjects; a score unmatched overall. Your results also caused quite
a stir: newspapers published special articles; even professors sent
us cards, everyone congratulated you, and your uncle wrote a kind,
heartfelt letter to ensure your success doesn’t go to your head. I
do the same: don’t become conceited, and remain the good-natured,
humble, cheerful, hardworking boy you are now.'

At the awards ceremony at Ghent University, Marcel received a book
prize worth 180 francs. The boy’s choice of books reveals his broad
interests. Jozefina proudly listed them: '1. Het Leven der Planten
(The Life of Plants), in two volumes;

2. Het Leven der Dieren (The Life of Animals);

3. Schriften und Dichtungen by Richard Wagner, 10 volumes;

4. Leerboek der Chemie (Chemistry Textbook);

5. Leerboek der differentiaal- en integraalrekening en van de eerste
beginselen der analytische meetkunde met het oog op de toepassing
(Textbook on Differential and Integral Calculus and the First Principles
of Analytical Geometry with an Eye on Application) in the field of
natural sciences by H.A. Lorentz;

6. Treatise on Rational Mechanics;

7. Textbook of Musical Composition.’

In short, ‘a beautiful addition to your library.’ The sequel took
place at the Académie des Beaux Arts in Brussels. He received books
that Jozefina disdainfully dismissed as ‘of no significant content,
books meant for decoration, or of a Catholic nature, namely 1. The
Life of Saint Louis; 2. Electricity (we already have it); 3. Poems
by Hilda Ram; 4. Belgian Writers of the 19th Century; 5. History of
Belgium in two volumes by Pirenne (we already have it).’

This disdainful assessment must have come especially from Marcel,
who as a Flemish nationalist had no interest whatsoever in the History
of Belgium as presented by Pirenne or in the foolish Louis XI of France.
By then he had become a Lion of Flanders, of the Battle of the Golden
Spurs and De Goedendag, of the struggle against French domination.
Finally, there was a ceremony at the Conservatory with the usual prize
books. Jozefina wrote modestly: ‘We were truly celebrated now and
set to work with renewed vigor.’

The following year, these events repeated themselves. He chose, among
others, the Complete Works of Charles Darwin. In 1910, he graduated
from the Athénée Royal with the best grades ever achieved. The farewell
evening with a selection of classmates was, as always, in Marcel’s
room. There they promised to meet every Second Easter Day. Jozefina
pondered: ‘Will everyone keep their word? Indeed! No. Who will be
missing? Will death have already taken some of them? Will differing
opinions on politics, religion, or language bring division into your
currently so pleasant and friendly gatherings? I am very curious.’

\section*{The cohabitation of Jozefina and Marcel}

Jozefina wanted to continue in the old way, but Marcel decided for
himself with whom he would engage in work. The princely balcony room
played a key role here and deserves to be described by its decorator:
'A room of four by five meters, a carpet (to keep your feet warm)
made of Tournai tapestry on the floor; upon it, a large table: one
by two meters. A smaller table at the window; a wonderful view of
the trees in the park; on the walls: all bookcases with books, many
covered and numbered; a cabinet with shells, curiosities, collections,
etc.; above the fireplace, instead of a mirror, a frame with fifty
portraits of great men: 'Mirror yourself in them!' I said to you.
- On the mantelpiece: three statuettes: At the Harbor, a replica of
a statue by C. Meunier. The Thinker, a replica of a statue by Michelangelo.
The Youth Removing a Thorn from His Foot, a replica of a statue from
the Vatican. I deliberately chose these three representations of man's
triple labor. In a corner, on the wall to the left of the fireplace,
hang weapons, products from the Congo; and in the cabinet to the right
of the fireplace is the entire music treasure! A tortoise stove provides
warmth and a lamp, a reading light, illuminates your room in the evening.
It’s cozy there. If you open the window on a fine day, you step out
onto the balcony and enjoy a wonderful view of the park and the surrounding
city. You love your room: you’re always there, except during meals,
when we are together, and that time is so short! But for you, it’s
good to have a place where you can work in peace and tranquility.
In that room, you work with pleasure and diligence; there, the editorial
board of De Goedendag gathers to assign articles for the publication.
There, you receive your friends and companions, and on the table lies
that comfortable disorder of someone who rummages through books and
magazines.'

The account contains the reproach: 'So short!' Too short as far as
Jozefina was concerned. She decided ‘to focus more and more on my
house and occupy myself with your education.’ She gives me a great
deal of satisfaction, affection, and wonderful promises for the future.’
The young man spread his wings and flew where Jozefina could not follow.
Father had found it unavoidable to lead Marcel strictly. Mother tried
to bind Marcel to her with a precious shower of projects. She had
furnished his study, which was open to his friends but remained closed
to her. She had to show understanding and seemingly lowered the bar.
By offering hospitality in her home, she got to know Marcel’s friends
from school and from De Goedendag, allowing her to continue playing
a key role. The close bond between Marcel and his mother was confirmed
by their daily life and was the highest commandment of his father.
As a young intellectual, Marcel claimed his place in Flemish life.

In 1910, De Heremans' Zonen had existed for twenty-five years. Group
17 organized the Jong Vlaanderen congress. Minnaert wrote as host:
'In our good city of Ghent, so full of memories from the times of
Flanders' glory and power, the spirit of our ancestors has continued
to live on and still inspires us today with fiery zeal and youthful
courage. Come visit our fatherland, friends; come walk in the broad
lanes of our park and in the narrow, medieval streets, come admire
our Belfry, our Castle of the Counts, and our Shipmasters' House;
come see the wonders that a free people once created! From every tower,
from every stone, the soul of our Flanders will sing to you like a
rustling symphony. Here you will feel how deeply, how intimately connected
you are to our beloved people.' By 'our beloved people,' Marcel did
not mean the people of Belgium.

New forms and ideas

From this nostalgic appeal, it is clear that Marcel was intoxicated
by the rhetoric of the old Flemish Movement. Only gradually did he
allow modern insights from progressive Flemish thinkers like Julius
Mac Leod, August Vermeylen, and Hippoliet Meert to sink in. For instance,
zoologist Julius Mac Leod, against the entire scientific community,
had chosen a Dutch-language path. His lectures in Dutch for teachers
caused a stir. After his appointment as professor of botany at Ghent
University, he continued these lectures in an extracurricular program:
the university extension. He had founded the Dutch-language biological
society Dodonaea (1887) and, alongside the Flemish-Dutch Linguistic
Congresses, which had been running for half a century, The Flemish
Nature and Medical Congresses (1897) were established.

The work of Mac Leod had laid the groundwork for August Vermeylen's
{*}Critique of the Flemish Movement{*} (1895), one of the initiators
of the literary magazine {*}Van Nu en Straks{*}, where he joined forces
with Karel Van de Woestijne, Herman Teirlinck, and Stijn Streuvels.
The journal had created a unique Dutch-oriented focal point within
Belgian culture. In his {*}Critique{*}, Vermeylen argued that an emancipation
movement could not indulge in nostalgic battle songs and declamations,
or in commemorations of the Battle of the Golden Spurs or other manifestations
derived from the glorious Middle Ages. He demanded that the Flemish
Movement become more individualistic, social, and cosmopolitan, more
European. Vermeylen criticized pan-Germanic racism and condemned the
anti-French sentiment dripping from the columns of many pro-Flemish
journals. He wrote passages about the flamingants that lent themselves
to radicalization: 'Flanders, they feel it, they want it; Belgium
leaves them cold: we could happily experience the fact that the Brabançonne
was booed.' For the time being, Marcel Minnaert was still infatuated
with the nostalgic movement.

A modern element was also the General Dutch Union, Hippoliet Meert's
creation, which Marcel and Jozefina had joined in 1906. In Leiden,
they attended the national congress in 1908. The goal was 'to bring
together in a powerful organization all members of our tribe who feel
something for our common mother tongue.' It united people from South
Africa, Flanders, and the Netherlands and received responses from
all parts of the world. Their 'Greater Dutch' consciousness initially
related to language and culture, not political unity. In its formative
phase, it had established a Committee of Inquiry to create a Dutch-language
university in Ghent. Julius Mac Leod would write the {*}Report{*}
(1896) of this committee, and commission member Fredericq would oppose
the monolingualism of the Flemish University, becoming an outspoken
opponent of Mac Leod. Marcel's statements from 1910 were thus fifteen
years behind. In 1909, as chairman of De Heremans' Zonen, he had still
granted the Ghent historian Fredericq honorary chairmanship of the
circle. He must have quickly discovered that this did not reflect
the radicalism for which De Raet had championed at the Young Flanders
congress. Over time, Marcel Minnaert shifted his main focus from culture
to politics. He learned about the Belgian laws won through Flemish
struggle but barely enforced. He advocated for students' right to
education in Dutch and thus came into contact with socialists.

In the final decades of the 19th century, socialism had become a powerful
movement in industrial Belgium. The Belgian Workers' Party (BWP) was
founded in 1885 and became relatively the strongest socialist party
in Europe. Socialists, broadly speaking, supported Flemish emancipation.
Many progressive Flemish-minded individuals belonged to the anti-socialist
wing of liberalism. Socialist agitation for universal suffrage led
to the abolition of tax-based voting rights (1893). Every man over
25 gained the right to vote, though literates could cast two votes
and the wealthy three. Belgium was 25 years ahead of most European
countries. The 1893 elections yielded 28 socialist members of parliament,
over 15\% of the seats, exclusively elected in the industrial, Walloon
regions. The Walloon city of Liège had elected Eduard Anseele from
Ghent.

With the social question, the Flemish issue also emerged. Flemings
within the Catholic and Liberal parties, supported by the socialist
bloc, managed to enforce the Equality Law (1898), placing Dutch on
an equal footing with French as a national language. The principle
was recognized; the practice had to be enforced step by step, inch
by inch. The language struggle at Marcel's atheneum, where Flemish-minded
students demanded the application of the law, was one of thousands
of skirmishes during those years. The privileged did not want to acknowledge
that Flemish aspirations were justified. Dutch was for them the language
of peasants and laborers, and a handful. Masochistic teachers and
notables. The idea that professors and Walloon officials in Flanders
should learn the national language seemed monstrous to them. Their
resistance grew stronger after the Equality Law. Cardinal Mercier,
unlike some priests and many assistant priests, exclusively chose
French: Dutch remained taboo in colleges, seminaries, and universities
controlled by the Church. After all, the requirement to teach certain
subjects in Dutch applied to State Schools, not to free schools.

In the first half-century following 1830, many Flemish-minded individuals
had felt Belgian. This began to change after the turn of the century.
The Flemish-minded no longer contented themselves with a certain degree
of recognition for Dutch in Flanders and no longer sought a bilingual
regime in those regions. They began to demand the complete Dutchification
of Flanders: monolingualism. Leaders spoke of their 'beloved Flanders,'
of the 'popular strength' of the 'dear Flemish people.' When Mac Leod,
De Raet, Vermeylen, or Meert around 1910 spoke of 'our people,' they
meant the Flemings. This spiritual separation from Belgium would intensify
in the struggle for the Dutchification of Ghent University as the
resistance from professors, Walloons, and Francophiles became more
stubborn and unreasonable. Some interest in Flemish demands arose
in the Netherlands. Flamingants sometimes became correspondents for
Dutch daily newspapers. The growing self-consciousness of the Dutch
people unfolded against the backdrop of the economic and cultural
upsurge in both the Netherlands and Flanders. The undermining of France's
authority, which had been humiliated by Germany (in 1870) and had
to cede Alsace-Lorraine, played a role. The appreciation for Wagner,
with his pantheon of Germanic and Nordic gods and heroes, was promoted
as such by the ANV. Marcel went to the University of Ghent in 1910
to study biology under Julius Mac Leod. The battle for the Dutchification
of the university reached its climax at that very moment. Marcel would
choose a uncompromising tactic in this struggle and become one of
the most radical student leaders.\\

1 Everaert 1997 shows figures like Vuylsteke, Anseele, Julius Mac
Leod, Andries Mac Leod, Masereel, Basse, Borms, Clevers with a reference
to the Verdurme case, Debeuckelaere, De Clercq, De Gruyter, De Keyser,
Van Vlaenderen, Fredericq, Fris, Goossenaerts, Minne, Van de Woestijne,
Heremans, Jacob, Meert, Minnaert, Rooses, Van Hauwaert, and Van Puyvelde.

2 Van der Plas, 1990, 201.

3 The 1907 volume shows full support for Eugeen De Bock, the later
publisher of De Sikkel. De Smedt, 1954.

4 The volumes of De Goedendag (DG) from 1906 and 1907 give space to
Achiel Petitjean. His speech on Flemish Education from April 20, 1907,
is printed in full.

5 The quotes from DG, 1907, 105.

6 War Rhapsody by Linding, DG, 1910, 15. A work by Marcel, The Bell
of Wengelen, was performed at the celebration of De Heremans' Sons
on February 29, 1912, with a recitation by Michel Van Vlaenderen (DG,
1911-1912, 93). The Annual Congress of Young Flanders on May 11 and
12, 1913, in Ghent, presented Marcel as the writer and composer of
the play Zanger (DG, 1912-1913, 155).

7 That was, at least, my experience with the images and sounds of
a student procession in Leuven in the 1930s.

8 Minnaert had this motto placed on his house in Bilthoven as an exile:
Blauwvoet House.

9 Speech by De Raet, August 9 and 10, 1908, in Antwerp, published
in DG, 1908, 3-4.

10 Marcel Minnaert, DG, 1909, Wagner's Word-Tone-Drama I and II, 5-6,
5-15; 7, 4-9. Guerber, 1925: the British William Morris says about
the Volsunga Saga: 'This is the great history of the North, which
for our race must be what the story of Troy was for the Greeks - first
for our entire race, and further, if the change in the world has not
made our race anything more than a name for what has been - also a
history that must be no less for those who come after us than the
story of Troy was for us.' Marcel must have discovered the Ring of
the Nibelungen in the same way.

11 DG, 1910-1911, 157-158.

12 Marcel Minnaert, The Hildebrandslied, DG, 1909-1910, 71-73.

13 Marcel Minnaert, People's Song, Almanac 't Zal Wel Gaan 1911, 156-159.

14 Hermodios and Aristogeitoon (DG, 1910, issue 5). In 1910, after
six issues, it switched to annual publication.

In it of Minnaert: Aan Lesbia (10-11), Salamis (27-28) en Zeegroet
(149-151).

Refused were De liefde in het Nibelungenlied, De dennenboom, Aan de
maan, Hymne aan Pan, Er is niets

nieuws onder de zon, etc.

15 Marcel Minnaert, Het Ideaal, De Goedendag, 1910, nummer 2.

16 Rhaeto-Romance, De Goedendag, 1910-1911, 12-14, 32-34.

17 DG, 1910, 79-80. The company of De Gruyter, the Flemish Association
for Theater and Recitation Art, performed Lessing's Minna von Barnhelm
there. See the final sentence of H. Conscience's{]} The Lion of Flanders
(1838):

'You Fleming, who have read this book, consider, amid the glorious
deeds it contains, what Flanders once was---what it is now---and
even more what it will become if you forget the holy examples of your
fathers!' Mac Leod in the Encyclopedia of the Flemish Movement.

18 Mac Leod commemorated in the Association of Sciences, October 1935.
The new congresses also stood alongside the Dutch Natural and Medical
Congresses.

19 Vermeylen, Brussels 1895 (1905, 2nd edition).

20 'University' was identical to 'university' until 1970, even in
the Netherlands. Therefore, a Flemish University of Applied Sciences
was demanded instead of the French-speaking university.

21 Mac Leod, 1897.

22 In Belgium, 28 out of approximately 180: later comparable in the
Netherlands with 2 out of 100. Capiteyn, 1991, 156-157, mentions 184
(in 1914).

23 De Schaepdrijver, 1997. Chapter 1, Before the Storm.

\chapter{Another Mighty Assault}
\begin{quote}
'Flemish youth, noble eagle, do not shrink from the light, but fly
straight to the sun.'
\end{quote}

\section*{Botanical society Dodonaea}

Marcel immersed himself in his biology studies. He completed both
his propedeuse and candidate years with la plus grande distinction.
He joined the Botanical Society Dodonaea, founded by the biologist
Julius Mac Leod. In December 1910, he gave his first lecture on 'The
defense of leaves against animals through thorns and spines.' A summary
appeared in the Botanical Yearbook: 'The flora of a region depends
more or less on the fauna. Where there is the least food for an animal
species, the plants are best armed against that species.' He would
continue with his own research on this Darwinian theme.

Marcel delivered numerous lectures for Dodonaea. The number of participants
ranged between ten and thirty. In 1911, he attended fourteen meetings
and excursions and gave four introductions. The one on April 4th already
concerned his doctoral research on 'Light shade leaves in Pinus laricio.'
That year, besides his lecture on Gossypium, an essay on Light-emitting
Organisms was included in the Yearbook. He cited the case of Schistostega
osmundacea, a crustose lichen whose protonema consists of 'spherical
cells that collect all the light on the few chloroplasts contained
in the cell; these illuminated particles gleam emerald green, as if
they themselves emit light.' Marcel dedicated a fairy tale to this
moss in De Goedendag before he managed to find it during a mountain
hike. In 1912, he gave six lectures, including one on Light- and Shade
Leaves in Ilex aquifolium, which also appeared in the Yearbook. He
investigated whether the number of spines on that tree depends on
the intensity of the incident light. He picked several hundred 'light
leaves' and 'shade leaves' from the Botanical Garden and Laboratoire
de Botanie and found that the shade leaves have more than 3 spines.

The following year, he delivered a lecture at the Flemish Natural
History and Medical Congress on The Influence of Light on Geotropism
in Aquatic Plants. He sent the members of Dodonaea a reprint of his
speech. He demonstrated that aquatic plants shoot upwards when it
becomes dark: they become lighter because they develop more air spaces.
Marcel referred to this as 'adaptation.' On another occasion, he enthusiastically
reported the locations in the province of Namur where Carum verticillatum,
Ophrys apifera, and the rare orchids Aceras anthropophora and Loroglossum
hircinum could be found.

In 1913, he became a board member, attended ten meetings, and gave
as many lectures. On December 22, he delivered a 'warmly applauded'
lecture on The Auxiliary Sciences of Botany. Prior to his promotion
in the unfortunate summer of 1914, he introduced, among other topics,
Charlton's investigations into the origin of life, The Opening of
Anthers, and Fairy Tales and Legends about Animal Behavior. During
this period, Minnaert and Cesar De Bruyker were the pillars of the
Society: the latter was Mac Leod's right-hand man. Unintentionally,
De Bruyker would determine Marcel's scientific career.

In De Goedendag, Marcel occasionally ventured into natural science
territory. He wrote about probability theory and the solar eclipse
of 1912. He organized a competition about the zone clips and received
two responses that he could not approve of. His fellow editors criticized
him for making the questions too difficult. His defense was: 'The
first part---apparent motion---aimed to determine how well the students
of the atheneum knew how to use their eyes; it was simply a matter
of observational spirit. The second part---the explanation of apparent
motion---was a question that should already have arisen in the minds
of every student and been answered long ago. Would one let such a
phenomenon, which one might presumably observe only once in their
entire life, pass by without even thinking about its reason? The very
first principles of astronomy are the only requirements for finding
the solution, and... a little scientific way of thinking.' Following
this was his complicated answer, which he considered 'the simplest.'
Marcel was intellectually miles ahead of his contemporaries.

Marcel's obituary of the Dutchman Jacobus Henricus van 't Hoff anticipated
his career choice. He felt that the first Nobel Prize winner in chemistry
(1901) had embarked on a new path in Berlin. He quoted Van 't Hoff:
'Should there not be men whose duty it is to investigate and who,
when they have the desire and time, may also teach?' The profession
of researcher did not yet exist, but Marcel already felt called to
it. He expertly discussed Van 't Hoff's life's work: 'Everywhere we
encounter his clear language, his genius comprehension, the elegance
of his experiments, the imaginative power combined with practical
sense.' Van 't Hoff's Dutch-language education was a recommendation
for the Dutchification of Ghent University. A curious passage deserves
mention: 'Perhaps it is important to note that, neither on his father's
nor his mother's side, foreign blood flowed through his veins.' At
the time, Marcel was fascinated by ideas about the purity of the Germanic
race. He dedicated the last two years of his studies to his own research,
allowing him to both graduate and earn his doctorate.

\section*{The aspiring researcher}

MacLeod had studied the effects of variations in sunlight intensity
on vegetation. He utilized the statistics of his Brussels colleague
A. Quételet and emphasized the reproducibility of biological measurements.
This was a novel approach to his field. He was working on his magnum
opus about quantitative biology and left Marcel, at that time his
only PhD student, to his own devices. This fostered Marcel's inventiveness.

Marcel would graduate in 1914 with a dissertation titled {*}Contributions
à l'étude de la photobiologie quantitative{*}. He investigated the
influence of sunlight on a pine tree, specifically the light- and
shadow-needles of {*}Pinus laricio Poir{*}. During his first year
of study, Marcel discovered that in this pine tree, the number of
resin canals in the needles varied significantly with exposure to
light. He selected a tree situated on a slope, partially shaded from
sunlight by shrubs. At the base of the tree, he collected 'shadow
needles,' at three different heights above, 'light needles I, II,
and III,' and at the top, he took 'top needles.' He measured and counted
various characteristics: shoot length, number of buds, number of dwarf
shoots, number of needles per dwarf shoot, the rotational direction
of the needles around their length axis, as well as the length, width,
and thickness of the needles. He made cross-sections and determined
under a microscope the increase in the number of resin canals on both
the flat and rounded sides of the needles. He also counted the number
of stomata on the needles. He was searching for the characteristic
that changes most strongly under the influence of light intensity
and introduced the variation coefficient for this purpose. His hypothesis
was confirmed: 'The increase in the number of resin canals with light
is evident from all our data with the greatest clarity. Equally clear
is that this increase occurs more rapidly on the flat side than on
the rounded side (of the needle), so that this correlation too is
influenced by light.' He believed he was the first researcher to point
this out.

Minnaert utilized this result in a second series of measurements.
He designed a setup that allowed for more precise measurements than
those taken on the slope. He planted four batches of seedlings in
the ground and placed a cage over each batch. Marcel used wooden slats
that were one centimeter thick and wide, leaving a constant space
between them. In this way, he created cages that allowed through 7/8,
5/8, 3/8, and 1/8 of the sunlight. Marcel measured an increasing mass,
more buds per shoot, an optimum number of dwarf shoots, more clusters
of three needles, relatively more right-turning needles, an increase
in the length, thickness, and width of the needles, and a rise in
the increase of the number of resin channels. The number of stomata
remained virtually constant, which he found 'very remarkable.'

Using mathematical processing and necessary assumptions, he was able
to convert the increase in the number of resin channels into a growth
rate. After plotting the 'time interval between the appearance of
successive channels on the convex side,' as he had postulated, against
this 'growth rate,' he obtained a curve in the form of a slowly rising
slope with the appearance of the fifth resin channel at its peak and
a steep decline afterward: 'We have thus found a curve that truly
represents the course of a growth curve.' This combination of quantitative
measurements and theoretical considerations seems like a creative
and original construction.

Finally, he compared his results with the values in international
literature. His experiments showed variable values that he had averaged,
while they were usually considered constant. Many colleagues apparently
counted with preconceived ideas and were satisfied with few measurements.
They used terms such as 'full' and 'dimmed' light without specifying
what they understood by 'light' and 'shadow.' This jeopardized reproducibility.
Marcel admonished: 'Shadow must be determined by measuring light intensity.'
Both in his choice of subject and in his emphasis on the quantitative
method, he proved himself to be a disciple of Mac Leod. He graduated
summa cum laude in the summer of 1914 and was also planning to graduate
in zoology.

That Minnaert was an odd fellow among his fellow students is evident
from the biographical sketch by his free-thinking friend Ciessen.
Printed next to a 12 caricature of Marcel examining under a microscope,
it reveals something about his youthful spirit. When Ciessen 'vivisected'
Marcel's 'deep psychological mind,' he couldn't help but think of
a song:
\begin{verse}
'He doesn't drink beer.'\textquotedbl{}

Does not have fun,

has never given a kiss.

Does not smoke tobacco,

does not know the jail

Therefore, is not worth living.'
\end{verse}
Ciessen found Marcel's urge to distinguish himself 'an incurable disease':
'Where do you see a studious person on Sundays with a set of tin cans
around their neck and armed with a selection of nets, in the vague
hope of catching a rare butterfly? Where do you see a studious person
attending public auctions, only to return with a pile of firewood,
which, glued together at the cost of months of work and patience,
eventually has to resemble a valuable violin?' Ciessen called him
the best student of the Alma Mater, already saw him as a professor
and member of the Royal Flemish Academy, and had a series of distinctions
parade by 'like a kinema film that represents the same person.' He
added a note about an obsession: 'Oh! I can already see Marcel at
the moment he has to choose his housewife. I see him measuring her
facial angles and taking samples of her hair, not to preserve them
in a silk paper, but to study their cross-section, length, and color.
And I hear him decide: \textquotedbl Miss, even though you seem to
be of the white race, learn from my mouth that you are Mongolian,
because the cross-section of your hair is circular. Leave me for the
sake of higher Science, because the mestizos of whites and yellows
are an inferior race\textquotedbl{}

His friends joked about Marcel's preference for the Germanic race.
He was an outsider, but also a true student. He shared the idealism
of youth. He had thrown himself into the Flemish Movement and, along
with it, the idealism of Lebensreform. With his promoter, he not only
shared a dedication to science but also his fanaticism regarding the
emancipation of Flanders and the Dutchification of the university.

\section*{Mac Leod on the sidelines}

Julius Mac Leod had written in the Report of the First University
Commission in 1897: 'Only in three regions of Europe was Latin not
replaced as the vehicle of Higher Education by the mother tongue,
but by another foreign language: in Russian Poland, in Romanian Hungary,
and in Flemish Belgium.' Among the peoples of Central and Western
Europe, only the Flemish people were deprived of higher education
in their mother tongue. In normal countries, there is a fine-meshed
network between all types of education: from the knowledge of the
agricultural worker to that of the engineer. In the Flemish regions,
this unity was broken because both ends of the spectrum spoke different
languages.

A Dutch-language university had to create a Flemish upper layer that
could break the fatal cycle of underdevelopment. The Dutchification
of Ghent was a right that eventually had to be granted. The Walloons
and most Ghent professors unleashed a storm of criticism on this Report.
In doing so, they received the support of the French-speaking bourgeoisie
of Ghent. Within the Flemish Movement, MacLeod's starting points were
not disputed. But the tactics were controversial. The Committee wanted
monolingualism in the four faculties. New lecturers had to teach in
Dutch; the existing ones would be explicitly invited to do so. French
would die out gradually through its phased introduction over ten years.
Francophone education would temporarily continue at the Technical
Schools. Some Flemish-minded individuals felt that monolingualism
went too far: for example, the historian Fredericq had emerged as
an advocate for a bilingual university. Others felt the Report did
not go far enough. The sociologist De Raet had written in July 1892,
when he was a student, about The Dutch University: 'A people must
strongly organize itself intellectually if it is to resist its enemies
in the struggle for life from nation to nation. Intellectual power
creates material strength.' De Raet considered technical education
to be of vital importance. He emphasized that higher technical, commercial,
and agricultural education were essential for Flanders, and precisely
those areas were being left to Frenchified education! MacLeod had
also stressed the importance of 'technical knowledge,' but he had
reached a compromise. The ANV-Belgium, which had developed as a Flemish
discussion forum, had voted against Fredericq in 1903 with 31 votes
against 3. MacLeod had written triumphantly: 'That vote closed the
'era of systems.'' It turned out to be wrong. De Raet chose a frontal
attack. The clash of the personalities of Mac Leod and De Raet, the
founders of the social and economic directions in the Flemish Movement
respectively, ruled out any compromise. In a 1902 letter, De Raet
had complained about flamingants who ‘died before they were dead,
stripped of all enthusiasm and resilience, blocking the way and holding
back the young,’ clearly targeting Mac Leod. De Raet triumphed in
1906 at the Language and Literary Congress, where he advocated for
a serving university as an instrument of culture and economic power
and as a strengthener of ‘people’s power.’ Mac Leod withdrew in anger
and from then on limited himself to his influence over student youth.
In his final years, he mentored Marcel Minnaert.

A Second Higher Education Commission was established, including six
members from the First, among whom its chairman, the art historian
Max Rooses. A key sentence from the 17th Report (1909) read: ‘Between
the moral humiliation, economic minority, and intellectual backwardness
of today and the desired complete harmonious development of our people’s
power, the pious goal of the four generations of flamingants who preceded
us, there is an abyss. A bridge must be built between the two, and
that bridge is the Flemish University.’ The content could be summarized
in four words: Language interest is material interest. The first year
of Dutchification would take effect five years after the law was passed.
After that, a step would be taken each year until full Dutchification
would be achieved after eleven years. The technical faculties were
also included in this operation. The Commission presented itself as
the center of the movement: the Great Staff. In the dark back room
of Hippoliet Meerts' home, alongside Neerlandia (ANV), the monthly
magazine {*}De Vlaamsche Hoogeschool{*} was also edited.

The economic argumentation made the Flemish University popular: the
people missed it as if it were part of their very being. This approach
largely lifted the divisions. A political collaboration emerged between
flamingants from the three parties, led by the Antwerp socialist Kamiel
Huysmans, the Antwerp liberal Louis Franck, and the Antwerp Catholic
Frans Van Cauwelaert.

On a meeting on December 18, 1910, Huysmans said: 'We will not stop
crying out like three Flemish roosters until Ghent University is made
Flemish!' The socialist leader Eduard Anseele wrote in March 1911:
'We tried to avoid the issue, but it forces itself upon us powerfully
and we can no longer close the door on it.' Translations and popularizations
of the Report appeared. In 1911, 368 meetings were held alongside
mass demonstrations in Antwerp, Bruges, Ghent, Leuven, Brussels, Mechelen,
and Hasselt. The campaign leadership organized a petition with more
than 100,000 signatures; two thousand university graduates spoke out
in favor of Ghent University.

The Great Staff observed that the core of the argumentation of those
who continued to resist revolved around the alleged bilingualism of
Flanders. In a memo dated November 19, De Raet demonstrated that Flanders
was unilingual. In 1910, there were 47,000 people for whom French
was the cultural language and nearly three million for whom Dutch
was. The Flemish people demanded the university, which had never been
anything but Flemish. The French-speaking circles were pushed onto
the defensive but defended themselves vigorously. They received full
support from France, which well understood that its hegemony in its
outer region of Flanders was at stake. Some interest arose from the
Netherlands. Minnaert would get to know several flamingant Dutchmen
well.

\section*{Three Dutchmen: Geyl, Bolland, and Domela}

Marcel, on behalf of the ANV-Ghent, organized several Great Dutch
Student Congresses. These also attracted Dutch students, making a
great impression on them. The Third Congress in 1911 took place in
Ghent. Young Pieter Geyl, later the designer of the Great Dutch historical
view, stayed with his uncle, Protestant minister Jan Derk Domela Nieuwenhuis.
It was an initiation, he wrote afterward: 'I was deeply struck by
the moral seriousness that there emerged a sense that this was not
merely a manifestation of nationalism but that these young people
were aware of a significant societal task they could fulfill only
if the abnormal language relations in their country were rectified.’
Geyl encountered Minnaert at this Congress as a tangible example of
such moral seriousness.

A short-term helper proved to be the Leiden philosopher Bolland. Ghent
students visited him in the summer of 1911 and invited him ‘to crush
with your powerful, beautiful, and spirited words all the vermin that
spits on our language.’ Bolland promised to speak about Dutch as a
vehicular language in science. He performed for packed halls in Brussels,
Antwerp, Bruges, and Ghent, where, according to {*}De Goedendag{*},
he received stormy applause. Bolland lashed out at the francophiles,
calling them ‘friendly, fawning followers of the foreign woman,’ ‘clumsy
bastards,’ and ‘linguistic donkeys,’ and scolded his audience in the
way only a Dutch preacher could: ‘You tolerate it---that so-called
fellow countrymen, who speak not a proper language but a regional
dialect, Koeterwaals, a slang that forces them to stick to French---first
and foremost in the military and even in higher and highest education---have
pushed aside, suffocated, and obscured the only language that can
be called a language in Belgium!’ No language could express Pure Reason
better than Dutch. French belonged in the opera and gossip magazines
but not in science and philosophy: ‘Afterward, the hall was in raptures,
and the cheering and waving of hats and handkerchiefs seemed endless.’
The Leidenaar appeared not to realize that his audience was conducting
an unparalleled campaign for the Flemish University. 22

The historian Fredericq noted: ‘Unheard-of scandal. Bolland cursed
and spoke with foam on his mouth and growls in his throat. I was convinced
he was drunk. Oratory skills like those of a filthy socialist.’ Fredericq
had noticed that not everyone shared his opinion: ‘On the contrary,
there was a clique of students and Flemish-minded individuals who
loudly cheered and, at the end of the sour prophet’s curses, gave
the Leiden professor an ovation.’ Fredericq left ‘crushed’ in the
company of Van Hauwaert and Fris, two of Marcel’s former high school
teachers.

23 The editorial board of Marcel’s journal rejoiced: ‘We have had
the rare good fortune and honor to welcome a man within our walls
who occupies the very first place among all our great minds in the
field of knowledge, regardless of which one.’

Which country or people? A man who stands at the forefront of the
great thinkers of the centuries, such as Aristotle, Plato, Spinoza,
Kant, Fichte, Schelling, Hegel, Schopenhauer---the giants par excellence
in the realm of intellect.’ Marcel Minnaert and his friend Gaston
Mahy 24 wrote to Bolland ‘on behalf of Your listeners’: ‘You have
appeared to us as an Apostle, giving us willpower and self-awareness,
pride and manliness, teaching and instructing us; also reproaching
and chastising us, as a father does with his children, for You came
out of love... Out of love You came, - and love You have awakened.
Your strongly pronounced personality shone upon us like the sun, and
all beings indeed turn towards the light. A power emanates from You
that inspires and elevates.’

The letter was in Minnaert’s handwriting. Both young men needed a
father figure, an idol, an inspiring force, a Sun. Minnaert invited
Bolland 25 to provide a philosophy curriculum. He wrote: ‘Is it even
necessary to tell You what a deep impression we have all retained
from Your series of lectures on Pure Reason?’ Andries Mac Leod and
he later formed a group. Every Thursday at half past five, the Flemish
Philosophical Society would gather at Minnaert’s home, with its chairman
being Minnaert: this included secretary Edzard Domela Nieuwenhuis,
Paul Van Oye, Gaston Mahy, Andries Mac Leod, and Paul De Keyser. Bolland
accepted the honorary presidency. They read Kant and Plato and likely
practiced dialectics in the style of 26 Hegel and Bolland.

The Dutch Reverend Jan Derk Domela Nieuwenhuis also entered Marcel’s
life. He had the stature of a prophet and resembled his uncle Ferdinand,
the Dutch anarchist, in other ways as well. At Fredericq’s suggestion,
he was appointed by the Protestant community on Brabantdam, which
had many Germans and Dutchmen among its members. The reverend, like
Marcel, served on the board of the ANV. Characteristically, for a
clergyman, his notable statement was: 27 ‘And furthermore, I am of
the opinion that Belgium must be destroyed.’ He was a vegetarian,
non-smoker, and total abstainer. Minnaert adopted that ascetic disposition
from him, which stood in stark contrast to the abundance at his mother’s
home. In the course of 1912, friends Gaston, Edzard, and Marcel emerged
as founders of the total abstinence lodge De Orde der Goede Tempeliers.
This initiative was supported, among others, by the Luxembourgish
philosopher Peter Hoffmann, who taught in Ghent. The pastor found
it pleasing that Minnaert followed his example in many respects but
regretted all the more that the young man persisted in the absolute
rejection of the Christian faith.

\section*{Just one more powerful assault}

In the autumn of 1910, Marcel had written: ‘Perhaps since the 16th
century, no greater or more stubborn struggle has been waged in Flanders
for our own people and sacred rights than the daily battle being fought
around us: the struggle for the Flemishization of Ghent University.
The Flemings have become aware and now the flower of the Flemish soul
is unfolding in the most glorious splendor; that Flemish soul full
of poetry, strength, and will, full of courage and fire, full of untamable
stubbornness and a drive for heroic deeds.’ He became one of the most
impatient activists: ‘At this moment, it may be enough to launch just
one more powerful assault so that the bastion of Francillonism will
fall into our hands. Now is the time to steel our courage and willpower.’
He pointed to the Czech national movement in Austria-Hungary and praised
the stubbornness of the Czechs, ‘his people, his language, nothing
matters more to him.’ Minnaert identified with the Slavic Czechs,
even when that nation resisted Germanic domination! His perspective
was thus not so much about race but about the liberation of the oppressed
nationality, whether Flemish or Czech.

The Czech movement was more radical than the Flemish: ‘But what we
can learn from them is that we will achieve everything if only we
want it. We are not even asking for as much as the Bohemians, and
we are in much better circumstances than they are. Could we not attain
our rights, our rightful demands? Yes, it’s true, we do not have the
higher classes on our side; but then we will manage without them!
And soon they will be Flemishized against their will through our Flemish
University of Ghent.’'It will surely go!'

In March 1911, Minnaert became a member of the Ghent board of the
Algemeen Nederlandsch Verbond (ANV). This was an initiative by his
former teacher Meert, who hoped for more decisiveness. As a result,
he came into weekly contact with Reverend Domela and Catholic Flemish
activists like medical specialist Speleers, who played a prominent
role in the Great Staff. In Ghent that year, large meetings were held
on February 19 and July 11. On Guldensporenday, tens of thousands
of demonstrators had gathered: a true triumph for the movement. Yet,
from a party political perspective, things looked bleak in Ghent.
Many municipal councils in Flanders demanded the Flemish University,
but the Ghent council did so least of all.

In 1911, Fleming Alfons Sevens asked the Ghent city council for a
position. Only six of the 38 council members deigned to respond; three
in favor and three against. The liberal mayor Braun ensured that the
University was not even on the agenda, while he himself publicly advocated
for a bilingual university. Following this, Sevens founded De Vlaamse
Blok together with Hector Planckaert and Alfons Van Roy. The 1912
campaign earned the party 1,800 votes, or 4.3\%. In their magazine,
they pleaded for Flemish-national independence. This last point was
a new sound that resonated with radical Minnaert like music to his
ears. That year, Walloon socialist Jules Destrée had argued in a Letter
to the King that there were no Belgians, only Walloons and Flemings.
The only way to get rid of the Catholic reaction, which had won the
elections again in Belgium, seemed to Destrée to be a federal Wallonia
where socialists and liberals had long held the majority. He advocated
for transforming Belgium into a federation of the two peoples, thus
for an administrative separation of the country. Incidentally, he
rejected monolingualism, which he considered normal for Wallonia,
for Flanders.

The Great Staff saw no point in addressing this administrative separation,
as they considered it a diversionary maneuver from the campaign for
the Flemish 33 College. Rebuttals like those from Hippoliet Meert
emphasized that the Flemings were good Belgians fighting for just
demands, though they did not fear administrative separation. Alfons
Sevens of De Vlaamse Blok, however, sided with Destrée: 'Administrative
Separation is the inevitable death of the Francophones. Finally, we
will breathe pure Flemish air in Flanders and live in a disinfected
house.' Sevens' hateful terminology indicated growing resentment.
The flamingants suspected liberal Freemason lodges of conspiring against
Dutchification. Brussels lawyer Josson then founded an imitation,
De Vlaamse Veem, to 'free the Flemish people from all harmful foreign
influence of France's interference in Belgium's domestic affairs,
and to make the Dutch language reign as a queen over the entire expanse
of Flemish land, in administration, justice, army, and school.' The
Ghent branch of this Veem met monthly at Minnaert's place. Among the
moderates were Hippoliet Meert and Boudewijn Maes; among the radicals
were Marcel Minnaert, his mother Jozefina Van Overberge, teacher Antoon
Thiry, ship's doctor Jules Van Roy, lawyer Alfons Van Roy, and atheneum
teacher Jan Oscar De Gruyter. The moderates opposed administrative
separation, while the radicals supported it.

In December 1913, Minnaert was among the initiators of the Association
for Civilized Pronunciation, which wanted to connect more closely
with intellectual life in the Netherlands. Together with the headmaster
of his lower school on Onderstraat, De Hovre, and his friend Andries
Mac Leod, Minnaert pointed out a 'regrettable gap': Flanders lacked
a generally civilized spoken language.

\section*{The goedendag and the Dutchification of the athenea}

Marcel began to focus again intensively on De Goedendag as a 'high
student.' Student Michel Van Vlaenderen had resumed the radical flamingant
stance at the Athénée Royal and started a 'grievance chronicle' in
1911. His complaints about French texts on attendance slips and school
fee statements nearly led to his expulsion from the Ghent atheneum.
Franskiljonse study prefect Eugène Clevers was his opponent. Minister
Poullet of Arts and Sciences had a conversation with Van Vlaenderen
about this issue, upheld the student's case, and insisted on Dutch
texts for all circulars of the Flemish athenea. The prefect dared
not take disciplinary measures. Marcel now saw De Goedendag as a means
by which students could enforce their rights. That was literally what
Vermeylen had pleaded for.

In 1912, the magazine, with Minnaert as editor-in-chief, entered its
20th year. Starting in September, the editorial board moved from Antwerp
to Ghent, Citadellaan 73, where the management would now be based
and the editors would reside. These included Marcel, Miss M. Ingels
of the Hélène Swarthkring, the graduated biologist Paul Van Oye, and
the high school students Michel Van Vlaenderen and Evarist Verdurme.
The latter was a poet and used the pseudonym Ledegouwer in the editorial
board. Van Vlaenderen continued his struggle, and Minnaert could keep
him as a 'high student' under cover. In the confrontations, there
was often uncertainty about the stance of the Flemish-minded teachers.
There was also division regarding bilingualism.

Minnaert paused on this in the Jong Vlaanderen editorial. According
to him, fierce opponents stood within their own ranks, betrayers of
their own people: 'If we do not cast them away as poisonous brood,
will they not be trampled by avenging feet?' His terminology resembles
that of De Vlaamse Blok. Marcel considered victory near, but nonetheless
wrote: 'They will stifle us in bilingualism.' He had finally noticed
the danger posed by Fredericq and his followers against the Dutchification
of the university. Marcel had no sense of nuance: you were either
his hero or a traitor to your country and people. The cowardice of
the elders would reign 'until the day when a youthful generation,
fully aware of its strength and omnipotence, will rise; until that
day there will be a crowd of Flemings who will be Flemish, down to
the most tender strings of emotion, down to the finest fibers of the
heart. Until the day when that Flemish youth will sacrifice everything
for the salvation of our tribe and the greatness of our culture.\textquotedbl{}
He evoked the image of the chosenness of Flemish youth. Marcel knew
himself to be chosen and called upon his companions to live in this
great era: 'Your meetings must be councils of war and training grounds.'
The students had to free themselves from Frenchification and develop
in 'the radiance of immaculate national culture.'

On October 16, 1912, August Vermeylen spoke in Brussels about Belgium
and European civilization. In it, he stated: 'To be something, we
must be Flemish. We want to be Flemish to be Europeans!' The misery
with Belgium was that the state poisoned the sources of healthy nationalism,
so there could be no question of internationalism: 'In Belgium, there
is not a single scholar who has heard a word of Dutch at the university.
What influence can such people have on the people?' Such a people
could not develop style, play a role in Europe. On the other hand,
Vermeylen praised the thorough knowledge of languages among the flamingants,
who were the only cosmopolitans in Belgium, predestined to promote
harmony between the Romance and Germanic spirits: 'We must be Flemish
to become world citizens.'

Marcel thought it an excellent article, as it was printed in {*}De
Goedendag{*} in two installments starting from the front page. Shortly
thereafter, a headline appeared stating that 'comrade Ledegouwer of
the Atheneum has been expelled because he wrote: \textquotedbl bare
breasts\textquotedbl ! Scandal! Scandal!' It was really something
to cry about.

\section*{A screaming, wild hatred}

It turned out that Verdurme had once assisted his comrade Van Vlaenderen
when he again pointed out a violation of the language laws. Prefect
Clevers had snapped at Verdurme that he should especially not follow
in Van Vlaenderen's footsteps, as it would soon be over for him. On
November 15, 1912, Verdurme's poetry collection {*}Heoos{*} appeared
under the pseudonym Ledegouwer. Immediately, drawing teacher Van Puyvelde,
an art historian and self-proclaimed flamingant, had complained to
the prefect about a poem by a student mentioning 'bare breasts.' This
performance is all the more astonishing because Van Puyvelde, the
same man who had encouraged the poet to make his debut, was involved.
Unlike the prefect, Van Puyvelde was aware of Verdurme's pseudonym
as an editorial member. On the very same day, Prefect Verdurme expelled
him from school. The editorial board wrote: 'It is an unheard-of and
incredible fact. And yet our colleague, our friend Ledegouwer, has
been cast out of the Atheneum. Is he removed like rotten fruit from
the 'good'? Is he shown the door as one infected with plague, with
dirty fingers and a contemptuous mouth...?'

The editorial board of {*}De Goedendag{*} could not express their
'boundless contempt and unspeakable aversion for those who committed
this knavish act.' Verdurme had been a good, quiet student. His collection
had nothing to do with the school. Minnaert mocked Prefect Clevers
and teacher Van Puyvelde for pages on end. There were no reasonable
arguments for Clevers' actions: 'Did he see an opportunity to take
revenge on one of the editors of {*}De Goedendag{*}, or did he want
to strike at the Flemish nationalist through Ledegouwer?' Was it the
revenge of a francophile? Minnaert could not get over it: 'I can't
find the words, and my thoughts swirl like autumn leaves in a whirlwind...'
He knew Clevers well from the board of {*}Dodonaea{*}; he had followed
Van Puyvelde's art history course at university. A young, talented
person had been betrayed.

Minnaert envisioned a small house, a boy approaching slowly. A mother's
laugh, suppressed sorrow, the words 'I have been expelled,' and the
thud of a falling body: 'I see, Mr. Prefect, a household full of inhuman
sorrow; - I see, Mr. Prefect, a philistine sitting in an armchair,
lecturing on morality. I feel the despair of the parents who had spun
golden dreams with tenderness and love, but whose souls are now shrouded
in cold black night. I feel the hatred, the screaming, wild hatred
of him whom they trample with their feet, kicking away the deepest,
most sincere, and most honest thing he possessed, like rotting filth.'
Marcel identified with the victim.

He called on young people to buy Ledegouwer's collection and discover
its profound humanity. The 'entire Flemish youth' stood by him, on
the side of sincerity, courage, and light! Later issues mentioned
the support from schools and the vengeful actions of some pro-Flemish
teachers. Minnaert: 'Their \textquotedbl opposition\textquotedbl{}
makes us laugh! What can they, those small-minded sycophants, do to
harm De Goedendag, the journal of light? If only you knew how I despise
you, so-called flamingants, who for the sake of a position as a teacher,
professor, or member of an academy, lick political boots. Oh, if only
you knew how I despise you, who for the sake of your parvenu politics
want to oppose true flamingants. We are not afraid of you!'

In {*}De Vlaamse Gids{*}, Maurits Sabbe wrote a review in which he
called Ledegouwer 'particularly mildly gifted' and 'a strong force
for the future.' The editorial team of issue 42 picked up on this
assessment.

This affair radicalized Marcel. It must have fueled his deep hatred
towards Flemish-minded individuals like Van Puyvelde and intensified
his disdain for the ruling generation.

\section*{From bilingualism to monolingualism}

Minnaert soon after wrote a treatise on bilingualism, the tone of
which in the country drew attention. He gave great credit to the socialist
Destrée and his Wallonists: 'They showed with unflinching courage
that Belgian patriotism fundamentally does not exist. We feel nothing,
absolutely nothing for Belgium, where we are treated as lackeys, as
slaves; the Walloons feel nothing either: they say so themselves.'
Here, the young Vermeylen resonated. Minnaert criticized the historian
H. Pirenne: his attempts 'to show that Belgium had already formed
a unified whole from very early on are truly ridiculous and remind
one of a beetle crawling against a glass pane. Pirenne's entire theory
is destroyed by a small, seemingly insignificant remark by Prof. De
Vreese: the medieval Flemings never made gallicisms in the construction
of their sentences.' Minnaert, a third-year student, called the Walloon
Pirenne, the celebrated praise-singer of Belgium and the scientific
Jupiter of Ghent University, a crawling beetle!

Marcel could get terribly worked up over bilingualism: 'If I see on
all public buildings in Ghent, on municipal documents and forms, etc.,
the two...If I were to see languages, sometimes I could become furious,
thinking to myself how ridiculous and insulting those bastardized
signs are in the free, proud, ancient city of Artevelde. We want no
bilingualism! Flemish must be the language here. Flanders for the
Flemings.\textquotedbl{} Anyone living in Flanders must learn Flemish:
\textquotedbl Whoever does not know Flemish will find all public
offices in Flanders closed to them, will not understand what they
read or hear spoken. It is their fault. This is the case in every
country in the world. The coercion to learn Dutch will thus be indirect
but no less forceful. It must happen, because anyone who is a citizen
of Flanders and does not know Dutch cannot fulfill their civic duties
toward their people; they can therefore also make no claim to their
civil rights.\textquotedbl{}

Incidentally, Marcel, like a Cato, believed that the Frenchified Flemings
formed a center of corruption that had to be eradicated: this was
an echo of Reverend Domela.

His first demand, besides a Dutch-dominated university, was also for
a Flemish army: \textquotedbl We want a Flemish army where our soldiers
do not hear a single word of French, where all communications and
commands are expressed in one language alone, where they learn {*}De
Vlaamse Leeuw{*} and not {*}La Brabançonne{*}. We want that everyone
born and living in Flemish Belgium is necessarily enrolled in Flemish
regiments; if they do not know our language, they must learn it.\textquotedbl{}

Minnaert wrote: \textquotedbl It is better to get nothing---do you
hear!---than the systems of Fredericq, Pirenne, and others for the
university, and the amendment by Buysse-Pêcher-Persoons for the military
law.\textquotedbl{} He concluded with an ode to pure youth: \textquotedbl There
you have our program, there you have our ideal, which is just as noble
and sublime because we young people, in the full awareness of our
youthful strength, dare to gaze upon its clear radiance entirely and
completely. Flemish youth, glorious eagle, do not shrink back from
the light, but fly straight toward the sun.\textquotedbl{}

\section*{A spark of immortality}

This argument by editor Minnaert was a sign of the times. In Antwerp,
his article made a great impression. Young radicals were fed up with
Belgium, wanted to get rid of the Moloch that trampled their ideals.
At the end of this academic year, a whistleblower named {*}Qui tuba
sonat{*} published an article about {*}Soldaterijen{*}. In 1913, the
first conscripts were called up. Of the 104 enlisted recruits, 48
had requested instruction in Flemish, which was refused. Fifty students
had shown Flemish leave passes, which were torn up by the captain.
The writer called on the students in the army to fight for their rights
so that their less educated brothers could benefit from it. 'Decisive
Flemings' could bring about a change in discipline: 'Comrades, don’t
forget, conscription is near and no less is expected of you.' The
May 1913 article was unsigned and thus appeared under Minnaert’s responsibility.

Marcel ventured again at the start of the new academic year with a
poem:
\begin{verse}
'But we want life, we want pain,

We want to fight, we want to free. O

h dearest, come, press me wildly to your heart,

And let us run on flying horses,

Drinking strength from each other’s gaze,

Amidst the howling wind and the roar of cannons,

Through the screaming storm of the somber night,

Until new horizons rise far away.'
\end{verse}
Four months later, a page with a black border appeared in memory of
Evarist Verdurme, 18 years old, student in literature and philosophy,
who died on January 15, 1914, in Paris after a painful illness. Marcel
spoke at his funeral and wrote an In Memoriam for De Goedendag:

'He was so pure, in the glow of his youth, boldly advancing like a
young hero, singing his fiery song, striving for what is noble and
good. He was a poet at heart; he dared to deeply feel what stormed
in his heart and his thought was stronger than prejudice and convention;
he dared to fight for Life, with deed and word. He was the born reformer.
He proclaimed the future of mankind. But in hours of calm domesticity,
how could his eyes look at you, so tender, so clear as a child’s!
Then he spoke softly and questioning, telling about his village and
his simple country people and the trees in the field. Then he wished
to be cared for; then he felt at home; then he let himself be called
Eva. Poor Eva, good comrade, what drove you to the distant land? Was
it the drive for action and love for the beautiful wide world?' Was
it the longing for a blue sky and warmer sun? Poor Eva, Eva with your
golden hair, Eva with your clear eyes, oh, did you want to cast one
last glance at life in all its beauty and joy before... dying? Now
you float on wings of light in the radiance of the growing dawn. And
we people of Earth, who mostly did not understand what sang in your
warm heart, still feel that something great has lived among us, a
soul that carried within it the spark of immortality. Your memory
stays with us, pure, uplifting, and inspiring.\textquotedbl{}

Marcel chose from {*}Heoos{*}, the fleeting {*}Dawn{*}, the poem \textquotedbl My
Life\textquotedbl :
\begin{verse}
\textquotedbl My life was dreaming in prayer,

In tender reverie;

My life was love, and devout melody.

Now I await Death, and sing

With a simple mouth, like a child,

Of the many, many things

I have loved.\textquotedbl{}
\end{verse}
Here Marcel showed his loving side. His rhetoric had vanished. Perhaps
the loss of Evarist touched the nineteen-year-old so deeply that he
could let the words come from his heart.

At the end of 1913, ANV-Ghent established a Youth Division with Minnaert
and Michel and Martha Van Vlaenderen as board members. The orientation
of the ANV was one of collaboration between liberals and Catholics
for concrete goals. For Marcel, this meant a break with the liberal
camp.

\section*{Marcel breaks with the liberals}

From April 4 to 6, 1914, the Fifth Great Dutch Student Congress was
held in Ghent. The organizers were the liberal group 't Zal Wel Gaan,
the ANV student section (ANSV), and the Catholic Higher Education
Association (KHSV). Minnaert served as general chairman and wrote:
\textquotedbl It must be the major event of the year; it must be
a festive and grand symbol of the powerful unity among all young intellectuals
from the Netherlands, South Africa, Indonesia, and Flanders. We must
show the gentlemen franskiljons that we are aware of the support we
have in Dutch science and Culture.\textquotedbl{}

The congress was timely due to the parliamentary handling of the Hogeschoolkwestie:
the students had to demonstrate that political differences played
no role. Minnaert pointed out the close cooperation that existed between
the liberal and Catholic organizers.

On the eve of the congress, a protest meeting took place. After it
had seemed that there was a parliamentary majority for the Dutchification
of Ghent, this appeared to evaporate in the parliamentary committees.
There was only talk of doubling the courses, thus a bilingual university.
On April 1, 1951, comrade Leo Picard welcomed about a hundred students
from Ghent. The report states: \textquotedbl Then Mr. Minnaert speaks.
An end must come to the destructive nonsense of bilingualism, which
is a means to impose French on us! (applause). If the doubling is
approved, then we will make revolution! (long applause). We must resist
half-measures and boycott the Flemish courses if they are placed alongside
the French ones! (thunderous applause).\textquotedbl{}

On Saturday, April 4, at the start of the congress, there were no
fewer than 110 students from the Netherlands: eighty gentlemen and
thirty ladies. According to {*}Neerlandia{*}, chairman Minnaert had
aroused \textquotedbl general enthusiasm\textquotedbl{} and Karel
Van de Woestijne had praised his \textquotedbl inexhaustible eloquence.\textquotedbl{}
At the end, they sang among themselves: \textquotedbl Nothing went
well/If Minnaert doesn't take care of the main thing.\textquotedbl{}
At the end of Edzard Domela's jubilant report, Minnaert had jumped
up and asked: \textquotedbl Do you all swear, you Greater Netherlands
supporters present here, that the French university will disappear,\textquotedbl{}
to which a prolonged \textquotedbl yes\textquotedbl{} echoed from
hundreds of throats. The report in the members' journal of the liberal
organization 't Zal Wel Gaan was entirely different in tone. Minnaert
had announced at the opening session that 't Zal Wel Gaan had withdrawn
and gave foolish reasons for it. Those present responded with an indignant
\textquotedbl Hou.\textquotedbl{} The socialist Kamiel Huysmans had
spoken excellently, while a second speaker, one Lambrichts, had complained
about the betrayal in the Flemish Movement: 'It was regrettable that
the speaker had not understood where he was speaking.' 'T Zal Wel
Gaan had invited the participants to a 'tonzitting,' which was a colossal
success among the Dutch.

The dispute had been widely discussed in the Ghent press. There was
disagreement over the kroegjool, which Edzard Domela had characterized
as a relic from the time when the Flemings were still half-wild. Everything
that smelled of tobacco and alcohol had been removed from the official
program by Minnaert and Domela. The liberal students could appeal
to the praise in Propria Cures. Moreover, 'T Zal Wel Gaan felt that
they had been given too few chairmen at the meetings. The protest
had not helped. The liberal camp had withdrawn thereafter. The liberal
columnist Lamme Goedzak believed: If the Catholics wage the struggle
with the slogan All for Flanders, Flanders for Christ, then the liberal
Flemings will understand that they must fight with the slogan: All
for Flanders, Flanders for Progress.

This last remark was a reproach to Minnaert, who was criticized in
a follow-up article: 'A student who does not belong to the clerical
party recently cried out, \textquotedbl T Zal Wel Gaan must be destroyed!\textquotedbl{}
And why? Because he thinks that there is only one fruitful flamingantism
that rises above all parties and wants to work outside all politics.'
According to the liberal students, this 'neutrality' was solely directed
against them: 'Have we seen them take action against the Catholic
Flemish-minded students? They too engage in that abhorred politics.
On the contrary; before, during, and after the Great Dutch Congress,
these gentlemen were shamelessly favored by the organizers for the
sake of neutrality.'

There was bitterness in the liberal camp, although this conflict was
formally settled. The fact that the Flemish University was off the
table because the liberal parliamentarians had not wanted to support
the vernacularization was discussed dismissively in the same journal.
The chairman's conduct seemed arrogant and unnecessarily offensive.
The reason was that Minnaert was furious with the liberal party leaders
who trampled on their promises. His radicalism compelled him to break
with his comrades from 'T Zal Wel Gaan, whom he had publicly humiliated.
He was tired of interacting with the liberal-minded students who covered
liberal politics with the cloak of love. The friendly contacts with
his peers turned out to have been purely functional for Minnaert.
Those who had previously been 'good' could suddenly become 'bad.'
His former allies had become traitors in Marcel's eyes. He couldn't
understand why his peers saw things differently than he did, yet they
remained Flemish-minded. Within a few months, he shifted from a Flemish-liberal
to a Flemish-nationalist viewpoint. His promoter, Mac Leod, was also
the behind-the-scenes promoter of this political development.

\section*{The study group around Mac Leod}

The political year 1913-1914 brought one disaster after another for
the flamingants. Nothing came of the Flemish regiments. The July 1913
law required officers to train recruits in Dutch over time, but the
commands remained in French. In February 1914, compulsory education
up to the age of twelve came into effect, but parliament rejected
the principle of territoriality---Flemish in Flanders and French
in Wallonia. A wave of Frenchification of primary education followed
in Brussels. Finally, even the Dutchification of Ghent seemed to fail.
The response of Belgian democracy to five years of Flemish mass movement
was to deliver three heavy blows.

Young radicals like Minnaert felt deeply offended. Their trust in
parliamentary democracy dropped below zero. Mac Leod became the inspirer
of a group of young people with whom he could prove his point that
the greatest danger for Flanders lay in the fragmented party politics.
According to his niece, the writer Virginie Loveling, he possessed
'a miraculous ability to fanaticize the very young.' Mac Leod weekly
gathered a dozen selected young people, including Minnaert, to discuss
his plans with them. According to disciple Frans Primo, true leaders
had to abandon their party colors and henceforth only seeing ‘the
black on yellow field, a symbol of our Flemish national consciousness’
shining. Each of them had to choose a subject they would specialize
in: ‘These circles and societies in their shared work aimed at one
and the same goal, which according to him would be the great study
room of Flanders, from where liberation would slowly but surely be
achieved through actions.’ In short, Mac Leod preceded Marcel in Flemish
nationalism.

MacLeod's efforts in this way confirmed his dispute with the Grote
Staf by forming a new leadership for the Flemish Movement. His achievements
were closely monitored, as evident from an exchange of letters in
June 1913 between chairman Max Rooses and Lodewijk De Raet, mentioning
a group of 57 malcontents in Ghent: this group ‘consists of older
leaders who have been pushed to the background with the creation and
operation of the new Higher Education Committee, and some younger
individuals. They seem to want to disregard the authority of the Higher
Education Committee, the Grote Staf.’ During the period of growth,
they had no chance of success: ‘In the face of the glorious achievements
obtained by the Committee in the struggle for the Flemishification
of Ghent, they had to remain silent; but now that the Committee, due
to circumstances, the 1912 elections, and the battle begun in Parliament,
cannot continue and is experiencing a period of idleness, they raise
their heads again. They push forward the ‘younger and radical elements,’
the ‘impatient ones,’ the ‘incautious.’ Rooses particularly had Mac
Leod in mind.

Mac Leod’s plan required great perseverance if it were to lead to
results in a democracy. In these months, some of his followers initiated
publications that were not based on scholarly work at all. In May
1914, the anonymous monthly magazine {*}De Bestuurlijke Scheiding{*}
appeared: ‘There is something fermenting, something brewing, something
cooking in the hearts of the Flemish people! An action, an action,
an action!!!’ the first issue blared from the rooftops. The magazine
opposed both the Belgian state and Flemish politicians as well as
the leadership of the Flemish Movement with equal vehemence. The historian
Maurits Basse, Marcel's former teacher, later referred to a magazine
‘compiled by some quarrelsome malcontents who follow every party with
mouths full of disapproval, like the abusive slave following the Roman
triumphal wagon.’ They were mostly young people almost in hiding,
Conspiracy lovers.' Basse was concerned with the same group of dissatisfied
individuals around Mac Leod, among whom was Benjamin: Marcel Minnaert.

\section*{The administrative Separation}

{*}De Bestuurlijke Scheiding{*} had anonymous editors including Antoon
Thiry, former editor of {*}De Goedendag{*}, who managed the administration
with his wife Marthe Van Ael at the Flemish House on Sint Baafsplein,
Jules Van Roy, former chairman of 't Zal Wel Gaan, Reimond Kimpe,
also a former editor of {*}De Goedendag{*}, and Minnaert. Thiry and
Kimpe were friends with Felix Timmermans and had both published literary
work. Thiry later said about the magazine: ‘In it, party politics
were relentlessly attacked, the fatal ambiguous role of the parties
toward Flanders was condemned as criminal, and the administrative
separation was presented as the only acceptable goal of the struggle.’

The editorial members called themselves Testor I, II, III, and so
on. On the front page of the May issue, there was a creed: ‘In the
pus of your rotting party politics, you are suffocating, unworthy
Flemish-minded politicians! Step back, you cowardly believers and
despicable traitors! Sweet-talking daily newspaper writers who still
dare to say, “we wanted what is right, and we got what we wanted,”
those words on your overly praising lips, you wretches, must die.
Fools, all of you!’ The school law of May 1914 had been a stab in
their hearts: ‘We will shout even louder inside when the Flemishization
of Ghent University also fails due to the submerged rocks of treacherous
politics, and it will.’ The Great Staff could do with it: ‘You are
not breeding lions but dogs from your people, dogs maltreated and
trained by animal torturers, the Francophiles and their Flemish-minded
henchmen, who will only bite when the torture finally drives them...
to madness.’

The editorial team was focused on action: ‘Yes, it must come to action
if Flanders is truly destined for redemption!’, ‘it is not the laws
that make a people! The people create their own laws!’, ‘we see no
salvation other than in a National Flemish party, which will not allow
itself to be involved in petty clique politics. It will bring you
deliverance with one strike, with one word.’ From this prose, it is
clear that the circle around Minnaert, Kimpe, and Thiry was willing
to justify authoritarian politics. Their disappointment in democracy
turned into a rejection of its principles. They were vague about the
alternative regarding the administrative separation, they neatly wrote
that it ‘will be thoroughly studied by us’ once the place has been
cleaned up, but ‘we will frequently return to the same topic, with
ever new evidence and arguments.’ The metaphor of the broom is timeless;
MacLeod's discourse resonated loudly.

Testor I expected the salvation of Flanders from a willingness to
resist and the inner strength of the Flemish people: 'The Flemish
movement must stem from a great principle: Die Wille zur Macht of
a people that still feels its own vital force unbroken.’ The struggles
of the Poles, Czechs, and Norwegians were led by literary figures.
Why couldn’t this happen in Flanders? Where was the artist who could
move his people like Hendrik Conscience or the composer Peter Benoit:
‘We await him, he will come to Flanders as a new Messiah!’

This is how these radical young people rushed forward. They expected
their salvation from personal study, charismatic appearances, and
a sudden, authoritarian change in circumstances: a Messiah! Only a
miracle, a deus ex machina, could provide an escape. Fanatical youths
like Minnaert finally wanted action and results, no matter the cost.
The August issue of De Bestuurlijke Scheiding focused on France, Flanders'
age-old enemy: military censorship prevented its publication.

Josef Minnaert had warned Marcel in his Farewell Letter about passion
and fanaticism, a family flaw. His brother Gillis, nearly eighty and
Marcel's godfather, completed his life's work in 1914---a kind of
Layman's Mirror for the Flemish people. He too indirectly admonished
Marcel: ‘Imagination is like a soft, penetrating fire whose light
and glow illuminate and warm us when it is fed and maintained with
tender caution, but it consumes and destroys if allowed to flare up
uncontrollably.’ This passage could have been in the Farewell Letter.
However, Marcel hardly involved himself with his godfather, nor as
a 21-year-old adolescent did he need the commandments of his father.
After all, these were the same principles that had laid the foundation
for his extreme resistance stance.

On July 24, 1914, Marcel graduated cum laude with a doctorate in biology.
A few weeks earlier, shots rang out in Sarajevo. People unaware of
what was to come went on vacation. Jozefina, Marcel, and cousin Emiel
Minnaert would continue their 1912 Scandinavian journey and traveled
at the end of July to the southern part of Norway and Sweden. Jozefina
noted in her diary: 'Trollhättan: The storm brewing on Europe's political
horizon was indeed there, but we didn't notice it at all. Untroubled,
we continued our journey. Belgium was so far away; surely nothing
could happen to Belgium.'\\

Endnotes:

1 Minnaert's contributions appear in the Botanical Yearbooks of Dodonaea
from the years 1911 through 1915. In Paul Van Oye's retrospect of
October 1935, MacLeod's influence on his students is reflected in
Minnaert's role. Other contributors to MacLeod who were remembered
included Prof. Dr. C. De Bruyne and Prof. Dr. A.J.J. Van de Velde.
On October 21, 1935, a bust was unveiled at the student house Huize
Mac Leod.

2 Minnaert, M., Botanical Yearbook, 1911.

3 Minnaert, M., Light and Shadow Leaves in Ilex aquifolium, Botanical
Yearbook, 1912, 23-26.

4 Minnaert, M., Proceedings of the XVIIth Flemish Natural History
and Medical Congress, 1913.

5 Minnaert, M., Science: Our Competition on the Sun Eclipse, De Goedendag,
1911-1912, 131-133. Idem, Something about Mathematical Chance and
Mathematical Hope, 73-76.

6 This explanation was contested by a mathematics teacher: More on
the Sun Eclipse, 157-159.

7 Minnaert, M., Jakob Hendrik Van 't Hoff, De Goedendag, 1910-1911,
117-120. 8 Minnaert, M, 1914. The French dissertation is untraceable.
However, the Dutch version from 1918 is available in many places.

9 In 1911, he had already conducted these experiments earlier. 10
His pride in this construction became apparent later when, as an astronomer,
he referred to a graphical construction of a entirely different form
as a 'growth curve' (1934). In modern astrophysics, 'growth curves'
are still determined.

11 The quote is almost prophetic. In Utrecht, he was tasked in 1919
with quantitatively measuring the intensity of Fraunhofer line shadows.

12 Ciessen, M, Almanac 't Zal Wel Gaan, 1914, Marcel: Natural Philosophy
Student; Musician; Lecturer, 182-185. Including a caricature by E.D.
(Edzard Domela?): see Vanacker, 1991, 20.

13 In De Goedendag, there was a section called Kinema, which Marcel
likely edited.

14 MacLeod, 1897. He lived from 1857 to 1919: Marcel was his last
PhD student and accompanied him when he was nearly sixty.

15 Lamberty, 1961.

16 Lamberty, 1961, 75. De Raet, second edition 1911. De Raet lived
from 1870 to 1914. Remarkably, many lives of prominent progressive
Flemish activists ended prematurely between 1914 and 1925, such as
those of De Raet, Meert, MacLeod, Fredericq, De Gruyter, De Bruyker,
and Rudelsheim. The concept of 'volkskracht' was borrowed by De Raet
from the Dutch Flemish-minded Leo Simons, director of the Wereldbibliotheek.

18 De Schaepdrijver, 1997, 30-31.

19 Lamberty, 1961, 11-112.

20 Geyl, 1958, 107.

21 Otterspeer, 1995. The student who visited him was the later philosopher
Lucien Brulez. Bolland traveled to Flanders on December 16, 1911,
for a first series of lectures. In Otterspeer, chapter 45, Bolland
and Belgium. Three centuries before Bolland, Simon Stevin had already
pointed out the unique suitability of Dutch for scientific work. Minnaert
would later focus on this as well.

22 Fredericq, Diary, December 21, 1911.

23 Bolland's lectures, De Goedendag, 1911-1912, 36-37. 24 Bolland
Archive, letter of January 17, 1912. 25 Bolland Archive, Minnaert's
letter of October 2, 1912.

26 Minnaert to L. Buning, May 17, 1970, mentions studying Kant and
Plato. Buning Archive.

27 Fromme, 1942. This book contains childhood memories of Domela.
28 Minnaert, M., The Struggle for the Flemishization of the University,
De Goedendag, 1910-1911, 85.

29 Minnaert, M., Czech Movement and Flemish Movement, De Goedendag,
1910-1911, 41.

30 Dedeurwaerder, 154. Speleers was an ear, nose, and throat specialist.

31 Vanacker, 1991, 12.

32 Destrée, J., 1912. Brouwers, J., 1988, takes his title from this.

33 Meert, 1912. The ANV distributed his response in a large print
run.

34 Vanacker, 1991, 21.

35 Vanacker, 1991, 20.

36 The Van Vlaenderen issue was major news in De Goedendag, 1911-1912
volume. The conversation with Poullet took place on July 16, 1912.\textquotedbl{}

37 Vermeylen, in 1895, wrote that enforcing the application of the
1883 law was the task of every Flemish-minded person.

38 Paul Van Oye was the son of Dr. Eugeen Van Oye from Ostend, once
the beloved pupil of Guido Gezelle. Van der Plas, Chapter V, Poësis.

39 Minnaert, Jong Vlaanderen, De Goedendag, 1912-1913, pp. 2-4, Minnaert's
main editorial for those two years.

40 Vermeylen, A., Belgium and European Civilization, October 7, 1912.
Appeared in DG, 1912-1913, pp. 17-19, 33-35.

41 The adventures of Evarist Verdurme in De Goedendag, 1912-1913,
The Ledegouwer Case, pp. 49-55, 65-66, 68. Leo Van Puyvelde had been
appointed at the University a few months earlier. Marcel followed
his academic course in art history and took the exam on July 15, 1913.
Van Puyvelde became the art pope of Belgium, curator of the Royal
Museums in Brussels, an expert on Rubens. Therefore, due to his profession,
he would have to deal with many bare breasts and worse; let it be
granted to him.

42 Sabbe, M., The Flemish Guide, A Youth Book, January 1913; cited
in De Goedendag, 1912-1913, pp. 80-81.

43 The same Van Puyvelde plays a remarkable role in Dedeurwaerder,
p. 194.

44 Minnaert, M., Tweetaligheid, De Goedendag, 1912-1913, 83.

45 De Smedt, 1954.

46 Qui tuba sonat, Soldaterijen, 16 mei 1913, De Goedendag, 1912-1913,

47 Minnaert, M., Vers, De Goedendag, 1913-1914, 4.

48 De Keyser, P, 1936.

49 Minnaert, M., In Memoriam Evarist Verdurme, De Goedendag, 1913-1914,

50 Minnaert, M., Orgaan 't Zal Wel Gaan, 1914.

51 Protest Meeting, Orgaan 't Zal Wel Gaan, 1914, 24. In the 1920s,
students would indeed boycott the Dutch courses of the bilingual system-Nolf. 

52 E. Domela Nieuwenhuis, Neerlandia, 1914, 101. 

53 Neerlandia, May 5, 1914; Organ ZWG, The Great Dutch Student Congress
(25-26), A Heated Disagreement (29-31), Klauwaert and Geus by Lamme
Goedzak (35-36), Klauwaard and Geus (51-53), Echoes of the Fifth Great
Dutch Student Congress with quotes from Propria Cures (53-55). How
unexpected his step was is evident from the biographical sketch in
the 1914 Almanac of ZWG. The 1913 Almanac, pages 62-63, mentions Minnaert's
presence as a representative of ZWG at the Fourth Great Dutch Student
Congress in Amsterdam on February 1, 2, and 3, 1913, where he must
have met the medic H. Burger.

54 On April 24 the governance of the ANV deciced that the dispute
had to be resolved. A committee decided that they would collaborate
‘More united than ever’. Everything had been based on a misunderstanding:
Minnaert had merely interpreted an unofficial statement as official.

55 Quote from V. Loveling in Vanacker, 1991, 24. 

56 Obituary of Mac Leod by Frans Primo in {*}De Toorts{*} of March
10, 1919. 

57 Quote from a letter by M. Rooses to L. De Raet from June 1913 in
Lamberty, 140. 

58 Quotes from {*}The Administrative Separation{*} of May, June, and
July 1914. The third issue on bilingualism seems to have come from
Minnaert's pen, as well as the discussion of the performance of Peter
Benoit's Rubens Cantata. The call in {*}Through the people, for the
people{*} for a Flemish Messiah would rather have come from the literary
corner. 

59 Basse, M., 1930, I, 71. 

60 Characteristic description in Vanacker, 1991, 22-23. Quote from
Thiry, 1943. 

61 Minnaert, G.D., 1913-1914, Part I, 109. {*}Mind and Aesthetic Sense{*}
is an encyclopedic work for ordinary people, covering all aspects
of social life. It is a product of the pursuit of popular elevation.
Fredericq mocks it in his diary. 


\chapter{Flanders free under German authority}
\begin{quote}
‘All the so-called “theories” of these “scholars” do no harm, and
they may write entire inkwells full of them, as long as they do not
hinder fresh, lively action.’
\end{quote}

\section*{The start of the war for the Minnaerts}

The Minnaerts arrived in Norway at the end of July. On August 1, 1914,
they packed their suitcases for a trip to Trondheim. They read that
Belgium had mobilized and that Germany had declared war on Russia.
They had intended to continue their journey but the Norwegians also
mobilized. On August 3, the newspapers reported: ‘Germany declares
war on France and Belgium.’ They took a ship to Newcastle and arrived
in Ostend via Folkestone on the {*}Leopold II{*}. On the evening of
Monday, August 10, they were back in Ghent.

The Germans had demanded free passage through Belgium. King Albert
rejected their ultimatum. His proclamation of 4 August ended with:
'Remember, Flemings, the Battle of the Golden Spurs, and you Walloons,
that at this moment the honor of the 600 Franchimontois falls to you.'
At that time, the King was aware that Flemings and Walloons lived
in the country. The invasion unleashed a wave of Belgian patriotism.
Some 20,000 volunteers enlisted to strengthen the conscript army;
mostly Flemings. The Belgian defense stalled the German war machine.
The Germans committed war crimes against the population: executions,
including of women, hostage-takings, reprisals, looting, and even
mass murder. This terror was meant to paralyze the Belgians, so that
the communication lines to the French front would remain untouched.
The Germans destroyed Liège, Aarschot, Leuven, Mechelen, Andenne,
Dinant, Ypres, Tamines, and Aalst. The burning of the Leuven library
became world news. By brutally attacking a small country, Germany
turned world opinion against itself. Cyriel Buysse portrayed the desperate
panic of the people in {*}The Two Ponies{*}. In the first weeks, more
than a million Belgians fled to the Netherlands: one in seven inhabitants!
France also took in hundreds of thousands of Belgian refugees. The
army only had to surrender the fortress of Antwerp on 2 October.

Minnaert did not want to fight for Belgium. He had set up a hospital
in the Normal School and served as its head for two months. When the
Germans stood before Ghent, that field hospital was evacuated. He
fled to Sluis in the Netherlands on 12 October and considered studying
in Amsterdam under biologist Hugo de Vries, a friend of MacLeod. Ghent
was spared from war violence because the city surrendered: the Germans
behaved relatively correctly. He returned on 17 October when it seemed
that the Germans would not take measures against young men. He continued
his research at the Plantentuin, worked on a publication about botanical
walks around Ghent, and began self-study of Russian. Primary and secondary
education resumed that month, but the university remained closed.

In mid-October, the Belgian army regrouped in the extreme southwest
corner of Flanders behind the Yser line. Flemish civilians had pointed
out to the military leadership the inundation techniques used by Maurice
of Nassau three centuries earlier. The area around the insignificant
stream transformed into a marshy, kilometers-wide trench system. The
fame of Brave Little Belgium was immense: the entire world was engaged
in Belgian works. The pacifist writer Romain Rolland compared the
spirit of Belgium to that of Charles De Coster's Tyl Uilenspiegel,
the legendary Fleming who had always resisted the 3 oppressors. The
occupier banned passenger traffic between municipalities. Ghent lay
in the Etappe, the zone between the military operational area and
the rest of the country, where further restrictions applied. Hatred
for the occupier was universal.

The only Flemish group that secretly pursued an anti-Belgian course
was that of the Ghent dissidents from De Bestuurlijke Scheiding. Julius
Mac Leod, their spiritual leader, had pinned on the Belgian cocade
and fled to Scotland. De Bruyker, Mac Leod's right-hand man, would
remark: 'The seed sown by Mac Leod has grown: Minnaert, Picard \&
Co are his fanatical followers.' The orphaned youth found a new leader
in Reverend Domela.

\section*{The oath-sworn of Jong-Vlaanderen}

Defending Belgium had been different for the Flemish-minded volunteers
than choosing sides for the Allies. The coalition of arch-enemy France
and tribal enemy England, exterminator of the Boers, with the tsarist
Russia of 'butcher' Nicholas II could stir little enthusiasm. The
most radical Flemish nationalists, including Minnaert, did not see
Belgium as their fatherland and refused to defend it.

The group of De Bestuurlijke Scheiding had remained intact. During
the warm late summer of 1914, they often gathered at Minnaert's home.
The group expanded with telegraph officials Jozef Boulangier and Frans
Primo, architectural draftsman Lodewijk Pintelon, 'exchange agent'
Omer Steenhaut, and teacher Robert Rens. Two striking personalities
joined them. The first was the historian Leo Picard, a student of
Pirenne and a friend of the Dutch nationalist Frederik Carel Gerretson,
alias the poet Geerten Gossaert. The second was the older Dutch reverend
Jan Derk Domela Nieuwenhuis Nyegaard.

The young people had hoped for a miracle for Flanders, for a Messiah.
Their wishes were fulfilled. Domela had a long beard, piercing eyes,
and a prophetic presence. He presented them with his bold plan inspired
by Paul: 'If this World War shatters the monster that oppresses the
Flemish people, and forever crushes the Belgian-Brussels-Royal-ministerial-parliamentary
Walloon clique, Flanders will awaken.' In 1918, he described his own
role: 'In God's strength, I grasped a dying nation with a mighty hand
and breathed the life-giving idea of an independent Flemish State
into those who had been half-dead for centuries, devoid of the capacity
to form a state.' He had fulfilled Ezekiel's vision from chapter 37,
which Flanders' poet Vuylsteke had already celebrated:
\begin{verse}
'But no!... withered bones,

grow together again into men!

Let sinews and muscles wrap around your bones once more,

and flesh cover you!

And you, O Spirit of God,

breathe life into these dead ones,

and they shall live! Obey His command.'
\end{verse}
These statements perfectly capture the Protestant zealot Domela.

The group chose the name Jong-Vlaanderen on October 30, 1914, borrowing
it without permission from the circles around Minnaert's De Goedendag.
Domela became the leader. His program proclaimed 'the necessary attachment
from a Germanic perspective from the Dutch-speaking parts of Belgium
and French Flanders to Germany.’ The name and the state of Belgium
had to disappear. The second language would henceforth be German.
The fate of Wallonia was ‘of no concern to the Flemish people’. Flanders
had to become a dam ‘through which all Romanic influence will be stopped
forever.’ Domela presented himself as a ‘pan-German’.

Minnaert was the first to speak ‘about the victories of the Germans
and the importance that the Flemish people have in this.’ He distanced
himself from Domela's argument: ‘The goal must remain Greater Netherlands;
cooperation with the Germans is not really in line with the Flemish
people’s approach.’ Still, he accepted the pro-German majority and
became the second secretary in a board of four. The eight men signed
a Declaration of Principles. At the meeting on November 13 at the
presbytery on Coupure, Domela ‘reminded the gentlemen of the seriousness
of Young Flanders’ conspiracy against Belgium, of the life dangers
that arise from it, and of the necessity of unity.’ Domela made them
‘swear loyalty to each other in times of need and death until the
end.’ That all this took place outside the half-dead Flemish people
was no problem for them. An elite had to show the way to the people:
they had learned that from Mac Leod.

They had wanted to stir this people into action. That was a passed
station: they had to liberate the passive Flemish people with the
help of the Germans.

\section*{Minnaert's dreams and the Flamenpolitik}

Minnaert seemingly provided a critical voice in this company. At his
insistence, ‘annexation by’ was changed to ‘rapprochement with’ Germany
in the secret program. At Christmas 1914, Domela sent an anonymous
Message to Wilhelm II on behalf of Young Flanders. For Flanders, after
the trial, the hour of deliverance from Romanic compulsion had come:
‘Then we will become loyal border guards of the mighty Germanness
under the high leadership of Your Majesty.’ Shortly thereafter, the
group sent concrete wishes to Berlin: the Dutchification of Ghent
University and administrative separation.

Minnaert's own viewpoint is outlined in a letter he sent to the philosopher
Bolland. He was apprehensive about the future, particularly when considering
the nonsensical way the nations of Western Europe are weakening each
other while the Slavic threat looms; we become wildly desperate at
the thought that Flanders might lose its last piece of Germanic tribal
consciousness and fall permanently under French influence. He realized
that a German defeat would be disastrous: 'If the Allies return, then,
as we know, a terrible reaction will occur; a professor told us: \textquotedbl It
will be a blow to the Flemish Movement, even heavier than 1830: it
will be its end.\textquotedbl{} The Flemish sympathizers will be persecuted,
and French influence will reign supreme.' Nevertheless, Flemish interests
dictated seizing the opportunity offered by their German brothers:
'When they understand that, in the interest of all Greater Germany,
Flanders must be incorporated into strong Germany, but with respect
for its own nature and language, then---we greet them as our saviors!'
He notably used Domela's term 'inlijving' and added hopefully: 'There
are signs indicating that this will happen.'

Perhaps Minnaert tended to divide the world into a good camp and an
evil camp, making Germany as white as France was black. In any case,
he cherished this illusion: 'The heroic struggle Germany is now waging
as a higher cultural power against barbarian hordes is our struggle.
That's how we think, despite everything. And in distant, distant imagination,
we see a great Germanic alliance, where each nation remains itself
but still merges into the greater whole, and where our Flanders would
finally stand again as a South Netherlandish state alongside the North
Netherlandish brothers.' Marcel apparently believed in Domela's vision.
He cautiously asked Bolland to handle the letter, which he found 'possibly
life-threatening.' The reply was to be sent to Dr. H. Wirth, interpreter
at the civil administration of the Etappeninspectie in Ghent, 'without
mentioning my name on the address.' At the time, the difference between
Domela and Minnaert was not significant. Minnaert's fear of the Slavic
and Romance threat also seemed to have racist, pan-Germanic roots.
He appeared to have forgotten his sympathy for the Czechs and ignored
the fact that the Franks were also Germanic. He grossly misjudged
the character of German Flamenpolitik, in which he had become a pawn.

At the time of the German invasion, Chancellor Bethmann had stated
that Germany would not make any claims on Belgium and would compensate
for the damage. After encountering unexpected resistance, the political
right wing opted for annexation: the German Empire could not relinquish
what so many soldiers had fallen for. On December 16, 1914, even before
Jong-Vlaanderen's letter, Bethmann instructed Baron Von Bissing, the
German governor of the occupied part of Belgium, about the opportunities
offered by the Flemish Movement: 'The German Empire can acquire the
position of a natural protector and reliable friend for a significant
portion of the Belgian population.'

The British, who portrayed themselves as the ultimate defenders of
Belgium, urged, 'Remember Belgium, enlist today.' German propaganda
now portrayed Belgium as if it were a French territory oppressing
the Germanic Flemish people. Pius Dirr, Von Bissing's advisor, saw
in Flanders on a small scale what was unfolding across Europe: the
undermining of Germanic folk strength. Dirr's ideological vision was
also shared by Wirth, the interpreter in the German administration
who became a friend of Minnaert. These intellectuals indeed viewed
the German invasion as an act of liberation for an enslaved, racially
related people.

Their idyllic vision clashed with the policy of the occupier, which
sought to maximize profit from the country. The Flamenpolitik aimed
to transform the Flemish element into a bridgehead that, regardless
of the war's outcome, would strengthen German influence in Belgium.
In January 1915, a committee for Flemish affairs was established,
and in February, Von Bissing set up a political department responsible
for propaganda, censorship, and language policy. Under the leadership
of Oscar Freiherr Von der Lancken Wakenitz, reports were produced
on the administrative division of the country according to the language
border and on ending Belgium's 'international legal personality.'
There was political support for Jong-Vlaanderen from Protestant Domela,
who, as a chaplain to wounded German soldiers, was a well-known figure.

\section*{Fredericq interrogates Minnaert}

Minnaert could start working at the Atheneum in early 1915 as a substitute,
possibly due to the shortage of teachers resulting from the war. He
also received a part-time position at the girls' atheneum. He was
occupied, aside from his teaching positions and administrative work
for Dodonaea, by Jong-Vlaanderen’s intention, supported by the Germans,
to establish a daily newspaper. The Ghent press had disappeared due
to their refusal to work under German censorship. There was a need
for printed material, if only for death notices. The future editor-in-chief
Picard believed that after a cautious start, the paper could spread
Jong-Vlaamse ideas: it should become ‘the splitting fungus of a Belgium
inclined towards its downfall, the dawn of an independent Flanders.’

Jong-Vlaanderen decided to consult several leaders of the Flemish
Movement to gain support. This required them to take public action.
On February 4, 1915, they organized an evening event at Hotel Royal
for about twenty Flemish activists. Alfons Sevens from De Vlaamse
Blok called accepting German help treasonous, ‘a horrendous crime
against Flanders.’ Those present, including Minnaert’s uncle Gillis,
advised against the venture. Minnaert chaired the meeting and abruptly
cut short the heated debate: ‘The paper has been founded and will
be published.’ A week later, a second gathering failed, and Jong-Vlamingen
decided to make personal contact with several leaders. That is why
Professor Fredericq found a note on February 14 from ‘Dr Marcel Minnaert.’

In his Diary, the historian Fredericq had cursed the Emperor on August
20, 1914, as a ‘murderer of peoples.’ He and his colleague Pirenne
had been taken hostage in December 1914. On February 3, 1915, he had
heard from his liberal friend Van Hauwaert that young radicals wanted
to establish a ‘league.’ The next day, Pirenne had informed him that
his student Picard wanted to start a weekly magazine. He had noted:
‘If Picard and the Van Roys collaborate with Marcel Minnaert, it will
become a beautiful paper of troublemakers and wastepaper.’ On February
10, he was given an oral account of the meeting at Royal. He had written:
‘Minnaert, who has already called his nephew Marcel to order, will
speak to him again tonight to beg him not to commit this moral suicide.’
The next day, the 79-year-old Minnaert visited him to say that his
nephew, supported by his foolish mother, aspired to martyrdom and
was not to be persuaded 15: 'He claims that the magazine will be a
very moderate and neutral publication and that he will only deal with
science and music in it. It will appear anyway, no matter what.'

Reverend Domela had been present at both meetings. On February 12,
Fredericq had written: 'They are committing suicide, those poor boys
of the first trousers; but Rev. Domela must fly.' So Fredericq was
informed when Minnaert visited him on February 15.

Fredericq's Diary reports: He says, 'In the main article, we want
to explain our purpose in more detail: defending rights, upholding
truth,' ('That is not possible under German censorship,' I interrupt.
'Indeed, but within the bounds of that censorship.'!!!), 'defending
Flemish language rights with the Flemish language laws in hand, which
are respected according to The Hague Convention.' (Those Germans would
not accept a Flemish university from those good gentlemen because
it would be against The Hague Convention.) So he first sells some
lengthy and complicated nonsense. I let him continue almost without
interruption. Then I ask him the following questions: Who are the
founders and contributors? He answers evasively without mentioning
names. - Where does the money come from? - Again, he answers evasively.
Some Flemish friends are doing it. Not much money is needed. They
will certainly make money, etc. - Additionally: a friend has gone
to the Netherlands to collect money there. I suddenly interrupt him:
'Domela??' - 'Exactly, Professor.' (He had fallen into the trap. That
confession is worth gold.) Another question: What are the relationships
with the Germans? - 'Through Domela in contact with Dr. Wirth, lecturer
at the University of Berlin, but a Dutchman by birth.' - I: 'No, German
on his father's side, naturalized German, officer in the German army,
an armed enemy of Belgium, not a Dutchman.' Minnaert finally declares
that they have received 'all conveniences from the German authorities
for the establishment and distribution of their magazine, car, Schein
to go to Ingooigem Stijn Streuvels to buy a serial, etc.'

Now I knew enough and had had enough of it. I told him that he could
no longer defend his fatherland behind an inkwell, but rather should
join his university comrades by the Yser or in the ranks of the Germans,
if he felt nothing for Belgium. 'Nothing for Belgium,' he said, 'everything
for Flanders.' I stood up and told him: 'I find your attitude disgusting,
even though I know you are acting in good faith. I can no longer respect
you, and I ask you to leave my house. My blood is boiling, and it's
better that we part ways.' He desperately grabbed his hat. The maid
was in the hallway: I told her, 'Please show the gentleman out,' and
turned my back on him without shaking hands, as I had done at the
beginning of our conversation. He could do as he pleased. But now
Domela has been thoroughly unmasked in his hypocritical scheming.
Fredericq called Minnaert's mother a \textquotedbl fool\textquotedbl{}
in his diary.

\section*{Removed from the ANV administration}

{*}De Vlaamse Post{*} appeared on February 21. The newspaper announced
that it would publish as much as censorship allowed: \textquotedbl However
the war may decide our fate in Europe, our goal remains the same:
Flanders above all and Flemish within Flanders.\textquotedbl{} Riots
followed at the atheneum, leading to Minnaert's dismissal. {*}Echo
Belge{*} reported on this with great relish: \textquotedbl From his
very first lesson, this pedantic joker has taught his students to
accept the hand of the Germans. Immediately, a hail of projectiles
rained down on the Ghent Zannekin: inkwells, rulers, chalk, notebooks.
Result: forty broken pots and Minnaert covered in shame and projectiles.\textquotedbl{}
Shortly afterward, Minnaert lost his position at the girls' atheneum
and was reappointed to an atheneum in Brussels by late April.

On March 2, Fredericq wrote that {*}De Vlaamse Post{*} was a filthy
rag. He quoted: \textquotedbl There is a strong conviction that it
will be impossible, no matter what happens, to drive the German troops
out of Belgium.\textquotedbl{} He added as commentary: \textquotedbl The
Minnaerts, Picards, \& Co., who send such things into the world in
these days, are worse than traitors; they are beasts without hearts
and without a fatherland.\textquotedbl{} Along with board member Van
Hauwaert, Fredericq set out to expel the young Flemings Domela and
Minnaert from the Ghent administration of the General Dutch Union
(ANV).

On April 7, Van Hauwaert informed Fredericq that he had forced \textquotedbl young
Minnaert\textquotedbl{} to resign from the ANV administration \textquotedbl despite
a nervous breakdown by Rosa De Guchtenaere and the ambiguous stance
of Dr. Speleers. On April 13, Van Hauwaert received a ‘dry impolite
note’ from his former student, withdrawing his resignation. Now, Fredericq
gathered the liberal men of Het Volksbelang to launch a counterattack:
these included at least teachers Basse and Van Hauwaert and writer
Pol Anri. They threatened to resign their ANV membership if Domela
and Minnaert were not removed. On April 21, Fredericq heard that the
ANV board had considered that there was no evidence against Minnaert
and that it was not authorized to make exclusions. Fredericq also
pressured Hippoliet Meert. His ANV magazine Neerlandia had congratulated
De Vlaamse Post on its appearance with its ‘robust spirit.’ He warned
Meert that ‘if M.M. and D.N. are not barred, a triumph of the Allies
would shake the ANV to its foundations!’ A special meeting of the
board brought the resolution.

Van Hauwaert revisited a statement by Minnaert, which ‘even the calmest
mind would find disturbing.’ Minnaert had said that most volunteers
in the Belgian army regretted having gone to war. He compared them
to a flock of sheep being led to the slaughter and included ‘our brave
fellow members Van der Haegen and De Kesel, who fell on the field
of honor.’ He called Minnaert and Domela a danger to the Ghent ANV:
he threatened that many members would resign if the two remained on
the board.

Domela responded that ‘a large number of intellectuals in Holland
are on our side’ and claimed not to understand the board’s stance.
Minnaert challenged those present to name articles in De Vlaamse Post
that contradicted the principles of the ANV. Chairman Speleers then
explained to him that legality excluded any political collaboration
with the Germans. Ultimately, Domela withdrew, followed shortly by
Minnaert: ‘I will resign under the same circumstances, although I
believe I am wrong to do so.’ Fredericq noted with satisfaction: ‘MM
and DN were present and challenging. But everything was done properly.
The outcome was that both gentlemen had resigned after refusing to
do so twice. None of those present uttered a word to support them.\textquotedbl{}
He felt relieved that the board had rid itself of the two Albochen:
\textquotedbl Marcel Minnaert was downcast and declared, 'I wanted
to play a heroic role among you, but I feel not everyone is prepared
for such a role.'\textquotedbl{} Like Domela, he had just shaken hands
as a farewell. During the meeting, the small Anabaptist had even led
Domela by the nose. \textquotedbl All’s well that ends well,\textquotedbl{}
Fredericq thought.

\section*{De Vlaamse Post and de Vlaamse Stem}

Picard had become editor-in-chief of {*}De Vlaamse Post{*}, while
an editorial board consisting of Domela, Minnaert, and Steenhaut would
monitor the course. Picard had also reestablished contact with his
Dutch friend Frederik Gerretson, who was the editor of the exile magazine
{*}De Vlaamse Stem{*} published in the Netherlands. Both newspapers
began publication in February 1915, and both men shared the perspective
of a free Flanders within a federal Belgium. Picard would dissociate
himself from Gerretson’s tactics and confront Domela.

On April 1, 1915, Picard had written: \textquotedbl We have always
accepted Belgian state unity. The Flemish people must claim their
full freedom in the state that governs or will govern it.\textquotedbl{}
A few days later, he provoked Domela again: the goal was ‘not to form
a small state alongside French, German, English, Dutch states, or
Wallonia or any other small state.’ The German civilian administration
also disliked Domela’s radical strategy, which they considered counterproductive
to their Flamenpolitik, and instead supported Picard. On the other
hand, the German military leadership saw an ally in Domela. Minnaert
did not seem involved in these skirmishes. He reviewed Beethoven's
symphonies for the newspaper and, when asked, stated that Picard’s
approach seemed ‘more practical’ to him than Domela’s.

These intrigues unfolded against the backdrop of the stalemate of
1915. The Belgian army, under King Albert, did not need to participate
in meaningless trench assaults, but a large majority of Flemish soldiers
were still commanded by Francophone, Walloon officers. The idea gained
ground that the 'Flemish Lions on the Yser danced for Latin culture.'
Some Flemish-minded individuals began to attach conditions to their
loyalty: Belgium's war goals must serve Flanders. {*}De Vlaamse Stem{*}
wrote that the truce of God was being violated by Francophone publications
and urged a 'liberating word' from the king on the 21st.

Editor Gerretson, also Berlin’s trusted man, had used strawmen and
the help of German funds to gain influence in the Flemish journal.
His actions led it to take a 'federalist' course. In his view, Flemish-minded
individuals who refused to let the language struggle rest were definitely
not unpatriotic. On the contrary, they were the patriots who, like
William the Silent, sided with their people. Gerretson managed to
lure chief editor Alberic Deswarte into a confrontation with the Belgian
government in Le Havre. In June 1915, Deswarte raised the issue of
Flanders' autonomy within Belgium. A promise had to be made for after
the war. That 'word' was provoked by the telegram Deswarte and editor
René De Clercq sent to King Albert on July 11, 1915: 'Flemings and
Dutchmen, united in their thousands at Bussum, commemorating in the
Battle of the Golden Spurs the first foundation of Flanders' and Belgium's
independence, pay their respects to Your Majesty in confidence in
Her wise policy to preserve an independent Flanders within independent
Belgium.'

This was a cunning text. If the response were negative---and that
could be anticipated---it would undermine the loyalty of the Flemings.
If it were positive, it would set a premium on striving for Flemish
goals during wartime. On the same day, the Utrecht student section
of the ANV, led by lawyer Anton van Vessem, urged Von Bissing in a
motion to take up the Dutchification of Ghent University.

The king’s response was: 'His Majesty believes that the legal authorities
of the country, when the nation shall have recovered the free exercise
of its sovereignty, will take all measures necessary to safeguard
the aspirations and interests of his people. Meanwhile, the King makes
an urgent appeal to the Belgians that they should have no other goal
or concern against the enemy than the liberation of the territory.
Later, this would be summarized as 'Fight and Keep Silent! For now,
the editorial team thanked the king but added that the Flemish people
could hardly renounce their say in matters of vital importance. Flemish-minded
individuals who prioritized Flemish loyalty were henceforth called
activists, a term coined by the germanist Antoon Jacob, editor of
De Vlaamse Stem. The passivists were the flamingants who maintained
the truce until after the liberation of the country.

At the same time in Ghent, it was about Picard's position, who followed
the federal line in De Vlaamse Post. Domela wanted to use the newspaper
for propaganda for his anti-Belgian program and managed to win over
half of Jong-Vlaanderen. He advocated for a General Flemish Council
of radical groups, a nascent government that would negotiate with
the occupier to arrange the future of Flanders. On August 3, 1915,
the split occurred: 'We, Steenhaut, Kimpe, Thiry, Pintelon, Rens,
Domela, demand Picard's resignation.' Minnaert tried once again to
resolve the conflict but found no support: 'He claims to be just as
radically anti-Belgian as Mr. Domela and his followers but sees in
Mr. Picard's attitude merely a divergent tactic and does not wish
to acknowledge the existence of a deep rift.'

No one could follow Minnaert. There was an abyss between Picard's
and Domela's views. Minnaert did not let the conflicts affect him
and did not bring forward a new perspective. In June, he wrote a few
challenging editorials like 'Has Our Direction Changed?' and 'Our
Struggle,' in which he swore to continue the struggle 'until death.'
Shortly thereafter, the 'domelists' took over the newspaper. Picard
left for the Netherlands and became a correspondent for the pro-German
Het Vaderland. Minnaert worked in Brussels, so perhaps he could not
follow the Ghent troubles as well. He informed the school administration
that after the summer he would go to the Netherlands to study.

In the spring of 1915, the border with the Netherlands was equipped
with a wire fence charged with 50,000 volts. Minnaert had the papers
to cross the border normally. His friend Wirth had wanted to stand
bail for him. He did not go to Amsterdam, where the biologist Hugo
de Vries was based, but to Leiden. He went to the Netherlands for
physics.

\section*{Initiation into physics and didactics}

Minnaert and De Bruyker had picked up the thread of the Botanical
Society in Ghent again. From November 27, 1914, to July 4, 1915, thirteen
meetings took place, eleven of which Marcel attended and six of which
he chaired. Two of his introductions were about Darwin's {*}The Origin
of Species{*}, while his lecture on X-rays and the structure of crystals
testified to his interest in physical chemistry. His last talk on
June 18 dealt with {*}The Propagation of Light and the Laws of Reflection{*}.
The meeting on July 4, 1915, was the last one attended by the 26 members.

Minnaert must have learned from Wirth that the Germans would continue
the Dutchification of Ghent. In that case, De Bruyker would make himself
available for biology; Minnaert might be able to play a role in physics.
On September 10, he arrived in Leiden, where his mother helped him
settle on Pieterskerkhof. The first few months in the peaceful Mecca
of physics, the city of Nobel laureates Lorentz and Kamerlingh Onnes,
did him good. He was warmly welcomed as a Belgian into the circle
of physicists around the Ehrenfest-Afanasjeva couple. At the faculty,
he got to know the Bosscha Reading Room, a creation of Ehrenfest.
Students had access to the latest books and journals there. The Ehrenfests
had built a sunny study room at home, with walls lined with books
and a large blackboard recessed into them. Twenty-seven domestic and
foreign researchers organized a weekly colloquium there. Minnaert
was a welcome guest. He became a member of the Christian Huygens disputation
group, where he made lifelong friends such as mathematician Dirk Jan
Struik, fluid dynamics theorist Jan Burgers, historian Jan Romein,
and theoretical physicists Dirk Coster and Hans Kramers.

The wife of physicist Ehrenfest, the Russian Tatiana Afanasjeva, had
ideas about didactics which she presented to a circle of interested
individuals, Struik recounted in his {*}Memoirs{*}: 'During the monthly
evening gatherings, Minnaert and I were introduced to the trends of
educational reform, inspired at the time by Klein in Göttingen, Jan
Ligthart, and R. Casimir in the Netherlands, and not least by Tatiana
Ehrenfest herself.' Minnaert later said: 'The colloquium at Ehrenfest’s
home was a true experience for all of us. It was a time when the theory
of relativity was stirring up the world, while Einstein struggled
to develop the special theory into the general one. It was also the
time when Bohr's atomic model opened entirely new possibilities for
understanding matter; Sommerfeld had further developed it in a magnificent
way---somewhat to the irritation of some Huygens members who disliked
both the Germans and classical systematic methods.'

Annie Verschoor, historian and literary scholar, wife of historian
Jan Romein, remembered Minnaert as an ardent advocate of the Flemish
University. Christiaan Huygens had convinced socialists in Dirk Jan
Struik, Jan Burgers, and Dirk Coster, whom Romein had converted. However,
their Marxism was not for Minnaert. Yet Marcel must have discussed
this a lot, since his address, Pieterskerkhof 34, was Struik’s. His
room was on the ground floor: when the windows were open, friends
could simply step inside. The dispute brought Minnaert into contact
with brilliant contemporaries who had a critical eye and were not
prone to hero worship or idolatry. Struik, for example, left no stone
unturned in criticizing Bolland. It also seemed as though Marcel only
became aware of social contradictions in Leiden, which were equally
glaring in Ghent. During an excursion to a factory district in 'a
neighborhood you rarely visited, whose existence you hardly knew,'
he recalled 'the contrast between the gray poverty of the streets
and the lovely splendor of Rapenburg or Breestraat.' And, of course,
Pieterskerkhof as well. His Flemish leanings were accepted, and he
introduced the couple Romein, future writers of {*}De Lage Landen{*},
to the Flemish emancipation struggle. As for the actual reason for
his stay in the Netherlands---his upcoming lectureship at the Ghent
University---most people had their doubts. 'Despite our respect for
Minnaert, most of us could not approve of his willingness to serve
under the Germans in an occupied country'

Preparing for physics had seemed like a personal sacrifice. In Leiden,
Minnaert became acquainted with the discussion on the foundations
of quantum mechanics through Ehrenfest. He immersed himself in those
few months in Christiaan Huygens, which, like Dodonaea, was a scientific
society of about twenty men and women that met every fourteen days.
A session lasted five hours: first a lecture with a discussion, then
the actual introduction, 31 the minutes, an improvisation, and a shorter
essay. Tea, lemonade, and cookies were served between sessions. For
the Ehrenfest couple, as for Minnaert, nicotine and alcohol were taboo.
Minnaert later found it strange that the topics---from Plato's mathematics,
The Sewing Machine, and Death Movements to The First Reading of Bolland---did
not involve politics, while World War I raged around them. However,
Struik thought that Coster, Burgers, Ehrenfest, and he himself realized
at the time that a discussion about the global situation, whether
pro-German or pro-French, would have torn the group apart.

This introduction to many new friends did not prevent him from playing
his role in a confrontation that would lead to the downfall of De
Vlaamse Stem.

\section*{Domela and Minnaert torpedo De Vlaamse Stem}

After the clash with the Belgian government, Deswarte distanced himself
from the course of events. De Clerq and Jacob took over the editorship
of De Vlaamse Stem and were subsequently dismissed from their teaching
positions by Royal Decree. 32 De Clerq then wrote his To Those of
Havere, because they did not know 33 that Flanders lay within Belgium,
with the refrain:
\begin{verse}
'Do I have no right, I have no land;

Do I have no bread, I have no shame;

Flanders, Flanders, with hand and tooth

I stand up for you, Fight for You!'
\end{verse}
Activism had its first martyrs. There were protests, for the Flemish
folk poet René De Clercq was a great figure! A petition was signed
by 34 young writers such as Paul van Ostaijen and Willem Elsschot.
From then on, {*}De Vlaamse Stem{*} advocated for the unitary state
of Belgium to be replaced by a federal system of self-governance.
The magazine promoted the slogan 'First Flemish, then Belgian.' No
one knew that Gerretson actually controlled the publication.

From these months dates a curious letter from Minnaert, in which he
explains to the historian Gerretson why he disliked historical writing:
'There are indeed laws in history; but no one knows them. Yes, not
even rules are known. History so far has been nothing but a science
par après. It cannot predict. And the proof is that one predicts this
and another that.' He went a step further: 'All the so-called 'theories'
of these 'scholars' do no harm, and they may write entire inkwells
full about them, as long as they do not hinder fresh, lively action.
If Belgium were to disappear now, they would prove a few years later
that it was inevitable and had to be so, etc. One engages in history
when one has nothing better to do. (With Paul I say: 'It is still
better than burning or stealing!') But let history stay out of the
Flemish Movement.' Minnaert, perhaps under the influence of Nietzsche's
critique of Hegel's historicism, wanted nothing to do with psychology,
sociology, or history.

Meanwhile, Gerretson had presented his {*}Flamenpolitik{*} proposal
to the German envoy in The Hague. His memorandum, titled {*}Deutschland,
Flandern, Holland{*}, was a carefully devised action program for the
occupier. {*}De Vlaamse Stem{*} would advocate for federalism and
neutrality for Belgium, which was attractive to both the Netherlands
and Germany. For the Belgian government, this would be unacceptable:
'Its refusal would make it clear to the Flemish people that they had
to find their salvation elsewhere, namely with Germany and the peace
conference.' A select group of well-known flamingants would then have
to draw up a 'minimum program' and organize a 'people's petition.'
After repeated rejections by the Havere government, the German authorities
could agree with it. {*}De Vlaamse Stem{*} had to play an essential
role: 'the newspaper is officially completely free, and apart from
the drafter of this Memorandum, all shareholders, editors, contributors,
and administrators are fully convinced that they are dealing with
a Dutch-Flemish enterprise. It is of the utmost importance that this
remains so.'

But Domela was also still involved. At the beginning of September
1915, the Domelians formulated their Seven Points. These contained
new elements. An independent Flanders was to become a 'kingdom' that
could accommodate German fortifications and warports. Domela presented
the Seven Points to Governor Von Bissing, who urged patience. On November
1st, the First National Congress of Jong-Vlaanderen took place. Thirteen
local 'branches' were established, overseen by a national board with
Dr. Eugeen Van Oye from Ostend as chairman and Domela as second-in-command.
On December 5th, about seventy people publicly signed the Seven Points.
With this document, Domela traveled to Amsterdam to seek declarations
of support from some editors of {*}De Vlaamse Stem{*}. He hoped to
use these to convince the German government in Belgium of his importance.
He enlisted the help of the eloquent Minnaert.

In Amsterdam, an irritated Gerretson tried in vain to explain Domela's
tactics regarding {*}De Vlaamse Stem{*}: naturally, he couldn't reveal
everything. Meanwhile, Domela succeeded in getting the chief editor
and the secretary-editor of {*}De Vlaamse Stem{*} to sign a letter
written by Minnaert on December 15, 1915, in which they expressed
support for the establishment of a Flemish state: 'The increasingly
sharp hostile attitude of the Belgian Government toward the Flemings,
and the systematic rejection of our justified, loyal demand, lead
us to the conviction that final peace and a beneficial future for
Flanders are impossible within a Belgian state framework and that
a Kingdom of Flanders is the only solution.' This was signed by L.
Brulez, M. Minnaert, J. Eggen, R. De Clercq and E. Rietjens, along
with four out of five signatories, renounced their 'conditional' loyalty
to Belgium. Domela spread rumors that the editorial boards of {*}De
Vlaamse Post{*} and {*}De Vlaamse Stem{*} were aligned with the Jong-Vlaamse
line. On the evening of December 16, Marcel arrived in Ghent for his
Christmas vacation. His mother wrote: 'This evening at half past five,
I finally saw your eyes and pince-nez shining under the gaslight of
the station.'

Gerretson was furious. The political foundation had been pulled out
from under {*}De Vlaamse Stem{*}. At the end of January, Gerretson
stopped the publication. By 1916, the withdrawal of German support
would also cause {*}De Vlaamse Post{*} to fold. Domela had blown both
publications sky-high. For a year, {*}De Vlaamse Stem{*} had been
an authoritative propagandist for Flemish aspirations. Its distribution
among soldiers was allowed until the editorial changeover, so soldiers
on the Yser front could take note of calls for self-organization around
Flemish demands. After the changeover, loyal Flemings in the Netherlands,
led by the Catholic Frans Van Cauwelaert, had founded {*}Vrij België{*}.
At the end of 1915, the monthly magazine {*}Dietsche Stemmen{*} appeared
in Utrecht, the publication of De Dietsche Bond, an activist alternative
to the passivist ANV-Nederland. In 1916, two Greater Dutch and pro-German
monthly magazines, {*}De Toorts{*} and {*}De Toekomst{*}, also emerged,
with German money involved. In {*}De Toorts{*}, Minnaert polemicized
about the collaboration of Dutchmen in the Flemish University. Because
in December 1915, Von Bissing had declared that the German administration
would rapidly Dutchify Ghent University.

\section*{Minnaert recruits professors in the Netherlands.}

Von Bissing's decision caused a stir among the Ghent elite. The struggle
over the opening of Ghent, over the recruitment of professors and
students, became a test of strength. In late 1915, a secret policy
document had been produced within the Jong-Vlaanderen circle. This
Report regarding the opening of the Flemish University in Ghent was
signed, among others, by lawyer Jan Eggen and drawn up in consultation
with philologist Willem De Vreese. It provided an open-hearted sketch
of all the difficulties and made suggestions for appointing professors
and administrators. As major opponents, it noted the professors Pirenne
and Fredericq. On January 8, a dignified protest declaration had appeared,
initiated by Antwerp's Franck, with 38 prominent signatories who denied
Germany the right to interfere in domestic affairs. Only seven of
the fifty members of the Second University Commission had signed.
Jong-Vlaanderen mocked that this 'swan list' of Old Flanders had nothing
to say to the fresh Flemish forces.

Minnaert's position in Leiden became controversial when he emerged
in January as an upcoming lecturer and public spokesperson for the
recruitment of Dutch lecturers: 'When it became known in Leiden that
Minnaert had accepted a call to Ghent - directed at him by Dr. C.
De Bruyker and Prof. W. von Dyck - and would go there to teach, he
immediately experienced a cool reception from many professors and
students who had previously always expressed their high esteem and
friendship for him.' Ehrenfest considered cooperation with the Flemish
University as collaborating with the enemy and no longer wanted to
receive Minnaert at home. Minnaert found Ehrenfest's attitude unjustified
but continued to regard him as the most valuable educator and physicist
in the Netherlands.

The anti-German De Telegraaf, with cartoonist Louis Raemaekers, was
at the forefront of combating German politics and the Flemish University.
The newspaper gave geographer J.F. Niermeyer a platform, who recounted
that through a Dutch intermediary, an acquaintance had been approached
to see if he would be willing to occupy a chair in Ghent: 'That's
all we need! When there is a group of Flemings, so consumed by hatred
of the French that they do not realize the shame of accepting a university
from hands stained with the blood of thousands of their compatriots,
from hands that have destroyed their villages and cities and their
artistic treasures have been shattered by those who officially declared
that Belgium would not emerge unscathed from the witches' cauldron---then
surely there will be no Dutch scholar outside The Future Group who
does not consider themselves above this mess.’

47 Against this, Minnaert responded: ‘If truly capable Dutch forces
were to eventually refuse to come and teach at the Ghent University,
if they were to truly abandon the Flemish people in this decisive
moment, the consequences would be doubly horrific: the danger that
German forces would occupy the chairs rejected by the Dutch; and lasting
resentment from the best of the Flemish towards Holland.’ In reply,
the pro-Flemish Leo Simons, director of the World Library, stated
that a North Netherlander accepting a professorship in Ghent would
either be an out-and-out pro-German, a careerist, or such a convinced
Greater Netherlands advocate that he would sacrifice everything: ‘If
Dr. M.M. truly wishes to bring Flemish and Dutch people closer together
for better mutual appreciation and understanding of each other's insights,
let him explain this side of our situation to his compatriots who
would otherwise become seriously disgruntled with us.’ 49 Minnaert
argued that the measure taken by the German authorities was in line
with The Hague Convention. His opponents' arguments seemed to pass
him by.

The Germans pushed the University forward. A German advisory committee
of officials and professors was established under the leadership of
Walther von Dyck, the former rector of the University of Munich. He
aimed for the replacement of unwilling teachers. The rector became
Peter Hoffmann, a Luxembourgish philosopher advocating for Dutchification
and a moderate Flemish nationalist. In February 1916, the German administration
sent a questionnaire to professors about teaching in Dutch. Were they
‘capable’ and willing? Fredericq jokingly replied, {*}capable, mais
pas en mesure{*}, which many others adopted. Out of eighty forms,
only eight positive responses came back. Von Dyck's committee branded
Fredericq and Pirenne as leaders of a resistance that needed to be
broken. They were deported to Germany. Worldwide resistance, from
the Pope and the Spanish King to the President of the United States,
had paralyzing effects on the eight teachers who had wanted to cooperate.

A regulation by Von Bissing from March 15, 1916, amended the Belgian
Royal Decree of December 9, 1849. A circular from March 22 reminded
them of the principle of the 1878 language law: 'Dutch in Flanders.'
German propaganda emphasized formal legitimacy: they were carrying
out what, according to Belgian law, should have been done long ago.
Chancellor Bethmann pledged support to the fraternal people on April
5, 1916: 'We must provide ourselves with real guarantees that Belgium
cannot become an Anglo-French vassal state and is not established
as a military and economic stronghold against Germany. Germany cannot
again abandon the long-oppressed Flemish people to Frenchification.'

The legal arguments seemed like nitpicking to most Dutch people, as
the Belgian government rejected the Dutchification after all. From
Leiden, physicist Kamerlingh Onnes sent a German-language letter on
June 6, 1916, to his equally famous colleague Arnold Sommerfeld in
Munich, who had asked if his Leiden colleague Keesom would be interested
in the vacancy in Ghent. Kamerlingh Onnes wrote that Keesom would
accept an appointment in Germany but not in Ghent and added his unsolicited
opinion that most Flemings 'would only accept a Flemish university
if they obtain it through Belgian legislation. We think that people
in Germany do not fully understand the true views of the Flemings.'

Kamerlingh Onnes and Ehrenfest, alongside Lorentz, were the professors
on whom Leiden physics relied. Both distanced themselves from the
Flemish scientist, whom they had earlier welcomed hospitably.

\section*{The success of Flamenpolitik}

The recruitment of teachers was progressing meanwhile. Von Bissing
requested financial guarantees from Berlin for the new professors
and received them. In June, he suspended the positions of the erudite
refusers. The reopening of the university seemed to offer prospects
for economic activity in a city that was withering away under military
occupation. Flemish support for the University unmistakably came from
a broader circle of flamingants.

August Borms found some support in Antwerp with his activist Het Vlaamse
Nieuws for the Flemish University and a federalist program for Belgium.
On June 12, the board of the Catholic Flemish Alumni Association unanimously
approved the Dutchification, except for one member. Alongside Dutch
and a few German professors, mostly Flemish academics who had called
for the acceptance of the 'rectifying' university in manifestos were
appointed. These calls appeared at the beginning of September. The
one from the University Union, where Meert and Rudelsheim, the secretaries
of the University Committee, resumed their work, was signed by more
than a hundred Flemish academics. A Catholic manifesto counted sixty-six
people, some of whom appeared in both manifestos. Nineteen out of
fifty members of the Second University Committee signed. Basse wrote
to the deported Fredericq: 'The announcement of those two manifestos
naturally makes a great impression... You see: it is an important
movement.'

Minnaert prepared for his physics lectures. In a letter to his mother
on June 20, 1916, he wrote about herborizing with Burgers, Struik,
and Coster in the dunes of Egmond. She wrote: 'You will stay in Leiden
until the end of July; the theoretical lessons have ended; now you
are woodturning and filing iron: practical education for your field
of study! This is a difficult time for me, especially when I hear
the cannons roaring day and night, sitting alone here in your room.
Yet I am glad to be here among your friends and books! Now I see you
returning here within four weeks. It’s quiet again outside and also
in my heart.' She added that relatives and acquaintances disapproved
of their conduct: 'Your name was dragged through the mud, and you
were slandered as a traitor to the country along with Thiry and Kimpe...
Many no longer greet us... Still... You remain loyal to the Flemish
people.'

Jozefina Minnaert stood firmly behind her son. At the end of July,
Marcel returned. Recruiting professors was not easy, but attracting
52 students was much harder. Student Remi Bosselaers said: 'Last year,
you had to be at least a hero to dare go there. You had to go against
everything. Against your parents, teachers, pastor, mayor, against
your entire village; against the invalidity of diplomas, which even
very intelligent people, so sure of themselves, threatened with; against
a resolute boycott for the rest of your life; against yourself, because
Ghent was so far behind the lines and food there was so scarce and
the Allies' bombs were so abundant, and in the end, you had to guess
whether you would be there with four hundred or forty people, and
whether your gigantic efforts under all these precarious circumstances
would be somewhat rewarded.'

Minnaert traveled that late summer with a propaganda group through
Flemish villages to recruit students. The first copy of 'Berichten
van de Hogeschool' stated: 'Dr. Minnaert has made himself exceptionally
meritorious with regard to the preparatory measures for receiving
students in Ghent and spreading correct ideas about the new university
to be opened in Ghent among the students.'\\

Endnotes:

1 Extensive travel report in Jozefina's Diary.

2 De Schaepdrijver, 1997, chapters 2 and 3.

3 De Schaepdrijver, 1997, chapter 4. The writer Rolland, a pacifist
who had emigrated to Switzerland, was friends there with Masereel
and Stefan Zweig, persona non grata in France. Romain Rolland was
nonetheless mentioned in the British King Albert's Book, 1915, 107.

4 De Bruyker according to Fredericq's Diary, May 15, 1915.

5 Faingnaert, 1933, 96-101. Buning, 1977, chapter V, In de voorhoede,
63.

6 Buning, 1977, 70.

7 Domela's complete speech at the Ghent Academy on May 18, 1918, is
in Van de Velde, 1941, 211.

8 The minutes and statements of Jong-Vlaanderen and Domela's heroic
deeds in Van de Velde, 1941.

9 Minnaert to Bolland, December 31, 1914. Bolland Archive. Bolland
wanted to cooperate with Jong-Vlaanderen.

10 Picard had heard this from his promoter Van Houtte, who then broke
off contact.

11 De Schaepdrijver, 1997, 147. The German H. Wirth was promoted on
the Dutch national anthem.

12 De Schaepdrijver, 1997, chapter V.

13 Vanacker, 1991, 39-46.

14 Fredericq's Diary, February 14 and 15, 1915. On Marcel's card was
the editorial address of De Goedendag, Citadellaan 73.

15 The relationship between Fredericq and Gillis Minnaert, national
chairman of the liberal Willemsfonds, must have deteriorated---probably
due to differences in their relationships with Marcel Minnaert. Dedeurwaerder,
268, reports that G.D. Minnaert called Fredericq a ‘babbelmuil’ (chatterbox)
and a ‘vuiltong’ (dirty tongue) that year.

16 Echo Belge, A Gand, undated clipping. In Fredericq's Diary is a
pamphlet by J. Bidez, président de l'action patriotique, against FR.
DR. SC. M.K.P.Z. Mullaert: 'I even threw a stone at his head and broke
a windowpane. Then our teacher retreated to seek fortune elsewhere,
somewhere in the Athenaeum for ladies, where he achieved strange success
thanks to his proud male figure and his bright, blue, truly Germanic
eyes.' Minnaert's eyes were brown.

17 Report of the board meeting of April 29, 1915, in Fredericq's Diary.

18 Dedeurwaerder, 2002, 265-267.

19 Dedeurwaerder, 2002, reports that ANV-Ghent would be on good terms
with Minnaert again in August 1915. It should be noted that the ANV
board members Speleers, Meert, Wannyn, and De Guchtenaere would join
activism in 1916 when the Germans decided to Dutchify the university.

20 Vanacker, Gerretson's Voice, 52-53. This alignment is confirmed
in the correspondence between Picard and Domela's biographer Buning,
such as in letters 458 and 459. ARA-Den Bosch. Gerretson in De Vlaamse
Stem of May 11, 14, and 15, 1915.

21 A notorious French-language pamphlet had turned against the 'betrayal
of Antwerp' by Flemish city administrators. It stated that 'the time
for flamingantism was over forever.' Lode Wils considers it a German
forgery in his Flamenpolitik, 1974, but does not substantiate this.
Fredericq, however, found it necessary to write a French-language
counter-pamphlet on February 3, 1915. His closing sentences were:
'Turn away from these wretched people with disgust. Unity makes strength.'
Fredericq, Diary, February 3, 1915.

22 Gerretson's role in Vanacker, 53, 63-65.

23 Deswarte, De Vlaamse Stem, June 28, 29, and 30, 1915. Faingnaert,
131-132.

24 Antoon Jacob, like Minnaert a former editor-in-chief of De Goedendag
and former chairman of De Heremans' Zonen, De Vlaamse Stem, November
4, 14, and 16, 1915.

25 The Flemish Mail of June 18 and 28, 1915.

26 The first post-war meeting was on November

27, 1920. Tatiana Ehrenfest-Afanasjeva in Klomp, 1997, The Geometry
Education and Euclides' Axioms, 164-179. Offereins, M., Tatiana Ehrenfest-Afanasjeva,
in NVOX, 9, 2000, 490-492.

28 Struik, D.J., typescript on the Leiden study years of the seventies.

29 Minnaert on Ehrenfest in the lecture University and Didactics,
October 14, 1961.

30 A. Romein-Verschoor, Looking Back in Wonder, I, 129.

31 Minnaert, Huygens Lecture, manuscript, 1970.

32 Royal Decree of October 9, 1915.

33 Vanacker, 1991, 55.

34 De Schaepdrijver, 1997, 158.

35 Minnaert to Gerretson, November 27, 1915. Gerretson Archive, ARA-The
Hague.

36 Nietzsche, 1964, On the Use and Disadvantage of History for Life.
An essay in which Nietzsche warns against a historiography that burdens
those who want to make history. History is only 'objective' for those
who will never make history. Minnaert's view may have been influenced
by Nietzsche but is not its consequence. See Janssen Perio, 1953.

37 Memorandum of October 7, 1915.

38 Vanacker, 63-65.

39 Picard, I, 211. Van de Velde, 1941, 19-21. The Seven Points of
December 5, 1915, in Domela's manuscript, are printed at the beginning
of this book but chronologically belong on page 101, writes Vanacker
the author.

40 Vanacker, 66.

41 Von Bissing's decision on December 31, 1915, unofficially in the
newspapers.

42 Report regarding the opening of the Flemish University in Ghent,
Christmas 1915. Letter from Jong-Vlaanderen to Von Bissing, integral
in Van de Velde, 1941, 107-120.

43 Faingnaert, 402-403. Possibly learned from Minnaert; he was temporarily
his neighbor in Soest in 1919.

44 Minnaert to Jan Burgers, February 2, 1919: 'The main character
in Leiden for me is still Prof. Ehrenfest, even though his conduct
towards me and his stance on activism are unforgivable. But he is
the ideal professor and almost the ideal human being.'

45 See Raemaekers, 1914-1917. A selection of his anti-German drawings
on the CD-ROM Chemistry and Society by L. Molenaar, 1998, Nature \&
Technology.

46 Faingnaert, 398-401, summarizes the core of the controversy. The
mentioned Niermeyer is indeed the erudite geographer, former teacher
at Erasmiaans Gymnasium, and the man behind the Bos and Niermeyer
Atlas.

47 Minnaert, A Danger to the Dutch Race, De Vlaamse Stem, 18 January
1916.

48 Leo Simons, De Vlaamse Stem of 21 January 1916.

49 Minnaert, De Vlaamse Stem, 28 January 1916.

50 Vanacker, 114-115. De Schaepdrijver, 160-162.

51 Kamerlingh Onnes to Arnold Sommerfeld, letter of 6 June 1916. Archive-Sommerfeld
Munich.

52 Vanacker, 1991, 340.

\chapter{The 'Min' in Full Armor}
\begin{quote}
'What a treasure of new perspectives and discoveries, new inspiration
and new feelings can we expect now that women are taking their place
alongside men in all fields?'
\end{quote}

\section*{Mutual Service and Division of Labor}

During these war years, De Min, as he was called in Ghent, underwent
an interesting development. He initially adhered to Domela's views
and his biological racism. But gradually, he tried to formulate his
own positions. This must have been encouraged by the Leiden student
environment, where Domela and Bolland's racist and chauvinistic ideas
encountered strong opposition.

Many wondered, whether they were named Fredericq, Gillis Minnaert,
Struik, or Ehrenfest, why De Min identified so strongly with Flanders
and the fate of the Flemish University? He wanted to answer this question
himself and set his earlier positions straight. Hence, he wrote a
series of articles in Dietsche Stemmen, which were also published
as a brochure. Upon closer analysis, it was a personal defense, a
Pro Domo, in which he does not appear himself but is nonetheless the
subject. Probably, he had to consider political censorship in a neutral
country that wanted to stay out of the war and therefore could not
write concretely about the position of Germany and Flanders. The development
process of the young Minnaert can be analyzed through this brochure
from spring 1916: The Division of Labor 1 and the Principle of Nationalities.

He began by clarifying his 'biosociological' starting point. The great
Master, Darwin, had written about the struggle for life. This concept
had been adopted by the politics of modern Europe. The necessity of
war was, so to speak, clothed with the authority of science. Darwin's
argument was taken out of context but also led to misuse: he had explained
that those individuals who were best adapted---i.e., the strongest,
wisest, or fastest---were selected. The Russian Kropotkin modified
this central thesis: those individuals are best equipped in the struggle
for existence who practice mutual aid to the highest degree. Precisely
groups of animals with a strong development of this mutual aid, such
as humans, are the most widespread and show the highest intellectual
abilities. Minnaert called Kropotkin's proposition 'the first law.'

Thanks to this mutual aid, language arose as a characteristic and
condition for development. The naturalist did not overlook that reciprocity
also played a role in the development of each individual: all higher
plants and animals consist of cell complexes that exhibit an exemplary
pattern of mutual aid. These cells are of unequal kinds, so 'a second
law' emerged: 'Mutual aid goes hand in hand with the division of labor.'
This was another key point: 'If this principle truly has universal
validity, we should find its opposite in associations, not of cells
but of entire individuals among themselves.' Where cooperation between
individuals increased, a specific division of labor indeed emerged,
and 'specialization' arose. Kropotkin spoke in this context of the
abolition of competition through mutual aid and mutual support. This
phenomenon is observable everywhere in nature. Darwin's 'divergence
of characteristics' was therefore closely linked to the division of
labor and specialization.

Such views were common at the time and were propagated by pacifist-minded
intellectuals. Mac Leod had early on devoted a lecture to Kropotkin's
ideas. Mac Leod's wife, Fenny Maertens, was friends with Kropotkin
and had translated his work into Dutch. Their nephew, the Flemish
graphic artist Frans Masereel, had drawn lifelong inspiration from
it. Even the Dutch socialist Ferdinand Domela Nieuwenhuis had written
a brochure on this subject.

Minnaert proposed applying 'the law of mutual service' to the human
species: 'It is the only means by which we can somewhat free ourselves
from our usual overestimation of ourselves and through which we learn
to perceive the spirit of society with greater sharpness.' War did
not arise from overpopulation, as was often claimed: 'Instead of fighting
each other, people should rather join forces against the greatest
enemies of our species; the Bacillus tuberculosis, the Spirochaeta
pallida, and alcohol!' Marcel picked up ideas that were in the air
and tackled human society with his 'laws.'

\section*{Women's Movement and National Striving}

When applying the law of the division of labor to society, 'international
cooperation' was the appropriate means to stand strong together. This
cooperation was not based on 'equalization,' but rather on utilizing
the specific characteristics of the partners. Effective cooperation
arose through specialization. Minnaert did not have technical specialization
in mind here, but rather a division of labor related to 'perspectives
and methods.' He cited the Swedish educator Ellen Key: 'Not in the
subject that man studies, but in the way he comprehends the world
and feels it resonate within him, thereby creating the highest division
of labor.' Only the boundaries that favor an effective division of
labor between groups of people have a reason to exist: the physiological
boundary between men and women and the psychological boundary between
people of different nationalities. He elaborated on this through the
striving of the two fundamental movements of his time: the women's
movement and the national movement.

Half of the human race dedicated its efforts to creating a good home,
demonstrating great talent and rendering immense services to civilization.
However, women remained almost entirely aloof from the pursuit of
art and science as well as from socio-political life: 'What a treasure
of new perspectives, new discoveries, new enthusiasm, will we not
expect now that women are taking their place alongside men in all
fields?\textquotedbl{} Within art and science, there were no female
or male boundaries: \textquotedbl Why would a woman lose her femininity
by engaging in astronomy or sculpting marble? Woman must unlock the
world for herself, and that is in the interest of all humanity. And
she will remain a woman in all her thoughts and actions: only then
can she truly be a woman.\textquotedbl{} As long as women did not
develop their strengths in all these areas, they had not yet found
themselves, and \textquotedbl as long as we humans are only half-conscious
of our humanity.\textquotedbl{} The Swedish pedagogue Key had called
on women to participate in all life's tasks while also making use
of their specific gifts. Therefore, the women's movement was essentially
about applying the laws of 'mutual service' and 'division of labor.'

When these biosociological laws were applied to national movements,
they resulted in the concept pair 'pacifism' and 'nationalism': \textquotedbl The
entire origin of the nationality idea is inextricably linked to that
of world citizenship and the brotherhood of all people.\textquotedbl{}
Minnaert sprinkled no less than seven quotes from German early romantics.
He cited Herder, who had argued that the 'natural state' consists
of a single people with a national character. The essence of patriotism
lay enclosed in world citizenship. A nationalist does not impose his
civilization because he knows \textquotedbl that nothing is more
dangerous for the pure and original preservation of our own people
than incorporating foreign elements.\textquotedbl{}

Minnaert advocated for the inviolability of Flemish soil but opposed
chauvinism: \textquotedbl Who loves their own country the most will
also best appreciate the good qualities of other countries.\textquotedbl{}
Some rejected chauvinism so vehemently that they denied the sense
of nationality altogether. A 'denationalized state' uproots people:
\textquotedbl Strong, on the other hand, is the man who rises from
among the ranks of a people, carrying the strength built up through
generations; he brings new shimmering treasures to all humanity, new
jewels and new swords...\textquotedbl{}

No great man, De Min thought, exists without being marked by his nationality.

National boundaries also existed in the pursuit of science: \textquotedbl As
long as doubting, searching people struggle to gain a portion of truth,
science remains national. Our entire personality lies in our scientific
work as well as in our social work or art.' National differences were
expressed in the way questions were posed and addressed; in the choice
of methods; in the observation and selection of facts themselves;
in the conclusions drawn from these facts; in the theories resulting
from the research; and in the manner in which the findings were communicated:
'And, in general, throughout the entire investigation: greater or
lesser rigor; a greater analytical or synthetic ability; bold imagination
or cautious prudence; extensive familiarity with the work of others
and foreign civilizations or greater originality; respect for existing
theories or polemical force.'

For history, it was natural to be 'national': 'Please note, however,
that precisely the most interesting parts of scientific research often
remain entirely unknown to us in most cases: in what way, through
what thought process, through what qualities and associations of ideas
did the researcher arrive at what he communicates to us? There we
would observe the very strong influence of national feeling.' Without
realizing it, scientists reasoned from national limitations and assumptions.

'Truth' may thus arise from the interrelation of relative truths determined
by sexual and national divisions of labor. The peaceful competition
of these 'national contributions' to 'international cooperation' guarantees
truth and progress, Minnaert seems to suggest. Fundamental differences
must be expressed. Hence, Minnaert believed that within a nation,
there should be no mixing of perspectives, as this would harm the
clarity of the 'national contribution' to the global community. Because
such mixing occurred abundantly in Ghent, even if only as a result
of the World War, this construction of 'purely national' contributions
seems artificial.

\section*{Differentiation of humanity}

Minnaert's argument led to political conclusions. If a nation was
to be equipped for its contribution to art and science, it had to
be able to develop freely. The first condition was learning the mother
tongue at school and effortlessly using that language in intellectual
work. That was precisely what was being withheld from the Flemish
people. If Belgium suppressed this Flemish linguistic community, the
Belgian yoke had to be cast off , for Minnaert was not concrete!

Language was the main thing: 'Thus, language is the first factor for
humanity that provides continuity, connecting us to the past and the
future. In this way, it develops alongside mankind, keeping pace with
our advancing spirits, taking something from everyone, and giving
everything to everyone, uniting every people in a powerful spiritual
communism.' Minnaert's 'communism' was a communism of the spirit,
of language, and of the unity of a people. Language was the first
principle for the 'distribution of labor among humanity': knowledge
of the mother tongue ensures that the energy inherent in a people
can be utilized. This was also a moral imperative, for which everything
else had to yield. This reasoning could implicitly justify collaboration
with the Germans if they gave space to the mother tongue. Minnaert's
views on humanity's striving for differentiation and specialization,
for achieving 'more diversity,' as it were, toward a grand entropy,
guided him.

In an era of rising workers' movement and political struggle between
Catholics and socialists, also a time of friendship with socialists
like Struik, Coster, Burgers, and Romein, both Marxism and Catholicism
remained fundamentally alien to him. Religion and class were only
conditionally relevant to Minnaert: if they divided a country, cooperation
was hindered. 'Social class can be very important in many respects.
However, for the sake of obtaining a harmoniously differentiated humanity,
division into hostile social classes, with the blurring of other boundaries,
is certainly not suitable. Whether someone emerges from the ranks
of the proletariat and dedicates themselves to science, or whether
someone from the bourgeoisie does so---their nature (all else being
equal) will not differ fundamentally. Improving the fate of workers,
even radically reforming society, can have many good consequences
for humanity; more capable and productive individuals will emerge
in every field; cooperation among all people will also be promoted.
But more diversity will not be achieved through this...’ And it was
this last point that mattered to Minnaert. The goals of class struggle
and religion were essentially identical: making everyone equally entitled
or equally believing. If they succeeded in this, the religious or
social principle of division would lapse, consequently, it was not
fundamental and had to fade away. More practical principles should
be sought. Marcel decidedly did not want to follow the socialists
in their concept of 'class struggle.'

According to this iron logic, Minnaert should have approached racial
diversity positively. That did not happen. However, he did distance
himself from his earlier fixation on Germandom. He found 'language'
as an expression of culture far more important than race: 'It is high
time that people stop fantasizing about the Germanic race or the Latin
race. As soon as a nation adopts a particular language, it also acquires
the corresponding way of thinking and feeling. The influence of race
is probably infinitely small, or even zero according to modern research.'

That was a bold statement, an overcorrection to the other extreme,
in which he conveniently skipped self-criticism. Had he not, in his
1914 Christmas letter to Bolland, identified both the Slavic and Romance
races as threats? Yet, on this point, he intellectually distanced
himself from pan-German Domela.

\section*{Against the Belgian state}

Minnaert arrived at his treatment of attitudes toward the state through
these pure thought steps. The state was indispensable in disseminating
intellectual property and functioned through a network of officials
for that purpose. The state had deep cultural value as 'an organization
of cooperation and division of labor among people.' The state became
the link between nationalism and internationalism. From this, two
principles could be derived: 'the necessity of disarmament' and 'the
disappearance of wars,' which indeed hindered international cooperation
to a great extent.

On a national level, the state existed for the people. It could foster
the characteristics of the nation. The history of the state showed
that its role had been increasingly misunderstood over time. The state
had become an end in itself: 'The State abuses its financial and economic
power to favor a small number of capitalists, who in turn become the
pillars of the State; it directs the entire education system towards
inspiring awe and reverence for itself; its officials become salaried
defenders of the existing state form: decorations, national anthems,
official displays are introduced. And finally, the State wastes an
immense amount of energy on organizing armies, fleets, fortresses,
which serve no other purpose than to strengthen its own authority
by hypnotizing the people and conquering power over other states.
Unfortunately, this criminal, misguided state policy, based on violence
and deception, still prevails around us...'

In its territorial drive, the State does not care about nationalities,
even tries to eradicate them if it deems it desirable for its unity:
'Where peoples of roughly equal strength inhabit the State, one can
usually not be completely suppressed by the other; and there the State
makes every possible effort to build an artificial, half-hearted state
people ad hoc through mixing and to promote language confusion, not
hesitating to make all true civilization and noble culture impossible
in its territory for centuries.' The last part referred to Belgium,
though everything had to remain implicit.

He noted that even national sentiment was made subservient to the
State: 'More than once we have seen how the most intimate, sacred
national feeling is used by the State to stir up a people against
their own brothers and interests, despite all bonds of language or
history and religion; and how one calls upon the songs and legends
of that people, the poets and thinkers, to whip it up, while flattering
national pride. This creates a disastrous confusion between People
and State. In this way, more than once a people has been murdered,
destroyed by means of her best affections.'

According to Minnaert, millions of people faced an unavoidable conflict.
They had to choose whether to remain loyal to the existing state form
to which their people were being destroyed or to remain true to their
nationality.

He was referring here to four million Flemish people. The State is
based on war and diplomacy and obeys the wishes of a few selfish individuals:
'Holy patriotism cannot be confined to a State that has been delineated
at the green table. It does not form part of the inventory transferred
from the former owner to the incoming victor. It remains forever and
always loyal.'

Minnaert assumed at the time that Belgium, like the great powers,
was also to blame for the outbreak of World War I by playing into
French diplomatic cards. That was a lesson from activism, but it lacked
any foundation. His conclusions were: 'The serious solution to the
nationality question is the first and indispensable condition for
world peace, and thus an international rather than an intranational
affair.' And: 'The solution can only be: the complete parallelism
between civilization groups and state structure.'

Minnaert advocated in this closing sentence for a separation of nations
based on language, as in Belgium between Flanders and Wallonia. On
the terrain of international relations, such a stance could promote
annexations. Would German-speaking Swiss people really want to join
Germany to fight for Wilhelm II? Once those cantons had left the German
Empire to avoid permanent war conflicts.

He finally propagated a world of mutual service and division of labor.
He added that he had consciously placed himself on an ahistorical
standpoint because he was focusing on the future rather than the past.
If one wants to curb the 'fresh energy and drive of youthful striving
minds,' then history is misused to halt evolution. And that happens
all too often! Minnaert had written this to Gerretson at the time,
and on this occasion, he joined a speech by the socialist Henriette
Roland Holst in which she had pleaded for 'fearless idealism' to tackle
unacceptable social conditions and shake off the slave mentality:
'We know what immeasurable energy humanity can dispose of when a great
goal is in sight.'

\section*{The min in full armor}

It was a defense document from a young Flemish intellectual: hyperindividual,
abstract, apodictic, straightforward, and irreconcilable. Bolland's
lectures on 'pure reason' had borne fruit. Min was therefore considered
in activist circles as the man of unwavering theory. He accounted
for his reckless actions and presented a strange brew, borrowed and
self-made, alternately stimulating and suffocating views. He justified
his activism with a conclusive construction that compelled voluntaristic
and radical action: a personal armor.

This young scientist transformed the reality of the World War into
a hyperindividual, personal reality. He drew large circles of magic
around himself of pure thoughts, which he did not test against the
reality of the war. He could abstract himself from the destruction
of Flemish cultural heritage, from the corrupt background of German
Flamenpolitik, from the gas warfare that the Germans had begun in
1915 near Ypres, or from their 'total' submarine warfare. His circles
of belief shielded him from reality so that he could live safely in
the world of his constructs. From the armored Minnaert, a short circuit
between fiction and reality was not to be expected. The collapse of
his constructs could only come about through a harshly intruding reality.
In December 1916, he would still congratulate the Emperor on his brilliant
victory in Romania. How would he react to the reversal in Germany's
chances in the war?

In the letter to Bolland from late 1914, it became clear that he was
still aware of the great risks that Jong-Vlaanderen was running, but
gradually he seemed to believe that the vast majority of the Flemish
population 'actually' supported the activist leaders. After all, he
had irrefutably proven that the activists had the salvation of Flanders
in mind. Belief in these constructs gave De Min the necessary certainty.

Finally, it is characteristic of Minnaert that he needed this justification;
at this point and in this form. Other activists, such as Antoon Jacob,
Lodewijk Dosfel, or Herman Vos, also justified themselves during the
World War, but they dealt with concrete circumstances and made real
considerations.

In October 1916, Minnaert threw himself into the realization of the
Flemish University, and that was concrete enough. He was also a doer,
a fanatical fighter for an ideal that finally seemed to be fulfilled.
That perspective could also have persuaded former fellow fighters
from the Ghent ANV board, such as Speleers, De Guchtenaere, Wannyn,
and even Meert, to participate in activism.\\

Endnotes:

1 Minnaert, 1916a.

2 On the fifth Vlaams Natuur- en Geneeskundig Congres in 1901 in Brugge,
Mac Leod had lectured on the Fight for existence and Mutual Service:
- A Factor of Evolution, 1902 was translated from English by Fanny
Maertens.

3 Van Parys, 1995, 25-27. The woodcarver Masereel is four years younger
than Minnaert. The Ghent native Masereel fled to Switzerland in 1914
and engaged in pacifist agitation there. He was only allowed back
into Belgium after the 'muzzle law' of 1928. Masereel and Minnaert
share many similarities, such as their peace activism and their love
for the poetry of the American romantic Walt Whitman.

4 De Rooy, 1995, also mentions Ferdinand Domela Nieuwenhuis, 1910.

5 Nowadays, 'mutual service' as a basis for human evolution is a respectable
scientific standpoint. Think, for example, of Edward O. Wilson's Sociobiology.
The New Synthesis (1975), On Human Nature (1978), and The Future of
Life (2002).

6 Key, E., Kärleken och Äktenskapet, Stockholm 1911. Minnaert read
the original and referred to it. It was translated in 1916 as The
Love and Marriage.

7 This early 19th-century German movement focused on the unification
of Germany. Bismarck's military solution (1870-1871) put an end to
the democratic character of this German nationalism. Minnaert did
not realize that the philosophers he cited had not gained prominence
in Germany. More respect for history could have served him well.

8 Romein, 1967, chapter VII on The Legacy of Chauvin.

9 One source was Pierre Duhem, who in his La Théorie Physique, Paris
1906, demonstrated with numerous examples that the same expressions
of nationality could be found in physics as in literature, legislation,
and history. For the Netherlands, Minnaert wrote this passage: 'The
equation of state by Van der Waals, the asymmetric carbon atom by
Van 't Hoff, the electron theory of Prof. Lorentz are fundamentally
expressions of the same Dutch striving: to penetrate the deepest causes
behind seemingly mystical facts and to think them simply, almost to
grasp them.' The respectful 'Prof.' probably had to do with the fact
that he had actually met Lorentz in Leiden.

10 Minnaert quoted from Wilhelm Wundt's Völkerpsychologie, Band I,
Die Sprache, Leipzig 1900. He referenced the work of the Groningen
psychologist Heymans and Wagner's praise for alliterative verse in
Oper und Drama. His argument was also a display of erudition.

11 He cited Fr. Boas, Kultur und Rasse, Leipzig 1914.

12 Minnaert to Bolland, January 31, 1914, Bolland archive.

13 Wils, L., 1974, refutes the 'fact' that Walloons and francophiles
massively violated the truce of God during that first year of the
war.

14 First lecture for the Leiden Association for the Study of Socialism,
March 30, 1915. Perhaps. had he seen Henriëtte Roland Holst-van der
Schalk's speech on Struik's table. Etty, 1996, chapter 13, about the
euphoric mood of HRH in the spring of 1915.

15 Basse, 1930, I, 174. Letter dated December 7, 1916, on behalf of
Jong-Vlaanderen.

16 A clarifying passage about this 'hour U' by De Deurwaerder, 2002,
286.

\chapter{Radicalism to the extreme}
\begin{quote}
'I feel great pleasure in spreading as much science as I can with
the knowledge I possess.'
\end{quote}

\section*{The late reaction of the Belgian government}

Only at the end of August 1916 did the Belgian government respond
mildly to the German decision to Dutchify the university. Rector Hoffmann
and his adjutants would lose their royal honors. They probably didn't
lose any sleep over that. The opening took place on October 21. The
day before the opening, the government announced that after the war,
parliament would resolve 'la question de la transformation de l'Université
de Gand.' The Belgian Documentation Office in The Hague interpreted
this as the government envisioning the Dutchification, but promptly
contradicted it.

In his speech at the start of the academic year, Von Bissing emphasized
that he was giving Flanders the university it legally deserved. The
rector highlighted the serving role of the university in society.
In the auditorium, five hundred Flemish sympathizers gathered with
the lecturers in their brand-new robes, German officials and military
personnel in civilian clothes. Apart from Minnaert, his mother and
godfather Gillis were also present. The festive joy coincided with
the first raids for the German war industry. On October 20, workers
in Ghent were herded like cattle and driven to the station. A new
phase of the World War had begun.

On August 28, 1916, Hindenburg had become chief of staff and his right-hand
man Ludendorff chef of the Oberste Heeresleitung. Peace had become
an impossimal topic, the soft Flamenpolitik was transformed into a
harsh occupation policy. This concentration of military power undermined
the civil government. Submarines torpedoed cargo ships carrying aid
for Belgium, regardless of the Relief for Belgium in the flags. Under
these circumstances, the decision to collaborate with the Flemish
University was an irreversible step.

The pacifist Vermeylen wrote that anyone who wanted to take up the
Flemish cause after the war would need 'pure hands.' He loathed a
university opened by 'gallooned murderers and exploiters of our people,
including our Flemish people.' And he addressed Minnaert and others
when he wrote that after a German defeat, 'the great exiles will return,
and the cheers they will receive will be your verdict... Because your
work at this moment must - yes, has to be destroyed. We want a Flemish
University, but one not burdened with German original sin.'

\section*{Minnaert preaches rebellion against the fathers}

At the university, 43 professors and lecturers were appointed, of
whom 36 were new. Out of 25 Flemings, the majority had signed the
university manifestos. Classes began with 38 students and two female
students. During the course of the academic year, this number grew
to 138. In the second year, it rose to nearly 400 enrollments: 366
boys and 26 girls. Some men escaped German forced labor in this way.

Even Flemish nationalist parents kept their children at home because
they wondered what would happen to them if the Germans lost the war.
Minnaert addressed a reproachful word to these parents in the activist
De Goedendag. 'Where are the parents who now speak to their children
about loyalty to the Flemish ideal, about the duty of intellectual
youth towards the people? Where does one hear the manly language that
befits a Flemish father when he speaks to his son? In a much higher
degree, girls suffer under this constant lack of freedom, and it is
no coincidence that Bernard Shaw says: 'Home is the girl's prison!'

In the middle of his argument, he suddenly switched to a plea for
private space: 'Such a little room of your own, with your stuffed
bookcase and the inkwell with the pile of pens and pencils, and all
the familiar things---the color of the table and the train whistling
in the distance---late at night, the wonderful fresh air when the
window has been open; that whole atmosphere, that piece of yourself.
Don’t parents feel that their children have just as much right to
a room of their own---as small as it may be under the roof---as
they have the right to their own thoughts, their own joys and sorrows,
their own personal lives, yes, to solitude?'

For Minnaert, who must have thought of his own balcony room, these
two punitive predications belonged together. He was concerned with
the conflict between educators who wanted to muzzle children and children
who wanted to fight for their freedom. He cited from {*}The Clouds{*}
by Aristophanes:
\begin{verse}
'How glorious it is to go into battle with new and clever ideas

And to despise prevailing opinions!'
\end{verse}
Minnaert wrote that lack of freedom was passed down from generation
to generation. Fathers imposed duties on sons. Children who had rebelled
eventually continued in the same way. Young people who had chosen
pedagogical reforms at university soon belonged to the worst school
tyrants. He urged the students to preserve 'the memory of their youthful
desire for freedom as a precious treasure,' so that they could call
out to their own sons:
\begin{verse}
'Sei Du! Sei Du!

And if your old father ever speaks to you of filial duty,

My son, don’t obey him, don’t obey him:

Listen to how the Föhn wind awakens spring in the forest.'
\end{verse}
Did this also convey indignation over the commandments his own father
had imposed on him? Or was he merely fanning the flames of rebellion
so that young people would push aside their cowardly parents and enroll?
What mattered now was action; the future would decide who was right.

\section*{A New Physics Education}

Lector Minnaert thoroughly turned the physics curriculum upside down.
There was no compassion for the 'old wig' whose function had been
stolen by a young man with only a short year of Leiden physics under
his belt. After all, he believed he was doing it much better than
his Francophone predecessor.

In a letter addressed to the Natural Science Students, he assured
them that the exact sciences were not 'dry': 'They are the study of
reality and the entire nature surrounding us; reality and Nature are
not boring but full of mysterious wonders whose explanations provide
the highest satisfaction for our minds and sense of beauty.' He recommended
Dutch, German, and English books, trusted they could read French fluently,
advised them to learn Swedish, Norwegian, or Danish ('three months
for a Fleming'), while knowledge of Italian was actually necessary
for electrical engineering. The students must have been amazed. Just
like with his eclipse prize question, Minnaert created an idealized
image of a Flemish student. In the 1918 student almanac, there is
an overview of the first two years' courses: notably, he introduced
not only familiar physics topics but also 'Principles of Practical
Meteorology.'

Minnaert assigned a more important role to experiments than his predecessor.
His unpaid assistant Gaston Mahy, himself a student, helped repair
equipment and design modern instrumentation. The facilities of his
father's boiler factory came in handy, and Minnaert's experience with
home practices must have paid off. The open field also provided opportunities.
During an outing with students, they searched for whistling echoes
and found 34: 'Almost the entire city of Ghent will become a whistling
echo!'

Soon enough, Minnaert had to admit he had set the bar too high. Some
students had never had physics before, and their math skills were
insufficient. As a result, his teaching level lay between secondary
and higher education. Why wasn't practical physics being taught at
the atheneums? Reforming secondary education seemed to him 'a life
question for the scientific revival of Flanders.'

The letters to Jan Burgers testify to his efforts. After three months,
he reported: 'We have turned everything upside down; new orders must
constantly be made to provide for everything that is lacking; a modern
spirit is replacing the old, stuffy school atmosphere that lingered
in the building. Even the engineers are being released from their
former discipline and are almost becoming human again. We have set
up a beautiful reading room for mathematics and inorganic natural
sciences, which (note well!) is open in the evenings from 9-11:30
and during vacations! I have introduced physical practical exercises
for medical students (who previously had none) and for pharmacists
and biologists (the same), and I have extended it to two years for
mathematicians, physicists, and engineers. Everything I learned in
Leiden is now serving me wonderfully; of course, I greatly regret
not having had the opportunity to study for my own enjoyment any longer,
but I feel great satisfaction in fulfilling a role that no one else
could fulfill at this moment and in spreading as much science as I
can around me.'

He enjoyed the students. The student association was set up in consultation
with him, following the example of the Leiden Corps, so without separation
into liberal or Catholic. Teachers invited students to their homes
for a cup of tea: Minnaert had also done this repeatedly. He complained
about the quality of textbooks and the lack of schematic drawings:
'One would say that the authors do not dare because then one would
too well notice the unclear aspects in their representations.' He
pleaded with Burgers for a reprint of the good physics book by Julius,
a physicist from Utrecht. Minnaert and Mahy that year created advanced
demonstration equipment, which made the lectures lively: the alternating
current dynamo made its debut, as did experiments with liquid air.
In 1917, the physicist Arnold Sommerfeld came from Munich to give
a lecture on the atomic model and stayed with us for a few days as
our guest.'

When everything was in order, Minnaert wrote to Burgers, he wanted
to 'investigate the teaching of physics in our various schools. I
already know that I will discover 'unheard-of scandals' (Ehrenfest's
term): in a state teacher training school for female teachers, there
is not a single piece of physical equipment; when they talk about
a magnet needle, they cut one out of paper and use that 'see' how
he 'tunes in.' He wanted to organize vacation courses for teachers
and establish a 'school museum.' To drastically expand the inventory
of physical devices in schools, he considered setting up a special
workshop: 'There are future plans. Let there first be peace, then
everything will be fine.' In the second year, he also began to focus
on his 'favorite subject': the didactics of physics. Three senior
teachers introduced exercises in practical physics at Normal Schools,
which turned out to be a great success. In between, he worked on converting
his French-language dissertation into a Dutch-language book, which
appeared as one of the Works of the Flemish University.

His mother also contributed to the enterprise. Although Minnaert was
a lecturer and she was not his wife, she became chairwoman of the
Senate Ladies' Circle and led the Society for Social Works at the
university hospital. Supporting needy patients was intended to enhance
the university's reputation among the population.

\section*{An intimate friendship}

Minnaert could move relatively freely as the head of the University.
The front was unrealistically far away. He could attend performances
of plays by Norwegian playwrights Ibsen and Björnsen. His work completely
occupied him. He could imagine himself in a normal world. Within the
boundaries of the occupied city and within his total abstinence lodge,
he was known as a vegetarian and pacifist, feminist, and internationalist.
Writers like Ibsen and the Swede Strindberg, whom he could read in
their own languages, called on women to tear apart the conventional
lies of marriage and throw off the yoke of men. Gaston Mahy introduced
him to the work of the American free spirit Walt Whitman and the graphic
work of their former schoolmate Frans Masereel.

Minnaert was particularly charmed by the oeuvre of the Swedish Ellen
Key, following Havelock Ellis's critique of traditional marriage,
transformed it into a new morality of marriage and education. At the
time, numerous movements had emerged aiming to reorganize society;
their collective name was Lebensreform. Followers were characterized
by seriousness and asceticism, behaving like members of a secular
clergy. This seriousness suited Minnaert well, as humor and self-deprecation
were not his strongest traits.

Minnaert wanted to share his new insights with a woman. The mixed
lodge Licht en Liefde (Light and Love) of the Order of Good Templars
provided him with the opportunity. He could ask the 'brothers and
sisters' to read certain books, which would be discussed during meetings.
One of the girls was Jet Mahy, Gaston's sister. She was a regent and
had also enrolled as a student in natural sciences and mathematics.
She confided her view of their friendship in her Diary of 1918. Before
the war, she had given her consent to Rudy Hoffmann, the rector's
son, who had moved to the United States. Nonetheless, she wrote that
outsiders considered her and Minnaert as 'engaged.' Minnaert even
offered her an assistantship during her second year of study. She
refused because the work 'under such a strict teacher' would be too
heavy for her, but she felt flattered.

Jet was enraptured by Minnaert's lectures. During a meteorology lecture,
she experienced moments 'of pure scientific pleasure; understanding
a difficult question, the solid outcome of an experiment puts me in
sweet ecstasy and I feel deep joy.' On April 16, she reported that
she had been shaken out of her 'inner balance': 'I no longer have
the same peace with myself that I once knew; lasting unrest torments
me.' This was due to a visit from Minnaert and the 'delightful' collaboration
that followed. Minnaert had assigned her, as a lodge member, to read
Key's The Ethics of Marriage and The Century of the Child.

Jet had an authoritarian Walloon father who opposed his children's
Flemish inclinations repulsive, she found. She recognized the family
patterns Key had described from experience. She copied long passages
about the equality of women. The fact that many men consider their
wives as their property, she learned to see 'as a remnant of lower,
erotic feelings, summoned by desire for power, vanity, cruelty, and
blind passion.' Only complete, mutual freedom between man and woman
could open the way to love and 'complete reciprocal surrender.' These
were Minnaert's ideals, with which Jet agreed: 'I believe more and
more that our current marital life is nonsense, totally contradictory
to the natural inconstancy of men. As a rule---especially these days---every
married man engages in amorous or sexual relations outside his marriage.
The households where everything goes normally are easily counted;
yes, where everything goes well, they form the exception. I hear that
several of our Dutch or German professors avoid their wives and children
in their homeland to start fresh here. That breaks my respect for
them.'

She added dreamily: 'Sometimes I dream of the delightful simplicity
in which I would later want to see everything around me. I am a passionate
advocate of simple beauty. Lots of air, lots of light in the house;
modern, practical furniture amidst harmonious wall and window decorations.
Bright, harmonizing colors that bring sunshine; none of the excess
that disrupts unity and leaves a heavy impression.' She embraced Key's
educational ideals, which were at odds with what she herself had experienced.
Educators had to become children themselves if they wanted to educate;
not directly intervene when a child made a mistake; focus on the environment
in which the child grew up; not force a child to show affection and
reserve touches for important moments. She underlined Key: 'Those
families that send the morally strongest and most active sons and
daughters into the world are those where children and parents are
work companions and equals, in the same way as an older sister considers
a younger brother or an older brother considers a younger sister,
where parents, by being children with the children, young with the
youth, spontaneously support the upcoming girls and boys in their
development into people, by always treating them as people.'

Key's views were also based on adult illusions. The educator is not
on the same level as the child and should not be a work companion.
Her books were nevertheless, worldwide, an impetus toward a more child-friendly
pedagogy. Minnaert perhaps recognized the upbringing of his mother;
for Jet, it was all new. She discussed these intimate matters with
Minnaert; undoubtedly in all modesty. For she had promised fidelity
to her young man in America. Meanwhile, her teacher and she were swept
up in the radicalization of activism.

\section*{Resoluteness to the extreme}

The activism in Antwerp and Brussels had gradually overshadowed that
in Ghent; even the Young Flemish groups were stronger there. The star
of August Borms, a teacher at the Antwerp atheneum, rose. On January
21, 1917, Minnaert participated along with twenty-three professors
from the University in a first tribute to Borms. This event led to
an 'activist day' on February 4 in Brussels, where a tactic was established
that 'must unite us all.' At this activist day in the Flemish House,
125 Flemish militants were present, including Minnaert. They approved
a manifesto To the Flemish People and delegated the formation of a
Council of Flanders to Borms. He assembled a council of forty-six
members. Among the ten Ghent members, seven were professors at the
University. The Young Flemish leaders Domela and Minnaert were absent.
Domela felt excluded and protested vehemently. Incidentally, the composition
of the Council remained secret. A delegation of seven men made a trip
to Berlin in March. The composition became publicly known through
German press photos. The Belgian public showed itself shocked.

That year, the resistance of Flemish frontline soldiers had led to
the establishment of the Front Movement. The long-standing calm of
the Belgian military at the front enabled their leaders to build an
organization. In an open letter to King Albert I on July 11, 1917,
they wrote: 'Due to the fault of our government, the Germans have
been able to establish a Flemish University in Ghent. The Flemings
have accepted it; they have done well.' In the mud of the Yser Front,
the later Flemish Nationalist Front Party was born.

German propaganda could thus drum up support for the Flemish nationalists.
The right to self-determination seemed to be of particular importance
to the Americans. Nevertheless, the unrestricted submarine warfare
was the reason for the American declaration of war in April 1917.
In contrast, there was the German diplomatic success in Russia on
November 7, 1917, which suddenly eliminated the Eastern Front. The
German troops were immediately transferred from East to West. The
American war machine still had to get into gear. Therefore, Germany
had to force a victory on the Western Front in the first half of 1918:
the Ludendorff offensive. The precarious situation of Germany called
for further radicalization among the activists. The Council of Flanders
proclaimed the independence of the state of Flanders at the end of
1917, much to the displeasure of the Germans who wanted to retain
control over Belgium's future.

Minnaert was not only involved in this radicalization but strongly
promoted it. In March 1918, he joined the party when the Ghent activist
Jan Wannyn founded the Nationalistische Voorwacht. This marked the
start of gatherings, sometimes attended by thousands, where Minnaert
acted as a speaker and agitator. He even became involved with Wannyn's
militant magazine De Vlaamse Smeder in April. The first editorial
proclaimed: 'We want to grasp the enemies of our people with an iron
fist and strangle them mercilessly because we know it is a life-and-death
struggle in Flanders between the powerful French-minded oppressors
and the suffering, weak, misguided Flemish people.' Wannyn even found
the establishment of a Flemish Nationalist Party undesirable. Minnaert
protested: 'If we renounce political, therefore parliamentary action,
we renounce the strongest weapon that the Flemish Movement should
have used to achieve victory.\textquotedbl{} Minnaert had seen this
clearly, but why was he then part of a group that had taken authoritarian
leadership as its starting point?

Domela felt offended, wanted to turn his back on activism, and threatened
to make his archive public. Minnaert managed to keep him within the
movement. He took charge of the Ghent chapter of Jong-Vlaanderen and
organized the resurrection of the 'father of the movement.' Minnaert
wrote: \textquotedbl You can be assured that in our Group, the old
spirit of resolute action to the utmost still lives on. With you as
our leader, as the spiritual leader of our Movement, we want to move
forward more resolutely than ever before, to achieve the realization
of the 'free independent Flemish State.'\textquotedbl{} The national
congress of Jong-Vlaanderen in May 1918 offered Domela an honorary
chairmanship and a seat on the Council of Flanders. The man then delivered
a rousing speech about his leadership role: \textquotedbl The spearhead
of all Flemish people is the Jong-Vlaamse Beweging, and as soon as
this spearhead loses its sharpness and becomes dull, it can no longer
penetrate all obstacles.\textquotedbl{} And: \textquotedbl A pure
idea is strong; stronger than any alliance with impure ideas. The
pure idea of 'a Flemish state,' without any added Walloon elements,
will ultimately triumph, even if the entire world, with all its emperors,
kings, presidents, governments, and half-Belgians, oppose it.\textquotedbl{}
These passages make clear why Minnaert could not do without the pastor:
he embodied for him the pure principle, the sharp spearhead that had
to remain cutting-edge.

The impending defeat could hardly penetrate Minnaert's awareness.
Ludendorff's offensive had been halted in July. While the Council
of Flanders, on June 20, 1918, tied its fate to that of the German
Empire through an Appeal to the German people, the Germanic brother
roughly shook off his Belgian sister. On Guldensporendag, Chancellor
Von Hertling declared that he was willing to guarantee the restoration
of Belgium as an independent state. This was the work of the advocates
of a 'compromise peace,' primarily the social democrats. For the activists,
this was, according to Hippoliet Meert, the most cowardly betrayal
ever committed against a Germanic brotherhood. The Chancellor promised
during the peace negotiations amnesty for the activists but that was
also his final commitment. The Germans reserved an amount from the
Belgian state funds (!) to pay the promised compensation to the professors
of the Hogeschool.

All the more remarkable, the correspondent of {*}Het Vaderland{*}
from The Hague, likely Leo Picard, found Minnaert’s speech of August
27 to the 34 Young Flemings of Antwerp. He had concluded as follows:
'We must not waver. Young Flanders sets a goal. And note this: we
ask for everything to achieve everything. If everyone does their duty,
we will reach our goal.’ The reporter was flabbergasted: 'That is
what Dr. Minnaert said. Simple, straightforward, even simplistic,
was his speech..., but I did not get the impression that a politician
was speaking. We heard a lot about the goals and ideals, nothing about
the means to achieve them in connection with current events. I know
that a public speech is not very suitable for discussing such matters,
but it seemed to me that the speaker had barely grasped the importance
of the issue.’ Perhaps the realization of defeat only set in after
Minnaert, on behalf of the Council of Flanders, visited the front
in Northern France.

General Ludendorff had ordered Governor Von Falkenhausen, the successor
of Von Bissing, on July 8 to 'give some insightful Flemings the opportunity
to visit the front. They will better understand the value of a firm
future union between Belgium and Germany. This refers to a visit to
the salient near Ypres and a trip via Saint-Quentin to the Chemin-des-Dames.’

At that time, the trips were still intended as German propaganda,
but due to the shift in military initiative, it became clear to the
selected individuals that Germany was going to lose the war. Minnaert
visited the front lines at Chemin-des-Dames from September 14 to 16,
along with De Clercq, Wannyn, and Faingnaert. The Brussels native
Arthur Faingnaert wrote: 'Now, we have seen the battlefield south
of Laon...! Upon our return, we communicated to the German authorities
on Wetstraat that they did not need to prolong the war for Flanders.
After a short storm, the mood in Flanders would clear up, and the
self-confident part of our people would continue their struggle, even
without foreign help.’

The independence of Flanders was soon over. On September 26, the German
administration dissolved the Council of Flanders. On October 15, 1918,
the University opened its gates for the third academic year. Five
days later, classes were suspended. Even Minnaert realized that Germany
had lost the war and that the Flemish University was at the mercy
of victorious Belgium.

\section*{Jet Mahy and the furiously pro-French}

Minnaert left for the Netherlands around October 20 with his mother.
They were able to save themselves, which means Jozefina must have
withdrawn or transferred part of her assets to the Netherlands. Minnaert
was convinced that the storm would pass after a few weeks and he could
return to Ghent. He had likely placed the most valuable musical instruments
and books with his aunt, Nathalie Van Overberge, but did not expect
looting. Other activists, such as his friends Jet and Gaston Mahy,
refused to flee.

On October 23, 1918, in The Hague, Professor Bodenstein and jurist
De Koning, members of the activist organization De Dietsche Bond,
met with Reimond Kimpe and Minnaert from the Ghent group Jong-Vlamingen.
The Dutchmen believed that the activists should remain in Flanders.
Minnaert thought that Flemish nationalism was still too underdeveloped
for martyrdom: those who fled also made sacrifices. It also had to
be considered that in case of imprisonment, 'the best forces' would
be immobilized.

However, the Antwerp native Borms had said in the Council of Flanders
the previous year: 'I think you will not commit the cowardice of fleeing
and leaving behind the poor devils we have gathered in the activist
army to be stabbed and sacrificed. If we were to do that, we would
commit a cowardly act.' Borms would act accordingly. Minnaert had
still pleaded for final victory in August, while now he confidently
justified his flight. It was a metamorphosis that is hardly comprehensible.
His friend Jet Mahy, at least, had great difficulties with it.

The Allies' victory was within reach from late October onward. On
November 11, the armistice was signed and the Germans withdrew. How
Jet Mahy experienced those weeks, she wrote in her Diary. 37

Jet wrote on October 24, 1918: 'Those dear Allies have meanwhile come
to station just a few kilometers from Ghent; they promise us day after
day that we will see them fully, but it seems that cannot happen so
soon. You cannot believe how our hearts rejoice and cheer. They are
inside and we see before us the great iron gate on the Coupure being
unlocked. We are called 'political criminals' in their respectable
mouths and must atone for the crimes they have promised. Our leaders
have polished the plate for a long time - actually, much too early,
as it turns out - as is fitting for their role, right? There, the
Temple of Science now stands lonely and abandoned, ready to be stormed
by the former rulers. We will still suffer greatly from the rabble
who do not know anything about the matter; we can feel that already:
in prison, we will finally be safest. The German troops still present
here make themselves deeply hated through the atrocious destruction
they seem to carry out with real pleasure.'

November 13, 1918: 'The neighborhood was stormed today and many shameful
acts took place under the impulse and encouragement of Belgian soldiers.
We too are threatened and, as true apostles of peace, we will all
go to bed armed tonight. We no longer know rest, for we now live under
a vile regime. Bought-off scum and drunken soldiers roam unpunished
and plunder one house after another; we experience dreadful lawlessness
here. And there is a whole series of miserable regulations that we
have to read. Oh! How sunny the future seems that our saviors have
opened for us... Full of bloody plans, we stood ready this afternoon
- after being warned by neighbors - to repel the attack on the house.
Water, tar, and sturdy iron cleats, everything was prepared; the men
in front, behind the door; we, the rear guard with the sharpest and
heaviest kitchen utensils. The alarm was once given when the plundering
procession passed by; the shutters were torn open and we all stood
there, pale with rage, on guard.'

November 14, 1918: 'The night is noisy; the shouting and screaming
from stumbling drunken fools persist endlessly, even for days on end.
The barking of a dog in the neighborhood, the stamping of a horse,
everything startles you and you gaze with tense eyes into the clear,
cold November night. But we will live with resentment in our hearts
and eventually, inevitably, demand revenge...\textquotedbl{}

Like Cassandra, Jet foresaw the drama of the Interwar period.

\section*{Great sorrow and the plundering of 72 Parklaan}

If Minnaert had stayed on Parklaan, he would have experienced something
similar. The Minnaerts had left their house at Parklaan 72 undefended.
Their furniture would not escape the wrath of the mob. A reconstruction
of the 38th plundering later also gives an impression of the rich
interior of their home: 'A soldier stamped through the shutters and
windows of the ground floor with his heavy boots; then they stormed
into the two rooms on the same level, smashing the marble fireplaces
and the stained-glass windows. The nearly new Pleyel et Wolf grand
piano from Paris was battered to pieces with rifle butts and such
violence that its panels and parts soon flew around the room. An old
harpsichord made of solid walnut wood met the same fate; all the windows
in the house were shattered, the mirrors above the fireplaces were
broken, and not a single piece remained whole in any of the upstairs
rooms. Hundreds of books were torn apart or carried away, clothes
pulled from the closets were ripped to shreds, the musical works that
filled the library were torn to pieces, and when nothing else seemed
to satisfy the drunken soldiers, who appeared to be mainly searching
for money and wine bottles in the cupboards and drawers, the gang
of looters, led by neat little officer types, withdrew. However, much
had escaped the destruction: the house was large, fully furnished,
and the library was extremely rich, the cherished possession of the
house, assembled with heaps of money and careful dedication by three
studious and passionate book lovers: books in many languages, valuable
dictionaries, scientific standard works, and a treasure trove of literary
masterpieces from world literature, along with an important music
library and a few more musical instruments that the family had not
been able to save in time.'

On the afternoon of November 11, a group of civilians broke in: 'The
wrought-iron chandelier, the heavy oak wardrobe, the mahogany clothes
cabinet, the beds, the kitchen stove, the Cadé heating stove---nothing
was too heavy for the scavengers. The entire house was literally filled
with torn-up papers, and even the bath heating system was crushed
and thrown into the garden, the lead pipes ripped out and taken away,
the copper faucets stolen, and along with the large remaining copper
musical instruments of the refined music lover Dr. Marcel Minnaert,
one could see the scoundrels calmly leaving the house in groups. The
heavy carpenter's bench was thrown downstairs, and the beautiful staircase
railing was smashed to pieces.

The French-Flemish fury also turned against pacifist Flemish sympathizers.
Thus, the interior of MacLeod's country house was equally devastated.
Alone in Ghent, 156 homes were burned and plundered. Minnaert had
summoned Jet to come to the Netherlands, as witnessed by his undated
letter: 'Dear Sister! Listen for a moment, no more jokes! You want
to return to the lion's den; but now that you've experienced how things
stand there, only one thing remains to be done: go North, here. Consider
that much more is at stake now than ever before; remain calm, think
carefully, but act quickly! For truly, if you fall into their clutches,
we are all powerless.' However, Gaston and Jet Mahy were caught. Jet
was held in solitary confinement for a month, after which she was
suddenly released.

In January 1919, she received a comforting letter from Minnaert. He
had chosen the metaphor of 'The Tree of Great Sorrow' from a poem
by Henriëtte Roland Holst: 'A dark tree, with a murmuring crown, represents
great sorrow; most travelers flee from it; some are lured by the illusion
of faith (=religion); others go to dry, dreamless paths: since they
fled the Tree of Great Sorrow, the Great Longing no longer awakens
in their hearts.' Sometimes there is an individual who sits in the
shadow and 
\begin{verse}
'listens and weeps and ponders, but does not despair'.
\end{verse}
These individuals gain the strength to chew the fruits and then feel
\begin{verse}
'the power that arises in humans only from the bitter food of great
sorrow,'
\end{verse}
and when they understand the meaning of the song rustling in the tree,
they stand up and step toward a bright horizon:
\begin{verse}
'They are the ones who lead the children of men

and sow in their pale minds

The flame seed of high-flying wishes,

And fertilizing with a ray of courage.’
\end{verse}
The question is whether the Flemish Jet found comfort in the Dutch
Jet. Gaston would receive a five-year prison sentence. She wanted
to escape from 'the flame seed of high-flying wishes' and emigrate
to her Rudy in America.

Marcel picked up the thread again in the Netherlands. He never complained
about the loss of his treasures. For Minnaert, the present and future
were always more important than the past. However, his confusion became
visible in the tactics he published in {*}De Toorts{*}.

\section*{Three tactics in two months}

After his return, King Albert I made three promises. The loyal socialists
cashed in on two: universal suffrage for men and the right to strike.
The third concerned the Flemish Movement. The king said on November
22, 1918: 'The government will propose to parliament to lay the foundations
for a Flemish University in Ghent, subject to the right of the Chambers
arising from elections to regulate the specific forms and methods.’
He added that amnesty for extremists was unimaginable: 'The Flemish
population itself has already disavowed these agitations, but the
guilty must undergo the severity of harsh punishment.’

Minnaert initially viewed it optimistically. At the beginning of November,
before the plundering, he devised a tactic, almost a strategy: the
activists actually took the same position as the socialists toward
the Belgian government. After all, the socialists also sought to overthrow
the existing conditions. Activism had 'a clearly defined path’ ahead:
its representatives could be proud and need not be ashamed of changing
their tactics. After the plundering, he remained optimistic. The Flemish
question had been internationalized, and a 'wonderful crowd of young
intellectuals and convinced propagandists’ had been trained to fan
the flames of national consciousness. The Flemish University had 'its
splendid possibility of existence and forever proven.’ The Francophones
and Walloons had settled the destructions: ‘The people in Flanders
generally hold favorable views towards activism and activists.’ He
seemed to have no understanding of the population's feelings or the
severity of the offense committed by Jong-Vlaanderen.

The Conseil Académique of Ghent University accepted an address in
December rejecting the Dutchification 'forever.' Only two professors
voted against it. Rector Fredericq, back from deportation, went along
with the majority. On January 21, 1919, he opened the university in
French and announced that some courses would be taught in Dutch. After
44 months, he resigned his position. The Dutch specialist J. Vercoullie,
a lonely supporter of Dutchification, wrote:

‘This strong resistance to activism should not be surprising, especially
in Ghent, where people have been so provocatively mocked and ridiculed
by activism. Whoever is even slightly compromised by activism must
patiently await the turning point.’

The Ghent scholars made a list of traitors and asked foreign universities
to reject them. Administrator Eeman expressed the prevailing opinion:
‘Our professors left their chairs, which they could no longer occupy
honorably, due to the betrayal of the sold-outs, the opportunistic
careerism of some cancers, and the hyena instinct of some disgusting
Batavians. Woe to the victors. Glory to the defeated.’ A group of
46 teachers, including Minnaert, responded: ‘Our goal was to change
the center of Frenchification that Ghent University has been for Flanders
for over 75 years into a center of Dutch civilization.’ They faced
their trial confidently, it seemed. In September 1919, Minnaert received
the message that all books and instruments he and Gaston had acquired
and developed were destroyed by the French professors! The same would
have happened with De Bruyker's new plants and trees in the botanical
garden.’

At the first meeting of the General Board of the liberal Willemsfonds
on January 5, 1919, the four attendees, including Vercoullie and Fredericq,
took note of the resignation of general chairman Gillis D. Minnaert.
He passed away a few months later. Biologists Cesar De Bruyker and
Marcel Minnaert were expelled. Nevertheless, the Fund published a
brochure denouncing the fact that Walloon activists could move about
freely.

Every Flemish initiative was paralyzed. Even the passive resisters
received the slur 'flaminboche,' 'vlamenmof.' Passive resistor Kamiel
Huysmans, secretary of the Socialist International, observed as early
as January 4, 1919, that the Francophiles were exploiting the 'activist
error': 'The fury of the Francophiles was no longer directed at those
who had committed blameworthy acts. No distinction was made anymore
between the signatories of a manifesto for the Flemishization of Ghent
University and those consciously contributing to the destruction of
Belgium.' Huysmans warned: 'We will not be strangled!' The warning
lay in the word 'we.'

\section*{The Process of the Flemish University}

Minnaert wrote a third article for De Toorts in January 1919: To Our
Friends in the Netherlands. He had come to realize the gravity of
the situation. He began with a defense: 'Imagine sitting alone late
at night in your cozy living room, surrounded by familiar surroundings
built up piece by piece over the years, and facing a decision on which
your entire future depends; to remain silent or to speak out, to endure
or to act, to be a passive resistor or an activist. There is no longer
a middle way; life’s reality confronts you: choose! And in one pan,
you first throw the respect by which you are surrounded in society,
the greetings of acquaintances, the handshake of a friend or family
member, your good reputation in respectable circles; you throw into
it everything you have acquired in the past through the long struggles
of youth: money, possessions, position; you throw in the future that
could be beautiful, the future of your wife and children, your life’s
task which you may no longer be able to fulfill, perhaps the peace
of your old age. And on the other side of the scale... No, no! It
is not a novel; no, on the other side of the scale there is not the
proud awareness of bringing salvation to your people and fatherland;
no---on the other side lies doubt! Life is not so simple, the circumstances
are so complex and confusing that one cannot be certain what is {*}now{*}
good, {*}what{*} is the way out. And now, for the love of Flanders,
everything must be sacrificed for the sake of a solution that many
warn against and which one does not immediately consider much better
than the other. Faced with this choice, the activists have chosen
what Flemish duty commanded.’

Minnaert thus dramatized the dilemma of the radical Fleming who had
chosen his country, even though it was an impossible choice. It was
a transparent lead-up to a call for support for his fellow sufferers
in the Netherlands: ‘There must be some way to provide decent people
of good will with work that allows them to earn enough for the simplest
lifestyle. Honor the soldiers who have returned from the front for
the Dutch cause for a while!’ The Minnaerts contributed a substantial
amount to a solidarity fund for former Ghent professors and students.

Minnaert refused the Germans' compensation. He had fulfilled his Flemish
duty, but that did not mean he had served the Germans. He alone made
it a matter of principle. A settlement list from a Berlin archive
dated March 27, 1919, shows that 38 professors from the University
received several years' worth of salaries as severance pay. It states:
‘Professor Minnaert has waived any guarantee payment and merely requested
confirmation from the German government that he had not received any
payment from them.’ He could apparently afford to take this stance
financially.

Minnaert followed the Belgian legal process and wrote to his friend
Burgers, who had since been appointed a professor in Delft: ‘The first
verdict that has been pronounced imposes the death penalty on a colleague,
Dewaele, who played absolutely no significant role in the movement.
You understand that all of us, through these terrorizing measures,
are more unyielding and stubborn than ever in upholding our ideal;
the fact that Flemish soldiers in general are very much in an activist
mood gives much hope.’ About this verdict, the socialist Vooruit judged:
‘From the bad and mean, one moves to the absurd and equally dangerous.’

After the state of siege was lifted on April 30, 1919, the cases were
brought before the Courts of Assizes. August Borms had been arrested
in Belgium and sentenced to death. This sentence was also commuted
to life imprisonment after protests from the Vatican. Extradition
requests to the Netherlands regarding the condemned linguist Willem
De Vreese, who was now a municipal librarian in Rotterdam, yielded
nothing. The Netherlands granted asylum, even to Kaiser Wilhelm II.
At that time, Belgium was also at war with the Netherlands because
the Belgian government claimed not only Luxembourg and German territories
up to the Rhine but also parts of Dutch Limburg and Zeeland Flanders
as spoils of war.

It wasn’t until July 1920 that the Court of Assizes in East Flanders
handled the case of the Flemish University. This coincided with the
appointment of Henri Pirenne as rector of the university. Newspaper
reports make it clear that the trial was tumultuous, with defendants
and their defenders accusing the Belgian government and its military
security services. The Catholic leader Dosfel said: ‘I do not consider
it a disgrace, but an honor, to be imprisoned by Belgium for Flanders.’
Four professors defended themselves: Dosfel received ten years in
prison, and Cesar De Bruyker received five years. Among the exiles,
Minnaert was sentenced to 15 years of forced labor, Speleers to 12
years, student Bob Van Genechten to 8 years, and artist Jozef Cantré
to 5 years. Student Wies Moens was sentenced to 4 years and wrote
his {*}Celbrieven{*} (Cell Letters), which became literature. After
the final hearing, the ‘cart with the convicts’ could not pass through
the crowd. People shouted: ‘Long live the Flemish University.’ Cesar
De Bruyker believed that the sentences placed a heavy mortgage on
Belgium’s future because they suggested ‘that the interests of Belgium
and Flanders are irreconcilably opposed to each other.’

A year later, the leaders of the activist press and Jong-Vlaanderen
were next: Domela was sentenced to death, Leo Picard received life
imprisonment, and Jules Van Roy received 20 years. By mid-1922, there
were a total of 268 convictions, 168 of which were by default, and
45 death sentences. Not a single death sentence was carried out. The
convicted lost their civil rights and were expelled from the government
apparatus. Those condemned by default could no longer return to Belgium
and played no further political role.

Those who had defended themselves in court, such as August Borms,
Lodewijk Dosfel, Cesar De Bruyker, Antoon Jacob, and Roza De Guchtenaere,
were martyrs. Minnaert's view that Flemish nationalism was too young
for this was disproven.

\section*{The activism and the young Minnaert}

Before the World War, the Flemish Movement had matured. Groups of
young people had radicalized and concluded that they had nothing to
expect from the Belgian state. The Flemish historian Maurits Basse
wrote afterward: 'If one of these directions ripens into sufficient
power to deserve one of the leading roles and is underestimated by
those holding the dominant opinion, extremism is born out of resistance
on one side and impatience on the other.' The Ghent historian Capiteyn
sighed that the course of the World War would have been different
if the Belgian parliament had decided in 1914 to make the university
Dutch: what a moral disaster Belgium, and Flanders, would have been
spared if the government and parliament had understood the people's
interest and their duty! Then Jong-Vlaanderen would not have been
viable: after all, the only achievement of the Flamenpolitik had been
the university.

The Dutch historian Pieter Geyl later wrote that activism, which he
opposed, was inevitable: 'Injustice causes bitterness. The activists'
actions were unwise but were provoked by the foolishness of the Belgian
regime.' Geyl's student Arie Wolter Willemsen observed that decades
of disappointment among part of the Flemish Movement had cultivated
a latent anti-Belgian sentiment, which had served as fertile ground
for activism: 'It was only one step to hostility.'

Activism had fled forward and lost contact with the population. The
obsession with the Flemish ideal made the activists blind to the World
political relationships. They desired not only the victory of Germany
for the benefit of Flanders but accepted everything that could contribute
to it. The historian Vanacker therefore concluded that the activist
adventure was 'understandable, almost inevitable, but unwise.' Vanacker
also incorporated the generational aspect into his analysis: circles
of young artists and writers saw activism as a resistance against
the bourgeoisie, as a heroic, anti-authoritarian movement. This was
an aspect that had played a significant role in both {*}De Bestuurlijke
Scheiding{*} and {*}Jong-Vlaanderen{*}. The young people in the occupied
cities were given an exciting task through this movement. Writers
such as Felix Timmermans, Antoon Thiry, Reimond Kimpe, Paul van Ostaijen---and
at a greater distance, Willem Elsschot and Richard Minne---were involved
in activism. Activism created a deep divide within the Flemish Movement.
The watershed between activists and passivists in the interwar period
can be compared to the schism between communists and socialists in
Flemish relations

\section*{Snapshot 1919}

Marcel Minnaert, who fled to the Netherlands, turned 26 in February
1919. He is 1 meter 90 tall, has dark, thick hair, and brown, piercing
eyes. He takes great care of his appearance, wears a monocle, and
sports a bushy mustache in the style of Groucho Marx. After the death
of his father, he dedicated himself to culture and science, and his
youthful knowledge and abilities have drawn attention. He is accustomed
to being at the center of attention, can interact functionally with
people, deliver speeches to large audiences, and agitate passionately.
Within the Flemish Movement, he adopts a didactic and instructive
style. He is versatile, restless, and has an insatiable work ethic.
He can push acquaintances and friends away. According to his father,
stubbornness, obstinacy, and temper are hereditary traits of the Minnaerts:
his son makes no effort to prove the opposite.

Minnaert uses a vocabulary with terms such as \textquotedbl maximal
decisiveness. He has read Nietzsche and claims to have been 'steel-like'
through him. He glorifies both the youth and the older teachers. For
his decisiveness, he needs pure starting points. The philosopher Bolland
appeals to him because of pure reason. From the biologist Mac Leod,
his initiator in the natural sciences, he adopts not only his 'biosociological'
views on mutual service but also the distrust towards politics and
the humanities. He remains loyal to Domela and his pure line. The
sabotage of Flemish rights and desires ignites radical activism in
Marcel. He clings to the principles of Jong-Vlaanderen, even though
they are discredited during the war. This raises questions about Minnaert's
personality. Is there a psychological explanation for his behavior?

A few observations seem possible. His hyper-individual, strict upbringing
spares him confrontations with Sinterklaas, Snow White, and Bluebeard,
keeping unpleasantness at bay. His frustration threshold is therefore
low. It is conceivable that the reluctant attitude of the Belgian
state frustrates the young man more than others and that he is quicker
to radical resistance to get his right. The world-alienating nature
of some of his convictions corresponds to the way his parents shielded
him from the outside world. His simplistic solutions to the problem
of the exiles' tactics may indicate that he lives under the illusion
that he can mold things to his will. It is difficult for him to confront
self-constructed images of reality with reality.

However, there seems to be more to it. For instance, the death of
his strict father may have awakened a desire to follow older individuals
like Mac Leod, Bolland, and Domela, who, just like his father, show
the way with great certainty.

A psychoanalytic interpretation is also plausible. At the moment Marcel
becomes a radical flamingant, he is 17 years old. That is the age
at which an adolescent separates from their father or mother, often
triggering intense conflicts to free themselves from them.\textquotedbl{}
is father, however, has passed away and, due to his final wish, makes
it impossible for Marcel to engage in a harsh conflict with his mother.
The anger he cannot direct at either his father or mother can be sublimated
into a choice of the most radical activism, in the struggle for the
destruction of Belgium and the independence of Flanders. The child’s
striving ‘to destroy the father’ is then shifted toward Belgium. The
liberation of the 'self' is deferred to that of Flanders. Identification
with Flanders is a safe way to express his anger; the unsafe way could
harm his mother.

His mother is also trapped in this triangle: Jozefina binds herself
unconditionally to Marcel. She loses herself in him and 'possesses'
him through her love. Marcel fulfills the promise of societal brilliance
and takes his father’s place. Jozefina goes along with everything
her son does out of fear of losing him. She also accompanies him to
the Netherlands, where their symbiotic relationship continues.

In line with this, an explanation for his extreme radicalism is possible.
Father and mother throw themselves at the child and block his emotional
development. The father’s commands hang around his neck like a millstone.
The child feels betrayed by adults and enters permanent rebellion.
When those in power destroy things, he erupts in savage anger over
yet another betrayal. Marcel identifies in that case with an oppressed
person, party, or cause: with Evariste Verdurme, Flanders, 'the Czech,'
'the child,' 'the woman,' and the Flemish University.

Minnaert is a man of deeply rooted resistance. He develops immensely
on intellectual and cultural levels. His drive is restless because
he cannot consciously confront the source of his anger and aggression.
He remains loyal to his great ideals and has the ability to sacrifice
himself. At the same time, he is selfish in his striving, struggles
with nuance, and finding empathy for others' feelings. His eager interest
in pedagogical and didactic issues may serve as a way to ward off
reflections on his own upbringing and give it a positive spin.

Under Mac Leod Minnaert had become a lauded biologist in Flanders/
His father had encouraged the natural investigator within him and
stimulated his capacity for self-examination. At his mother's home,
he ran an electrical, optical, and chemical laboratory. He had learned
carpentry and at the university of applied sciences, he had specialized
in metalworking and the construction and operation of modern physical
demonstration and precision equipment. He was ambidextrous and had
an unlimited dedication to natural science. In Leiden, he had learned
to think back and forth between advanced physical theories and phenomena
and had seen an elite of physicists and educators at work. He had
developed a broad cultural interest and, in addition to German, French,
English, Greek, and Latin, he also knew three Scandinavian languages,
Italian, and elementary Russian.

Someone like him should be able to find work. The Netherlands and
its colonies could once again attract an appealing brain drain of
capable, overzealous Flemish intellectuals. The country had once welcomed
Stevin, Lipsius, and Dodoens; it still knew how to appreciate such
gifts.

66 Marcel looked at the sun, which, unlike him, still came to greet
Flanders. He sought her advice. She was warm and clear. He followed
Klaas's advice and tried to be as good as she was warm and as honest
as she was clear.\\

Endnotes:

1 Vanacker, 1991, 132.

2 Vanacker, 1991, 230, mentions Gillis Desideer Minnaert as an activist.
Minnaert claimed in 1924 that his uncle was not a member of an activist
organization but wanted to attend the opening of the Flemish University.
The Encyclopedia of the Flemish Movement, including the new version
from the 1990s, mentions Gillis' resistance to De Vlaamse Post and
implicitly characterizes Gillis D. as a passivist.

3 Daane, 2000, 96. Vanacker, 137.

4 De Schaepdrijver, 1997, 169.

5 Minnaert, 1916b. The activist monthly was under the leadership of
Michel Van Vlaenderen.

6 Aristophanes, The Clouds, lines 1399-1400.

7 Dehmel, R., Lied an meinen Sohn. The core: 'Be yourself! And if
your old father ever son's duty speaks, my son, obey him then.'

8 Minnaert, To the HH students in Physics. At Vanacker, 138.

9 Letter to Burgers 11 April 1917 The whistling echoes in the first
edition from 1939 of {*}The Physics of the Free Field{*}, II, 29,
but then in Baarn and Utrecht. There is an echo under the arches at
the Stropbrug in Ghent and under the Ghent-Wetteren railroad bridge
over the Scheldt, II, 26.

11 The complaints of Minnaert and De Vreese in Vanacker, 144.

12 Letter to Burgers from December 28, 1916. Archive of Burgers.

13 Because he would become Professor Julius's assistant by the end
of 1918, this observation is significant.

14 Minnaert would later correspond with Sommerfeld. Schulmann and
Kox (1998) published a letter from Ehrenfest to Einstein dated March
27, 1918, in which he announces that he will inform the misguided
Sommerfeld 'about some gentlemen whose chatter his impressions are
based on. We know them here in Leiden.' Ehrenfest alluded to Minnaert's
position in Ghent: the prominent quantum theorist Arnold Sommerfeld
was not lectured to by either Kamerlingh Onnes or Ehrenfest and maintained
a friendly relationship, along the lines of German politics, with
the people of the Flemish University and with Minnaert. In Utrecht,
he would still visit Julius and Minnaert in the early 1920s.

15 Minnaert, 1918.

16 Vanacker, 256.

17 Minnaert to Burgers, April 11, 1917.

18 Gaston Mahy introduced Minnaert to Whitman's {*}Leaves of Grass{*}.
For Masereel in Ghent, see Van Parys, 1995, {*}At the Foot of the
Belfry{*}, pp. 19-40, and {*}Arènes de Lutèce{*}, pp. 41-51 with drawings
by Jules De Bruyker.

19 Key, 1900 and 1911. Key is a main character in Romein, 1976, which
provides the title for {*}The Century of the Child{*}, Chapter XLI.

20 Nottingham, 1999.

21 Jet was born on July 18, 1894, a year and a half younger than Marcel.
She only became a member of Jong-Vlaanderen in 1918.

22 Mahy, 1918, July 20. The quotes come from Jet's Diary of 1918,
which she wrote for Rudy Hoffmann in the US. Simon-Van der Meersch,
1982, with a photo of Jet Mahy, the first female student. Vanacker,
177, with a photo on page 145. Van de Velde, 196, mentions a Declaration
of the Jong-Vlaamse Wacht to the Council of Flanders on August 13,
1917, with Jet as the second author, urging the proclamation of the
State of Flanders. In her Diary, she cannot conceal her joy when the
Council indeed did so.

23 Vanacker, 170. De Schaepdrijver, 1997, 262.

24 Vanacker, 170-172.

25 De Schaepdrijver and Charpentier, 1919.

26 If Jong-Vlaanderen had the charismatic reverend Domela as its leader,
among the soldiers, the priest Cyriel Verschaeve played a corresponding
role.

27 Churchill, 1926. On April 1, 1917, the Aztec was torpedoed, resulting
in the drowning of 28 Americans. The next day, President Wilson requested
Congress's approval for a declaration of war against Germany. He received
it on April 6. In 1918, a million Americans came to Europe, and several
million were expected in 1919.

28 This proclamation took place on December 22 and 23, 1917, but was
only made public a month later.

29 Vanacker, 302-305. Minnaert's signature is missing from the magazine,
except under an article about the Battle of the Golden Spurs on July
11, 1918.

30 Vanacker, 313.

31 Buning, 123-125. Vanacker, 241-243. Van de Velde, 191-192, 206-248,
contains the letter of December 6, 1917, to Domela ‘in deep humility
and high veneration, with Flemish sincerity,’ and Minnaert's handwritten
letter of April 30, 1918. Jet Mahy was also involved in this rehabilitation.

32 Domela, Vlaanderens Ontwaken, Van de Velde, 211-248. Previously
mentioned in the establishment of Jong-Vlaanderen in chapter 4.

33 Faingnaert, 817-818.

34 Het Vaderland, correspondent, August 27, 1918.

35 Dietsche Stemmen, November 1918.

36 Faingnaert, report of the bizarre meeting of the Council where
independence was proclaimed, 694-722. Borms' statement is from December
18, 1917.

37 A play on words by Jet, mixing alliés (Allies) with aliénés (alienated).

38 'Flor' (Jan Wannyn) in De Noorderklok, May 25, 1930, The Belgian
terror in Ghent in 1918; the plundering of Mrs. Wed. Jozef Minnaert's
and Dr. Marcel Minnaert's home. The report will provide an image of
the destruction based on the testimonies of neighbors/family members.
Flor's article was part of a series that, following the disappointing
'extinguishing law' of 1929, aimed for real amnesty and economic compensation
for the activists.

39 The destruction of the Mac Leods' villa is mentioned in his biographical
sketch in the Encyclopedia of the Flemish Movement. In other biographies,
this is often not mentioned.

40 Minnaert to Jet Mahy, undated. 41 Minnaert to Jet Mahy, January
12, 1919.

42 Minnaert, De Toorts, November 30, 1918.

43 Minnaert, De Toorts, December 7, 1918.

44 Fredericq (1850-1920) could no longer endure the poisoned atmosphere.
In his obituary note, the rarely one-sided H. Pirenne wrote in 1924:
'After the war, Professor Fredericq was disoriented due to the excesses
of the Flemish activists who had followed activism during the war
and the reaction that followed after the liberation of the territory.'
On March 23, 1920, Fredericq died of a stroke.

45 Vanacker, 359.

46 Vanacker, 358.

47 Eeman, E., Le corps professoral de l'Université de Gand sous l'occupation
Allemande, Ghent 1919.

48 The Open Brief dates from February 21, 1919.

49 Minnaert to Arnold Sommerfeld, September 14, 1919. Sommerfeld Archive,
Munich.

50 Vanacker, 356.

51 Roemans, 1961, 190-191.

52 Minnaert, To Our Friends in the Netherlands, De Toorts, January
9, 1919.

53 Minnaert to Burgers, February 2, 1919. The verdict was from January
23. Burgers Archive.

54 Vanacker, 361.

55 Luykx, 1969, 278-280. Kossmann, 1979, After the Armistice and the
Belgian-Dutch Conflict, 429-433.

56 Newspaper article from July 11, 1920.

57 Reports from July 19, 1920. Vanacker, 361-362.

58 Prof. C. De Bruyker and the Flemish University before the Belgian
Court, 1920, 102-103. Dedeurwaerder, 2002, 404.

59 Even in Ireland, the Easter 1917 rebels only became popular after
their trial by the British. There, however, it was about the external
relationship of a colony to the mother country. The relationship between
Flanders and Belgium was one of internal colonization.

60 Basse, 1930-1934, I, 71, 156/157.

61 Capiteyn, 1991, 156-157.

62 Geyl, Nederlandse Figuren 2, 29. 63 Willemsen, 1969, 72, 74.

64 Vanacker, 368-370.

65 'Groucho' Minnaert is a creation of his friend Gaston Mahy.\textquotedbl{}

66 Back to the Leitmotiv of Part I, Minnaert in Flanders, the opening
quote from De Coster's Tijl Uilenspiegel.

\part{(1919-1945) Minnaert in the Netherlands\protect \\
Proclaimer of the Salvation of Science}
\begin{verse}
'Oh blessed are the spirits who first strove for knowledge

And rose into the star-studded heavens!

They must have been elevated above life, Human joy, and human frailty.

Neither Venus nor wine ruled their hearts,

Nor political bustle, nor rough soldier's work,

Frivolous ambition not, nor idle glory,

They did not know the thirst for money and great wealth.

They brought the distant stars closer to our eyes

By encompassing the universe through the power of their genius.'
\end{verse}
Ovidius, Fasti I, 297-306. 1 (translation: Marcel Minnaert)

\chapter{Flanders from Afar}
\begin{quote}
'There is only one thing that can help: radicalism and once again
pure national sentiment.'
\end{quote}

\section*{Asylum in Zeist and Soest}

Minnaert had settled with his mother in Zeist by the end of 1918.
They apparently had access to cash. He could immediately start working
at the Physical Laboratory of the solar physicist W.H. Julius. He
wrote to his friend Jan Burgers that he, along with his mother and
a colleague, had found a spacious accommodation at Slotlaan 70 in
Zeist. He already announced that he would miss a reunion of the fraternity
Christiaan Huygens. He mentioned having little free time, not wanting
to leave his 'old, brave mother' alone, and needing to save money
to help comrades in the Netherlands 'and in the occupied territory.'
With that last term, he delivered a concise judgment on the events
in Flanders.

After several months, they moved to Soest, where they settled at Malvahoeve
on Boschstraat 2. This was located in the humanitarian colony Chreestarchia
of Lodewijk van Mierop and Felix Ortt: it was one of the idealistic
enclaves that the Netherlands was rich in at the time. In this circle
of vegetarians, non-smokers, and teetotalers, Minnaert felt at home.
The stream of Belgian refugees in 1914 had caused much discussion
in the colony about the Flemish question. The Belgian village near
Amersfoort, De Vlaschakkers, was located a half-hour's walk away.
Chreestarchia had taken in activists such as the tailor Arthur Faingnaert
and the family of Jef and Marie Hinderdael. Jef, like Minnaert, had
been a contributor to {*}De Vlaamse Post{*}. At Ortt's home, Flemish
concerts took place with Lieven Duvosel at the piano and singing by
the Flemish minstrel Geert Dils. The debates about Flanders had inspired
Ortt to write {*}Staat en Volk{*}: according to Ortt, the Belgian
national spirit had no connection whatsoever with the Flemish folk
spirit. The so-called traitors of 1918 belonged to the finest of the
Flemish people. In a postscript, he paid 'respectful tribute' to the
condemned individuals, 'high-standing figures, some of whom I proudly
call my friends.' Among these figures were the Minnaerts, who were
his closest neighbors.

Minnaert reported to Burgers: 'We are living wonderfully here amidst
forest, dunes, heath, and meadow, in a peaceful and invigorating environment
where one can work quietly and diligently; my mother is thriving here,
putting on weight delightfully; by chance, we have been able to rent
half of a furnished villa, magnificently modern-built, a house that
is a pleasure to live in---and all for little money. There has been
some inconvenience because we have taken in a young activist lady
from Ghent, but she is wonderful company for my mother, and I won't
deny that I also enjoy her presence.' They thus offered shelter to
Jet Mahy while awaiting the formal arrangement of her crossing to
her fiancé in the United States.

Every morning, Minnaert walked through dunes and heath to Soesterberg;
half an hour later, he was on the train to Utrecht. At six o'clock
in the evening, he was home again. He didn't waste that hour of walking.
He outlined the ripples in the dune sand for Burgers: 'I might perhaps
find some aerodynamic laws in their course and maybe even derive a
measure from their mutual distance for the pressure exerted by the
wind.' That interested Burgers, who had become a professor of fluid
dynamics in Delft. The heath was full of interesting phenomena, such
as the witches' circles formed by Carea arenaria. Minnaert sketched
these and explained how only seven mysterious rings remained after
withering: 'Finally, I should tell you about a small investigation
I conducted at home regarding the tones of bubbles forming in water
and other fluids; as far as I know, it's an entirely new subject,
rich, and yet applicable everywhere around us.'

Marcel, at 26, was a busy man. A love affair seemed inappropriate
during this hectic time. Yet, his longing for a love life was evident
from his joy over Burgers' marriage to the physicist Nettie Roozeschoon:
'What a pleasure it must be to completely renew oneself, adopt new
habits, and lead a new emotional and intellectual life entirely in
line with what one had desired as a modern idealist!' Incidentally,
that modern idealism for Jan Burgers, like other Leiden comrades such
as Dirk Jan Struik and Jan Romein, was the communism of the CPH.

\section*{Reemergence of the Flemish nationalism}

Minnaert dedicated himself to providing financial aid to exiles and
restoring contact with the home front. In early June 1919, a first
meeting of the Flemish Committee took place in The Hague, which he
attended. The nationalists achieved an electoral success by the end
of that year. Minnaert wrote enthusiastically to Burgers: 'You will
have heard about the Front Party; they are the soldiers of the Yser
united with our activist fighters. For the first time, they appeared
as a national party in the November elections and won five seats.
That's beautifully beyond all expectations for those who know our
situation.' Minnaert wrote that the five Flemish representatives were
feted in Ghent and openly acknowledged declared that they owed their
victory to the activists: 'A large procession forms and demonstrates
in the city, under the cries: \textquotedbl Long live the activists!
Long live prison!'

It became increasingly clear to him that the nationalists would win:
'I am particularly pleased that the Bolsheviks immediately took a
national stance and solemnly assured the Afghan envoys of liberating
the smaller nationalities. Where a people is free, authority must
naturally come into the hands of democracy. Meanwhile, the exiles
overcome numerous difficulties but mostly keep their heads above water.'
Minnaert's remark about the Bolsheviks recognizing the 'right of nations
to self-determination' had to encourage Burgers. That this recognition
in practice would have to give way to the higher interest of the world
revolution, in this case the Soviet Union, would only become clear
in the coming years.

The demand for amnesty for the convicted activists became the focal
point of the activity of the Flemish nationalists. A movement also
emerged among the relatives of the fallen. On hundreds of graves of
Flemish boys stood Celtic crosses, designed by the Irish-Flemish artist
Joe English with the inscription 'All for Flanders/Flanders for Christ.'
After the war, the government had these replaced by simple stones
with the Belgian tricolor and the inscription 'Mort pour la patrie.'
In one case, the crosses were crushed and incorporated into the pavement.
This would in part become the reason for organizing annual pilgrimages
and building the IJzertoren in Diksmuide.

King Albert's broken promise regarding the Dutchification of Ghent
would leave its mark on the 1920s. The joy about the end of the occupation
was initially general. Because the Belgian state consciously confused
activism and Flemish emancipation, its actions were increasingly experienced
as suppression of Flemish-mindedness. The activists' negative judgment
of Belgium seemed justified in hindsight. The resentment towards Belgium
grew.

It seemed as if the Flemish-minded population of Belgium had not belonged
to the winners of the World War but to the losers. The country became
torn apart over the Flemish question.

\section*{Minnaert sacrifices his family name.}

In the summer of 1919, Marcel's godfather Gillis Desideer Minnaert
passed away. The Belgian state had still summoned him to court, although
no organizational involvement in activism was known about him. His
daughter Marie, who had married a Dutchman, was not allowed to visit
him. In a farewell letter, he complained: 'How is it possible that
the gentlemen inspectors show so little humanity toward their own
countrymen, that they will only give you a passport if I am in mortal
danger!' The obituary mentioned his two knighthoods but remained silent
about his presidency of the liberal Willemsfonds.

In 1924, Marcel Minnaert used the magazine Vlaanderen to claim the
legacy of his uncle for activism. This radical exile journal of classicist
Josué De Decker and priest Robrecht De Smet, the organ of the Federation
of Flemish Nationalists De Blauwvoet, preached irreconcilability toward
Belgium. Every issue contained the Ten Commandments of the Flemish
Nationalist. These included: 'You shall believe in one fatherland:
Flanders'; 'You shall accept all effective help for the liberation
of your fatherland'; 'You shall promote the Greater Netherlands striving
with word and deed' and 'You shall forsake Belgium with all its pomp.'
The commandment regarding assistance justified collaboration with
Germany in the past, present, and future. Vlaanderen hunted down every
Fleming who showed even the slightest trust in a federal path toward
Flanders' independence. Minnaert became a close friend of De Decker.

Minnaert claimed that the liberal men of the Willemsfonds had silenced
his uncle, even though he had been responsible for the great flourishing
of this institution: 'He was not a member of any activist association.
But his sympathy was with us and in his faithful Flemish heart, he
so fervently wished us victory! Did he not have the right to his own
thoughts?' For Minnaert, the Flemish-minded liberals were finished:
French was spoken at home by most of them. The Flemish nationalists
were now the bearers of Flanders' future: 'The work I attempt to do
for the Flemish-national movement, I consider to be the direct continuation
of my uncle’s work.' Minnaert’s claim that Gillis Desideer had secretly
chosen activism was a new revelation. He would not have discussed
this with his aunt Marie and his cousins Marie and Helena. Minnaert
sacrificed his family name on the altar of Flemish nationalism. Perhaps
it seemed to him the least he could do as an exile for his comrades.

The Minnaerts had their own scores to settle with Belgium. Jozefina
Minnaert’s house on Parklaan, after being destroyed, was sold for
half its market value. She had not been persecuted or convicted, so
she demanded compensation for the loss in value from the War Damages
Court. The case dragged on, and the outcome is unknown. The Minnaerts
were financially stable enough that Jozefina must have sold her remaining
houses by 1920. During the course of that year, they decided to build
a new house in Bilthoven, again on Parklaan. Minnaert had an observation
tower installed on the roof so he could enjoy the starry sky and sunrises.
On two gable stones, numbers 16 left and right, neighbors could read
Albrecht Rodenbach’s battle cry: \textquotedbl Does the Bluefoot
fly? Storm at sea!\textquotedbl{} Starting in 1921, their mail went
to \textquotedbl Huize De Blauwvoet.\textquotedbl{} According to
a neighborhood boy, Jozefina, with her gray hair, spent the whole
day tending to her garden: \textquotedbl In summer, it was like a
waving sea of colorful flowers.

\section*{Discussions among exiles}

Some preserved letters and cards give an impression of what occupied
Minnaert. Notable is a prison letter from the fall of 1919 by Roza
De Guchtenaere, the former activist and director of the Ghent girls’
Athenaeum, who was also one of Jozefina’s 17 former students: \textquotedbl Pity,
Marcel says, and you too want to improve my situation. Oh! I wish
I could share my joy and calm happiness with you, but my heart flies
toward you in gratitude for your strong, enduring affection that shines
through both of your letters. I had also imagined it all very differently,
much more terribly. My cell has already become dear to me like a home,
within whose walls I enjoy a far greater freedom than life usually
grants us. Solitude does not oppress me; on the contrary, I view the
turmoil of ordinary life with anxiety.’

This composed and unwavering Roza became a loyal visitor to Huize
De Blauwvoet after her release in late 1921.

Minnaert re-entered the strategy discussion in 1920. He wrote in {*}Federalisme
of Nationale staat{*} that there were two types of federalists. The
principled federalists distanced themselves from state interference
and advocated for decentralization. Among these, he counted friends
such as Jacob, Herman Vos, Gerretson, P.H. Ritter Jr., and Leo Picard.
However, Minnaert believed that a centralized Flanders could better
fulfill its positive role within a future World Federation. These
people made mistakes, ‘but ultimately, how dear they are to me because
of the honesty of their conviction.’ The tactical federalists, however,
recoiled from the ideal: ‘Why should the path that leads the Poles,
the Irish, and the Jews to their goal be bad for us?’

Minnaert himself pleaded for an independent Flanders, thereby opposing
federalists of every stripe: ‘The main thing remains to combat mixed
states, which threaten our national character with destruction and
from which it is high time we save ourselves.’ He consistently chose
the straight line, the most difficult path to follow, and the most
anti-Belgian policy.

Minnaert repeatedly wrote to the imprisoned Cesar De Bruyker, the
biologist who had replaced Mac Leod at the Vlaamse Hogeschool. He
reported that a Ghent flamingant had told him that the passivist Kamiel
De Bruyne had addressed a packed hall: ‘Such a thing is sad. That
you are not optimistic does not surprise me; the situation is far
from rosy, and I assure you that, seen from afar by an observer, the
view becomes increasingly gloomy and miserable. There is only one
thing that can help: radicalism and, once again, pure national sentiment;
perhaps it would be best of all to organize a series of theater performances,
acted by enthusiasts, perhaps poorly, but compelling and romantic;
and on a large scale, a hundred performances every week. Art speaks
to the people, awakens dormant forces, elevates man above his own
mediocrity. Daydreams? Maybe...'

At the beginning of 1924, the Catholic Flemish nationalist Frans Van
Cauwelaert came to Utrecht to speak about Flanders within the context
of Faith and Science. Minnaert, representing the Utrecht-based Flemish-Dutch
Association, had arranged with the Catholic association Faith and
Courage that an opponent would also be allowed to speak, albeit at
a subsequent meeting on the 20th. He tried to persuade the philologist
Willem De Vreese, who was then the municipal librarian of Rotterdam,
to accept the challenge. Ultimately, he found in the physician Reimond
Speleers, the second rector of the Flemish University, a capable defender
of the Flemish-nationalist viewpoint. Minnaert was deeply committed
to ensuring that the debate about Flanders would take place and that
the arguments of the Belgian government and the passive resistance
opponents would be refuted.

Minnaert became the librarian of De Dietse Bond. In a note from 1924,
the exile Jules Spincemaille wrote that he would send Minnaert his
list of Flemings in the Netherlands and abroad. Apparently, Minnaert
was officially responsible for maintaining the administration of the
activist exiles.

\section*{Board member of De Dietse Bond}

De Dietse Bond, established on June 23, 1917, aimed to replace the
passive resistance General Dutch Union (ANV). It adhered solely to
'the pure Great Dutch interest' as its guiding principle and rejected
'any foreign influence that conflicted with this interest.' Its members
could decide for themselves whether to emphasize the political or
cultural aspects of Greater Netherlands. Ultimately, Belgium was to
disappear. The Bond was neutral in matters of religion, not aligned
with left or right, and focused solely on what was best for Flanders:
'The preservation and development of Dutch civilization and the societal
welfare of the Dutch people as a whole are vital interests for each
individual part.' Minnaert had contributed his series on the national
question to Dietse Stemmen, the activist publication founded by the
Bond. After the war, the journal ceased publication. During the General
Members'

Meeting on October 1, 1921, Minnaert was elected as a board member.
He thereby became part of a company of high-ranking officials, cultural
figures, and scientists, with South African General J.B.M. Herzog
as honorary chairman, Dutch lawyer mr P.W. de Koning as chairman,
Flemish poet dr René De Clercq and South African dr Ph.R. Botha as
vice-chairmen. In addition to Minnaert, the Utrecht-based lawyer dr
A.J.M. (Anton) van Vessem, editor-in-chief of {*}Vlaanderen{*} J.
(Josué) De Decker, Ghent native Boudewijn Maes, and geographer prof.
mr S.R. Steinmetz were board members.

In February 1922, Minnaert undertook to investigate whether Flemish
children could have a holiday in the Netherlands: he initiated a modest
exchange of children between families of refugees and Flemish sympathizers
in the Netherlands and those of nationalists in Flanders. The board
established the Diets Studenten Verbond in 1922, at the suggestion
of Utrecht journalist P.H. Ritter Jr., as an alternative to the student
section of the ANV. A committee with Minnaert as secretary awarded
several scholarships to refugee students.

Activist Roza De Guchtenaere had proposed organizing greeting days
between refugees and nationalists in Flanders. Minnaert decided to
take charge of organizing them. She spoke at the 1923 Guldensporendag
in Utrecht and stayed at De Blauwvoet house. A month later, the first
Greeting Day took place in Hansweert, where Roza was one of the speakers.
Minnaert wrote: 'For the Flemings from Flanders, the greeting day
means: that the Flemish struggle is placed under the sign of activism.
For the refugees, it means: that they remain loyal to Flemish ideals
and to the brothers who are fighting there. For everyone: that love
for Flanders can bridge all oppositions, reconcile disagreements.'
According to the romantic Minnaert, there were only two occasions
where the Flemish sentiment was so strong that everyone joined hands:
the great Borms protests and these Greeting Days. In the summer of
1923, Minnaert also organized a meeting of Dutch nationalists in Vlake;
in addition to organizing it, he was also responsible for the content
of the program. This would apply to most of the Landdagen and Begroetingsdagen
that followed.

The board reports sometimes mention Minnaert's interventions. At his
suggestion, the Board decided to complain about the un-Dutch behavior
of the Commissioner of North Brabant, as he had greeted the King of
Belgium in French in Ghent. On another occasion, there was a difference
of opinion regarding the assessment of the Belgian-Dutch Treaty. 'Messrs.
Minnaert and Besse judged that the Netherlands was relinquishing part
of its sovereign rights, while Mr. Minnaert also feared that the Netherlands
had fallen too much into the sphere of interest of Belgium and France.'
On such occasions, Minnaert invariably chose the most anti-Belgian
stance. The imprisoned August Borms became an honorary member of the
board on January 26, 1925. Starting from the 1926-1927 academic year,
De Bond began publishing the monthly magazine De Dietse Gedachte.

\section*{Lectures for Flemish-Dutch audiences}

Minnaert gave numerous speeches to exiles in the 1920s, even as far
as India. His lectures, for example, are about the composer Benoit,
the poets Vuylsteke and Rodenbach, the writer Hendrik Conscience and
his 'Leeuw van Vlaanderen,' the work of the imprisoned Dutch scholar
Antoon Jacob, or 'On the Necessity of Cultural Rapprochement between
Flanders and Holland.' These lectures consistently evoke memories
of the abandoned land and inspire hope and optimism. Minnaert could
see the silver lining in a thundercloud. He wrote such stories by
hand and did not date them. An article in which he pleads for Jacob's
release must have been written before November 21, 1923. Many lectures
were adjusted and updated.

One of these lectures, for Flemish-Dutch Associations and related
societies, is about the composer Peter Benoit: 'I would like to invite
you to follow me in thought to the land that lies south of your borders
and is called Belgium on the map.' In the Flemish landscape, Benoit
was born in 1838: 'Who tells us how the landscape surrounding us unconsciously
influences us during our childhood? Who explains the secret ties?
During this period of the unconscious, during his youth, Benoit gradually
developed that deep love for his birthplace, for his Flemish fatherland
and his Flemish people, which would later make him Flanders' outstanding
national composer. His further life would increasingly come to be
marked by nationalism. A tough twenty-year struggle with the institutions
followed, which in 1898 resulted in the establishment of the Royal
Flemish Conservatory.

The Czechs and Norwegians incorporated folk music into their national
music. Benoit did not use those songs because he had so deeply immersed
himself in the spirit of the Flemish folk song that he could create
his own folk songs: 'He had become a part of the people himself.'
Minnaert valued this higher than transforming a folk tune: 'Benoit
awakens the listener, grips you. Each of his works is marked by his
lion's claw. They have an impact. Clear melodic lines, fluttering
like flags; simple, muscular harmonies; compelling rhythms in the
spirit of folk songs and dances; telling stories of courage, self-confident
strength, noble pride.'

In the cantata {*}De Schelde{*} with text by E. van Hiel, 'a poem
like a bread,' Minnaert incorporated political current events. One
of the 163 lectures took place during the time of the Belgian-Dutch
Treaty, when the opening of the Scheldt was at stake. Minnaert admonished:
'The Scheldt has become infamous in more than one way in recent months.
If they still tell you that the Belgians want the Scheldt, answer
them: yes, but the Flemings will resist it to the utmost. For them,
the Scheldt is the river of the Netherlands. That is the best response
to the annexation plans.' On such evenings, Minnaert evoked nostalgic
memories of Flanders, used the piano to introduce Benoit, connected
art and culture with nationality, addressed political current events,
and stirred up resistance against Belgium. In these circles, he became
a beloved introducer.

\section*{Flanders Fermenting}

31 In a lecture for students from the late 1920s on Nationalism and
Internationalism referred Minnaert to the influence of Bolshevism
and fascist leaders such as Mussolini. They set unprecedented forces
in motion, which had to be made subservient to the pursuit of a harmonious
world federation and disarmament. He called for idealism and fearlessness,
speaking about ‘being one’s own cause’ and the omnipotence of humanity.

Nationalism also gained ground in Flanders. The historian Pirenne
still referred to the activists as a ‘pathetic group amidst a population
that repudiated them with disgust.’ This proved no longer to be the
case ten years after 1918, specifically on December 9, 1928.

By-elections were held in Antwerp due to the death of a liberal MP.
Socialists and Catholics refrained from fielding a candidate. The
activist Herman Vos, spokesperson for the Front Party, nominated the
imprisoned August Borms as a candidate, although he could not become
a people’s representative. The Brussels elite assumed that the liberal
candidate would win effortlessly. However, Borms received 83,000 votes
compared to the liberal’s 44,000, while 58,000 ballots were blank
or invalid. The people of Antwerp had taken their revenge on the anti-Flemish
spirit of the Belgian state: Tijl Uilenspiegel had played a new trick
on the authorities.

On January 17, 1929, Borms was unconditionally released. On Sunday,
February 3, 1929, tens of thousands marched in Antwerp for the ‘king
of Flanders.’ Exiles such as René De Clercq crossed the border in
large numbers. After all, the Amnesty Law had come into effect on
January 19. Former activists could cross the border again without
being arrested. However, the law primarily caused bitterness. No restoration
of rights took place: the convicted individuals could not exercise
official professions or accept political functions. The confiscations
were also implicitly confirmed. Minnaert signed the Declaration of
Convicted Flemish Nationalist Activists. This document rejected this
‘abomination’ and addressed Borms with a militant ‘You bear the standard.
Our homage is that of the warrior: deep and short.'

There followed Borms' acknowledgments in a series of Dutch cities,
including the one in Utrecht on March 16. Minnaert became a member
of the Committee 35 Justice and Reconstruction, which was established
in Brussels on May 4, 1930.

Minnaert began to involve himself with the leadership of the Flemish
Movement, which was hopelessly divided. He had argued at the Landdag
in Breda in 1927 for an overarching leadership of the Flemish nationalists.
The release of Borms gave impetus to the idea that a highest authority
should be established, a Council to prepare for the independence of
Flanders. On November 23, 1929, a meeting took place in Antwerp: Minnaert
made his opinion known in writing. Two weeks later, a secret Flemish
People's Council was formed. It sent out 139 invitations to prominent
individuals, nearly half of which went unanswered: 53 selected individuals
accepted and 20 people refused. After this setback, Borms was personally
tasked with making contact with organizations and people so that a
new Council of Flanders could be established. In the course of 1931,
a Declaration of Principles was drawn up by Borms and René De Clercq.

Only Borms knew the names of those who were invited for membership:
the secret member had to return a written oath. The Council had established
36 secret committees. The whole endeavor resembles undertakings such
as the Flemish Veem and Jong-Vlaanderen. Minnaert also would have
signed up. In 37 letters he sent to De Vreese, he told him that he
had become the chairman of the education committee of this Council
of Flanders.

The spring elections of 1929 had yielded gains for the Flemish nationalists.
The law of April 5, 1930 finally provided for a gradual Dutchification
of Ghent University. It was adopted with 154 votes in favor and 6
abstentions. Rector August Vermeylen was able to open the university
in Dutch on October 21, 1930. The students had asked Speleers, the
second rector of the Flemish University, to create a shadow opening
in {*}De Uilenspiegel{*} No. 38, which he gladly provided. The principle
'regional language is the leading language,' rejected in 1914, was
finally accepted.

In 1931, the students of Leuven treated King Albert and Queen Elisabeth
to a flute concert and a volley of cooked apples. The king's ambiguous
tactics had failed. The price for Belgium was high.\\

Endnotes: 

1 Ovidius quote by Minnaert, 1946, 119, referred to by him as {*}In
Praise of Astronomers{*}.

2 This is explained at the beginning of chapter 8.

3 Minnaert to Burgers, February 2, 1919.

4 Interviews with Hans Littooij and Nanda Mierman-Ortt.

5 Ortt, 1917. Epilogue from May 1920.

6 Minnaert to Burgers, April 13, 1919. Minnaert begins work on {*}The
Physics of the Open Field{*}.

7 The result, published in {*}The Philosophical Magazine{*}, was according
to the Science Citation Index his most frequently cited article in
2002. Minnaert (1924b, 1933).

8 Minnaert to Burgers, August 19, 1919.

9 Alkemade, 1995. Romein-Verschoor, 1970. Stutje, 1999. Struik was
an observer for the Comintern during the founding of the Communist
Party of Belgium (KPB) in 1919 and met in Antwerp the fiercest Flemish
nationalist: it turned out to be the Dutch diamond cutter Paul de
Groot, later a leader of CPH and CPN.

10 Meeting on June 2, 1919, mentioned in a retrospective in {*}De
Dietse Gedachte{*} 2, 27.

11 Minnaert to Burgers, December 30, 1919.

12 De Schaepdrijver, 1997, is a contemporary hagiography regarding
King Albert. She has heard nothing about the annexationism of Albert's
government. Time does not necessarily enhance the objectivity of historical
writing.

13 Marie Minnaert wrote in 1915 from the Netherlands that her cousin
Marcel disgraced the family. The postcard caused a stir at Ghent City
Hall, according to Fredericq. The literarily gifted Marie van Zadelhoff-Minnaert
published the story {*}A Consultation in Waterveld{*} in {*}De Vlaamse
Gids{*} in 1912, p. 216.

14 Letter from H.C.J. De Decker in Molenaar, 1994, 393. The classic
De Decker at Brugmans, 1980, p. 46.

15 Minnaert, Flanders, November 16, 1924. In the anniversary edition
of the Willemsfonds from 1926, Gillis D. Minnaert was notably present.

16 Memories of F.W. van Milaan, a neighbor from Parklaan 50. Letter
dated June 26, 1998.

17 Roza De Guchtenaere to the Minnaerts, September 26, 1919.

18 Minnaert, Federalism or National State, De Toorts, 1920.

19 Minnaert to De Bruyker, March 18, 1922.

20 Van Cauwelaert spoke on January 30, 1924; Speleers on March 21.

21 Minnaert to De Vreese, January 19, 1924. Archive-Dousa. Dedeurwaerder,
2002, pp. 440-446. Speleers had to take an exam again as an ENT specialist
and had settled in Eindhoven.

22 Jules Spincemaille to Minnaert, June 30, 1924.

23 Minnaert, 1916a.

24 E. Besse, The Dietse Bond; A Look Back at Ten Years of Existence,
DDG 2, 1927-1928, p. 7.

25 E. Besse, DDG 2, p. 51, Minutes of the Board Meeting on February
11, 1922, and also the establishment of the DSV.

26 The support from the Dietse Bond to the exiled activists, Scientific
Announcements, 2000, p. 230.

27 Minnaert, DDG 1, pp. 34 and 121. There were about 300 people according
to Dedeurwaerder, 2002, p. 434.

28 Minutes of the DDB Board, DDG 2, pp. 84 and 87. The disputed greeting
took place on October 26, 1924.

29 Archive-History. It contains a folder of lectures from the 1920s,
such as the one about Benoit.

30 The Scheldt Treaty issue occurred in 1925. See Burger, 1932.

31 Minnaert, Nationalism and Internationalism, undated. Archive-History.

32 Pirenne, H., Belgium and the World War, Paris 1929.

33 Geyl, P., Herman Vos and the Flemish Movement (1952), in Historian
in Time, p. 147, Utrecht 1954.

34 DDG 3, from p. 139 onwards.

35 Dedeurwaerder, 2002, p. 492.

36 Dedeurwaerder, 2002, p. 465. Besides the one mentioned here, there
was apparently an education committee.

37 Minnaert to De Vreese, May 9, 1932. Archive-Dousa.

38 Dedeurwaerder, 2002, provides the full text: p. 486.

39 Balthazar, H., {*}Het maatschappelijk-politieke leven in België
1918-1940{*}, p. 148, in {*}Algemene Geschiedenis der Nederlanden{*},
vol. 14, Bussum 1979.

40 Suarez, G., {*}Nos seigneurs et maîtres{*}, Paris 1937. Interview
with a grieving Albert, published after his death, pp. 247-257. The
royal visit to the Catholic University was abruptly cut short.

\chapter{A Solar Physicist on His Way to the World Top}
\begin{quote}
'Doesn’t Plato also tell us about the shackled ones, how they are
blinded by daylight when they are finally made to look, how their
eyes hurt and how they would much rather return to the dark cave?'
\end{quote}

\section*{Minnaert at the Heliophysical Institute}

In Leiden, the Mecca of physics, Minnaert had become persona non grata.
He had deliberately sought out Julius' Heliophysical Institute in
Utrecht. An anecdote circulates about this: he supposedly told his
promoter MacLeod that he wanted to determine the strength of sunlight
quantitatively. MacLeod is said to have replied, 'Young man, get that
notion out of your head; measuring the strength of light is impossible.'
Julius had dedicated himself to this challenge and therefore Minnaert
would contact him. It’s possible that the physicist Arnold Sommerfeld,
who visited him at the Flemish University, pointed him to Julius'
work on the sun. In any case, he informed his friend Burgers in early
February 1919 that he was attending the spectroscopy lectures of physicist
Leonard S. Ornstein and was already fully engaged in work for Julius.

Willem Henri Julius was director of the Physical Laboratory and had
been granted permission in the 1910s to establish an experimental
research center: the Heliophysical Institute. Julius, without much
consultation with the Utrecht astronomers, was pioneering at the interface
of physics and astronomy and needed someone who was both theoretically
and technically skilled and also had two right hands to make his spectroheliograph
operational. Minnaert would have offered his services unpaid and likely
suggested to Julius that his salary be arranged later.

After the war, the victors of World War I forced a break in collaboration
between professional organizations of scholars. Their intention was
to isolate Germany and its allies and harm the practice of science,
which had been so crucial for warfare. Many French and Belgian scientists
proved irreconcilably hostile toward 'the Central Powers.' This effort
resulted in the creation of the International Research Council, established
in Brussels in 1919, with sections dedicated to fields such as astronomy---the
International Astronomical Union (IAU)---physics, and chemistry.
Researchers from sixteen countries joined: the losers of World War
I were excluded from the scientific community. Forty-seven members
of the Royal Academy of Sciences, including physicist W.H. Julius
and astronomer J.C. Kapteyn, had unsuccessfully appealed to the scientific
world to admit their German and Austrian colleagues. Minnaert likely
greatly appreciated this step taken by his director.

On June 30, 1919, Julius wrote in his annual report: 'In the astrophysics
department, Dr. M. Minnaert regularly worked, whose expert and meticulous
assistance in the extensive preparations for solar research was of
great value.' At Julius' suggestion, the curators appointed Minnaert
as an observer on December 31, 1919, with an annual salary of 2,600
guilders. Julius referred to the establishment of this position in
his 1920 annual report as 'an important gain.'

In 1919, Minnaert had accepted a second part-time position at the
Meteorological Institute for one afternoon. He found the work engaging:
'The approach in De Bilt is very physical; while other meteorologists
constantly seek accidental correlations between weather and certain
phenomena through mere description, here they aim to understand the
entire system of lower and upper air layers and derive forecasts from
it in a completely rational way.' Working on something as changeable
as the weather in such a 'completely rational' manner was truly suited
to him.

To understand Minnaert's later breakthrough into the world elite of
solar physicists, it is useful to outline some key principles of physical
optics as they were taught during his youth. Following this will be
a brief overview of the relevant experiments in this field until the
beginning of the twentieth century. Julius attempted to explain these
experimental results in a way that soon proved outdated. Minnaert
initially defended Julius' views and as a result was at risk of being
sidelined. This chapter from the history of science should clarify
how he struggled and why he ultimately succeeded. It is fascinating
that through his strong promotion of solid experimental research,
Julius could still play a positive role for Minnaert.

\section*{The spectrum; frequency and wavelength of light}

The nineteenth century had confirmed the understanding that, like
sound, light can also be understood as a wave phenomenon. A wave is
the result of the propagation of a vibration. In the case of sound,
a medium is required: the speed of sound waves in air at room temperature
is approximately 340 meters per second. If the source of the sound
is a tuning fork, when struck, it produces a tone. This tone has a
frequency, the number of vibrations per second, expressed in Hertz
(Hz). A higher frequency corresponds to a higher pitch. The lower
limit of human hearing is at a frequency of 20 Hz; the upper limit
for young ears is around 20,000 Hz. The 'length' of the sound wave
(\textgreek{λ}) is equal to the quotient of the propagation speed
(v) of the waves and the 'frequency' (f). A tone with a frequency
of 5000 Hertz has a wavelength of 340/5000 = 0.0680 meters. Thus,
a high frequency corresponds to a small wavelength, and vice versa.

Light waves are electromagnetic waves with a very high frequency that
do not require a medium to propagate. Just as the pitch of sound is
determined by frequency, in visible light this is the case with color.
Violet light has a higher frequency and thus a smaller wavelength
than red light. The speed of light in a vacuum is approximately 300,000
kilometers per second for all wavelengths. The wavelength is the quotient
of this speed and the frequency. Red light has a wavelength of about
0.000,000.70 meters; for violet, it is 0.000,000.40 meters. For these
small measurements, an additional---\textquotedbl{}

suitable unit has been introduced: the Ångström (Å). An Ångström is
0.0000000001 meters, so one ten-billionth of a meter or 10\textasciicircum -10
meters. Red light therefore has a wavelength of approximately 7000
Å, while for violet it is 4000 Å. Light waves with wavelengths longer
than 7000 Å are invisible: they lie in the infrared range. Wavelengths
smaller than 4000 Å are also invisible to the human eye and lie in
the ultraviolet range. Detectors exist for observing infrared and
ultraviolet light.

The Austrian physicist Doppler has lent his name to an important analogy
between sound and light. When the sound of a car horn reaches an observer
from a stationary car, the observer hears a certain pitch. If the
car is moving toward them, more vibrations per second reach the ear,
causing the observer to hear a higher pitch. After passing, the experienced
pitch becomes lower. This Doppler effect plays a major role in astrophysics.
If a star is moving away from Earth, light waves will have a slightly
lower frequency and exhibit a redshift compared to the same terrestrial
light source.

The physics of the sun had advanced in the 19th century through the
study of its spectrum. It had been known since Newton that sunlight
can be split into the colors of the rainbow. This happens, for example,
when it passes through a triangular piece of crystal glass at a certain
angle of incidence: a prism. Red light deviates the least in the glass,
while violet deviates the most. This phenomenon is called dispersion.
With the help of prisms, the solar spectrum could be studied around
1800 by displaying it like a small piece of the rainbow. The separation
of colors becomes clearer when the light enters in a very narrow beam.

\textbackslash figcaption The colors of dispersion and the formation
of a spectrum lie between them.

\textbackslash figcaption \textquotedbl A piece of the violet spectrum
between 4380 and 4399 Å with Fraunhofer lines is depicted as a narrow
strip of a rainbow. \textquotedbl{}

If the slit is narrow enough, it also turns out that the spectrum
is interrupted at thousands of places by a forest of vertical dark
lines, named after the German optician Joseph von Fraunhofer. He had
already discovered more than 500 of them in 1814. He marked the most
striking ones with a letter: from A in the red to I in the violet.
One Fraunhofer line is narrow and dark, another is wide and vague:
it’s ‘a full richness of chiaroscuro’ to quote Minnaert, an infinite
variety of shadows and halftones. The German physicist Kirchhoff meticulously
copied this spectrum by hand due to the lack of photographic means.
In 1860, he assigned a letter from a to g to each line's width and
a number from 1 to 6 to its darkness.

The combination of the three components---the entrance slit, prism,
and imaging system ---is called a spectrograph. The American Rowland
replaced the prisms around 1895 with a reflection grating or ‘grille.’
Such a grating is a perfectly flat glass plate with parallel grooves
at identical distances. Rowland managed to fit 600 of them in one
millimeter. The distance between the grooves is then one-six-hundredth
of a millimeter or 16,000 Å, which is close to the wavelength of red
light. Using this grating, he was able to photograph a bright spectrum.
Rowland's Atlas of the Solar Spectrum mentions 20,000 Fraunhofer lines
with wavelengths accurate to 0.01 Å, one-billionth of a meter! By
photographically enlarging the spectrum, it can naturally be made
longer, and thus Rowland published a spectrum that was thirteen meters
long. On those 13,000 millimeters, he accounted for 3000 Å, so each
Å was enlarged to 4 mm. Rowland gave each of these lines a number
that served as a measure of the blackness or ‘strength’ of the line:
that was a qualitative estimation. This Rowland scale became the standard
for the measuring the intensity of a 12 Fraunhofer line.

\section*{The strength of the Fraunhofer lines}

Naturally, physicists tried to explain the origin of the lines. In
the second half of the 19th century, experiments by the Germans Kirchhoff
and Bunsen made a first interpretation possible. When colored light
from glowing vapors passes through a prism, colored lines of the emission
spectrum appear against a dark background. They heated sodium vapor
and found around 5890 Å a yellow double line. Exactly at that location,
the solar spectrum shows a dark double line, which Fraunhofer named
the D-line. The Fraunhofer lines indicated the presence of atoms of
terrestrial elements in the photosphere, the gaseous surface layer
of the sun, which absorbs this light precisely.

Every chemical element, whether hydrogen, sodium, iron, or uranium,
when heated so much that it becomes gaseous, emits its own characteristic
set of lines. By this, one recognizes the element: it is as if the
barcode of the gas in question. The solar spectrum shows all the lines
of all elements that float in the outer, visible envelope of the sun.
The lines do not emit light but are dark instead. This phenomenon
had already been discovered by Kirchhoff: a cool gas, floating between
the observer and a hot light source, absorbs the light from the light
source at the wavelengths of the barcode of the gas. Thus, since Kirchhoff,
it has been clear that the Fraunhofer lines together form an absorption
spectrum, a fingerprint or barcode of the atoms of chemical elements
in the photosphere of the sun and other stars.

In the second half of the 19th century, such chemical analysis of
solar gas had made good progress. A special discovery was that of
a gas with an unknown barcode, which was named 'helium': the gas of
Helios, the sun god. A problem was formed by the emission lines of
the corona: the tenuous outer atmosphere of the sun, visible only
during total solar eclipse. These lines were not known from laboratory
experiments: did this indicate a new chemical element, Coronium? Also,
the visible protrusions or protuberances above the solar surface posed
difficulties: according to Doppler, the measured redshift of the spectral
lines suggested that these enormous masses would move at an almost
unimaginable speed of hundreds of kilometers per second. The fundamental
questions about the sun and stars made solar physics a favorite research
object for physicists like Julius, who ventured into the territory
of astronomers.

Rowland and others before him used a camera as the imaging instrument
of the spectrograph: the spectrum was photographed on glass plates.
These plates are therefore negatives: what emits light appears black
on the plate; the Fraunhofer lines, which are actually dark, thus
appear as light or less black streaks against a black background.
A German named Schwarzschild could detect the darkness and light of
a moving photographic plate using a photoelectric cell. Schwarzschild
died in 1914, putting an end to his pioneering work. His technique
was perfected during the war by Julius' associate W.J.H. Moll, who
replaced the photoelectric cell with a sensitive thermoelectric element,
which at the time provided greater accuracy and reproducibility.

\section*{The spectrograph at the Heliofysisch Instituut}

Julius immediately set Minnaert to work installing his spectrograph
in his Heliofysisch Instituut. Installing this expensive research
instrument had a quarter-century history (see the Appendix on anomalous
dispersion and Julius' sun theory). It was equipped with a Rowland
grating of 600 grooves per millimeter and, in the 1920s, needed only
to acknowledge the observatories at Mount Wilson (US) and Arcetri
(Italy) as its superiors. The right wing of the Physics Laboratory
was fitted with beams and floors of reinforced concrete to dampen
resonances. On the roof stood an instrument called a coelostat, which
consists of two heavy mirrors: one rotates with the movement of the
sun and reflects the light to a second mirror, which projects it into
the solar telescope. Thirteen meters below, a solar image with a diameter
of 12 centimeters appeared. At the same location lay the entrance
slit of the spectrograph.

\textbackslash figcaption Schematic drawing of Julius' spectograph
in Utrecht (1919).

Minnaert had to design the follow-up mechanism for the coelostat and
wrote to Burgers: 'It was a long-lasting task; every half minute,
a position had to be recorded from a little light line reflected by
the rotating coelostat mirror; from that, the irregularities were
calculated. Just for the worm wheel, the investigation lasted 48 hours,
and that on the roof while it was freezing so hard that the ink in
my inkwell had solidified! Then there came out a curve with a double
period, corresponding to the rotation time of two gears from the clockwork;
we had them remade to make the waves disappear. Then the Foucault
regulator had to be adjusted and made isochronic; what an interesting,
simple-yet-complex little device it is! After that, I went on to silver
the mirrors according to the Mount Wilson method; that worked fairly
well.' Minnaert loved solving technical puzzles and proved to be the
right man in the right place.

The most important part of the setup is the spectrograph. The entrance
slit is located in a plate at the top of a four-meter-long metal tube.
The grating was attached to the bottom of it. The tube can be tilted
over a steel ball in the basement pit: by hand, a part of the meter-long
spectrum can be selected by turning the grating slightly. Also, the
solar image can be photographed as a 'spectroheliogram' through a
second slit on the worktable at a selected wavelength by moving the
spectrograph under the solar image.

Minnaert's role becomes clear indirectly from an article in March
1923 where Julius proudly described his Institute. The story mentions
no names, but on the final page, there are two spectroheliograms 'obtained
by Dr. Minnaert with the Utrecht spectral equipment in two kinds of
light near the calcium line K, on September 11, 1919.' That must have
been a tribute.

Minnaert wrote enthusiastically to Burgers that he would have heard
about Julius' Sun theory. He had to substantiate this theory using
spectrograph 21 and the self-recording microphotometer of Julius'
assistant W.J.H. Moll: 'It was a complex mechanism, with slides moving
up and down, diaphragms opening and closing, the prism and the recording
drum turning a few degrees each time, and a counter shifting position
with every new stand.' In two hundred steps, which took about two
hours, the intensity of a portion of the spectrum was recorded fully
automatically. Moll was one of the first physicists to introduce an
automatic recording technique. Minnaert was one of the first to gain
experience with it.

\textbackslash figcaption Moll's microphotometer

\textbackslash figcaption A couple of angstrom of the profiles from
the microphotometer near the orange-yellow Na D-line.

A light beam thus traces the dark and light lines on the moving photographic
plate: a thermo-element measures the heat effect of the transmitted
light, which is converted via the deflection of a sensitive ammeter
into a profile of the Fraunhofer line on a moving paper roll. Minnaert
wrote lyrically: 'A small section of the spectrogram, just a few millimeters
long, is thus transformed into a beautiful registrogram about ten
centimeters in length, resembling an emotional mountain landscape
with peaks and rolling valleys. One can clearly distinguish single
lines and the places where two or more lines almost coincide and partially
merge; faint depressions that the eye would never notice when studying
the plate are flawlessly recorded by the microphotometer; and all
these details are represented in such a way that the transmission
of the plate can be read quantitatively and accurately at every point.
If the entire spectrum were converted into a registrogram on a reasonable
scale, it would result in an enormous curve approximately 100 meters
long.' He found it sensational: 'I remember that our first microphotometric
registration of a small part of the solar spectrum immediately gave
rise to the highly ambitious plan of creating a complete photometric
Atlas.'

These quotes are recollections. In the early years, much creative
work was required before the measurement of the photographic plates
would definitively reveal the intensity of the Fraunhofer lines. Minnaert
was at the forefront of this effort. His admiration for Julius seemed
boundless during those first Dutch years.

\section*{Minnaert sings the praises of his teacher}

In the fall of 1920, Julius fell ill. He handed over the leadership
of the laboratory to his younger colleague Ornstein. He somewhat recovered
and in 1921 was able to celebrate his 24th or 25th anniversary as
a professor. Minnaert's contribution to the Liber Amicorum appeared
in Physica, the journal of the recently established Dutch Physics
Association. It deserves ample mention.

According to Minnaert, Julius had proposed ideas that opened new paths
in solar physics. Due to significant differences in density and temperature
within the solar atmosphere, irregular refraction would undoubtedly
occur. The mirages and the 'flattened' setting sun in Earth's atmosphere
are attributed to this phenomenon. Additionally, anomalous dispersion
occurs near the wavelength of the Fraunhofer lines (see Appendix).

According to the optical formulas known at the time, these anomalous
phenomena should manifest in the refraction, spreading, and scattering
of the solar light in question. Through anomalous refraction, bent
bundles of a specific wavelength and color could unexpectedly appear
elsewhere, explaining numerous solar phenomena in their early stages.
This is why Julius focused on it. Unlike his new teacher, Minnaert
dared to present these preliminary ideas with admirable certainty.

Minnaert believed that Julius' theory had linked numerous physical
laws into 'a grand astrophysical theory,' which his teacher had sometimes
referred to as a 'preliminary study': 'One does feel compelled to
humility, however, when considering how essentially all the rays of
light that the sun radiates toward us have traveled thousands of kilometers
through glowing gas masses and could have been bent or scattered in
the strangest ways along their path. So that a protuberance we thought
we saw rising like an enormous flame at a speed of hundreds of kilometers
per second, now it appears there is a slight condensation that propagates
like the stirring of foam along a wave and refracts a little light
from the solar core into our eye. Everything thus becomes an apparent
image, an 'optical illusion,' and one would almost be inclined to
regard solar research as hopeless, were it not for the fact that Prof.
Julius has been able to indicate general laws that summarize the regular
consequences of this irregular refraction and scattering and point
out the safe path in deciphering solar phenomena.'

The writer of these vivid sentences shows the pupil who continues
to celebrate his teachers. No one had thought of this: Julius was
the only one who saw through it! And still, there were unbelievers
among the physicists! Minnaert therefore scolded them with the parable
from {*}Politeia{*}: 'An old comparison comes to mind, the famous
comparison with which Plato begins the seventh book of his treatise
on the State. - Men are in a cave, chained from childhood, able to
see nothing but a rock wall in front of them, illuminated by a fire
behind them; when their guards walk back and forth behind a low wall,
carrying objects that protrude above the wall, the captives see the
shadows of these objects moving back and forth on the rock wall. These
shadows are the only thing they perceive of the external world; through
habit, they have come to recognize the shapes of the shadows, give
them names, find a certain regularity in their movements and sequences.
And they imagine that those shadows actually move of themselves and
constitute the essential part of the phenomena.

Is it not remarkable how accurately Plato represents the relativity
of our sensory impressions with this image? And is it not peculiar
that he derives his comparison precisely from optics, where illusions
can be so frequent and deceptive? - The observers who count and catalog
thousands of sunspots, who stubbornly draw protuberances and make
spectroheliograms every day, they are inevitably inclined to regard
the shapes they 'see with their own eyes' as material boundaries on
the Sun; and naturally, we all sometimes find the ideas of dispersion
theory unpleasant, since it is much harder to find one's way in this
new, still unfamiliar line of thought than to continue preserving
the empirical rules of old. Does Plato not also tell us about the
captives, how they are blinded by daylight when they are finally made
to look around, how their eyes hurt and how they would much rather
return to the dark cave?

But he also tells how they gradually succeed in observing first the
reflections, then the dimly lit objects themselves, then looking around
in daylight, and finally contemplating the clear sun, 'not just any
appearance of it, but the sun itself, in its place and as it is.'
- This pious wish could certainly be taken literally by heliophysicists!
However, the philosopher's words also contain deep comfort for anyone
who might despair over the hesitations and uncertainties plaguing
the still-young solar physics: through all the doubt, despite detours
and obstacles, we are moving forward step by step. Justice is done
to everything; outdated theories gradually fall away, and the correct
positions are preserved. No solid work is lost. Yes, that's how it
is! And so the dispersion theory can certainly face the future with
confidence.' It was almost a political argument, which also ended
optimistically. Minnaert was not yet operating independently. He was
also not a trained physicist. As a party member, he turned Julius'
cautious prose into an irreconcilable polemic against physicists like
Lorentz, Ornstein, Zernike, and Einstein, who had serious reservations.
Minnaert would later regret that optical illusion. It was a striking,
uncertain, and therefore seemingly arrogant entry into the world of
Dutch physics. Julius, 33 years older than Minnaert, was apparently
delighted with his master student's contribution and did not urge
caution upon him. No more so than Mac Leod, Bolland, or Domela had
ever done.

\section*{Ornstein's international center for photometric research}

With thermodynamicist Leonard Ornstein, Minnaert developed a more
equal relationship. Ornstein was an imposing and energetic figure
who, like Minnaert, dedicated himself fanatically to political ideals.
As a 26-year-old doctoral candidate in 1907, he had been the secretary-general
of the Eighth Congress of the Zionist World Organization in The Hague,
and four years later, he led the Zionist Congress in Basel. He proved
to be a skilled agitator: 'After an inflammatory speech by Ornstein
in 1918, the audience spontaneously broke into the Zionist hymn, despite
explicit agreements to the contrary.

This Ornstein became curator of the controversial Hebrew University
of Jerusalem and thus a colleague of Einstein. This university had
to educate the elite of a future Jewish state, placing it at the heart
of world politics. After World War I, England and France had taken
countries such as Lebanon, Iraq, Iran, Syria, and Palestine away from
the Turkish caliphate. The British had made promises to Jews, Arabs,
and Indians that were mutually contradictory. Jewish colonists tried
to create facts on the ground by force of arms. Ornstein's combative
choice must have appealed to Minnaert, twelve years his junior, with
a similar love for Flanders and the Flemish University.

Ornstein had earned his doctorate under Lorentz on a theoretical subject.
In 1915, he was appointed as a theoretical physicist in Utrecht. He
had admired Moll's automatic microphotometer, designed for Julius'
experiments. Moll's instrument could perform quantitative measurements
of the intensity of spectral lines. In Leiden, everything revolved
around the cryogenic equipment for Kamerlingh Onnes' work at extremely
low temperatures. Ornstein, in turn, as director of the Physics Laboratory,
had acquired a unique instrument and understood that he had to capitalize
on this divine gift. He wrote in his 1922 annual report to the curators:
‘It is intended to expand the microphotometric department in order
to assist other institutions with this work for which there is now
great experience and an exceptionally fine instrument available, thus
making our laboratory an international center for photometric research.’

With this microphotometer, Utrecht could place itself at the heart
of modern physics. Quantum physics had provided a theory of Fraunhofer
lines that was more convincing than Julius' dispersion theory. The
atomic model (1913) of the Danish physicist Niels Bohr proved to be
the Rosetta Stone for solving the Fraunhofer script. When atoms are
given energy in the form of light of a certain frequency, electrons
can reach higher-energy orbits by absorbing ‘photons,’ light quanta
of a certain frequency; when they fall back to a lower energy state,
they emit light with a frequency corresponding to that energy difference.
This frequency matches the wavelength of the respective Fraunhofer
lines. When light from the sun’s interior falls on atoms of specific
elements in the photosphere, photons with exactly the energy needed
to move electrons to a higher energy level will be absorbed. For example,
sodium atoms in the photosphere will absorb the frequency corresponding
to yellow-orange light; the double D-line indicates a 'doublet,' two
slightly different electron energy levels. The electrons re-emit that
orange-yellow light when they fall back into their original energy
level, but scatter it in all directions. An observer will see an absorption
line in the yellow-orange region of the sun’s spectrum.

The British-Indian physicist Saha (1920) demonstrated that the diversity
in emission spectra at the sun’s periphery, from the chromosphere
and corona, was not due to extraterrestrial elements. Under those
physical conditions, atoms lose many electrons, resulting in multiply
charged ions that exhibit their own Fraunhofer lines. Initially, this
multiple ionization was attributed to extremely low pressure. It is
now known that elevated temperature, rather than reduced pressure,
promotes ionization. The 'unknown elements,' such as the coronal gas
Coronium, turned out to be earthly elements after all.

However, the spectral lines recorded in various laboratories depended
on the conditions. In radiation experiments, flames, glass discharge
tubes, or electric arcs could be used, with significant differences
in temperature, pressure, and concentration. The only constant was
the position of the emission line in the spectrum. The shapes of the
line profiles, obtained using different photometers---thus their
height, width, or symmetry---differed from one institution to another.
As many 'strength classes' emerged as there were research centers.
For the compilers of Atlases of spectral lines, this was a discouraging
fact. Here lay a great opportunity for an institution that could create
a standard.

The first task thus involved standardizing optical procedures so that
the measurements taken in Utrecht would yield results that could be
verified and reproduced elsewhere. Ornstein initially focused on photographic
materials: developers and fixing agents. His staff investigated the
light-sensitive layer of the glass plates, the size of the silver
grains after development, and the influence of impurities in the gelatin
layer. Those photographic plates for which a graph showing the relationship
between darkening and the intensity of the incident light appeared
to form the best straight line were preferred. The 29 plates had to
be calibrated before each experiment.

\section*{Minnaert's step attenuator and true intensity}

The new director incorporated Minnaert into his research program,
which seemed to conflict with Julius's program, to which Minnaert
had initially adhered. In practice, however, the microphotometer could
serve both programs. No wonder that Minnaert's first achievements
focused on work with this instrument.

Minnaert was assigned two major tasks: calibrating the measurement
errors of the spectrograph grating and calibrating the profiles of
the spectral lines produced by the microphotometer. Regarding the
former, Minnaert wrote: 'I have taken a side path in sun research
and am trying to learn something about the issue of ghost images with
a grating. The ultimate goal for which these and other already conducted
or pending measurements are intended is the investigation of intensity
distribution within a Fraunhofer line.' He reported the results in
1921 in a voluminous report to the Academy. The grating produced weak
spectral lines on its own due to small deviations in the regularity
of the grooves.

These 'ghost images' proved to be real nuisances, but Minnaert was
able to demonstrate that their integrated effect resulted in a uniform
veil across the spectrum. This veil is no denser than a few percent
of the continuous light, and its effect can be accounted for by subtracting
a fixed percentage of the light intensity of the continuous spectrum.
He provided an electromagnetic theory of the phenomenon, supplementing
it with extensive calculations using Rayleigh's and Voigt's classical
formulas for scattering and diffraction. He found himself here in
Julius's optical paradigm, where he attempted to hack a 32 quantitative
path.

His second assignment immediately led to a discovery. Both he and
his colleague Van Cittert took a step forward with the step attenuator.
The microphotometer does not directly measure the intensity of the
light that has fallen on the photographic plate: if that light has
ten times greater intensity, the emulsion of the glass plate does
not need to become ten times blacker. Minnaert designed a piece of
glass, 2 cm long, coated with platinum in six gradations, from black
to transparent. Previously, he experimentally determined that these
platinum steps, at a certain wavelength, transmit 11, 16, 26, 43,
65, and 100\% of the light intensity. He placed this step attenuator
over the slit of the spectrograph and covered the rest of the slit.

On the photographic plate, six spectra appeared on top of each other
with varying degrees of darkening. In each of these spectra, he determined
the transmission of a piece of continuous spectrum between the Fraunhofer
lines using the microphotometer. He then found six transmission numbers
which he plotted graphically against the pre-determined transmitted
intensities: this gave him a calibration curve. Subsequently, he had
to manually convert the transmission profiles automatically recorded
by the microphotometer into intensity profiles using this calibration
curve.

This cumbersome text corresponds to the time-consuming work that was
necessary before Minnaert and Van Cittert first obtained line profiles
showing the true intensities of Fraunhofer lines. The same method
could, of course, be used to find the true intensities of emission
lines from glowing vapors and gases in the laboratory.

With his brand-new equipment and methods, Minnaert was naturally a
pioneer worldwide. He abbreviated his new quantity, true intensity,
with 'i.' He immediately had a brilliant idea regarding the unit of
this quantity, which he presented in 1923 to the Dutch Natural and
Medical Congress: 'For absorption and dispersion lines, it is desirable
to adopt another determination for strength, namely the energy that
has disappeared from the spectrum, expressed with the unit of energy
within 1 Å of the surrounding continuous spectrum. This way is currently
used to calibrate the Rowland scale of solar lines at the Heliophysical
Institute in Utrecht. The strongest line from the solar spectrum has
a strength of 9.4 units; the weakest lines seem to have values of
a few thousandths of this unit.'

These were groundbreaking sentences for astronomy, which did not elicit
a single question from anyone at the conference.

\textbackslash figcaption Minnaert's principle of equivalent width:
the area under a \textquotedbl line profile\textquotedbl{} is converted
into a \textquotedbl width.\textquotedbl{}

The Fraunhofer lines, after registration, show a parabola-like curve
whose shape depends on the number of grooves in the grating and to
some extent on the slit width of the microphotometer. However, the
area of the curve is constant. This represents the energy removed;
after calibration, it gives the true intensity. Minnaert proposed
determining the hypothetical width based on a rectangle with one side
equal to the local continuous radiation and an area equal to that
of the processed line profile. He referred to the quantitative result
in milli-Å as the total absorption or true intensity of a Fraunhofer
line. Minnaert thereby extracted intensity from the intuitive realm
it had occupied since Fraunhofer, Kirchhoff, and Rowland---more than
a century earlier.

At that very moment, astronomers in the United States were joining
forces to revise Rowland's Table. However, the new values remained
based on estimates. Minnaert began promoting his project: converting
more than 20,000 Rowland values into true intensities, which he would
eventually refer to as equivalent widths in the late 1920s. It would
take a decade for the importance of his proposal to penetrate the
awareness of astronomers. However, research groups in Germany and
Australia quickly supported him. Minnaert's star began to rise considerably.
He now had to turn his idea into publications.

He once complained to biologist De Bruyker: \textquotedbl I spend
all my time on my duties; nothing remains for my own research. In
the late evening hours, tired from physical and mental work, I try
to make further theoretical progress. So many beautiful plans, almost
completed, lie waiting there. Occasionally, I hear that someone else
has carried out something I wanted to undertake. I try to economize
on time---in the morning, afternoon, while eating my sandwich, at
night; perhaps I am gaining ground after all. The piece I sent you
is not yet my dissertation; it’s a side path that caught my attention,
and along which I took some time to explore an interesting area in
depth. The dissertation, I hope, will be five times as extensive.
But when? '

It must be borne in mind that Minnaert was engaged in regular work
at the Physics Laboratory. Supervising the practicals for students
of natural sciences, astronomy, and mathematics, as well as for many
medical students, was certainly no sinecure in Utrecht. Since Minnaert
had quickly formed an idea of the life’s work that could be laid out
before him, the responsibilities must have weighed heavily on him
at times. He took the stance back then that writing a dissertation
was comparable to preparing for an exam: it should not happen during
his employer's time!

\section*{Minnaert defending Julius}

Julius involved Minnaert, who operated mathematically skillfully,
in his polemics and encouraged Minnaert to pursue his doctorate on
the defense of his sun theory. In 1920, Julius' assistants, Van Cittert
and Minnaert, had separately reported that the line profiles they
measured showed an asymmetric redshift postulated by Julius, although
their results were actually within the margins of error. In 1923,
Julius and Minnaert jointly undertook a so-called 'crucial' experiment
proposed by Julius. Julius had deduced in an incomparable way that
two adjacent Fraunhofer lines of a certain type must 'repel' each
other, so their centers should be slightly farther apart. Minnaert
calculated seventeen selected cases, which he graphically plotted.
The result could be briefly summarized: 'The theoretical expectation
is thus not at odds with the results of the observational material
known so far; but our conclusion cannot reach further for now because
the quantities under investigation lie on the edge of current measurement
precision.' That was essentially a repetition of 1920 and deadly for
a crucial experiment.

Remarkable in their text are the ambiguous, Julian passages that try
to save the situation. Julius then decided to comprehensively set
out his solar physics once again in a 'Textbook of Solar Physics.
When he died in 1925, he was halfway through his work. In the introduction,
he wrote that there could be no systematic growth in solar physics
in the sense that researchers would continually build upon foundations
laid by their predecessors. He needed this starting point to justify
the break with his predecessors. He repeated the condition for his
palette of beam curvatures: 'We will introduce the hypothesis that
a relatively large local variety of optical densities is found in
the rarefied mass.' Recently, he had drawn courage from Minnaert’s
dissertation, which had already succeeded, under certain restrictive
but highly plausible assumptions about the distribution of irregularities,
in calculating how an originally parallel beam of light would spread
out in terms of direction and intensity in such a medium. The devoted
student gave the old master false hope!

Einstein called his friend Julius in his eulogy 'one of the most original
exponents of solar physics' and hoped that his views on anomalous
dispersion would not be overlooked. He agreed with Julius regarding
protuberances, 'that it is incorrect to posit high speeds when explaining
solar phenomena.' However, he noted that Julius, in turn, had placed
no faith at all in redshift as a result of his theory of gravity.
Julius’s PhD student H. Groot hoped that one of Julius’ students would
attempt to complete his {*}Textbook of Solar Physics{*} 'as an act
of scientific piety in a manner worthy of the Master.' Minnaert naturally
felt called upon to do so.

\section*{Irregular Beam Curvature}

In the early 1920s, with the help of all possible exemptions, Minnaert
had graduated again in physics and devoted himself to his promotion.
He wanted to demonstrate with his second dissertation in mathematics
and physics that Julius’ ideas were indeed solid physics. In his Foreword,
he promised the deceased that 'it will be my endeavor to continue
working in Your spirit.' The substitute promoter Ornstein had been
willing to take over the cum laude already awarded by Julius. Minnaert
had picked up an article by Ornstein and Zernike, 41 in which they
had laid the foundations for the mathematical treatment of some of
Julius' concepts.

First, Minnaert once again defined the terms he used. When a parallel
beam of light falls on an optically smooth planar piece of glass,
with parallel front and back surfaces, it reflects back in one direction
(normal reflection) and the transmitted beam is refracted in a certain
direction (normal refraction). If the glass plate is provided with
small protrusions like frosted glass, the light reflects back in various
directions depending on how individual light rays (photons) bounce
off the protrusions (irregular reflection), and the same applies to
the transmitted light (irregular refraction).

If one considers the case where not only the surface but the entire
medium is irregular, then the density will vary from place to place.
Minnaert took as an example a non-homogeneous, concentrated solution
of kitchen salt. Then individual light rays would be deflected differently
at each location. The initially parallel beam of light spreads out
and becomes wider: Minnaert 43 calls this irregular beam curvature.
When light passes through a medium where the disturbing particles
are comparable in size to the wavelength of the light, such as cigarette
smoke in a living room or dust in the atmosphere, one speaks of 44
scattering. Minnaert's argument had to lay the mathematical and quantitative
foundation for two solar phenomena that Julius had explained qualitatively.

The first, the distribution of light across the solar disk, he could
not bring closer to a solution despite much impressive calculation
work. The second was the question of the sharp solar edge. If the
sun is a gas ball with a gradual variation in density, how does it
present a sharply defined disc? To this, he applied an ingenious calculation:
he used a formula for the 'spreading coefficient' that he himself
had derived, substituted his estimates, and concluded: 'Near the apparent
photosphere boundary, the Julius theory demands that the spreading
coefficient decreases 125 times over a height difference of 700 kilometers;
with the 'disturbance' in the atmosphere remaining the same, it is
sufficient for the density to decrease in the ratio of 11 to 1. This
is much more plausible than Schwarzschild's ratio of 10,000 (over
200 km) or Stewart and Russell's ratio of 100,000 (also over 200 km).
Thus, Julius' basic idea is completely confirmed.'

The numbers associated with Julius' assumptions, therefore, seemed
more realistic than those of his competitors. For such a great effort,
that seems like a meager result. It is remarkable that the spectrograph,
the instrument that was supposed to provide the hard data for Julius'
theory, plays no role at all in the dissertation.

Minnaert's thesis provided little evidence to revive Julius' theory.
Ornstein even wondered what he should do with the Heliophysical Institute.
He had completely overhauled the laboratory in the mid-1920s; it could
restart as a new facility serving as a center for photometry. The
ministry had sent two letters after Julius' death: one of condolence
to his family and another regarding the termination of his professorship
to the faculty. The board, led by Ornstein, refused to accept the
termination and even wanted to appoint a new professor. A compromise
between zero and two was reached by maintaining Julius' position.
A long-lasting intrigue ensued, against Ornstein's wishes, leading
to the appointment of the theoretical physicist H.A. Kramers.

The Heliophysical Institute was also a point of attention on November
19, 1925, in the confident letter to the minister. The faculty board
wanted to maintain it due to Minnaert's great qualities, who should
be appointed as a lecturer. However, the institute was formally subordinated
to the Physical Laboratory, and Minnaert's lectureship could remain
unpaid for the time being. The curators were pleased and agreed: 'We
strongly support the awarding of the personal title of lecturer to
Dr. Minnaert on the grounds presented by the Faculty.' A year later,
on November 5, 1926, Minnaert was indeed appointed as a lecturer and
privaatdocent with a yearly salary of 4400 hfl.

Minnaert completed the second half of Julius' textbook the following
year, making use of his publications and 'everything he shared with
me in our daily conversations.' The passage in which Minnaert compares
the wavelengths of 46 Fraunhofer lines with those of terrestrial emission
lines deserves attention. The centers of the solar lines indeed show
a tiny shift toward the red, varying from line to line, mostly between
0.0010 and 0.020 Å, increasing with wavelength. This was literally
in agreement with Einstein's hypothesis regarding redshift! However,
Minnaert bluntly dismissed this redshift for Julius and wrote literally:

'This explanation of the redshift leaves no room for the relativistic
shift posited by Einstein. Since the measurements already yielded
much smaller shifts than required by relativity theory, there is certainly
far too little left after subtracting the dispersion shift.'

Both Julius and Einstein were seeking a redshift on the order of 0.01
Å, albeit of different natures (see the Appendix). Minnaert rejected
Einstein's claim because there was simply no room for two effects!

Minnaert concluded the book with an ode to Julius' theory, which 'results
in a minimum of assumptions.' He seemed at first glance still a supporter
of Julius, but he was no longer one at that moment! He had spoken
in his public lecture on the history of solar physics over the past
century and allowed himself the freedom of new insights. He had cited
the Dane N. Bohr, the Brit A.S. Eddington, and the German K. Schwarzschild
as sources of inspiration and had outlined a fairly ambitious research
program for a lecturer. He wanted 'to calculate the entire composition
of the Fraunhofer spectrum and the ratios of the strengths of all
Fraunhofer lines from simple assumptions regarding temperature, pressure,
and composition of the layers.' That was a lifelong work and more
than that. He had the ability to set ambitious goals, thinking on
a timescale of decades, and also possessed the necessary perseverance.
He had already begun preparations for the attack on the generally
accepted Rowland scale of intensities.

\section*{Using equivalent widths.}

The year 1927 brought new success. The previous year, a four-month
eclipse expedition to Sumatra, including Amsterdam astronomer Anton
Pannekoek, had yielded nothing due to cloudy skies; an expedition
with Pannekoek and Minnaert to Lapland was brilliantly successful.
In 1928, the Academy published the extensive report by Pannekoek and
Minnaert. They succeeded for the first time in measuring the 'absolute'
intensities of emission lines in the chromosphere by comparing them
with a standard lamp developed by Ornstein. It was meticulous work:
they processed 1,900 measurements of emission lines between 4154 and
4768 Å into 30 pages filled with minuscule text. They were able to
identify many lines using Saha's theory as multi-valent ions such
as yttrium 10+ or iron 6+. They interpreted the emission lines of
the most prominent elements as electron quantum jumps. This detailed
explanation took up twelve pages with 300 details.

The two astrophysicists reasoned in terms of radiation transitions
between electron levels, 'summation rules,' 'multiplet intensities,'
and 'transition probabilities': these were terms from the quantum
physical work in which Ornstein's Physical Laboratory, along with
H.C. Burger and H.B. Dorgelo, had specialized in collaboration with
theorists like A. Sommerfeld, W. Pauli, H.A. Kramers, and N. Bohr.
Their book indeed referred to the work of Ornstein and his colleagues.
It contained not a single reference to Julius. The publication took
place in 1928, the year in which Julius and Minnaert's Dutch-language
textbook on solar physics appeared, and this one was in English.

During these years, Pannekoek imparted not only his knowledge and
experience but also his scrupulous attitude to his young friend Minnaert.

He wrote in his 51 diary about Minnaert: 'Here, in Lapland, as well
as during the Indian eclipse, I have come to admire Minnaert for his
practical skill and theoretical insight, as well as his complete dedication:
he has effectively been the soul of the work in both expeditions.
52 In his letters, he could sometimes reprimand: 'Since I saw your
original drawings, on which the deviations among themselves were visible,
I did not suspect for a moment that you would leave them out here
in the explanation.'

Pannekoek knew many prominent astronomers personally, such as E.A.
Milne, J.S. Plaskett, and H. Shapley. Some were comrades from his
time as a Marxist theorist. When Milne came to visit him in Bussum,
he invited Minnaert to join the discussion with the great man. He
wrote to Minnaert about the American H.N. 53 Russell: 'He has a talent
for skillfully handling large material and extracting the main trends;
clever and dominant; yet I have an instinctive resistance to all kinds
of treatment methods. He also has a broader collection of multiplets
than others because he has done much beautiful unpublished work on
it.' He taught the young 54 Minnaert to critically examine the giants
in the field as well. That must have been important for the enthusiastic
55 Minnaert.

In 1927, Minnaert published an article with Ornstein on the intensity
distribution in recordings of spectral lines. The article provided
an overview of how they worked with the spectrograph and the microphotometer.
They discussed the influence of slit widths on the profiles and that
of the developer on the photographic plates. It illustrated only their
intensive collaboration. A spicy footnote reveals Minnaert's authorship:
the plates for the article were 'taken and measured in a commendable
manner by Miss M.B. Coelingh and Miss J.G. Eymers.' This refers to
Miep Coelingh and her close friend Truus Eymers; Minnaert had just
become engaged to Miep.

He continued steadily working on a publication about his own experiments.
He compared his true intensities for weak Fraunhofer lines with the
intensities according to the revised Rowland Table by the Americans
and derived an empirical formula that converted Rowland's intensities
R into Minnaert's true intensities i. For the first time, through
extrapolation, he deduced that the Fraunhofer lines absorb approximately
15\% of the energy from the solar spectrum: 'All theoretical considerations
about the shape of the sun's energy curve were dangerous as long as
the depression of that curve caused by the Fraunhofer lines was not
taken into account.'

During final editing, he was unpleasantly surprised by an article
from German H. von Klüber on emission line profiles from the chromosphere.
It included measurements of the intensities of the K and H lines of
the Ca ion, which Minnaert himself had worked on with Pannekoek. Von
Klüber worked at the Einstein Tower in Potsdam, where director Freundlich
continued the photometric work of his predecessor Schwarzschild, who
had died in 1914. Minnaert hurried to submit his article and immediately
included harsh criticism of the German's methodology. The latter's
intensities differed too much from his own values. He advised taking
into account the 'ghost images' of the grating and urged better standardization.
However, their values actually matched quite well, while Von Klüber
used a peculiar 'half-value width.' This may have annoyed Minnaert.

The German had not criticized Minnaert, nor even mentioned him. That
could also have been a point of contention. Minnaert's reaction seems
haughty. This was painful because he had tried to find a theoretical
explanation for the ratio of the surfaces of the two emission lines.
But he did not succeed to connect the theory with the data. His article
had already given much food for thought. While correcting the proofs,
he received the first set of galleys from a contribution by Albrecht
Unsöld, a colleague of Von Klüber: he could just mention in a footnote
that he suspected that this had satisfactorily solved the issue.

\section*{An earthquake}

60 Unsöld had theoretically interpreted the results of his colleague
Von Klüber. His article must have shocked Minnaert for two reasons.
He had insufficiently prepared himself for the reality of absorption,
emission, and scattering as a result of atomic processes due to his
fixation on Julius' theory. In his calculations, the number of particles
in the chromosphere that cause emission played no role at all! According
to Unsöld, these provided an explanation for both the spectral lines
and the Fraunhofer lines. The more active atoms or ions per unit volume
of sun, the greater the effect. Unsöld worked with quantum mechanical
formulas for the 'scattering coefficient' and 'radiation damping'
as a result of colliding and passing particles. A sober consideration
in terms of really existing particles Minnaert had described in his
1916 brochure as a characteristic of the Dutch way of thinking in
physics!

Minnaert must also have been shocked for a second reason. He had moved
for years both in Julius's optical research program and in Ornstein's
quantum mechanical domain. Unsöld finally opened his eyes: his story
originated from a quantum mechanical world of thought, but in Juliaanse
terms, it came down to anomalous scattering in optical physics. Minnaert
had focused on anomalous refraction following Julius because those
refracted bundles of rays could 'in principle' explain so many solar
phenomena!

For Minnaert, solar physics got back on its feet. Protuberances, sunspots,
and chromosphere lines became real again. The asymmetry of the line
profiles might be a peculiarity of the 61 spectrograph, a waste of
useless work within the margin of error. Not these phenomena but Julius'
explanations were an optical illusion: food for psychologists! He
himself had honored optical illusions in the manner of those chained
in Plato's cave: his mockery was simply hubris. Minnaert must have
realized all of this during that July month of 1927.

Unsöld made it clear that the scattering activity depends on the number
of absorbing particles, 'resonators' or 'oscillators' N, and this
also applies to the energy lost in the line profile. Quantum physics
provides the data for the probabilities of electron transitions: for
the K and H lines of the +Ca ion, it turns out that they are in a
ratio of 2:1. The easiest transition brings about the greatest scattering.
Interestingly, Ornstein's Physical Laboratory was set up in the 1920s
to provide empirical evidence for these 62 quantum rules. It was the
pioneering work of Burger, Dorgelo, and Ornstein himself. Unsöld derived
a formula in which the intensities of two lines are related as the
square roots of the number of particles N and the square roots of
the ratio of transition probabilities. One surface had to be in proportion
to the other as v2:1, as 1.41:1.00, and that was correct. His calculations
approached the line profiles quite decently.

Minnaert had submitted his article on July 29th and it appeared at
the beginning of September. He had denigrated the methods of his German
rival. Unsöld had every reason to strike back sharply in his revised
article. However, Minnaert, as a Sunday's child, born exactly at noon,
had all the luck in the world: his German and American opponents would
manoeuvre him into a comfortable position.

\section*{An opportunity for an open goal.}

Unsöld explained that if the intensity of spectral lines is determined
by the number of particles, then inversely, the frequency or abundance
of those particles can be derived from the line profiles. He thereby
opened the way for the quantitative chemical composition of sun and
stars. For the first time, he made estimates known of the abundances
of metal atoms in the chromosphere:

'From the width of a line, one can calculate the total number of atoms
resting on a square centimeter of the sun's surface.' In a Nachtrag,
he addressed Minnaert, who 'criticizes the procedures applied by Von
Klüber and myself.' He argued against Minnaert's points of criticism:
'Without delving into Minnaert's theoretical considerations, I would
like to mention that, as far as I can see, Julius's beam curvature
does not play a noticeable role in our problem. From an experimental
perspective, the agreement between Minnaert's results and those presented
here seems more substantial than the remaining specific differences.'
That was well said and truly the case!

Unsöld had also stirred up the Americans. In July 1928, an article
by H.N. Russell, W.S. Adams, and Charlotte E. Moore appeared, in which
they established a connection between the number of particles N introduced
by Unsöld and the intensities R of their Rowland scale. They had two
advantages: their revised Rowland values and extensive quantum studies
on atomic spectra. However, the Americans also acknowledged that the
basic idea came from Utrecht.

Following preparatory work by Sommerfeld and Heisenberg, in the years
after 1924, Ornstein, Burger, Dorgelo, and De Laer Kronig established
rules for the relative extent to which different quantum transitions
in an atom contribute to the strengths of spectral lines. In technical
jargon, these relative contributions are called 'numbers of active
atoms.' These 'summation rules' pertain to lines that, from an atomic
perspective, are interconnected because they originate from the same
ground level in the atom. Such a group of lines is called a 'multiplet.'
Using these summation rules, Russell and his co-workers determined
the 'numbers of active atoms N' for 228 atoms and ions belonging to
several multiplets. They also estimated the Rowland intensities of
these lines. Thus, they could investigate how the observed strength
of the lines depended on these 'numbers of active atoms.' They attempted
to find a graphical relationship by plotting these N values horizontally
against a derived function of the Rowland values vertically, which
they also varied with wavelength. They might have expected a linear
relationship: twice as many active atoms would result in a line twice
as strong. That turned out not to be the case, however. They found
a complicated graphical relationship that did not lend itself to a
meaningful interpretation.

The Americans were aware that Minnaert wanted to measure the thousands
of surfaces of his calibrated line profiles to find the true intensities:
'The value and importance of such work are clear. But even if the
laboratory measurements were completed, the physical interpretation
will not be easy. It is absolutely not allowed to assume that when
a line blocks ten times as much solar energy as another, it will contain
ten times as many atoms. A satisfying theory of the widths and depths
of Fraunhofer lines is difficult to establish. It is likely that such
a theory will be the end product of a general theory of the solar
atmosphere rather than a step in its development.'

Their own results had not been encouraging, but they still believed
a priori that there must be a royal path. In that case, at a certain
wavelength, a given number of particles could be converted into the
intensity of a Fraunhofer line by deduction, and vice versa, without
requiring Minnaert's heathen measuring work! It was not so strange
that they thought in that direction, because five Americans, including
two of the three mentioned authors, had just published the Revision
of Rowland's Preliminary Table: 66 a standard work with the estimated
Rowland strengths of 21,835 lines. They were not the first to be called
upon to address Minnaert's criticism of their Rowland values!

Minnaert became the undisputed opponent due to the Americans. Through
this polemic, he was referred to his colleagues at the Physics Laboratory
by both the Germans and the Americans! His Utrecht colleagues could
provide him with quantum numbers for the number of particles N of
closely spaced lines. He was the only one with measured surfaces,
his equivalent widths, which were more accurate than the Rowland values.
And he could measure himself! Good old Julius, whose foresight had
resulted in a spectrograph that could still attract worldwide interest.

\section*{The discovery and explanation of the growth curve}

Minnaert alone was the only one who could attempt to establish a graphical
relationship between these two quantities, N and its equivalent width
i. This happened in an article he co-authored with his assistant Van
Assenbergh. Minnaert chose a blue-violet segment of the solar spectrum,
150 Å in length, representing 5\% of the visible spectrum. They measured
the equivalent width i for 57 lines between 4400 and 4550 Å. They
recalibrated the Rowland scale at this wavelength. By working within
a narrow wavelength range, the necessity to treat wavelength as a
variable was eliminated. Following the approach of the Americans,
they plotted the logarithmic values of i against those of the number
of absorbing particles N. A curve emerged, indicating that the intensity
of a Fraunhofer line does not increase linearly with the number of
'oscillators.'

\textbackslash figcaption This was the first 'growth curve' of the
Sun. Half a century later, it turned out that the general trend appeared
somewhat different.

In fact, three qualitatively distinct relationships seemed to emerge:
one for small values of N (weak lines), one for 'medium-strong' lines
with larger N values, and one for 'strong' lines with very large N
values. For weak lines, i appeared to increase proportionally with
N, but 'for medium-strong lines, a systematic deviation occurred in
the sense that the absorption lines were less dark than theory would
have suggested.' In the strong section, two Fraunhofer lines deviated
significantly from the flat middle section. They achieved a result
that, contrary to what the Americans had feared, invited theoretical
explanation.

Since no one had anticipated this relationship, Minnaert gained an
advantage over his rivals. He was convinced that the curve contained
essential information. According to him, it resulted from theories
proposed by Julius---'anomalous scattering' due to anomalous dispersion---and
quantitatively developed by Unsöld. Here, Minnaert proved himself
a street fighter, even daring to present Julius as the champion of
anomalous scattering. He knew very well that his mentor had promoted
anomalous -refraction-. Minnaert stood alongside the greats in the
field of astrophysics.

He had published hagiographic works about Julius only in Dutch 68,
which was not to his disadvantage. There was great anticipation for
his new findings. Could the curve be reproduced in other wavelength
regions? How could it be explained? His announced intention to create
an Atlas of all true intensities using the microphotometer met with
approval everywhere.

In 1930, Pannekoek published an article on the precise shapes of the
69 curves produced by the microphotometer. Changes in pressure and
temperature altered these shapes, as did electromagnetic effects that
caused line splitting, resulting in line bands. Sometimes, those physical
processes affected the wings of the profiles, while other times they
manifested in the center of the profile. Minnaert considered this
article at the time to be the best theoretical work ever written on
the Fraunhofer lines. However, one of his own theoretical articles
made at least as much of an impression.

Minnaert was eager to have the scoop on the theoretical explanation
of 'his' curve. For him, physics ultimately revolved around explanation
rather than phenomena. After two years of research, he and his Ph.D.
student Mulders published an article on the wavelength region between
5150 and 5270 Å. They determined the equivalent widths of 47 green
lines. Once again, they calibrated the Rowland scale: the equivalent
widths of the Rowland values were found to be, on average, 1.29 \texttimes{}
greater than in the blue region! This finding damaged the authority
of the Rowland scale. They rediscovered the familiar curve.

Initially, they had wanted to work in all color regions before attempting
a theoretical explanation. However, they abandoned this plan after
a publication by the German physicist W. Schütz, who, based on laboratory
experiments with spectral emission lines, had discovered a similar
curve. They calculated for days on end, from morning till night, and
obtained the first theoretical 72 growth curves for the sun. Minnaert
named his curve a 'growth curve' because its origin reminded him of
the hypothetical figure he had proposed in his biological dissertation
on the stomata of pine needles. In 1934, he wrote: 'A curve showing
the functional relationship between equivalent width and oscillator
concentration shall henceforth be called the growth curve of the respective
spectral line.' This is what it has been called for sun and stars
to this day.

The physical background of the core of a Fraunhofer line is the absorption
of an 'individual light ray,' a photon, by an atom and its subsequent
scattering when the photon is re-emitted. But the line also broadens.
There are two causes for this. One, even in very weak lines, is due
to the high speed of particles causing a Doppler effect, which results
in minimal variations in frequencies and thus wavelengths. The second
main effect is 'damping.' This is a consequence of the fact that an
absorbed photon remains absorbed for only a finite time. The shorter
this time, the broader the line becomes. The 'natural' lifetime of
the photon is shortened by collisions with other particles. These
broadenings are referred to as 'natural broadening' and 'collisional
broadening.' Together, they are called 'damping broadening.'

With a small number of atoms, the Doppler effect plays the main role,
and the absorbed intensity is proportional to the number of particles.
With a larger number of particles, there is a slight increase in absorption,
which is primarily attributed to damping. When this damping becomes
the dominant influence with a very large number of particles, the
absorbed intensity in the wings of the profile increases again with
the square root of the number of particles. Of the three parts of
the curve that one may theoretically expect, Minnaert and Mulders
wrote that Schütz had found only the last two in his experiments:
'The solar spectrum, on the other hand, shows us the entire course
of the theoretical curve in the purest completeness.' They could have
equally well used Unsöld's scattering theory as well as Pankoek's.
Here begins Minnaert's eclectic grappling with the results of colleagues
and his own theorizing based on formulas he creatively adapts to the
material to be interpreted. The manual calculations were murderous,
but Minnaert felt at home with them. Incidentally, they honestly noted
that the growth curve would also appear if, instead of plotting the
equivalent width i vertically, they had plotted the Rowland intensity
R!

In 1931, Minnaert and his collaborator C. Slob wrote a triumphant
article 74 for the Academy. They predicted that the equivalent width
would make a splash in astrophysics. In star spectra, the precise
profile of an individual line can hardly be determined, but its surface
area can! Minnaert and Slob illustrated, with the help of formulas
from Pannekoek, that the growth curves would prove to be universal.
By now, Minnaert's articles were teeming with synonyms: true intensity,
total intensity, total absorption, and equivalent width. Over time,
the latter term came to the forefront.

\textbackslash figcaption The change in the shape of the 'growth
curve' with the increase in the concentration of absorbing particles.
If Doppler broadening dominates, small \textgreek{α}-values apply.
This is the case in the sun: high temperature, high particle velocity,
thin gas, little damping. Then the growth curves shift to the right.

Minnaert allowed himself more liberal articles from then on. In the
Zeitschrift für Astrophysik, he pondered the remarkable residual intensity
of light in the heart of a 75 Fraunhofer line, as found by Unsöld.
Could the scattered light perhaps deviate slightly from the frequency
of the incoming radiation? He was no longer the journeyman of his
teachers, whether Julius, Ornstein, or Pannekoek. He became someone
who thought aloud about his field, dared to ask open questions, engaged
in dialogue, and knew how to appreciate the contributions of others.

\section*{Minnaert surpasses Rowland}

The debate between the Minnaert scale and the Rowland scale had not
yet been settled. On Ornstein's instructions, Minnaert also occupied
himself with the Foundation for Lighting Science and physiological
issues. He repeatedly encountered the treacherous aspects of color
perception with the eye. Based on a comparison of the results from
76 observers of the Orion star Betelgeuse, he concluded 'that an observer
of red stars should have their own eyes tested for color vision ability.'
Another time, he posed the question: 'If an image is so faint that
I can barely perceive it with my eye, how long do I need to capture
it on a photographic plate?' Minnaert asked a seemingly simple question,
which nevertheless had significant practical implications. It turns
out that such exposure times sometimes amount to several hours, meaning
that such phenomena are overlooked by instruments. Visual observation
is more effective in such cases. The eye proves to be insensitive
to violet and very sensitive to green: 'The eye is most sensitive
in the region where the photographic plate is least sensitive.'

\textbackslash figcaption The 'catastrophe' between green (5500)
and green-orange (5900)

Minnaert always understood that surprises could occur if the Rowland
scale were applied across the entire spectrum to its equivalent widths.
His PhD student, Mulders, addressed this question and announced their
discovery in 1934. During the graphical processing of the results,
Mulders plotted the equivalent width i in mÅ vertically against the
wavelength in Å. The graph now showed the Rowland intensities compared
to i, ranging from '-3' to '+4'. If Minnaert's entire operation had
been useless, eight more or less horizontal lines would have appeared
here. In that case, the intensities according to Minnaert and those
according to Rowland could have been converted into each other via
a simple proportionality factor. This was the case between 4000 and
5400 Å and between 5900 and 8500 Å. The R-values suddenly dropped.
Between 5400 and 5900 Å. That had to be made known to the entire world!
This unexpected drop must have dealt a decisive blow to the skepticism
about converting Rowland units into Minnaert units. Suddenly, the
lack of quantitative sense in Rowland's work and that of his followers
proved to be a major mistake---a game of misunderstandings between
the eye and the plate. No one except Minnaert could have imagined
such a green-yellow catastrophe; it was also a triumph for Ornstein's
calibration program! Minnaert's star rose to the zenith.

In 1936, Minnaert even received congratulations from Unsöld, by then
the undisputed theoretical expert on the sun, because he had succeeded:
'The fact that you managed to derive the theory of line wings from
the differential equation in such a simple way delighted me greatly.
I have always wished that these formulas, which are so simple and
intuitive in their structure, could also be derived in a straightforward
manner.'

\section*{A versatile astrophysicist}

Minnaert engaged with various aspects of his field. For years, he
had been focused on eclipse images of the solar corona and chromosphere.
He published extensively on the corona. In 1930, he wrote an article
about the polarization of its light. Scattering by 'free electrons'
would provide a good explanation for this polarization, but it conflicted
with other assumptions. Minnaert addressed a note by Einstein on this
subject and considered a suggestion by Ornstein regarding a 'recombination
spectrum.' He agreed with his discussion partners that the corona
must have a lower temperature than the solar edge of 6,000 Kelvin;
the consensus on a temperature of one million degrees came only ten
years later with the work of Bengt Edlén from Sweden. In a lecture
in Leiden, he said: 'The corona likely consists of strongly ionized
gas, i.e., a mixture of atoms, ions, and free electrons. The free
electrons should scatter the most strongly because their mass is the
smallest; this would align well with the Observation that Fraunhofer
lines in the inner corona are not visible, apparently because the
thermal motion of free electrons imparts strong Doppler shifts to
all lines, erasing them. If he had continued his reasoning based on
these facts, he should have predicted the high temperature of the
corona. He also struggled with explaining the continuous spectrum:
this only became possible ten years later with Rupert Wild's discovery
of the negative hydrogen ion in the solar atmosphere.

In 1932, Minnaert and his doctoral student Wanders addressed the theory
of 83 sunspots. The accepted view was that sunspots resulted from
rising gases that expand and thus cool down. When they photometered
and interpreted their recordings---a rare practice---they concluded
that radiation equilibrium must prevail, leading to a temperature
of 4,300 K. They referred to Julius's work, who had not found the
violet shift in the umbra of sunspots that should have accompanied
rising motion. Wanders could position himself at the heart of a long-standing
controversy with his dissertation.

Minnaert guided his doctoral students, formally Ornstein's, always
into the heart of debates and ensured that he also received credit
through joint publications. He did not forget himself; in his own
publications, he prominently presented his students' results. They
benefited from his questions, and he could use their results in syntheses
beyond their capacity: here too, there was 'mutual benefit.'

In the early 1930s, recognition of Minnaert's quality accelerated.
He corresponded with colleagues, attended conferences and observatories,
wrote an internationally accepted masterpiece, and became one himself.
Only a small part of his correspondence has been preserved. In 1930,
the Russian E. Perepelkin sent him some protuberance spectra from
Leningrad; he had a thousand of them. He wrote to Minnaert: 'Unfortunately,
I cannot process my extensive material now because the Pulkovo Observatory
does not yet have a self-recording photometer. A Moll has been ordered
for a long time, but when will we get it?' In response to an article
by Minnaert on protuberances, he wrote: \textquotedbl I think your
result is much more accurate than my earlier 85 measurements.\textquotedbl{}
In 1932, Von Klüber wrote from the Observatory of the Einstein Foundation
in Potsdam. He had made recordings of the red atmospheric bands of
the solar spectrum: \textquotedbl As I hear, you are working on the
same issue, and it might be good if we stayed in touch about it?\textquotedbl{}
Von Klüber had heard from his director Freundlich that Minnaert was
visiting their observatory: \textquotedbl We would all like to see
that.\textquotedbl{} Willi Cohn from Harvard Observatory asked Minnaert
that year for assistance with the eclipse expedition and help with
photometric standardization: \textquotedbl Could I get a standard
lamp from your lab, like the type you gave to the Potsdam eclipse
expedition of 1929?\textquotedbl{} He invited Minnaert for further
discussion on the polarization of the corona.

In 1933, he attended the congress of the International Astronomical
Union in the United States. He traveled around Canada and America,
making friends for life. Theodore Dunham Jr. from Mount Wilson Observatory
later wrote to him that he wanted the article by Minnaert and Slob.
He was working on a device that could radically reduce the reduction
time of the microphotometer, otherwise hoped that Minnaert would soon
return to California, and referred to the conversation between them
and 'Miss Payne.' Also dating from that year are the letters from
the Russian W. Barabascheff from Kharkov about measuring sunspots:
\textquotedbl I am very interested in Mr. Wanders' dissertation and
would be grateful if I could get one.\textquotedbl{}

At the end of 1933, Prof. Dr. W. Grotrian, editor of the Zeitschrift
für Astrophysik, wrote: \textquotedbl Here in Potsdam, a lot has
changed since you left. It has become quiet in the tower: a definitive
new arrangement is still pending. We have good news from Mr. Freundlich
in Istanbul.\textquotedbl{} He would publish two articles by Minnaert:
\textquotedbl We look forward to the Dutch colleagues continuing
to publish in our journal.\textquotedbl{} The Jewish scholar who fled
Hitler tried to keep his head above water in Turkey. The idea of a
publication boycott would not have occurred to Minnaert, even though
he was opposed to the German regime. Precisely in times of political
disagreement, scientists must continue to communicate with each other!

The fame of the 43-year-old Minnaert can be illustrated with a passage
from the letter that Gerald P. Kuiper, recently appointed as Assistant
Professor in Chicago, wrote in 1936 to the Utrecht astronomer J. van
der Bilt. The people who pass by 91 revue were, without exception,
giants in the astronomical world: 'Struve wanted to add another theorist
to his staff. I proposed two candidates, Minnaert and Chandrasekhar.
Shapley was enthusiastic about the idea of Minnaert and strongly supported
it. I wrote, among other things, that I thought M. was 42-44 years
old; that Chandrasekhar considered him the best solar physicist in
the world, etc. It then turned out that Struve felt very strongly
about Minnaert because he had been looking for a good solar physicist
for years; but the President preferred Chandrasekhar, who is only
25 years old (a young genius!). Van Biesbroeck then told Struve that
he thought Minnaert was much older than I had said, close to 50. That
completely rejected M.'s candidacy, as they only wanted to appoint
young people. I immediately felt the possibility that Van Biesbroeck
might be anti-Minnaert; so I asked you how old M. was.' Unfortunately,
the cards were already shuffled. Kuiper had found it regrettable:
'Minnaert is internationally highly regarded! I hope we can have him
for a year. (Of course, you don't want to talk about this Minnaert
issue!)' Minnaert was thus almost appointed in Chicago, where in those
years a world top of astronomers was being acquired. His Flemish past
had thrown a spanner in the works.

\section*{The curtain falls on anomalous dispersion}

In 1937, the British astronomer R. Woolley had determined that the
sun's density gradients are not significant. He concluded that Julius'
anomalous dispersion was therefore permanently ruled out. Minnaert
filed 92 objections against this.

According to him, Julius had distinguished three effects of anomalous
dispersion. The first effect was classical anomalous dispersion: regular
refraction according to Kundt. This could not be ruled out because
it was hard physics that had been around for a century. However, this
effect played no role in solar physics, even though Julius had hoped
to explain protuberances and chromosphere lines with it.

The second effect was anomalous dispersion due to spreading in a non-homogeneous
gas. Indeed, this effect too had proven to be an illusion: 'I myself
have demonstrated that in the solar atmosphere, in that case, very
sharp density gradients would have to occur, extending only over millimeters;
this is indeed unlikely.' He now interpreted this part of his dissertation,
unlike earlier, as a feat against Julius.

The third effect was anomalous dispersion in the form of scattering.
According to Minnaert, both physically and mathematically, two elaborations
of this phenomenon were equivalent. First, a quantum mechanical argument
was possible using the scattering coefficient k, which takes an abnormally
large value at the relevant wavelengths, and radiation damping. Additionally,
an optical consideration was possible, which states that the refractive
index near the two Fraunhofer lines is abnormally large (or small)
and thus applies the factor (n - 1) after which Rayleigh's scattering
law is applied. Minnaert suggested that the results of these two approaches
would lead to identical physical relationships. According to him,
it was childishly simple to convert the terms Julius borrowed from
classical physics into those of quantum theory.

This was Minnaert's retrospective on Julius that he circulated. He
would not gain any supporters. For the last bastion of anomalous dispersion,
his straw man of anomalous scattering, the curtain would fall in the
late 1930s. Scattering also plays no role in the solar atmosphere;
it is too sparse. With this, the anomalous scattering at the wavelengths
of Fraunhofer lines was out. The basis for understanding Fraunhofer
lines lies in the absorption calculated using Planck's quantum law.
Minnaert believed there should be a bridge between classical optics
and quantum theory. Such a bridge did not exist. Minnaert went to
great lengths in adhering to his preconceived notions. In a certain
sense, this is characteristic of a theoretically inclined natural
scientist, but his persistence bordered on dogmatism.

That Minnaert would continue to position himself as the defender of
Julius' ideas is nonetheless understandable and is not based solely
on psychological grounds. Without Julius' solar theory, no spectrograph
would have been installed. Without the spectrograph, independent solar
research in Utrecht would not have been possible. Without Moll's microphotometer,
which was made for Julius' experiments, the true intensities and equivalent
widths would not have been defined in Utrecht. Without the micrometer
measurements of alleged redshifts and line center displacements, Minnaert
would not have learned such extreme precision. In all these aspects,
Minnaert owed his success to Julius.

Additionally, in the 1920s, the microphotometer had become the basis
for Ornstein's photometric project at the Physical Laboratory. Minnaert's
breakthrough was also a result of his involvement in the calibration
work within Ornstein's research program. In Utrecht, Minnaert found
two ideally prepared research environments that offered him, as an
exile, the space to deploy his ambition, perseverance, and boundless
work ethic for Greater Netherlands. He would show the world that his
appointment to the Flemish University had not been a mistake.\\

Endnotes:

1 De Jager, 1993, 40.

2 Sommerfeld, 1914; 1919.

3 Minnaert to Burgers, February 2, 1919.

4 This issue is the subject of Otterspeer's book, 1997, 61. See also
the review by L. Molenaar in Jaarboek Buitenlandse Zaken, The Hague
1998, 148.

5 Julius to Curators, December 1919. Astronomy Archives.

6 Minnaert to Burgers, December 30, 1919.

7 Nowadays, the unit nanometer (nm) is used, i.e., 10\textasciicircum -9
meters, which is ten times larger.

8 Minnaert, (1947). See Literature Part III (1945-1970).

9 In the Kirchhoff year 1959, Minnaert wrote: 1859 - Kirchhoff explains
the Fraunhofer lines.

10 To observe the operation of a reflection grating, one simply needs
to hold a CD under a white lamp. Looking at the top of the CD, one
can see that the light is decomposed into all the colors of the rainbow.
The pits on the CD surface function like the grooves in a grating.

11 Rowland, 1895.

12 Vervaet, E., {*}The Birth of Spectral Analysis{*}, Intermediair,
43, 1987, 43.

13 Heijmans, 1994, 73.

14 Minnaert, (1947).

15 Moll, 1919.

16 Appendix at the end of the book.

17 Here and there, the present tense is used because the coelostat,
'the device that holds the sky,' is part of the museum setup of the
Utrecht Observatory, which was made operational again in 2002.

18 Minnaert to Burgers, April 13, 1919.

19 Julius, 1923.

20 Minnaert to Burgers, April 13, 1919.

21 Heijmans, 1994, 33.

22 Minnaert, 1936, 66.

23 Minnaert, (1947).

24 Minnaert, (1921a). The journal {*}Physica{*} with the subtitle
{*}Dutch Journal for Physics{*} marks the professionalization of physics.

25 Heijmans, 1994, 43. Minnaert received a postcard from Ornstein
in Jerusalem in the early 1920s.

25 Heijmans, 1994, 51

27 Bleeker, 1951. It seemed logical that gas layers farther from the
core would be cooler than the solar edge of 6000 K.

28 Heijmans, 1994, 81.

29 Heijmans, 1994, 64.

30 Minnaert to Burgers, June 17, 1921.

31 Minnaert, (1921b).

32 Minnaert, (1921c) mentions that he is working on attenuators. De
Jager, 1993, 99, refers to the year 1923.

33 Minnaert, (1923a), 103.

34 Minnaert to De Bruyker, March 18, 1922. He probably sent a reprint
of his 1921 article on {*}Ghost Images{*}.

35 Minnaert (1923b, 1923c).

36 Hentschel, 1994, 121.

37 Julius, 1928.

38 Minnaert, 1925.

39 Einstein, 1925. See the Appendix.

40 Minnaert, 1925, July 6, 1926.

41 Ornstein and Zernike, 1916-1917.

42 The astronomer De Jager was helpful in formulating a short summary.

43 A modern example is that of a laser beam aimed at the moon. Without
Earth's atmosphere, a spot with a diameter of 400 meters would be
illuminated on the moon; because the beam passes through the atmosphere
where density fluctuations occur, the diameter of the spot is several
kilometers.

44 Or the electrons of the corona around the sun.

45 Letter from H.J. Jordan (chairman) and L.S. Ornstein (secretary).
The latter wanted his colleague H.C. Burger to be appointed as a theoretical
physicist; the faculty received H.A. Kramers from Copenhagen, the
Mecca of quantum mechanics: an acquaintance of Minnaert from his time
at Christiaan Huygens. His friends Dirk Coster and Jan Burgers had
already been appointed as professors in Groningen and Delft.

46 See the Appendix.

47 Julius and Minnaert, 1928.

48 Minnaert, 1926, January 16th. In two installments that year in
Hemel en Dampkring.

49 Pannekoek and Minnaert, 1928.

50 Heijmans, 1994, Chapter 5, The sum and intensity rules, p. 73,
and Chapter 6, Further research on intensities in atomic and molecular
spectra, p. 97.

51 Pannekoek, Memories, 1982.

52 Pannekoek to Minnaert, May 2, 1928. Astronomy Archives.

53 Pannekoek to Minnaert, September 10, 1928.

54 In his speech at Pannekoek's funeral in April 1960, Minnaert referred
to him as his 'great teacher' and dear friend. For the first time,
he seemed to have a relaxed relationship with someone he highly valued
as a scientist and who was twenty years older.

55 Minnaert, Ornstein, (1927c)

56 Saint John, 1928. 57 Von Klüber, 1927. 58 See the Appendix. 59
Minnaert, (1927b).

60 Unsöld, 1927.

61 It ultimately proved to be so. Other spectrographs, such as those
of the American Evershed, showed an asymmetry toward violet.

62 Fascinating overview by Heijmans, 1994, 85.

63 Unsöld, 1928a and b, effectively a continuous article.

64 Russell, 1928.

65 Ornstein and Burger, 1926, 412.\textquotedbl{}

66 Saint John, 1928.

67 Minnaert, (1929b).

68 Minnaert, (1921a), 1926, 1928.

69 Pannekoek, 1930.

70 Minnaert, (1931e).

71 Schütz, 1930.

72 Minnaert in: De Jager, 1965.

73 Minnaert, (1934a).

74 Minnaert, (1931f).

75 Minnaert, (1932a).

76 Minnaert, (1931d).

77 Minnaert, (1932b).

78 Mulders, 1934.

79 Minnaert, (1936b).

80 Unsöld to Minnaert, August 27, 1936.

81 Minnaert, (1930a).

82 Minnaert, (1930b). Leiden, May 24, 1930. Note from De Jager

83 Minnaert, Wanders, (1932c).

84 E. Perepelkin to Minnaert, 1930. Astronomy Archive.

85 H. von Klüber to Minnaert, 1932.

86 Willi Cohn to Minnaert, spring 1932.

87 Theodore Dunham Jr. to Minnaert, 1933.

88 About Cecilia Payne-Gaposchkin (1900-1979), discoverer of the dominance
of hydrogen in the universe, M. Offereins wrote a Women's Portrait
in: NVOX 1997, 10, 516-517.

89 W. Barabascheff to Minnaert, 1933.

90 W. Grotrian to Minnaert, December 1933.

91 G.P. Kuiper to J. van der Bilt, March 15, 1936. Astronomy Archive.

92 Minnaert, (1938a). He responded with his Anomalous Dispersion in
the Sun to a lecture by astronomer R. Woolley from December 1937 for
the Royal Astronomical Society, which was printed in The Observatory
in early 1938.

\chapter{A marriage which couldn't be modern}
\begin{quote}
\textquotedbl For Mrs. Curie, marriage by no means implies saying
goodbye to scientific work.\textquotedbl{}
\end{quote}

\section*{Marcel Minnaert and Miep Coelingh}

Minnaert had lived with his mother for nearly a quarter of a century.
His father had advised him at the time to dedicate himself to work
and to be cautious in choosing a life partner. He had mentioned the
age of thirty. Minnaert was not far off from that.

Little is known about the blossoming love between the extroverted
Marcel Minnaert and the very reserved Maria Bourgonje 'Miep' Coelingh.
In 1923, at the age of seventeen, she began studying physics in Utrecht
after completing her HBS (higher civil service). They must have met
during their first year of study. The romantic relationship between
the student and her lecturer may have only begun after Minnaert's
first eclipse expedition, in the spring of 1926. He was then thirty-three
years old, and Miep, born on February 17, 1906, was twenty. She still
lived with her parents in Bussum and commuted to Utrecht. The enamored
Minnaert received the approval of her parents. Several postcards indicate
their relationship. The earliest was from Jet Mahy, who wrote in May
1927 from Blankenberge in Flanders: 'Glad that I at least caught a
glimpse of you and briefly met Miss Coelingh.'

Derk Coelingh, Miep's father, had received a scholarship as a mathematics
student from the municipality of Amsterdam, became a math teacher,
and later director of the 3rd HBS on Mauritskade. In 1900, he was
awarded a cum laude doctorate in mathematics and physics and married
a French teacher, Maria Bourgonje Smit. They had four daughters: Joop,
Wil, Miep, and Mia, who were one, four, and nine years apart in age.
They lived outside Amsterdam due to the mother's malaria. Derk Coelingh
became a member of the Education Council and was appointed Officer
in the Order of Orange-Nassau in 1924.

Minnaert and Derk Coelingh must have known each other even before
the relationship with Miep. Coelingh was, after all, the lifelong
secretary of the Dutch Nature and Medical Congresses, where Minnaert
had presented both his staircase attenuator and his equivalent latitude.
Coelingh knew his Flemish colleague Frans Daels, professor of medicine,
secretary of the similarly named Flemish Congresses, and initiator
of the Yser Pilgrimages. Coelingh must have understood his son-in-law's
passion for physics, didactics, and Flanders well and therefore approved
of his new passion.

The teacher’s family was not very warm. The father was always away
for science. The mother had given up her work outside the home, was
cool towards the children, and could not forge the family into a unity.
Miep had a difficult relationship with her mother. Two children studied
natural sciences: Wil studied biology and Miep studied physics. At
the time, girls actually made up half of the physics student contingent.

When Miep was fifteen, she met Felix Ortt, who was then Minnaert's
neighbor and directed the Liberal Christian Youth Community of Bussum
from Soest. Derk Coelingh was a member of the progressive Freedom
League and did a lot for the People’s University. He was an atheist,
and Miep's mother was non-religious. Still, the girls had to attend
catechism for five years as part of their general development. Miep
became a member of the Reformed Church and had herself deregistered
by a bailiff after her marriage.

Miep adored Marcel and shared all his ideals. She had been a teacher
at an industrial school for girls, where she had pioneered physics
for three student projects. She began giving lectures on the position
of women for Flemish-Dutch Associations. She lived for science. Her
first scientific publication was written at the Physical Laboratory
when she was twenty-one, together with her close friend Truus Eymers
and director Ornstein.4

\section*{A marriage and a divorce}

Miep and Marcel married on December 20, 1928. The story goes that
they were averse to tradition, wanted to live together unmarried,
and got married in raincoats. The story also goes that Ornstein had
threatened to have them removed from the Physical Laboratory if they
would live together without being married. Miep wanted to be addressed
by her maiden name after her marriage, which was highly unusual. A
laboratory servant understood that she no longer went by 'Miss Coelingh'
and did not want to be called 'Mrs. Minnaert,' so in desperation he
addressed her as 'Miss Minnaert.'

The couple moved in with Mrs. Minnaert at De Blauwvoet. This might
have been understandable for an inexperienced 22-year-old woman who
wanted to leave her parental home, but Minnaert could have known it
would cause problems. The Flemish widow adored her son and viewed
her daughter-in-law with skepticism. The casual way her son got married
probably didn’t sit well with her: a formal wedding and a new home
would have been clear signals that a different time had begun for
her.

Their honeymoon was the 1929 eclipse expedition to Sumatra: for Minnaert,
it was his second boat trip to Sumatra in three years. They set sail
in March and returned four months later. On May 3, Minnaert wrote:
'Often until 10 or 11 at night, I am taking readings, and Miep always
helps me and is tireless. Apparently for the double reason that she
doesn’t want to leave me alone and that she finds the work pleasant
herself.' They led their own lives, on the boat and in accommodation
on Idi. The story goes that Minnaert tipped his hat to the coolies:
that was unheard of. Miep and Marcel talked with the locals and heard
the old stories of the Aceh uprising. Even then, the idea must have
arisen among them that communism could contribute to a more dignified
world.

At the end of the trip, it turned out Miep was pregnant. They celebrated
this with an additional vacation in Switzerland. Koenraad was born
on January 28, 1930. A postcard from Miep's sister Wil from Pasoeroen
in the Indies has been preserved: 'Dear people, what fun pictures;
on the two large ones from March 2, 1930, he looks like Miep, I think.
Also nice of Father and Mother. Father is really holding him cleverly.
Mother's way of course, hers are also good looks, but she really needs
to get a little comb someday! Too bad you lost your girlfriend, do
you have another one already?'

Miep probably had a tough time on Parklaan. The mentality of that
bossy mother and the stubborn Miep differed greatly: the difference
between a Flemish matron and an independent Dutch woman also played
a role. According to Jozefina Minnaert, the daughter-in-law should
do what the mother-in-law says. Probably, a bad relationship arose
from the beginning. Miep devoted herself to science and continued
her studies, even after having children: she hired a maid to take
care of them. Jozefina opposed this, although at the time in Bruges,
before Marcel's father died, it had gone the same way.

Miep Coelingh graduated, to the dismay of many professors, with a
thick belly, three weeks before Boudewijn would arrive on April 5,
1931; thirteen months after Koen. The Germanic names reflected Flemish
resilience: 'Koen of counsel' and 'Boude wien' (brave friend). Postcards
from August 1931 from six Spanish cities show that Minnaert celebrated
his mother's 75th birthday together with her. From Madrid, he wrote
to Koen: 'In this palace, a king used to live, a very rich man. But
now the people have said that the king had to leave. And now many
other people live in the house, but no king anymore.' In hindsight,
this text preludes the civil war. Minnaert went on a trip with his
mother while his wife took care of the babies. That had little to
do with the 'fresh married life' that Minnaert had highly praised
to his friend Burgers. At some point,

Miep probably demanded that her mother-in-law live separately in a
small house on the property, which could only be reached from outside.
Maybe the Spanish trip had made up her mind to go through with this
housing separation. Jozefina Minnaert-van Overberge was joined there
by Miss Anna Secrève, who had been the companion of the flamingant
Roza De Guchtenaere.

Miep Coelingh wanted to get her Ph.D., but ran into a wall of disapproval.
Ornstein was not opposed to women in physics but believed that mothers
belonged at home. At the Physics Laboratory, she was mockingly referred
to as 'the second Madame Curie.' Miep could fortunately rely on a
group of six girls in physics with whom she could stand strong in
this male-dominated environment. Miep continued her promotion, and
her husband followed suit. She eventually earned her doctorate under
the chemist H.L. Kruyt. Ornstein allowed her to conduct experiments
at the Physics Laboratory after all.

\section*{The Curies: an exemplary marriage?}

The derogatory reference to Curie may have been partly due to a book
by Minnaert that dealt with both radioactivity and the life of Pierre
Curie. In it, he addressed the concept of 'nuclear reactions' and
drew a relatively early revolutionary conclusion: 'The astronomer
has found new possibilities in radioactivity to explain the source
of the sun's and stars' heat.'

Minnaert likely mirrored himself in his biographical sketch of Pierre
Curie: 'A silent man, often lost in thought. And yet, one could sense
a gentle kindness emanating from him, a great benevolence and helpfulness
toward everyone. He had firm notions of what was good and noble, and
he lived by those principles in everything he did, even if such steadfastness
brought him little advantage. He never wanted to compromise on such
points or let himself be carried away by half-measures. He did not
engage in superficial distractions, but he was extremely sensitive
to beauty and dedicated his entire life to science. Like many other
scholars, he could not entirely free himself from the very human desire
for recognition, fame, and advancing in the world. Curie did not think
about these things in his work; all his thoughts were directed toward
this single goal: understanding and uncovering phenomena. He never
rushed to publish a discovery to prevent someone else from being first;
he was indifferent to who received the credit as long as science progressed.'
The question is how Minnaert arrived at that last piece of wisdom:
in any case, he himself had certainly hurried to publish in order
to stay ahead of competitors.

Minnaert sketched several tableaux vivants of the Curies' lives. One
of these likely also concerned his own marriage: 'The old notion that
the man is meant to provide money and the woman for the house and
clothes, they do not want to acknowledge. Instead, they strive to
be companions in their life's work as man and woman. They also wish
to encourage and support each other and share both joy and sorrow
together. Is this form of love and fidelity not at least as noble
as what was once considered the duties of a housewife? For Mrs. Curie,
marriage therefore does not mean bidding farewell to scientific work.
She is a teacher at the Higher Normal School for girls in Sèvres near
Paris, thus contributing her part to the family income, and in her
free time, she dedicates herself entirely to physical research.'

Marya Curie-Sklodowska had been 'regentes' and remained dedicated
to science as the mother of two daughters. In the scene from 1906,
Pierre was run over by a carriage. 'Mrs. Curie continues her husband's
work faithfully,' Minnaert wrote casually, while earlier he had argued
that she had initiated the research into radioactivity. Marie Curie
would be the only one to receive both the Nobel Prize in Chemistry
and Physics, become a professor, and yet be rejected as a member of
the French Academy: 'Apparently due to a lack of advertising and political
friends,' Minnaert thought. He was blind to her being rejected because
she was a woman.

Was Minnaert's idyllic sketch of this scholarly marriage a blueprint
of his own desires? The French couple had come to appreciate each
other based on equal careers, without including a mother-in-law in
the deal.

\section*{A Diary about family life}

Minnaert started a diary in February 1934. He wanted to document the
early childhood of the children, just like his father did. Koen was
then just four years old and Bou almost three. Koen could suddenly
be wild, moody, and troublesome. He was sensitive, took harsh words
very much to heart, and spoke wisely for his age. Boudewijn hopped
around, ate plenty and often, it was balanced and cheerful. Both children
were often sick. The parents, like Minnaert's parents had done, consulted
numerous doctors who consistently provided contradictory advice. Doctor
Van Schaik vaccinated the children against colds. Another doctor advised
showering with cold water, during which the children had to shuffle
slowly for two rounds.

Koen was the first to attend juffrouw Schroeder-van der Kolk's Montessori
class near home. His upbringing wasn't without its challenges: 'It’s
a huge problem to get Koen to obey. He only does what you say if you
negotiate with him as equals. Force won’t make him give in. When he
plays with Bou, he’s the boss, Bou the cat; or he’s the captain, Bou
the sailor. Koen plays independently a lot, with ropes, planks, crates,
stones. Bou is more fond of people and harder to encourage to play
on his own. Koen calls his mother 'my favorite,' and Bou calls his
father 'my favorite.' Even though we carefully avoid making any distinctions.
Every evening the routine is: picking me up from the train, an evening
walk, treats, undressing, showering, a little dip in our bed, telling
a story, sleeping.'

Minnaert read aloud from Nils Holgersson by the Swedish Lagerlöf and
from the Fairy Tales by the Dane Andersen. He pondered: 'A lot of
trouble comes from jealousy, which manifests as an exaggerated careful
weighing and considering of everything to check if both boys are getting
exactly the same things. Along with that, Koen’s tendency to boss
Bou around, which Bou responds to by spitting, saying nasty words,
etc. If Bou doesn’t do what Koen wants, he hits him; something Bou
almost never retaliates against with hitting. At Miep’s urging, we’re
doing our best to combat these tendencies solely through kindness
and understanding reasoning.'

He wrote that in 1933 he had dressed up as Sinterklaas and had taken
off his costume to reveal who this holy man really was. He had apparently
adopted this approach from his father. The following year, there was
again knocking and stomping: 'The children were no longer afraid,
still somewhat impressed, but found it normal when I took off my suit
again. Bou cheered that he had now gotten a new daddy. But he didn't
dare go upstairs: \textquotedbl See, Koen, there is a real St. Nicholas,
because I heard his horse.\textquotedbl{} Magic, after all, is hard
to push aside with Enlightenment.

In the winter of 1934, the children were sick and coughing for half
the school time. For those staying home, the maid was available, so
the parents kept their hands free. They were visited by Doctor De
Kleyn, who prescribed calcium chloride and a diet without purines.
At Easter 1935, they went to visit family in Ghent for three days.
In the summer, with Koen five years old and Bou four, they finally
started playing on the street with other children. That was a year
later than Marcel's experience in Bruges: 'They enjoy it incredibly
there. Koen is often the leader, assigning all kinds of tasks to the
smaller boys. He always longs for more, further ahead. When he gets
what he wants, he immediately thinks about the next thing; he desires
things so passionately; he can talk so sensibly and defend his position
energetically.' About Bou, his father thought: 'He doesn't take pleasure
in others' misery. If someone takes Koen's side against him, he gets
angry and says that one should give in to Koen.'

Miep Coelingh practiced a sober upbringing, with sparing use of sweets
or gifts. On Sunday walks, she divided four pieces of the six-piece
chocolate bar; the rest was for another time. The story goes that
Miep had become a vegetarian after a youth leader told her, after
she petted a lamb, that she could have that creature on her plate
that evening. She had a fierce argument with her mother about it and
found a natural ally in her husband. She imposed her vegetarianism
on the children, who sometimes had little understanding for it. The
story goes that once the maid gave Koen a piece of ham, whereupon
Miep slapped it out of his hands. There were things with which one
could not joke in the Minnaert household. Playing with the children
from neighboring families taught the boys that different norms applied
elsewhere.

There was no cuddling in the Minnaert household; the contact between
parents and children was rational. In the winter of 1936, Minnaert
picked up his diary again. He apologized for a year of absence with
the characteristic line: 'The present and the future are more important
than the past.' For colds, Dr. Hettema had prescribed a new regimen:
'Cold shower in the morning, warm bath in the evening, no animal fat,
no egg, lean cheese, no beans, no peas, no peanuts. The result is
truly increased appetite. Eating raw vegetables in the morning also
contributes to this.'

The word of the doctors was the holy word of science for the parents.
On the advice of Professor Ten Bockel-Huininck, they spent a month
in Zandvoort during the summer: 'Once a week, I come back to Bilthoven
with the children to say goodbye to Grandma, otherwise she can't cope.'
The children received an autoped from Grandma. 'Occasionally, friends
visit. Permanent friendships do not form.' It seems that cozy children's
birthday parties are not the order of the day; just as they were not
common at Marcel or Miep's in the past either. The family seems somewhat
inward-looking.

Minnaert now had a family of his own, but his lifestyle had not changed.
He was occupied with his scientific work, which gained international
recognition in the early 1930s. For this, he had to spend several
months in the United States in 1933. His organizational and political
work for Flemish refugees also took up a lot of time. His pioneering
work in the field of physics education drew attention. He simply continued
the restless activity he had developed while living with his mother.\\

Endnotes:

1 Jet Mahy to Minnaert, May 28, 1927.

2 Derk Coelingh only laid down this function in 1941.

3 This book by Minnaert is central to the next chapter.

4 Ornstein, Eymers, and Coelingh, 1927.

5 Minnaert to Jozefina Minnaert, May 3, 1929.

6 Wil Coelingh to Miep from Pasoeroen, March 1930.

7 Minnaert to Koen, August 1931.

8 Stamhuis, J.H., M.I.C. Offereins, Two female physicists and their
promoter in the Interwar period; Lili Bleeker, Truus Eymers, and Leonard
Ornstein, Gewina 20, 1997, 98.

9 Minnaert, 1931, in the series Character-Knowledge-Art for Industrial
Education. The brochure was a spin-off from his work on the article
Science in Flanders, which is discussed in chapter 11.

10 Minnaert, Diary: several quotes from it.

11 Memory of Koen Minnaert, communicated by and through his wife Els
Hondius.

\chapter{Pioneer of Physics Didactics}
\begin{quote}
'Teachers form a conservative mass that opposes the development of
education.'
\end{quote}

\section*{Criticism of existing education}

Minnaert had been interested in education from an early age. 1 His
godfather, following Erasmus, believed that upbringing should shape
people harmoniously: 'Through ignorance, one can ruin the most precious
material, while from a worthless chunk of ore, through melting, hammering,
and striving, an immortal masterpiece is wrought.' At the time, there
was already much criticism of rote learning, performance-oriented
education, the limited attention to personality development, and education
for self-reliance. Minnaert's father had summarized these shortcomings
in his comments on Marcel's teachers. From the United States came
impulses to base education on practical training, which fit well with
the 2 pioneering spirit of that country, and the socially committed
Kerschensteiner advocated for an Arbeitsschule in Germany. Many authors
emphasized the rights of the child and found 3 that educators should
guide more.

Mrs. Ehrenfest-Afanasjeva had fueled Minnaert's love for didactics.
4 She had formed a group in Leiden that debated about her 'family
tree' of geometric propositions at her home. Minnaert wrote afterward:
'This new approach to elementary mathematics immediately gripped us:
here was an interesting logical problem to think about, but here also
lay the key to simplifying the school curriculum.' A spark had jumped:
'I still remember how it initially seemed like blasphemy in my ears,when
I heard her claim that Euclid's logical method, from a psychological
perspective, should be considered very poor because the student cannot
guess at all where it is heading at the beginning. I believe we then
understood that an original idea in the field of didactics can be
just as important as a scientific discovery.

As a teacher at the Hogeschool, he had let students experiment to
increase their self-reliance. He had supervised physics education,
which some teachers at teacher training colleges had implemented.
He read publications on didactic matters and discovered that the German
educator Kerschensteiner had made a connection between gymnasium education
and physics didactics. He then wrote: 'From a didactic point of view,
it is particularly remarkable to see how much the teaching of classical
languages can be turned into aesthetic enjoyment, into living beauty.
Just in recent days, I read {*}Wesen und Wert des naturwissenschaftlichen
Unterrichts{*} and, to my delight, found that this proponent of the
activity school also fully appreciates the value of classical languages.'

Minnaert wanted to realize his didactic ideals. As a child, he had
learned to identify flowers, indulged in chemical materials, learned
to understand the coherence in the nature of landscapes, observed
the sky day and night, and was well-versed in experiments with electromagnetic
and optical tools. If anyone was ideally equipped to become a pioneer
in physics didactics, it was Minnaert.

\section*{Minnaert and the New Education}

After the World War, the movement for educational renewal gained momentum.
There was a strong emphasis on cooperation and mutual assistance among
children, on striving for peace, so that humanity would not plunge
into the abyss again. This was expressed in terms like {*}Education
in the New Era{*} and {*}Reformpädagogik{*}. The establishment of
the New Education Fellowship (NEF) marked a breakthrough. Unlike international
scientific organizations, the NEF had French, English, and German
sections. Its congresses attracted attention: the number of participants
grew from 500 in 1925 in Heidelberg to 1,200 two years later in Locarno.
The NEF wanted to stimulate independent thinking 'instead of being
carried away by mass emotions.' World peace could be preserved if
education from now on would focus on cooperation between nations.
The NEF functioned 'as a catalyst in a global innovation process in
the field of education and teaching.'

Its Dutch leaders, such as Kees Boeke, Beatrice Cadbury, Cor Bruijn,
Felix Ortt, Lodewijk van Mierop, Tatyana Ehrenfest-Afanasjeva, and
Marcel Minnaert, shared a philosophy of life in which vegetarianism,
total abstinence, non-smoking, clean living, and pacifism played a
key role. In 1913, the 'humanitarian' Engendaalschool was founded
in Soest, which was partly based on the philosophy of writer-educator
Lev Tolstoy. Minnaert had lived next to the Engendaalschool, while
Boeke and Cadbury, with their four daughters, began De Werkplaats
in 1926, near the Minnaerts' house in Bilthoven.

In 1926, a Dutch branch of the NEF was established: De Nieuwe Opvoeding.
Of the twelve Dutch NEF schools, nine were Montessorian. General pedagogical
characteristics included the child's freedom of movement, 'the leader
as helper,' self-designed 'teaching aids,' responding to 'stages of
child development,' timekeeping, and school walks. At De Werkplaats,
students were called 'workers' and teachers 'collaborators.' Many
teachers worked without pay. The children received all the attention,
which appealed to parents. Sometimes people moved house to enroll
their children in these schools, and often children came from afar
to attend. The position of the collaborators was ambiguous. On the
one hand, they had to remain reserved to enable the child's own development.
On the other hand, they had to play a guiding role if they did not
want to be overwhelmed by national chauvinisms.

At the first annual meeting in November 1927, besides Kees Boeke,
Dr. Elisabeth Rotten from the International Council was also present.
According to the report, she visited the 'training course of the association
where, among other things, she attended lessons by Dr. M.G.J. Minnaert,
which focused on Physics and student self-activity.'

In this movement, Minnaert had indeed contributed with his {*}Physics
in 11 student experiments. In September 1923, he wrote to the biologist
Cesar De Bruyker that he had completed the book: 'This morning I sent
the entire stack of papers to the publisher (Noordhoff); the 136 self-made
drawings and photos cost a lot of work, but now it’s all behind me.'
With this, he had written the first book on student experiments in
the Dutch language.

The book began with his Credo: 'There was once a time when children
learned about nature through little drawings on the blackboard or
in a book: that was physics with chalk and eraser. Then came an era
where the most important feature of good education was considered
to be its visual nature. The teacher performs an experiment for the
class, and all the children are allowed to admire it as best they
can; sometimes you see a light ray pass by or a little bell ring,
other times there’s a beautiful fire phenomenon or a popping effect.
But how these nice things come about remains a mystery to many children:
they look at this demonstration as if it were a picture---albeit
an animated one! Since Lichtart, Kerschensteiner, and Montessori,
the school’s motto has become: self-reliance! And a group of enthusiastic
men and women have proclaimed, applied, and developed this principle
with dedicated devotion. Under this sign, the physical student experiment
developed.'

Minnaert thanked his friends in Soest, who for years had stimulated
the creation of this book during Sunday gatherings: 'You gave time
and labor; with moving dedication, you bought the necessary materials
from your own pockets; in your friendly, simple Engendaalschool, it
has now been shown how one can achieve excellent physics education
even with limited means.'

He himself was not averse to manual work. Former students from 1920
could testify how Minnaert, armed with wires and batteries, taught
physics lessons and used the seesaw on the schoolyard as a balance
during explanations of mechanics.

The three classrooms were equipped for twelve students. Minnaert:
'The class I envision working like this consists of at most 20 to
25 students. With a larger number, one cannot achieve good results
- not for physics nor other subjects.'

\section*{Physics in student experiments}

The book was 'a safe guide in seeking new paths for our 14 public
education system.' It was intended 'for the highest classes of primary
school, for teacher training colleges, agricultural, domestic science,
vocational, and evening schools, and for private education,' but could
also be useful for the lower classes of lycea, gymnasiums, HBS (higher
civil service), and Flemish athenea. He addressed a heartfelt word
to 'the boys and girls who love physics, who want to supplement what
is taught at school through their own tinkering and experiments. I
hold dear these young discoverers! I know how they are mocked by their
family members and banished to the attic; how bravely they save up
to buy small devices; how often they must seek advice from outdated,
silly booklets; but also how their eyes sparkle with the joy of the
first success, which makes them forget all the effort and trouble!
I hold dear these young discoverers!'

Minnaert showed how persistent collecting and manual labor, with the
help of local carpenters and a few handy parents, could deliver a
true instrumentation of physics materials. Saving metal plates, eyeglasses,
magnets, nails, rope, coil springs, pieces of candle, small mirrors,
cigar boxes, needles, cardboard, and corks brought about a wonder:
'All these worthless things turn into the most useful tools as soon
as they find their place in the cabinet.' His experiments reveal that
test tubes, beakers, glass tubes, flasks, and rubber hoses were also
needed. They weren’t free, but the costs were indeed minimal.

Minnaert adhered to four principles for his didactics. First {[}p.
15{]}, subject teaching is only meaningful if it is connected to real
life. Extracurricular experiences are both the starting point and
the endpoint: 'The development of physical concepts in children takes
place along the same paths that humanity has followed over the centuries.
We must convince the government that physics cannot be taught with
books. If we wish to discuss living nature in lower school, we must
have elements of that nature in the classroom: plants, an aquarium,
etc. Similarly, the physics teacher must be able to demonstrate the
most important physical phenomena in the classroom.' Physics cannot
be taught with computers either, a note that can be added here after
a small century.

Secondly, unaware of Piaget's work at the time, he presented developmental
psychological arguments advocating for early science education. The
child is busy collecting a treasure trove of material for memories
and can draw from it later. The young child enjoys conducting experiments:
'I saw a six-year-old boy spend an hour investigating the smallest
slope needed on a piece of cardboard to make a pencil roll off. Two
little girls played just as patiently on a balcony with a crumpled
newspaper hung from the end of a long rope, exploring the muted swinging
and falling movements. Could most children's games be anything other
than excellent series of physics experiments?'

Thirdly, he introduced new teaching methods from England, Germany,
and America, such as simultaneous student experiments. The children
carried these out in pairs using their own equipment. Kropotkin's
vision of evolution inspired Minnaert: it stimulated cooperation and
mutual assistance. Ideals and didactics coincided: 'In this way, the
desire for collaboration grows naturally---that noble, truly human
feeling---of such invaluable importance for life and society.' The
classroom becomes a unified whole that seeks, works, and decides together!
Lucky discoveries by one student benefit everyone else, just as much
as mistakes made: 'Comparing results builds confidence and introduces
understanding and appreciation of observational errors; creating class
averages significantly improves the outcomes and gives each student
a sense of responsibility. Finally, the teacher draws the necessary
conclusions from the collective work.'

Fourthly, he connected cognitive development with affective and motor
education. The physics lesson meant much more than just formulating
a law. The subject was not confined to the brain but also involved
collaborating, sawing wood, setting up experiments, and discussing
observations with partners and the class. Minnaert: 'The child must
directly engage with nature; it not only wants to see things, to see
them well, but also to feel, smell, and hear them up close; their
'being' must become tangible! Only then is the sense of movement exercised,
the most important of our senses, and through thousands of new experiences,
we learn how our movements affect the things around us. Therefore,
every student should have their own small apparatus, which they can
build and modify according to the play of their imagination and the
demands of their common sense; an apparatus with which they can perform
measurements, so that the numbers come alive for them. In this way,
every lesson can become a delightful hour of joy in discovery.'

Minnaert introduced the home experiment: \textquotedbl Why shouldn’t
one occasionally conduct real experiments at home or observations
in the open air? There are plenty of experiments for which the necessary
tools can be found in any household; and I wouldn’t even hesitate
to lend the children a more expensive instrument (a thermometer) for
a day. Experience has proven to me that such exercises are very much
appreciated.\textquotedbl{}

\section*{A \textquotedbl joint\textquotedbl{} student experiment}

In Minnaert’s book, the subjects pass by in the classical order: mechanics
of solids, fluids, and air, heat, light, sound, and electricity. As
Van Genderen notes in retrospect, Minnaert wrote as if he were addressing
the students directly. The experiments are sometimes original and
spectacular; he must have taken inspiration from Tom Tit, for example,
with the center of gravity tricks around ‘the egg that always stands
upright’ and the iron wire with balls dressed by the children that
can easily be balanced on a finger using a nail. An example from the
book gives an idea of Minnaert’s style and approach.

\textbackslash figcaption An empty eggshell, filled with clay at
the bottom.

\textbackslash figcaption A body with a low center of gravity will
automatically stay balanced.

The teacher proposes a ‘collective student experiment’ with classroom
processing. Previous experiments have introduced the thermometer.
The children measure the temperature of air and water: \textquotedbl A
beautiful autumn day, for example, is very suitable for these measurements.
Observing the air temperature is still not too...\textquotedbl{}

\textquotedbl Simple. Determine those who are always in the shade;
make sure the thermometer is thoroughly dry, then firmly hold it by
its top end and swing it back and forth for two minutes. Only then
can you be certain that it has taken on the true temperature of the
air itself. Read off the tenth degree!

Every two hours, between classes, we step out into the delightful
fresh outdoor air and measure the temperature in the shade; we need
to observe for 24 hours! For this evening and tomorrow morning, you
will take the thermometer home with you; who plans to stay up late?
Who wants to get up early? All in the name of science! Between midnight
and 6 a.m., just leave it as it is; I myself will make an observation
at 3 a.m.

The children create a neat little table, carefully chart everything,
and become curious to know what the thermometer will indicate in the
next hour! After a day, a night, and another day, we will represent
all our numbers with a drawing. The consecutive squares on our millimeter
paper represent the hours; for each hour, draw a line indicating how
high the mercury column stood at that time. Where we have skipped
hours without observing, you should also leave squares empty. '

\textbackslash figcaption Course of air temperature over a 24-hour
period.' d = day, n = night.

The same happens with water nearby: a river, a canal, or even a rain
barrel in the shade. The difference between the two graphs allows
for the conclusion: 'Apparently, water is difficult to heat and difficult
to cool.'

The book effortlessly crossed the boundaries between subjects. The
optical topics required the biological treatment of 'the eye,' and
the electric cells needed a bit of chemistry. The discussion of the
electromagnetic bell and the Morse alphabet ends with the climax:
A visit to the telegraph office. Groundbreaking were his photos of
20 girls experimenting. A girl handled the color disc that accompanied
a story about the solar spectrum! Twenty years later, a series of
books appeared in the Netherlands with titles like {*}Boys and Physics{*};
eighty years later, a Flemish TV program was called {*}Boys and Science{*}.

In the spirit of Kerschensteiner, Minnaert casually pointed out the
‘easy-to-read’ Greek text about Heron's ball! His cardboard models
of suction pumps and steam engines were certainly inventive. As a
non-chemist, he considered the brownstone in a battery to be unnecessary
and the conductive graphite as the effective component. But contrary
to what the proponents of rapid obsolescence think, this book continues
to captivate and amaze generation after generation.

\section*{Empirical and epistemological approaches: Minnaert versus Dijksterhuis}

In 1924, Ehrenfest-Afanasjeva published her vision on geometry education
for ‘non-mathematicians’. She replaced axiomatics and memorization
with self-activity and the intuitive ability to derive propositions
from geometric images. Minnaert was not alone in his advocacy for
student experiments: prominent physicists, chemists, and astronomers
such as Ehrenfest, Kohnstamm, Pannekoek, Kruyt, Zernike, Coster, and
Fokker believed that experimentation and self-activity should play
a greater role in physics education.

In mid-1923, the Dutch Physics Association (NNV) sent an Address to
the Ministry regarding laboratory hours for teachers in physics. Teachers
were to receive additional hours due to the ‘time-consuming effort
and care required outside of class hours to give experiments their
due place in lessons’. In 1927, an educational study committee of
the NNV was established, consisting of six teachers and school principals
under the leadership of Delft professor A.D. Fokker. On April 20,
their Propositions were discussed at a well-attended meeting of members,
interested parties, and education inspectors. Various views were discussed,
such as those of Kerschensteiner who demanded that students would
independently solve problems. The commission rejected this, even for
the HBS: 'The student must be trained in making simple observations
and measurements; only after that can they be allowed to independently
solve simple problems, a phase which, however, is likely to be left
to the university.' Minnaert's revolutionary proposal to combine laboratory
and classroom from then on was rejected due to 'practical objections.'
The present physics teachers agreed that student practical work should
become part of the program. It looked promising for the practicals
in this circle.

The opposing effort was equally successful. A group of teachers, led
by mathematicians Dijksterhuis and Beth, advocated for deductive education
in mathematics and physics. Dijksterhuis distinguished between empirical
knowledge based on experience and memory and epistemological knowledge
based on the interrelationship of facts and abstractions. Epistemological
education required students to account for the terms, formulas, and
methods used and would serve as an excellent selection tool for the
university. Education should not attempt to connect with ideas and
intuitions about nature that students had acquired in daily life.

On the border between mathematics and physics, these opposing directions
fought over 29 the control of mechanics, which was still an independent
subject. Should the incorporation of this subject into epistemological
mathematics, a favorite exploration field of differential and integral
calculus, be maintained, or should it belong to physics as an experiential
science? A compromise was reached between the Beth commission and
the Fokker commission: the four hours of mechanics would be divided
among the 30 teachers of physics and mathematics. The Groningen physicist
D. Coster presented this to the members of the NNV on May 19, 1928:
'The Utrecht physicists L.S. Ornstein and M. Minnaert definitively
caused discord by advocating against the advice of the Fokker commission
for the incorporation of all mechanics.' The NNV board adopted this
principled stance: mechanics should be taught as an experimental-physical
discipline. 31

The members of the Beth commission were present. Minnaert challenged
them by saying: 'If one asks what is more important, the physical
content or the mathematical formulation, one would perhaps say that
students can best reproduce the mathematical formulation. That is
what they learn by heart. But if one asks what stays with them later,
it turns out to be precisely the physical content.' The meeting tore
up the compromise and followed the executive board with five votes
against (the Beth committee) and six blank votes (the Fokker committee).
It was a Pyrrhus victory, as Fokker had warned beforehand.

The NNV wrote an Address to the Minister, in which they refused to
agree with 'the overly large classes and the excessive number of teaching
hours for teachers.' Some formulations must have been written by Minnaert
himself. The teaching method had to take into account the main requirement
of modern education: self-reliance. Practical education had to form
a whole with the lesson. Mechanics had to be what it had always been
everywhere and at all times: a part and one of the foundations of
physics. Historical development had led to mechanics being taught
by the mathematics teacher: 'In education, as in all scientific teaching,
deduction and induction should alternate. If one emphasizes too much
the axiomatic, abstract side, one also commits the pedagogical error
of not connecting to what is closest to the students' experience.'

Minnaert criticized Dijksterhuis in an educational journal: 'The mathematicians
who want to maintain mechanics as a separate subject often argue that
mechanics rests on a completely different basis than physics and that
its statements can be derived from fundamental axioms given to us
by reason itself. Nothing is more incorrect and dangerous than this
claim! The 'principles' of mechanics are so little self-evident that
they are not even correct! It is not true that a velocity u, combined
with a similarly directed velocity v, would equal a velocity of u
+ v; we know this from the theory of relativity and Michelson's experiment.
It is also not true that the acceleration of a body is proportional
to the force acting upon it; we know this from experiments by Bucherer
and others with fast electrons. The principles of mechanics are therefore
certainly not a priori truths.'

Minnaert believed that Dijksterhuis' mindset belonged in the Middle
Ages, 'as if one could learn about nature merely by theorizing about
it in the manner of Hegel, who in his youth attempted to prove a priori
that only seven planets \textquotedbl could\textquotedbl{} exist.'
He sighed: 'It is as if one wanted to teach a foreign language by
only having students learn word lists and grammar.' His polemical
style did not become any milder. The physicist and later Nobel laureate
F. Zernike joined him. Physics education lagged behind chemistry and
biology, which are 'much more modern, much livelier, than physics.'
The lecturers and professors who participated in the discussion still
knew what they were talking about because they acted as examiners
during oral final exams.

The NNV established a second study committee under the leadership
of teachers Reindersma and Denier van der Gon, with university lecturers
H.C. Burger, Zernike, and Minnaert. Their report was published in
1936 under the title Physics Experiments for Students. It was a collection
of 60 experiments with a separate booklet for students. A year later,
a second booklet with 69 experiments appeared. The texts were distant,
and the drawings looked too polished. Teachers could consult the professional
journal Faraday, which would focus on experimentation.

\section*{The Utrecht teacher training college}

The battle of pens had been unusually sharp thanks to Minnaert. The
victory was entirely for the mathematicians and traditionally oriented
physics teachers. The Beth faction represented an overwhelming majority
of the teachers. Thus, the critics got nothing instead of two hours
of mechanics. Minnaert moved forward.

Just as in Flanders in the spring of 1914, when he still expected
salvation only from fresh forces willing to overturn relationships,
he now expected nothing more from the teachers in existing education.
In the late 1920s, he took over the unpaid lectures on the didactics
and pedagogy of physics in Utrecht from the thermodynamicist Ph. Kohnstamm,
who was appointed as an educator in Amsterdam. In 1930, the faculty
recommended Minnaert to the Minister as a private lecturer in the
didactics and methodology of natural sciences. Both the addition of
‘methodology’ and the omission of ‘pedagogy’ were Minnaert’s doing.
According to U. Keller’s lecture notes, Minnaert began his lecture
with the statement: ‘Teachers form a conservative mass that hinders
the development of education.’ This was his conclusion from his defeat
against Dijksterhuis and others. Such an elitist attitude toward professionals,
despite all good intentions, would not have been conducive to creating
support for innovations.

During Ornstein’s rectorship in the early 1930s, a collaboration emerged
in Utrecht between the faculty of letters and philosophy and the faculty
of mathematics, natural sciences, and astronomy to establish a teacher
training program. At Minnaert’s instigation, by the mid-1930s, a ‘didactics
for teachers’ program was functioning, which consisted of a lecture
on ‘didactics and methodology’ in the chosen natural science and a
pedagogy course. Students were required to intern for three months
during two school periods per week, with practical training taking
center stage. A committee was established comprising professors of
mathematics, physics, chemistry, zoology, and botany, with the didactics
lecturers having an advisory role. The committee’s chair and two student
representatives formed a ‘bureau’ that administered the internships.
This was pioneering work. The involvement of students betrayed Minnaert’s
signature style.

His engaging didactics lectures attracted many students. He emphasized
the formative value of the subject: conveying attitudes and norms
such as ‘the value of truth,’ ‘trust in reason,’ ‘respect for causality,’
‘respect for labor,’ ‘sense of responsibility’ in student experiments,
‘skill in handling materials,’ and ‘sense of beauty.’ He warned that
transferring these values to other areas of life was not automatic.

The bulk of his lectures consisted of demonstrations, leading student
practicals, guiding didactically profitable teaching methods, analyzing
and structuring lessons. He taught experimenting in nature: ‘The laws
of physics apply outside the school as well!’ He discussed books and
professional journals, explored connections with mathematics, and
examined border areas with chemistry, astronomy, mechanics, and meteorology,
and introduced historical and philosophical works.

However, he did not refer to the general pedagogy course of the educator
M.J. Langeveld. His dismissive attitude toward the humanities had
thus not changed. On the other hand, Langeveld did refer to Minnaert.
The 39-year-old astronomer H.C. van der Hulst recalled that Langeveld
used examples from the work of psychologist Ludwig Klages when discussing
teacher types: 'Such a type was characterized by this Wille zur Verständlichkeit
der Welt. He said: \textquotedbl That's Minnaert\textquotedbl :
that absolute desire to understand the world and rejection of anything
that hints at the mystical.\textquotedbl{}

Ornstein and Minnaert had collaborated on establishing the Utrecht
teacher training program. Ornstein wrote proudly afterward: 'For years,
plans and laws followed one another, but the strong life that our
University fortunately possesses brought forth this new branch, and
with the valued help of teachers from secondary and higher education,
this training has been established and will grow for the benefit of
our students and the youth in schools.' The magazine Vernieuwing dedicated
a theme issue to the program and gave Minnaert the floor.

He argued that the university-educated teacher could not agree with
traditional school science but instead engaged in the didactic translation
of new insights: 'He feels inspired by the zeal of the reformer, the
life before him becomes rich in content and worth devoting himself
to.' Visiting the best practice schools was meant to confront future
teachers with the 'unforgettable impression' of modern education.
The influx of new people had to break through the routine of the old
guard: 'I wish that he will see this ideal image for himself, even
before starting his career; as a sunny gift from his youth, he will
receive it, in the period of his life when he is most receptive to
all ideals. Lack of inspiration is more dangerous than inexperience!'
Finally, according to Minnaert, true inspiration came from the didactics
of the subject: 'One captivates students the most by directly tackling
the central, practical problem: \textquotedbl How will I convey my
science to the pupils?\textquotedbl{} From there, we must elevate
ourselves to the more general pedagogical and psychological questions.
This path best protects us from abstract theorizing, which has so
discredited the older, more straightforward 'pedagogy.' Every sentence
expressed Minnaert's disdain for the existing education system, where
Dijksterhuizen's epistemological approach remained dominant.\\

Endnotes:

1 Gillis D. Minnaert, 1913-1914, I, 56.

2 Romein, 1967, 808. Ellen Key plays a role again.

3 Kruithof, 1982, outlines the principles of Pestalozzi, Fröbel, Kerschensteiner,
Montessori, and Dewey.

4 Lecture on University and Didactics from October 14, 1961. Astronomy
Archive.

5 Minnaert to Burgers, June 17, 1921. In the 1950s, he revisited this
topic in a radio debate with the classicist C. Spoelder: he then argued
that classical education overburdened students unnecessarily and stood
too far from 'the full life.' He advocated for a school 'based on
natural science and new art,' which could represent a higher stage
of classical education. AVRO, March 1951; discussion leader P.H. Ritter
Jr sided with C. Spoelder, who defended gymnasial education.

6 Morsch, 1984.

7 Bruijn, 1984, 95, 104.

8 Interview with Hans Littooij, son of the administrator of Chreestarchia.

9 Report in Morsch, 1984.

10 Minnaert, 1924.

11 Letter to De Bruyker, September 18, 1923. AMVC Archive. 12 Minnaert,
1924, 1.

13 Interviews with Hans Littooij and Nanda Ortt.

14 Minnaert, 1924, Foreword.

15 Quotes stitched together from the introductory pages.

16 Before the Swiss J. Piaget, Minnaert used the notion of developmental
psychology. His work began circulating in the late 1920s. It is questionable
whether Minnaert, with his reservations about the humanities, ever
read pedagogical standard works like {*}The Language and Thought of
the Child{*} (1923) and {*}The Origins of Intelligence in Children{*}
(1936).

17 Van Genderen, 1994.

18 Minnaert, 1924, 33.

19 Minnaert, 1924, 83.

20 Minnaert, 1924, 81,34,137. There are three photos with girls and
one with boys.

20 Minnaert, 1924, 63.

21 Minnaert, 1924, 73, 104.

23 Minnaert, 1924, 169.

24 A first review by J.M. Telders in the Dalton issue of Volksonderwijs,
June 12, 1924.

25 T. Ehrenfest-Afanasjeva, 1924. Klomp, 1997, 166.

26 Address from 1923 to the Minister of Education, Arts, and Sciences,
signed by J.M. Burgers and P.H. van Cittert.

27 Klomp, 1997, 186.

28 Minnaert on April 20, 1927, at a meeting of NNV. Report in Physica.

29 Klomp, 1997, 180-217, D. The great struggle over mechanics.

30 Klomp, 1997, 187. Dirk Coster was indeed Minnaert's friend from
Christiaan Huygens.

31 Report in Physica 8, 1928, 173.

32 Fokker had written to his colleague Ornstein in a letter of June
12, 1928, that Ornstein was right. He himself could 'not demand more
where I have accepted less.' However, he warned against 'general resistance
from teachers who feel threatened themselves and their subject.' Klomp,
1997, 255.

33 Address from NNV to the Minister, July 11, 1928, Physica 8, 1928,
183.

34 Minnaert, (1928d). Berkel, 1997, 142-150. My emphasis. Minnaert
and Dijksterhuis shared a common concern for teacher training: they
were not always at odds with each other. Dijksterhuis had delivered
a lecture on teaching at the Dutch Natural and Medical Sciences Congress
in 1925, which was published in full in the Proceedings.

35 Zernike, F., The old and new mechanics education, Weekblad 25,
1928-1929, 901.

36 Lecture notes by U. Keller. University Museum Archive.

37 Berkel, 1997, 135.

38 Heijmans, 1994, 133. Ornstein was rector in 1931-1932 when this
initiative began.

39 Interview with H.C. van der Hulst.

40 Ornstein, lecture on teacher training in Utrecht, 1939. 41 Minnaert,
(1941b).\textquotedbl{}

\chapter{The Fist of Moscow in Lage Vuursche}
\begin{quote}
'We no longer surround with prestige the somber pages in the history
of our people, the pages that tell of war murders, barbarism, and
violence.'
\end{quote}

\section*{The commemorative book 1830-1930}

Minnaert continued to commit himself to the fullest extent possible
to the liberation of Flanders. On the occasion of the centenary of
Belgium's existence, De Dietse Bond had taken the initiative for the
anthology A Century of Injustice and Oppression. 'Against the cynical
celebration full of champagne of the Belgian state, a dignified protest
from Flanders.'

Minnaert took charge of two topics: Education in Flanders 1 and Science
in Flanders. The Belgian state continued to deny the Flemish people
what they needed to develop as a modern nation: education and science
in Dutch. Comparisons with Wallonia, the Netherlands, and 'other Germanic
countries' invariably turned out to the disadvantage of Flanders.
The hypocritical 'respect for minorities' had resulted in numerous
French classes for Walloons and Francophiles: 'every French school,
every French class in Flanders will remain a center of denationalization
and must therefore disappear.' After fifty years of struggle, the
language law of 1883 had yielded an average of eight hours of Dutch
per week; 20\% of teaching time. When, in the 1920s, some lectures
were given in Dutch at the University of Ghent, the Francophiles had
established an École des Hautes Études where the same professors repeated
their lectures in French. As late as October 1929, a Royal Decree
had established that the rector of the University of Ghent did not
need to know Dutch. Although the law stipulated that 'the Flemish
language is the administrative language of the University,' all faculty
meetings were still held in French in 1930.

Moreover, all higher technical education remained in French. Lodewijk
De Raet had written: 'Now that Belgium's industrial center of gravity
has shifted to Flanders, good technical education becomes crucial
for us.' A quarter of a century later, Minnaert noted that nothing
had changed! The recent decision regarding the Dutchification of Ghent
was immediately undermined. The Dutchification of technical schools
would only be allowed to begin in five years, and professors could
continue teaching at the École des Hautes Études: 'We are also disillusioned
because we now discover how much more needs to be done to make all
higher education in Flanders Dutch.' Flemish-national legislation
would not tolerate the continuation of this French education, he threatened.

In his explanation of Flemish science, the comparison between France
and the Netherlands played an important role: 'Every Fleming who has
come into contact with Dutch science has known the wonderful feeling
of liberation from the one-sidedness of French civilization.' Minnaert
praised the connection of Dutch scholars to the world's top. He commended
the many Nobel Prizes, the work of the astronomer Kapteyn and the
biologist De Vries, who were 'world-famous,' and 'the Philips factories'
light bulbs where science and technology go hand in hand to an extent
that is hardly found even in the best American factory laboratories.'
Flemish science was held in lower regard: 'This is evident from the
quality of the publications; rarely does a Belgian article appear
in foreign journals.' Students worked more scholastically and did
less research: 'Our backwardness is the faithful reflection of France's
decline as a scientific nation since the beginning of the 19th century.'

The Belgian Academies had banned Dutch for a century and abused their
authority against Flanders. Now that Flemings were demanding their
own academies for medicine and natural sciences, those institutions
suddenly declared that Dutch could be freely used there: 'Not being
disturbed by this and continuing the constructive work quietly within
entirely Flemish organizations is the only method to command respect.'
Flemish newspapers still showed an 'enormous backwardness' in articles
about science. The dominance of French had prevented the establishment
of journals such as {*}De Natuur{*} and {*}Hemel en Dampkring{*} ({*}Sky
and Circle of Air{*}).

He only provided a positive turn at the end. The turning point had
been the collaboration between the Netherlands and Flanders regarding
the Flemish University: 'No Belgian government has appointed Dutch
scholars to the University of Ghent, even when it was 'Flemishized':
the Dutchification had to happen in a Belgian way! The spirit could
not be Dutch!' The Belgian government would not have been charmed
by Minnaert's specific accusations.

\section*{Antimilitarist nationalist}

In the late 1920s, Italian fascism and the rise of German National
Socialism forced every nationalist to make a choice. Minnaert expressed
his stance at the Van Speyk commemoration organized by the Vlaams-Hollandsche
Vereniging in Utrecht. This event was intended as a protest against
the positive reactions in the Netherlands to the celebration of 100
years of Belgium. The exiles reminded everyone that this secession
had been an act of mutiny, that many Dutch soldiers had fallen, and
that Commander Van Speyk in 1831 would rather have blown himself,
his crew, the attackers, and the ship to pieces than surrender.

Minnaert, however, wanted to view Van Speyk's action from the perspective
of antimilitarist nationalists: He distanced himself from authoritarian
currents: 'A \textquotedbl nationalist\textquotedbl{} in our terminology
does not mean a \textquotedbl chauvinist,\textquotedbl{} who wants
to conquer, who places the so-called holy self-interest of one’s own
people above all else; it also does not mean \textquotedbl fascist\textquotedbl .'
In Minnaert's view, there was no necessary connection between nationalism
and militarism. Because patriotism was nonetheless being misused to
whip people into a war frenzy, 'there was every reason for nationalists
like us to counter this looming danger by being not only nationalists
but also antimilitarists.'

Van Speyk had remained loyal to his prince and to the honor of his
soldiers. Minnaert’s idea of national ‘honor’ had nothing to do with
that: 'I will tell you when a country's honor is tarnished: that is
when people start flogging the Congolese and cutting off their hands
if they don’t produce enough rubber. But these stains cannot be washed
away by blowing up a ship!' Minnaert couldn’t even find Van Speyk’s
suicide action ‘brave’: 'We find heroism in the resistance of the
mild, the clear and rational, of love for humanity against bestiality.
A hero, to us, is one who, despite all dangers, whims, and fancies
of fate, and despite the smallness and meanness of his opponents,
remains true to his ideal of freedom, truth, and purity (...). We
defend our people against the greatest danger: the obligation to kill
others. Nothing outweighs that, nothing outweighs the sanctity of
life.'

Minnaert had already heard so many war slogans: Against British dominance
at sea; Against German militarism; For God, Prince, and Fatherland---they
were all hollow phrases. After the World War, American President Wilson
had determined that it had essentially been a trade and industrial
war. Minnaert likely had the movement of the then-popular Gandhi in
mind when he wrote: 'A resilient people are those who do not kneel
down in worship of violence, but who listen to their prophets and
artists speaking of a noble past and a glorious future: a people tainted
by a desire for justice, armed with the consciousness of their solidarity,
striving for the disappearance of human exploitation, morally upright
and living happily. Such a people are invulnerable to foreign violence
in our modern times. (...) Therefore, we must stop gilding the iron.
We no longer surround with prestige the somber pages in the history
of our people, the pages that tell stories of war murders, barbarism,
and violence.'

He put his words into action. As an assistant to Ornstein, he was
assigned tasks for the Foundation for Lighting Science. A captain
from the Corps of Engineers had requested an investigation into searchlight
beams, which took place in November 1930. Minnaert began his article
'Measurements on Searchlights with Simple Means' with a poetic reference
to the Iliad: 'The magical interplay of searchlight beams on a dark
night, under the high starry vault; with delicate color nuances of
yellow and light purple, sometimes like fingers pointing skyward in
the distance, then suddenly as powerful arcs of light stretching across
the entire sky...' He demonstrated that at low altitudes above the
ground, Rayleigh's scattering law could not be applied: the scattering
turned out to be 2.5 times greater than predicted, which Minnaert
attributed to polluted air. Yet, the purpose of the research was less
idyllic than it seemed.

He presented his friend Burgers with a moral dilemma: 'I refuse to
contribute to military purposes under any circumstances; however,
to what extent can such a conviction conflict with my duty as an official?'
Burgers' response is unknown. Minnaert refused to continue the research:
it was completed by two colleagues. Ornstein assigned him other tasks
in this field, given his publication on light scattering through milk
glass and his final editing of a standard book on Lighting Science.
He had meanwhile become a staunch advocate for an international language.
With his cycling companion on the Bilthoven-Utrecht route, economist
W.P. Roelofs he discussed grammatical issues. Esperanto is economical,
rational, and promotes international cooperation. The book on Enlightenment
studies appeared in five languages as well as in Esperanto.

Did Minnaert believe that an end should indeed be put to the belligerent
language of Flemish nationalism? Yes and no. For just as in 1916,
alongside his pacifist and internationalist arguments, he also maintained
a different narrative.

\section*{The man of the pure line.}

In 1930, the Flemish-nationalist De Noorderklok wrote about Minnaert:
'In our struggle, he was always a rock when it came to purity of line,
seriousness in striving, and persistence in work. He is not the man
of the popular movement; he is the young fellow who can sweep the
studying youth off their feet with his calm, icy-cold, but relentless
logic and extensive knowledge.'

Minnaert was apparently a member of the new Council of Flanders. On
May 9, 1932, he wrote to De Vreese: 'The Council of Flanders has appointed
a committee for the study of educational legislation; this committee
consists of subcommittees for primary, secondary, higher, and technical
education. The higher education subcommittee includes Prof. Dr. R.
Speleers, Prof. Dr. De Groodt, Mr. R. Van Genechten, Dr. A. De Waele
(Ghent, secretary), and Dr. M. Minnaert (chairman).' The letter was
confidential because some in Belgium could suffer 'the most unpleasant
consequences' from its disclosure. This Council rejected anything
that hinted at federalism and therefore could not play a conciliatory
role.

The 1932 IJzerbedevaart pilgrimage attracted as many as 200,000 visitors,
all advocating for full amnesty and a Free Flanders. All Flemish members
of parliament supported the Amnesty Law that year: the activists had
adopted a different attitude than the passivists, but they had still
been good Flemings. After Hitler's rise to power in early 1933, Flemish
nationalists and Greater Netherlands supporters were faced with the
concrete question of whether they once again expected salvation from
the Germans? In Belgium, in October 1933, the Vlaams Nationaal Verbond
led by Staf De Clercq was founded, which cultivated authoritarian
leadership and imitated fascist displays. In the Netherlands, the
National Socialist Movement (NSB) under Anton Mussert seemed to be
the only party embracing the Greater Netherlands ideal. Flemish allies
of Minnaert, such as jurists Anton Van Vessem and Robert Van Genechten,
held prominent positions in the NSB. Initially, the NSB did not seem
enthusiastic about antisemitism.

At its General Assembly on June 3, 1933, De Dietse Bond prioritized
the political union of the Netherlands and Flanders. Earlier, members
had been free to emphasize either the cultural or political aspect
of their Greater Dutch identity. The question arises as to why the
leadership believed this tightening was opportune. Minnaert's unwavering
stance that year could be read in {*}De Dietse Voorpost{*}, the magazine
of Roza De Guchtenaere, which carried the slogan {*}Delenda Belgica,
Neerlandia Una{*}: 'Therefore, it is important to clearly and purely
define the principal goal of our national struggle, without being
confused by temporary, accidental circumstances. Only in a Dutch state
form will Flanders ever be able to become itself. Bilingual states
are a constant cause of cultural decay and international friction;
they disappear one after another, and soon we will forget that they
ever existed.'

This had already been stated in Minnaert's 1916 brochure. How could
this irreconcilable stance be reconciled with his pacifist speech?
His militant rhetoric and his peaceful ideal of international brotherhood
still stood apart from each other. The dominant influence of Hitler's
National Socialism made a connection between the two unavoidable.
The moment and place of that short circuit can be precisely determined
in Minnaert’s case.

\section*{Minnaert barred from Belgium}

The Extinsion Law of 1929 had exempted him from prosecution; in 1932,
he naturalized as a Dutch citizen. He could visit his family and friends,
such as the Mahy's, as often as he wanted. According to Rudolf Mahy,
there were many points of contact between Minnaert and his father
Gaston: 'My father admired the idealism of {*}New Land Under the Plow{*}
by Sholokhov. The World Library published numerous Russian editions.
That was mandatory reading for us: \textquotedbl There they build
in a peace-loving way.\textquotedbl{} Father received a book from
Rabindranath Tagore from Minnaert at the time, which he read aloud
at the table. Gandhi's politics: that was their perspective: \textquotedbl Look
what that little man achieved without shouting like those of our Eastern
neighbors with their bellowing.\textquotedbl{} His father was also
friends with the graphic artist Frans Masereel, the Ghent pacifist
with whom Minnaert must have felt a deep affinity. This memory from
young Mahy places Minnaert’s ideal of nonviolence within a political
perspective shared by organizations like Church and Peace.

The Dietse Landdagen could also be held in Flanders starting from
1931. That year, the gathering in Ghent had taken place without incidents.
At the 1932 Landdag in Utrecht, Minnaert organized an evening program
on Modern Dutch Film Art. The critic A. van Domburg showed the films
{*}Zuiderzee{*} and {*}Heien{*} by Joris Ivens, as well as {*}Pierement{*}
by J. Theunissen, at the Arts and Sciences building. During the 1933
Landdag in Mechelen, the Belgian Security Service, without giving
any reason, expelled three Dutch nationals across the border: writer
Anton Coolen, historian Geyl, and Catholic MP Dr. H. Moller. The Belgian
press targeted Minnaert. {*}The Gazette of Ghent{*} wrote: 'Under
this fine company was a former Gentenaar and Belgian, one Marcel Minnaert,
for whom his father, a good liberal Fleming and leader of the Willemsfonds,
would blush... He obtained Dutch naturalization for services proven
during the war and thus became Dutch. Holland is not averse to accepting
such condemned traitors who betrayed their country when, as students,
they annually cost Belgium thousands, as citizens. Thank you...'

The Antwerp Gazette noted that 'a certain Minnaert' gave a speech,
'condemned to death in Belgium after the war for political offenses.'
These Flemish papers were unforgiving and took little care with the
truth. At the 1934 Landdag in Tilburg, there were 250 participants,
and the Flemings received a speaking ban from the Dutch government.
Minnaert discussed 'the special task of youth in the Dietse movement.'
Poems by the late René De Clercq were recited.

During the 1935 Landdag in Mechelen, Minnaert was expelled from Belgium
without specific cause; he received prohibition number 22, 'never
to cross the border of this state again': 'A Flemish-Dutchman can
scarcely receive greater honor,' De Dietse Gedachte believed.

\section*{The monument for René De Clercq}

On June 12, 1932, the Flemish poet René De Clercq was buried in Lage
Vuursche near Utrecht. Two years later, a committee for a monument
was established; a bank account was opened, and an advisory committee
of 125 people was formed. The Amsterdam physician H. Burger, chairman
of De Dietse Bond, chaired the Committee; Borms was honorary chairman,
and Minnaert was an ordinary member. The unveiling was scheduled for
Saturday, September 19, 1936. Naturally, the organization of this
event was in Minnaert's hands. The poet would receive a sculpture
on his grave from his friend Jozef Cantré.

Minnaert had received several signals indicating that difficulties
were looming. For example, Burger had addressed his 'dear friend Minnaert'
with guidelines on July 20: 'I live in fear of what Van Vessem will
prepare for us. It would not surprise me if there would be many people
on the Lage Vuursche cemetary, who would stand with their hand in
the air next to the Flemish Lion'

On 11 September, Minnaert received a letter from Boudewijn Maes in
Sint-Martens-Latem. In late April, Maes had spoken with Ria De Clercq
and several friends in Amsterdam: 'There, I first learned that certain
individuals were maneuvering to seize control of the unveiling of
René's memorial, intending to turn the ceremony into a celebration
of fascism or National Socialism.' Although Minnaert was aware of
Van Vessem and Van Genechten's fascist convictions and the pro-German
sympathies of many old comrades, he did not heed these warnings.

That afternoon, wreaths were laid by DDB, ANV, Raad van Vlaanderen,
the Grijze Brigade of the Vlaams Nationaal Verbond, 27 South African
groups, NSB, Nationale Jeugdstorm, and Zwart Front, all mindful of
the truce. Burger explained the symbolism of Cantré's sculpture: 'De
Clercq rising in the garden of Dietsland, with his face turned to
the sun, pressing The Noodhoorn against his heart, while his left
hand holds fast to the mother earth, Dietsland.' A choir from Utrecht
sang De Clercq's songs set to music by composer Jef Van Hoof. The
highlight was Magda De Groodt reciting several verses, including Belijdenis:
\begin{verse}
'Netherlands, my land has come.

Ensure that Flanders does not perish.'
\end{verse}
Uniformed youth stormtroopers then marched around the cemetery. During
the singing of the national anthems, the majority of those present
gave the fascist salute. In protest, Jozef Cantré and Minnaert raised
a clenched fist during the Wilhelmus and De Vlaamse Leeuw, as reported
by the social-democratic newspaper Het Volk, which added: 'In the
name of René De Clercq's memory, we strongly protest against the outrageous
imposition of these fascist enemies of freedom, and we seriously reproach
the committee for failing to prevent this desecration of De Clercq's
grave.' The paper called the Flemish folk poet De Clercq a social
libertarian who wanted nothing to do with fascism.

\section*{Minnaert's 'fist of Moscow}

During the meal, the National Socialists repeated their display, and
once again, the now isolated Minnaert raised his fist. The fascists
threatened him, and eventually, Minnaert left the dining hall. The
NSB newspaper {*}Volk en Vaderland{*} criticized Het Volk's commentary:
intimate friends of De Clercq, such as Borms, Jacob, and Van Vessem,
were in the front row and, along with hundreds of others, gave the
Diets greeting 'in the spirit of the poet.' 'We do not know if Dr.
Minnaert and Jozef Cantré had the shameless intention---entirely
without the knowledge of the family and the committee---to misuse
this commemoration for a manifestation of red-political sentiment;
in any case, Het Volk---perhaps out of regret over the failure of
such an unlawful annexation---takes the opportunity to also strike
a Diets tone, to expand its clientele.'

Chairman Burger, on the other hand, approved Minnaert's actions: 'His
clenched fist was certainly not directed against the princess or against
the Wilhelmus, but rather against the fascist greeting of his neighbors.
It should be said that this man is not a communist, but rather an
uncompromising idealist, with a deep-seated hatred of fascism.' Burger
believed Minnaert was in the right. Demonstrations for fascism, communism,
or any other 'ism' were inappropriate: 'However, based on our political
beliefs, if we cannot refrain from such demonstrations, justice and
tolerance demand that we also allow others to express their views
in an equally silent manner.' On October 2, Burger refused to accept
Minnaert's resignation as a member of the Bond: 'For De Dietse Bond,
Van Vessem and Van Genechten have resigned as members. Your withdrawal
from the Diets movement must under no circumstances occur. It is urgent
that alongside the fascists, a neutral Diets movement continues to
exist. None of us can be absent from this.'

At the end of 1936, De Dietse Bond was unpleasantly surprised by political
maneuvers in Belgium, where the Flemish-nationalist VNV of Staf De
Clercq had entered into a collaboration with the Walloon-fascist Rex
of Léon Degrelle. This was a shock to many Groot-Nederlanders. At
the same time, the NSB completely rejected any Diets sentiment. Under
the influence of the newly emerged ideologist Rost van Tonningen,
Hitler's policies had become the guiding principle for its actions.
Promoting Diets ideology could create a barrier to the acceptance
of Hitler's National Socialism. The NSB had to cover up this ideological
reversal, which it did by using Minnaert as a scapegoat, who, upon
closer inspection, had proven to be an ideal candidate for this role.

On October 16, the NSB newspaper Volk en Vaderland cheered: 'Flanders
against Moscow; Agreement between Rex and VNV; Peace between Flanders
and Wallonia.' On October 23, De vuist van Moskou focused on Minnaert's
performance in Lage Vuursche: 'The display of the fist of Moscow within
De Dietse Bond has taught us much and made many things clear to us.'
By November 13, the editorial board had added headlines such as In
the Grip of Moscow; De Dietse Bond Degenerated into a Front Organization.
According to the newspaper, 'Jewish Marxists' in 'our association'
suddenly played a leading role: 'Dimitrov's tactics triumphed over
vague nationalism and spineless intelligentsia, diverting this nationalism
toward Moscow.' The article directly attacked 'Dr. Minnaert, the Belgian
communist, whose unfortunate naturalization granted him Dutch citizenship.'

Burger mocked in De Dietse Gedachte: 'Poor Mr. Van Vessem! For years,
he sat harmoniously alongside a dangerous communist, Jewish Marxists,
and a group of spineless invalids within De Dietse Bond. Miraculously,
he escaped unharmed. Yes, thank God, the NSB opened his eyes and clearly
showed him that all his former comrades were dangerous individuals
and that his Dietse Bond had become a helpless tool in the hands of
Moscow. Take care!' Of course, Minnaert was no 'communist cell builder.'
Why didn't Van Vessem want to see that? 'Or has the cry Against Moscow,
like for so many Flemings, pushed all attention for our Greater Dutch
cause into the background?'

Minnaert also wrote 'A letter' denying any party affiliation: 'When
I raised my fist against the fascist demonstration at the René De
Clercq tribute, it was solely an anti-fascist gesture.' The impulsive
Minnaert had unequivocally chosen internationalism and anti-fascism.
The NSB exploited him to mask its own ideological flight. His name
was dragged through the streets for two months. He wrote in his diary
on page 242: 'The astronomy lectures are temporarily being given by
Van der Bilt alone, the future looks bleak for me.' Indeed, an investigation
had been launched into his political views.

\section*{Suspension due to communism cancelled.}

On November 5, 1936, the Board of Curators received a letter from
the Minister of Education, 'Cabinet Division,' urging the immediate
suspension of Minnaert's lectures and the initiation of a political
background investigation. In response, Secretary De Geer contacted
the Utrecht Police Commissioner (HC) 'to obtain more information about
the political inclinations of the individual concerned, if possible.'
The HC consulted his inspectors and informed De Geer by telephone.
He wrote a Note for the Board: 'At the time, it was rumored that M.
had sympathy for the Communist Party, after which attention was paid
to M. No evidence was ever obtained. There has never been anything
indicating this. M. does not attend Communist meetings. It is difficult
to gather information from the professors. Then came the incident
at the dinner. Prof. Burger wrote the article in De Dietse Gedachte
about it. For the Foreign Service, what was stated there is not implausible.'
The Police Commissioner warned against rumors often spread by people
with ulterior motives: 'In HC's opinion, there are too few grounds
to hinder M.'s career. The case is much too weak for that. Possibly,
the NSB would protest against admission.'

That 'admission' referred to Minnaert's request to make his unpaid
position as a private lecturer in the didactics and methodology of
physics, which he had been doing for years without pay, official.
He had written his Address on November 26, immediately after his suspension.
Probably, Minnaert wanted to force a decision about his future. On
December 7, the Curators received an extremely positive recommendation
from the faculty board. The College, in turn, fully supported Minnaert
and wrote to the minister: 'Since no objections to admission can be
raised otherwise, we have the honor of proposing a favorable decision
to Your Excellency in this sense, that the applicant is admitted as
a private lecturer in the didactics and methodology of physics, in
accordance with the statement in the faculty's official message.'

Minnaert's request seems to have accelerated the resolution. Minister
J.R. Slotemaker de Bruïne decided on February 10, 1937, in line with
the recommendation and personally informed Minnaert about it on February
19. The uncertain situation had lasted three months.

In 1936, after the death of astronomy professor A.A. Nijland, Minnaert
had become eligible for a lectureship in astronomy. At Ornstein's
request, he and Van der Bilt took over Nijland's lessons. Ornstein
explicitly told him that he was eligible for it. The suspension thwarted
this promotion.

In that anxious November, Minnaert also started a project that would
bring him great fame.\\

Endnotes:

1 Also published in DDG 4, numbers 10, 11, and 12. Minnaert, M., Het
Onderwijs in Vlaanderen, 57, and De Wetenschap in Vlaanderen, 67.

2 In De Dietse Gedachte, a controversy had just been fought out with
Ph. Kohnstamm, also a pioneer in physics didactics. Kohnstamm wanted
to introduce French as the second language in primary school. This
alienated Minnaert greatly from Kohnstamm. E. Besse, 3, 33, 49, criticized
the 'French-mad spirit' of the 'Nutsrapport'.

3 In his brochure on Pierre Curie, Minnaert criticized the state of
French natural sciences at the beginning of the 20th century by unfavorably
comparing it to her pioneering role of a century earlier. Curie was
then the exception.

4 The Dutch-language Academies would emerge in the course of 1938.

5 Minnaert, Van Speyk and his deed, seen from the perspective of anti-military
nationalists, DDG 5, 1930-1931, 157.

6 Romain Rolland, Mahatma Gandhi, Amsterdam 1930. Besides Gandhi,
his counterpart Rabindranath Tagore, whom Minnaert also admired, is
discussed.

7 Heijmans, 1994, 117.

8 Minnaert, (1931c).

9 Minnaert to Burgers, December 2, 1930. Heijmans, 1994, 127.

10 Minnaert, (1935a).

11 Standard work on Lighting studies, Utrecht 1936.

12 Interview with W.P. Roelofs.

13 Flor, likely a pseudonym of Jan Wannyn, A second Pro Domo for the
Minnaert family in: De Noorderklok of May 25, 1930.

14 According to DDG, which mentions the intention on February 11,
1931, the Council was established on March 15.

15 Letters from Minnaert to De Vreese, among others from May 9, June
8, and December 18, 1932. At the time, De Vreese built a unique collection
of and about Erasmus as a Rotterdam librarian. Dousa Archive RUL.

16 An Amnesty Law was only introduced on June 13, 1937: the confiscated
goods were not returned and the functions in social life were not
restored. The activists did get their civil rights back! Those sentenced
to death, such as Borms, could not be made eligible. The material
aspect of the law brought renewed bitterness among activists. Someone
like Minnaert could work in Belgium.

17 The literary figure G. Knuvelder still discussed this and was reprimanded
by the editorial board. DDG 11, 1936, 49.

18 Minnaert, October 1934. The Latin motto: 'Belgium must be destroyed
so that the Netherlands can become one!'

19 Molenaar, 1994, 20.

20 DDG 6, 1931-1932, 192. This detail is included because Minnaert
apparently had a soft spot for experimental sound films. Many film
enthusiasts at the time swore by the artistically superior silent
films and strongly opposed the introduction of sound, as shown in
the early years of Menno ter Braak's Filmliga.

21 DDG 8, 1933-1934, 66.

22 DDG 12, 1937, 96.

23 Committee for René De Clerq. He occupied a country house in Hollandse
Rading belonging to his friend Anton Pieck, DDG 7, 1932-1933, 1-3.

24 Call for September 19, 1936, in DDG 11, number 7.

25 H. Burger to Minnaert, July 20, 1936.

26 B. Maes to Minnaert, September 11, 1936.

27 Report in DDG 11, number 11.

28 The Noodhoorn was a collection of Patriotic Songs by De Clercq,
Amsterdam 1927 (second edition), which wrote: 'In holy admiration,
I dedicate The Noodhoorn to Dr. August Borms, Flanders' greatest man.'
De Clercq's first poem about Borms dates from 1917. The poet had passed
away before the division of spirits within Flemish nationalism.

29 Het Volk, September 21, 1936. 30 Volk en Vaderland, September 21,
1936. 31 Burger, H., DDG, number 11.

32 Burger to Minnaert, October 2, 1936.

33 Etten, H.W. van, The Germans and the Dietse striving, typescript
RIOD Amsterdam.

34 Volk en Vaderland, October 16, 1936.

35 Volk en Vaderland, October 23, 1936.

36 Volk en Vaderland, November 13, 1936.

37 The Bulgarian communist Dimitrov was acquitted of the Reichstag
arson and became the spokesperson for the Communist International
(Comintern) in 1935 regarding the united front against fascism. Communists
had to isolate and combat fascists in all 'mass organizations.' Volk
en Vaderland thus claimed that Minnaert had applied this tactic within
De Dietse Bond.

38 Burger in DDG, The NSB against Groot-Nederland, 11, 1936, 113.

39 Minnaert, in DDG 12, 1936. 40 Minnaert, Diary, winter 1936.

41 A report from Secretary De Geer to the Board of Curators followed,
which complied with the minister's request. Archive-Curatoren in Het
Utrechts Archief.

42 Note from De Geer dated January 11, 1937. What mattered was only
whether he was a 'communist.'

43 Minnaert's address is from November 26, 1936. On November 16, he
had written to publisher Thieme about The Physics of the Free Field.

\chapter{The Physics of the Free Field.}
\begin{quote}
Who loves nature observes its phenomena as naturally as they breathe
and live; driven by an innate, deep urge.
\end{quote}

\section*{On the way to a classic book}

\textbackslash figcaption Minnaert with a halo around his head on
dewy grass: a clear light radiating around the shadow of the observer's
head, holding his camera close to it.

\textbackslash figcaption Experimental setup for sounds of splashing
water

In the 1920s, Minnaert had begun collecting, cataloging, and documenting
observations and explanations of physical phenomena in free nature.

On a page of a notebook, he had scribbled twelve topics: Measuring
and Observing; Along the Water; At Sea; Play and Sports; The Sounds
of Nature; Light and Color on Earth; Heat and Cold; The Realm of Air;
Electricity in Nature; Precipitation; Colors in the Sky; On the Train.
He had gone through an impressive number of articles and even contributed
original work. One of his most successful contributions was 'On the
Sound of Splashing Water,' about which he shared that a water droplet
falling into water carries a small amount of air with it. This air
forms a tiny bubble that rises to the surface and creates a tone:
'The sound of splashing water in nature can be attributed to the unison
of millions of such sounds: the bubbling of streams, the splashing
of fountains, the rustling of the sea.' Everyone knows this phenomenon,
but it required Minnaert's uninhibited curiosity to want to investigate
it! In the early 1930s, he had taught a course on Physical Observations
in Free Nature at several People's Universities.

\textbackslash figcaption The palisade illusion: a rolling wheel
as observed through a gate

During those years, there must have been many people who knew stories
similar to that of young Van Milaan: 'Minnaert would stand in the
morning, waiting before the railroad crossing barriers for the local
train to Utrecht (Bello), and at the last moment, he would crouch
down to study the effect of the spinning wheel spokes through the
vertical bars of the crossing gates. For many, it was a comical professorial
display that took place every morning for weeks on end, for the benefit
of a piece from {*}De Natuurkunde van 't Vrije Veld.{*}'

It likely led to the text of ‘The palisade illusion’: ‘A rapidly rolling
wheel with spokes, seen through a fence, displays a surprising pattern:
it is as if all the spokes were bent.’

On November 16, 1936, Minnaert wrote to the director of the Thieme
publishing house that he had completed the manuscript for {*}De Natuurkunde
van 't Vrije Veld{*}: ‘The concept of the work is entirely original:
it is a collection of physical observations that can be made outdoors
without instruments. The intention is to show that the physicist,
just like the botanist or zoologist, can experience joy in the surrounding
nature and that even the interested layperson can fully enjoy this
field. (...) The book is written in a way that makes it understandable
to anyone who has had secondary education and has not completely forgotten
it. A few small, more difficult pieces here and there can be printed
in smaller font and do not disrupt the overall coherence. On the other
hand, so many little-known publications and my own original observations
are included that, in my opinion, even the expert will read it with
interest.’

Minnaert had formed an image of the potential buyers: ‘In these times
when education is increasingly emphasizing nature, I expect it will
be sold to teachers, school libraries, public reading rooms, scout
leaders, readers of {*}Hemel en Dampkring{*}, and all amateur meteorologists,
members of the Weather and Astronomy Association, voluntary observers
of the Meteorological Institute, physicists, tourists, geographers,
etc.’ The honorarium could not be a problem. However, he did want
to retain the translation rights himself. Two days later, he already
had a contract from the publisher regarding part II. Minnaert had
stipulated that he could have parts II and III published elsewhere
if Thieme decided to stop further publication. This quick success
must have been encouraging for him during his suspension in Utrecht.

The encyclopedic aspect was in Minnaert’s genes. His father, with
his aphorisms, had covered the entire world, and his godfather, Gillis
Desideer, with his {*}Geesteswereld en schoonheidszin{*}, had wanted
to offer Flemings a general development, a {*}Leekenspieghel{*}. In
nature, Gillis had found order, coherence, and harmony everywhere.
For a long time, he had marveled at the richness of form and color
in nature: 'Climb a hill or dune top, from where the beautiful natural
scene unfolds before your eyes.' He had written about flowers, trees,
birds, butterflies, and plants, much like Jacques P. Thijsse before
his time, but also about mountains and rocks. His godchild had inherited
this mindset: he merely needed to change 'the plan of the creator'
into 'the laws of the universe.'

Minnaert could, moreover, mirror himself in his older friend. In 1917,
the astronomer Anton Pannekoek had published {*}The Wonderful Structure
of the World{*}, which 'unveiled the foundations of the astronomical
worldview.' He had illustrated the texts himself and, just like Gillis
Minnaert, wrote for laypeople. The communist Pannekoek had wanted
to serve a political purpose: 'Ever larger masses of people are rising
from the traditional faith of their fathers to a scientific understanding
of the world. If they wish to orient themselves well in the current
struggle of worldviews, they must also form a clear understanding
of the actual structure of the universe.' Minnaert had thoroughly
absorbed this last sentence.

\section*{A magic wand touching the reader}

Minnaert had not pointed out these sources of inspiration in his manuscript.
He began his book with a quote from {*}Farbenlehre{*} by the German
poet and naturalist Johann Wolfgang von Goethe (1749-1832): 'First
and foremost, we consider the experiences related to this matter in
full daylight. We bring the observer into the open air before leading
him into the confines of the dark chamber.'

Just like Minnaert, Goethe had spent many years botanizing and drawing
before dedicating himself, as a universal man, to mineralogy, chemistry,
optics, meteorology, and literature. Goethe's physiological theory
of colors, which at the time had been dismissed in physical circles,
was considered by Minnaert as a necessary complement to Newton's optics.
In his breakthrough as a solar physicist, Minnaert had greatly benefited
from his {*}Fingerspitzengefühl{*}, which suggested that a difference
in color could also make a difference in the image on the photographic
plate. That the phenomenological\textquotedbl{}

\textquotedbl Goethe's view of nature, which was also Minnaert's,
could have been promoted by his studies in biology and zoology. Minnaert
would always draw attention to primary worldview, sensory perceptions
of smell, color, and form---thus qualitative aspects. Not only of
living things, but also of seemingly dead nature. Respect for creation
was instilled in Minnaert playfully: he later found it in the writings
of his father and uncle, as well as in Goethe.

A long-term collaboration began between author and publisher. At the
end of 1937, part I appeared with the subtitle Light and Color in
the Landscape. The book received enthusiastic reception from both
the public and the press. The publisher sent Minnaert the reviews.
Minnaert wrote that he had read Dijksterhuis's appreciative review
in De Gids 'with great pleasure,' as he was 'a rather demanding critic.'
In December, Thieme decided to publish part II, Sound - Heat - Electricity:
it only came out in 1939. By the end of 1938, a reprint of part I
had already been made and work on part III, Rest and Movement, had
begun. There was talk of a German translation, which the publisher
discouraged because he feared competing prices. The English translation
of part I took place in 1939. In September, Minnaert heard that a
translation had appeared in Poland: at that time, World War II had
broken out in that country due to the invasion by Germany and the
Soviet Union. Part III appeared in November 1940. Four years of feverish
work on proofs and experiments, correspondence about photos, drawings,
layout, captions, and dust jackets were behind him. In the first reprints
from the early 1940s, Minnaert consistently improved passages and
incorporated criticism and suggestions from readers, friends, and
colleagues.

Minnaert writes in his foreword about nature experience also in words
reminiscent of the American poet Whitman: 'Who loves nature observes
its phenomena as he breathes and lives; from an innate deep urge.
Sunshine and rain, warmth and cold are equally welcome opportunities
for observation; he finds his interest in the city and in the forest,
on the sandy plain and at sea. Every moment presents him with new
and significant events. With an elastic step, he wanders over vast
lands, eye and ear ready to capture the impressions coming from all
sides, deeply inhaling the scent of the air, feeling every temperature
difference, sometimes stroking a shrub along the way to be in closer
contact with earthly things. In this way, he feels like a lively human
being. Do not think that the infinitely varied moods of nature lose
any of their poetic quality for the scientific observer: the habit
of observing refines our sense of beauty, and enriches the emotional
background against which individual facts stand out. The connection
between events, the causal relationships between parts of the landscape,
create a harmonious whole out of what would otherwise be merely a
sequence of disconnected images.'

The human being does not notice much more than the things he already
knows. Therefore, the book had to become a magic wand that touched
the reader with the 'knowledge of what I should pay attention to.'
Biologists had their floras and faunas; now physicists had an equal
counterpart. Observations in the open air also served as a didactic
guide: 'They help us in our increasing effort to connect education
to life: they provide a natural reason to ask thousands of questions
and ensure that what has been learned at school is continually rediscovered
outside school walls. Thus, the omnipresence of natural laws is experienced
as an ever-surprising and impressive reality.'

The book added a dimension to observations with its rational explanations.
Indeed: what humans love, they want to know and name. According to
Minnaert, there was always a scientific explanation for a phenomenon.
If coincidences were encountered, these too were 'governed by fixed
laws.' This deterministic view is expressed in every paragraph of
the trilogy.\textquotedbl{}

\section*{The Last of the Mohicans}

The trilogy still makes an overwhelming impression on the reader.
On one hand, Minnaert was inspired by works in specific fields such
as 'Scenery and the Sense of Light,' 'Waves of Sand and Sound,' and
'Ocean Waves' by Vaughan Cornish, 'Meteorological Optics' by Pernter-Exner,
'On Ship Waves' by W. Thomson, 'Wind und Wasserhosen in Europa' by
A. Wegener, and 'Mathematik und Sport' by E. Lampe. Other sources
of inspiration included 'Physiological Optics' by physicist H. von
Helmholtz, 'Modern Painters' by art historian John Ruskin, Goethe's
'Farbenlehre,' and Leonardo da Vinci's 'Trattato della Pintura.' For
Minnaert, physics intersected with painting, physiology, and literature.
John Ruskin often speaks at length; his aesthetic views seemed to
align with Minnaert's. He also used contemporary dissertations such
as Feenstra Kuiper's 'The Green Ray' or P.D. Timmermans' ' Experiments
on the Influence of Waves on the Beach.' He processed thousands of
articles from journals and noted that most authors were unaware of
their predecessors' work.

On the other hand, Minnaert's own incidental experiences seem to determine
the depth and scope of the treatment. For instance, a subject like
'Damping of Sea Waves by Oil' begins with statesman and physicist
Benjamin Franklin, who read about this phenomenon in Plinius de Oude.
The American said, 'I made sure, every time I went for a walk, to
carry a little oil in the top hollow section of my bamboo walking
stick to reapeat the experiement and found it invariably successful.'
Franklin thus poured oil from his hollow cane onto the waves and observed
its damping effect on the water. Minnaert followed Franklin's example
and described the differences in effects of various oils from the
1930s. Another time, he verified the echo at Chantilly's castle staircase,
which Dutchman Christiaan Huygens described in 1693. Then, on page
18, he tells about the walk of his friend Felix Ortt, who between
Den Dolder and Huis ter Heide was struck by a whirlwind, a 'little
whirl.'

The Minnaert trilogy---soon referred to with this honorable substantive---dedicates
tens of pages to 130 poems and quotes from 70 literary figures. This
approach strengthens Minnaert's plea for an emotional surrender to
nature. Most of the quotes---13 in total---are borrowed from the
Flemish poet Guido Gezelle; he also cites his activist friends René
De Clercq (5), Felix Timmermans (2), and Wies Moens (1). The names
of Georges Duhamel, Herman Gorter, Walt Whitman, Goethe, Paul Verlaine,
Coleridge, Virgil, Henriette Roland Holst, and Jacques Perk appear
multiple times. Quotes from Fj. Gladkov, D. Merezjkovski, and M. Sholokhov
prove that he reads contemporary Soviet literature.

Minnaert seems to want to bring 'lifeless nature' to life, using 19
literary quotes for this purpose. John Ruskin writes about a mud puddle:
'That puddle is not the brown, muddy, ugly thing we think it is; it
has a heart like ours, and on its bottom are the branches of tall
trees, the leaves of trembling grass, and all kinds of changing beautiful
lights from the sky.' Georges Duhamel writes about a rock: 'A stone
is a beautiful thing, beautiful in every way: its grain, its color,
its fracture, its gloss, its hardness, its many properties that exercise
and satisfy our senses, inspiring us to think.' The great physicist
Helmholtz muses: 'The advancing waves with their rhythmically repeating
motion, which nonetheless constantly changes, create a strange feeling
of comfortable rest without boredom and evoke the image of a mighty
but ordered and harmonious life.' In Minnaert's work, nature addresses
people. Sometimes it seems as if the author attributes consciousness
to 'dead nature.'

Minnaert dedicates many paragraphs to describing cloud formation and
translates Goethe's ode to Luke Howard, the father of cloud classification:
\begin{verse}
'But you, Howard, with a clear mind,

give us the noble advantage of your science.

What we cannot grasp or reach, you make accessible

You now hold it firmly for the first time;

determine the indeterminate, confine it, name it aptly.

Therefore, the honor is yours!

As clouds rise, fall, and gather together, one gratefully remembers
your name!
\end{verse}
Someone who acquaints themselves with the physical backgrounds and
transitional forms of Cirri, Cumuli, and Strati can daily enjoy a
glorious spectacle! Where Henriette Roland Holst saw the spiritual
characteristic of Holland in those 'swelling cloud processions,' compatriots
who pass them by miss something invaluable.

The astronomer H.G. van Bueren emphasized, when asked, the cultural-historical
aspect of Minnaert, the longing for completeness and systematics reminiscent
of the 19th century: 'Minnaert approaches nature in a way that lies
between science and art. He enters nature with sketchbook and pen
in service of physics. The atmosphere he evokes is unique. He reminds
me of John Ruskin and, in my opinion, Goethe. With that, he is a great
man: the last of his generation; the last of the Mohicans. His trilogy
makes me think of Jakob Burckhardt, of Bernard Berenson. He was a
remnant of the past. That atmosphere is reflected in his writing style
and books. Minnaert is both an artist and a scientist. His {*}Natuurkunde
van 't Vrije Veld{*} reminds me most of D'Arcy Thompson's {*}On Growth
and Form{*}. Minnaert could have written that as well.'

How can one adequately convey the impression this epic still makes
on new generations of readers, who can now read the first part in
Russian, Hindi, Swedish, Finnish, Romanian, Polish, German, and English?
Imagine that Minnaert travels from Utrecht to Zandvoort in June 1937
to take measurements along the waterline and returns late in the evening
to Bilthoven. Multiply the sensations of this day by a factor of 150,
and you will start to see the contours of Minnaart.

\section*{From Utrecht to Zandvoort and back to Bilthoven.}

It promises to be a sunny Saturday. Minnaart walks at seven o'clock
after a night of working with the microphotometer of the Physics Laboratory
along the canals to Central Station. He stops for a moment on the
side of a bridge. The rippled water surface reflects the light of
the low sun and forms a network of curves on the vault. Where the
surface curves outward, the rays diverge and light-poor spots appear
on the brick arch. The concave surfaces converge and create strong
light patterns. This mercury-like spectacle on the span always fascinates
him.

He goes to Zandvoort to measure the difference in water levels between
low tide and high tide. Owning a vehicle is a waste of time, pollutes,
and disrupts the city. He listens to the sound of the train from that
time. The rumble arises because the wheels receive a jolt when transitioning
from one section of rail to another. The rails expand in warm weather,
so the gaps between them must bridge the difference between winter
minimum and summer maximum. Sometimes the rails are shorter, and the
number of jolts per unit of time increases. The seasons also create
varying sound patterns. He thinks about what his comrade Wies Moens
once wrote: 'The song of the wheels has comforted our great sorrow
so many times.' The rumble is a mixture of many tones, so the passenger
can listen to any desired melody. When the train stops, he hears the
screeching sounds of objects rubbing hard against each other without
lubrication: these are relaxation vibrations, like those of chalk
on a blackboard.

\textbackslash figcaption Measuring acceleration on a train

He has conducted many acceleration tests on trains: he then uses a
vertical pendulum attached to the luggage rack as measuring instrument.
Today, he observes physiological effects. When the train brakes, he
looks at chimneys, houses, and other vertical objects: it seems as
if the verticals lean forward, especially when the train has just
come to a stop. The passenger feels as though they are being pushed
forward, as if the direction of gravity has changed. Even the horizons
of the meadows seem to tilt. Once the train has stopped, everything
returns to being upright. At Muiden station, another train departs
on the opposite track, making it seem as though his own train is pulling
away. When the other train truly starts moving, Minnaert gazes dreamily
out the window at the ground rushing by: 'The train stops; and while
I am certain it has come to a halt, I still have an irresistible impression
when looking outside that it is slowly sliding backward.' The explanation
lies in 'our mind,' which has learned to mentally subtract a portion
of the speed from every part of the field of view and continues to
do so even after the motion has ceased.

The sun is low and shining from the east into the compartment on the
left. Minnaert walks back and forth from the left side to the right
side of the carriage, comparing. The difference in hue between the
grass fields on the sunny side, lit by diffuse light, and those on
the shaded side, lit by direct light, is clearly visible: in the latter
case, more reflected light can be seen. He uses a small mirror to
look at both sides and compares: 'This difference corresponds to the
well-known distinction among painters between the green tones of Willem
Maris in his backlit landscapes and those of Mauve, who preferred
to paint with the light behind him.'

The grass is moist with dew and therefore full of color: as soon as
a thin layer of water covers the objects, the surface becomes smoother.
The grass scatters the white light less, allowing its own hue to dominate,
making the color appear 'more saturated.' In some places, morning
fog still clings to the sunlit slope of the railway embankment, while
the shaded side remains free of mist. The dewy earth on the sunny
side releases water vapor, but the air above it is still cold: when
the air mixes, the fog condenses due to an over-saturation of water
vapor.

\textbackslash figcapation Near the horizon, Cumulus clouds overlap
one another, and one overestimates the degree of cloud cover\textquotedbl{}

Cumulus clouds aer visible in the sky, which will disappear during
the course of the day. The cumuli exist at a single altitude: 'When
the Earth is heated by the sun's radiation, columns of hot air rise
everywhere. Where this air ascends, it cools through expansion until
it reaches its dew point, and the moisture it contains condenses into
droplets.' Just like morning mist. The underside of the clouds forms
a horizontal plane at a similar temperature, where the rising air
condenses.

In Zandvoort, Minnaert chooses a walking path to the beach. In the
dune area, he looks back at the forest; against that dark backdrop,
the light scattering becomes noticeable. The farther he walks away
from it, the more 'bluish' the forest appears. The long layer of air
between the observer and the forest, illuminated sideways by the sun's
rays from the east, scatters light that 'superimposes' itself on the
background: in the forest, the contrast between light and dark areas
weakens, the light becomes more even and 'bluer.' Minnaert is not
yet familiar with the story of the scattering effect of terpenes,
the hydrocarbons exhaled by coniferous forests. The vegetated dunes,
rising like waves in the sea, ridge after ridge, farther and farther,
display increasingly intense bluish hues, 'as one often finds depicted
in the landscapes of our 16th-century painters: Van Eyck, Memling.'
For Minnaert, these are Dutch painters, just as the highest Dutch
churches for him are those in Antwerp and Bruges, not Utrecht or Delft.

Minnaert sees the blue sky above his head: from white light, the short-wave
violet and blue rays are scattered the most by collisions with air
molecules in the atmosphere, which is why the sun appears yellow.
Why do so few people wonder about this? If the air molecules were
a bit larger, the sky would be red! Rayleigh's scattering law, which
seemed to play a key role in Julius' theory of the sun, plays a role
here as well. When the sun dips below the horizon, the green and yellow
hues will be scattered more prominently in the longer atmosphere toward
observer 38, causing the sun to appear orange-red.

Minnaert is one of the fortunate individuals who observed 'the green
flash' dozens of times aboard a ship heading to India, above an ultramarine
sea. Many others have looked for it in vain. On one previous occasion,
he managed to extend his observation of the flash by several seconds
by running up Zandvoort's 39-meter-high dike: 'It all comes down to
perceiving that very last speck of light it emits.' 'The green flash'
refers to the emerald-green tint of the sun's final glimpse. Blue
and green rays are not only scattered more intensely in the atmosphere
but also refracted more strongly than red or orange rays. As the sun
sets, one should ultimately notice a green edge. Due to this scattering,
blue cannot be part of it.

\textbackslash figcapation The formation of the green flash occurs
when the upper parts of the setting sun are obscured.

On the beach, Minnaert verifies earlier experiments concerning the
difference in height between low and high tide. Tidal movement arises
because the moon attracts the movable water masses. This attraction
changes, among other factors, due to Earth's rotation. In Zandvoort,
this phenomenon is not symmetrical. The ebb lasts seven hours, while
the lowest water level persists for two hours. Flood movement, however,
reaches its peak in three hours and immediately begins to recede.
He places a stake with markers by a row of objects perpendicular beach
poles along the coastline and reads the height from the intersections
with an imaginary line to the horizon. The tidal range of approximately
is found to be well reproducible: this time it amounts to 1 meter
55.

\textbackslash figcaption High and low tide observations with simple
equipment.

\textbackslash figcaption Determing the level of the sea to research
tides.

Minnaert has often studied the ripple pattern of the sand waves on
the beach, the current ripples. He lets a handful of sea sand, 50
centimeters above the ground, trickle away in a fine stream. Like
separating wheat from chaff, the coarse grains separate from the fine
ones. The asymmetric pattern of the current ripple arises because
the fine sand is eroded upstream by seawater, even at low flow velocities,
while the steep lee side consists of coarse, settling grains. When
the water flows quickly, this separation can be reversed. At an even
greater speed, the apparent wave pattern disappears. Using a small
stick, the displacement speed, the creeping movement, of the ripples
can be measured. Then it becomes clear that the sand grains do not
vibrate around an equilibrium position but actually move. Where the
sand grains are all the same size---Minnaert claims this is true
for a five-meter-wide strip along the coast---the ripples are absent,
and the walker can expect singing sand.

Minnaert addresses the reader about his footprints at low tide: 'Wherever
you step then, you see it turn white around your foot, apparently
because it suddenly becomes dry.' Does the pressure of the foot compress
the sand? But in that case, water should appear on the surface in
the capillary spaces, while the opposite happens. So the foot pressure
does not compress the sand but causes it to expand. The sand grains
are shaken together as they settle and take on the densest packing.
Any change in this arrangement, such as from the pressure of the sole
of the foot, causes the sand to expand. The British scientist Reynolds
called this phenomenon 'dilatation' in the Philosophical Magazine
of 1885. Therefore, the water level drops, and the dry footprint appears.
When the foot disappears, the sand grains return to their densest
state, and the water reappears: 'If one considers the two hundred
thousand million men, women, and children who have walked on this
beach since the creation of the world and had asked: 'Does the sand
under your feet get compressed?', how many would have answered anything
other than 'yes' at the meeting of the British Association in Aberdeen
in 1885?

The sea is clear, transparent to depths of meters and emerald green.
The question arises as to why the interplay of absorption, scattering,
and reflection of sunlight gives the seawater above a latitude of
40 degrees that color and not the ultramarine of the seas under a
latitude of 30 to 48 degrees. At sunset, the play of light provides
the bathers of 1937 with a color sensation like that of the current
water in the Venetian lagoon. Minnaert sees the twilight colors over
the sky change minute by minute: all stages are effortlessly distinguishable
on this cloudless evening. The earth's shadow in the east and the
sunset in the west create the transition from gray-blue to orange-red
across the sky from east to west, finally to soft purple and yellow-green.

Whoever has never noticed that changing color splendor, that magical
word on the sky, that four-quarters-lasting spectacle of colors, truly
misses out a lot. The tide is coming in. The surf makes a roaring
noise. Where does the pitch of those falling water masses come from?
Minnaert was the first to pose this research question and provide
the beginning of an answer.

\textbackslash figcaption Bending of light through the scratches
on the windows

The day is over. He takes the train again. Lantern light falls through
the scratched windows, and the bending of light forms circles. The
temperature in Bilthoven seems to be lower than that in Utrecht. He
checks this by holding his thermometer outside the window at the stations
of the boemel: 'In a certain case, the temperature of a large city
on a beautiful summer day was approximately 1 degree higher than that
of its surroundings, while in the evening the city was 7 degrees warmer!
Apparently, the nocturnal radiation in the city was less, and the
heat capacity of the bricks and the warmth from thousands of fireplaces
in homes and factories should certainly not be underestimated.

Minnaert keeps busy, his notebook at the ready for drawings and text,
utilizing travel time for calculations involving self-devised physical
formulas. From his earliest youth, he transforms questions into elementary
mathematics. Only then can twenty years of observation and study culminate
in {*}The Physics of the Open Field{*}. Minnaert is perfectly happy
with himself and his work. For his fellow travelers, his restless
activity may be exhausting, but for the nature lover, it provides
an inexhaustible source of inspiration, a magic wand that opens the
eyes to the background of phenomena in nature.

Minnaert remains a child and a romantic, endlessly marveling and constantly
surprising himself anew. He embodies the researcher as one who must
have existed in all times. As a student, he could not inspire his
companions with his worldview, as evidenced by the prize question
regarding the eclipse. With the help of this book, he effectively
conveys his ecstasy about nature to ordinary mortals. Thanks to a
formidable effort, he makes himself understandable and creates a classic
work.\\

Endnotes:

1. Part I, 1937, 220, no photo. Part I, 1968, 267, plate XII opposite
241. The last 'halo' is used here.

2. Manuscript with notes, 1920. Astronomy Archives.

3. Minnaert, (1923d). Figure of the setup in Minnaert, (1933b).

4. A 1933 dictation by student M.H. de Jongh mentions ten themes:
Rest and motion; Waves, Wind, and clouds; Water and ice; Sound and
murmur; Reflection and refraction of light; Rainbows, circles, and
coronas; Phenomena resulting from peculiarities of the eye; Colors
in the landscape; Electricity in nature. Mechanics, as usual, precedes
optics. Eventually, Minnaert reversed this order. The course lasted
three months and was given in Amsterdam, Utrecht, and Arnhem at least.
Astronomy Archives.

5 Drs F.W. van Milaan, the boy next door from Parklaan 50, in a letter
dated June 26, 1998. The story about the physiological phenomenon
dates back to the mid-1930s and apparently led to the figure drawn
by Minnaert in a section of Part I, Afterimages and Contrast Phenomena,
under the subtopic The Fence Phenomenon, 111 (1968, Figure 96, I-144).
6 Minnaert to Thieme, November 16, 1936. Zutphen Municipal Archives.

7 Minnaert, November 18 and 27, 1936. Zutphen Municipal Archives.
8 Minnaert, G.D., 1913-1914, I, Conclusion.

9 Pannekoek, 1917. Quotes written consecutively.

10 The decisive breakthrough in his equivalent latitude was related
to the yellow-green catastrophe of the Rowland intensities (Chapter
8).

11 Dr F. Boer wrote in a letter of June 8, 1998, about the similarities
in worldview between Goethe, Minnaert, and the Swiss biologist Adolf
Portmann (1897-1982). This paragraph about Goethe would not have been
written without his intervention.

12 That took place on May 30, 1938. A German edition of Part I, Licht
und Farbe in der Natur, newly illustrated, was published in 1992:
more than half a century later!

13 In the revision of his trilogy (1968-1971), Minnaert introduced
Walt Whitman in the Introduction to Part I. In the first edition (1937-1940),
he quotes him several times, but Whitman is not yet a kindred spirit.
The Goethe quote remained neatly in its place in the revision: at
the beginning of Part I.

14 That is present in his didactics book from 1924. Quote from the
Foreword to Part I, VII.

15 Interview with astronomer P.J. Gathier from Molenaar, 1998.

16 Minnaert, Part III, Damping of Waves by Oil, 146.

17 Minnaert, Part II, Echo of a Staircase, 28.

18 Minnaert, Part III, 240. From About Felix Ortt, The Hague 1936.

19 Ruskin, J., Modern Painters III, 496, cited in Part I, 292.

20 Duhamel, G., La Possession du Monde, 106, cited in Part III, 199.

21 Helmholtz, H. Von, Tonempfindungen, 1863, 388, cited in Part III,
128.

22 Minnaert indeed went further. In The Unity of the Universe (1963),
he denies that there is a fundamental difference between inanimate
nature and conscious organisms. This worldview is implicitly present
in The Physics of the Field and is part of the enchantment of the
book.

23 Minnaert, Part II, 130-131. A brilliant piece by O. Krätz, {*}Goethe
und die Naturwissenschaften{*}, 1998, is dedicated to Goethe and Howard:
Luke Howard, the man who 'distinguished clouds,' with the original
poems in German, 188.

24 Henriëtte Roland Holst, Holland, Minnaert, Part II, Cloud Perspective,
147.

25 Interview with theoretical physicist H.G. van Bueren. Partially
in Molenaar, 1998.

26 Plate III, Part I, Refraction of light during the transition from
air to water, next to 30 (1968, Plate IIIA, next to I-48).

27 Minnaert, Part II, The sound of a train, 42; Expansion of rails
due to heat, 65.

28 Part III, Braking, 34, figure 20, 35. According to De Jager, in
the 1930s he measured acceleration during takeoff and landing of an
airplane in the same way: his pocket watch was tied to a piece of
elastic, and he studied how much the elastic stretched or shortened.
None of the members of the study association S-square had flown on
the 'giant airplanes' that could carry a dozen people. Minnaert hardly
looked out the window because he was busy with his experiments (1972,
better figure, III-48).

29 Part I, Optical illusion regarding position and direction, 134.

30 Part I, Illusions about rest and motion, 137.

31 Part I, The color of green leaves, 314.

32 Part II, Formation of fog and mist, 104.

33 Part II, Cumulus, 138.

34 There is no specific figure of clouds at the same altitude; but
it is implicitly used in a drawing about the apparently increasing
cloud cover of Cumulus near the horizon in Part II-126.

35 Part I, Aerial perspective, 228.

36 The terpenes were added by Minnaert in the revision of Part I in
1968.

37 Part III, The height of the main Dutch towers, 13. 38 Part I, The
blue sky, 227.

39 Part I, The green ray, 58. Feenstra Kuiper, {*}The Green Ray{*},
dissertation, Utrecht 1926.

40 Figure of the 'green ray.' The American Andrew T. Young specializes
in the 'green flash.' He wrote in {*}Zenit{*} about the green flash
as a celestial phenomenon, 1999, 248, and maintains a website discussed
elsewhere. His critical findings suggest that Minnaert should listened
more to geologist A Wegener and

less to his Utrecht colleague Feenstra Kuiper who earned his PhD on
{*}The Green Ray{*} in 1926 (1968, Figure 68, I-90)

41 In 1968, Taylor and Matthias photographed a sunset with a 'green
ray' from an airplane. The photo series appeared in Nature, looking
just as Minnaert had drawn it in 1937. The authors sent the photos
to Prof. Minnaert with admiration. Color print on the back of Molenaar,
1998.

42 Part III, Ebb and Flow, 102, figure 64, 104 (1972, figure 74, I-142).

43 Figure 63, 103 (1972, figure 73, III-141).

44 Part III, The Composition of Beach Sand, 163, figure 88 (1972,
figure 98, III-204).

45 Part III, Ripples, 173.

46 Part II, Singing Sands, 57. See also his predecessor Hertha Marks
Ayrton (1854-1923), whom Minnaert was unaware of, in the Miniature
of Women by M.I.C. Offereins in NVOX, 2002, April. Long live search
engines!

47 Part III, Dilatation of Sand, 169.

48 Part I, The Color of the Sea, 295.

49 Part I, Twilight Colors, 254.

50 Part III, The Surf, 136.

51 Part II, The Murmur of Water, 47.

52 Part I, Bending of Light on Small Scratches, 205.

53 Part II, The Temperature of the Big City, 83.

Chapter 13

\chapter{The Man of the New Observatory}
\begin{quote}
'But as he lies stretched out on his deathbed, the machines roar and
the electric current trembles through the thousands of wires in his
laboratory.'
\end{quote}

\section*{Appointment in Chicago}

The Utrecht astronomer Nijland had been an old-school astronomer,
very involved with amateur organizations and always busy with his
estimates of the brightness of variable stars. Developments in astrophysics
had passed him by. After Nijland's death, the curators wanted to divest
the Observatory and abolish the chair. Budget cuts were the order
of the day: it made sense that this College would draw this consequence
upon proven failure. Ornstein, leader of the physics community, opposed
this plan vigorously. 2

The matter started moving when Minnaert received a letter from Otto
Struve, Director of the Astronomy Department at the University of
Chicago, in March 1937. Struve invited Minnaert to accept a professorship:
'The most important thing is that we secure our university's access
to the services of a leading astrophysicist who would take on teaching
duties on campus and perhaps teach some other courses.' The teaching
load would consist of two courses per quarter, four hours a week.
One of the quarters was reserved for vacation and personal research:
'At this moment, our large spectroheliograph is not in great use.
I would welcome it if the solar work at Yerkes Observatory were to
be expanded.'

Struve could offer Minnaert a salary of \$5,000 and colleagues who
ranked among the world's top astronomers, such as the Indian Chandrasekhar,
the Swede Strömgren, the Belgian Van Biesbroeck, and Struve himself.
The University would acquire a unique observation post in 1937 with
the McDonald Observatory in Texas. The 82-inch reflector would be
at Minnaert's disposal, allowing him to have plate material recorded
as he pleased: 'Let me add personally that I would consider it of
the utmost importance for my own work if you were willing to accept
this offer. Your work on growth curves has been particularly valuable
to me, and it would be a great pleasure if we could collaborate.'

The next day, Minnaert received a letter from Gerald Kuiper, recently
appointed in Chicago, who recalled their collaboration during the
1929 eclipse expedition and the Leiden Student Peace Action. He wrote
that astronomer Hale had labeled their spectrograph as the best in
the world. There was no solar physicist in Chicago: Minnaert could
revive that line of research! The equipment was breathtaking: 'The
University of Chicago has invested two million dollars in Astronomy
through its Texas project, comparable only to Mount Wilson and the
200-inch institute.' The management was positive about research and
approved every reasonable request. A house was ready in Williams Bay,
where Mrs. Minnaert and the children would have a wonderful time.
There was a primary school and good secondary education available.
Lectures and concerts were organized, and there were parks and beaches
by the lake: 'It is certain that with the position now being offered
to you, the expansion has reached its limit, at least for the coming
years.'

Minnaert immediately consulted Ornstein, who suggested that he would
advocate for an extraordinary professorship in astronomy. The ordinary
professorship would elude him for the time being, which would align
with the budget-cutting trend. Minnaert agreed but wanted to transform
the observatory into a fully-fledged research institute. He demanded
that Julius' spectrograph be relocated. Ornstein concurred.

Minnaert thanked Struve for his offer. He didn’t mind the teaching
responsibilities at all, as he enjoyed giving lessons. If the appointment
were to proceed, he would consider it a permanent position. However,
he thought the journey would be challenging for his eighty-year-old
mother. He also mentioned plans to improve his position: 'I wouldn’t
want to make them impossible a priori, especially since I am deeply
attached to this University, this country and its people, the career
I have built here, and the research initiatives I have undertaken.'
The Faculty will meet tomorrow to discuss this matter and will certainly
contact the minister immediately; but before any decision is made,
some time may pass. I would like to ask you to inform me within what
timeframe I need to make a definitive decision.'

\section*{Appointment in Utrecht}

At the request of the faculty, Minnaert wrote the epistle 'Reorganization
of Astronomy Education in Utrecht,' which formed the basis for the
faculty board's letter of April 5 to the curators: 'Who will ensure
that any successor to Prof. Ornstein will maintain heliophysics as
an independent department within the laboratory? What will happen
to the work of Prof. Julius? What about the international spectrophotometry
work that the Heliophysical Institute had begun?' It was known that
the curators were wondering if the Leiden Observatory could take over
the solar research. Minnaert wrote indignantly that Utrecht had a
spectrograph, which was as essential for an astrophysicist as a microscope
is for a biologist! He had reviewed recent volumes of the Astrophysical
Journal: six out of seven articles relied on own spectroscopic work.
Leiden had no spectrograph and no self-recording microphotometers:
'As things stand now, each of the astronomical institutes in the Netherlands
has its own territory. None of these fields is better defined than
that in Utrecht, none is more important for the modern direction of
astronomical research.'

Meanwhile, letters from Struve and Kuiper continued to arrive. On
April 12, Kuiper wrote that he was pleased Minnaert was considering
the offer. In Chicago, they also collaborated with physicists. The
Yerkes Instrument Shop could construct any design: 'Working with the
largest instruments in the world in such a climate is a privilege
and gives the observer a sense of happiness that a European astronomer
does not know.' The dynamic aspects of America made life attractive:
'We all hope that you will join our group.' Kuiper was making it very
difficult for Minnaert. On April 20, Struve wrote that he understood
Minnaert's circumstances, both in family and scientific terms. He
could address all concerns regarding facilities. Minnaert could determine
the start date of the lectures himself. Near the Observatory, the
University owned a furnished house where Minnaert could spend weekends
and vacations free of charge. On the same day, Kuiper mentioned the
opinion of a Dutch colleague teaching at Harvard: 'Bok and I feel
the same way: America offers such opportunities for scientific work
that we cannot return to Holland (or Europe). There is also this point:
so much is happening in American astronomy that it is advantageous
to be in close contact with it and, if desired, to exert influence.'

Kuiper also appealed to national pride. The Dutch in scientific America
set an exemplary family life and, through their idealism, language
skills, and travel experience, were true ambassadors: 'I bring up
this point because I would not be surprised if someone in their home
country (or second homeland) is not aware of this aspect and could
fear harming the country by leaving it. I often feel that the opposite
will be true.' A good Dutchman simply could not refuse this position.
He really didn't want to persuade Minnaert but merely outline the
circumstances: 'Because this decision will be important for astronomy.'

However, it was a run race. The rescue operation succeeded completely.
The 10th Royal Decree came on June 16. The Utrecht Daily reported
with enthusiasm: 'There was great fear that Dr. Minnaert would follow
this call from Chicago, but now that he has been appointed as a professor
in Utrecht, this danger is fortunately averted.' Pannekoek congratulated
Minnaert and wondered 'that you patiently waited for the end of the
official sluggish process instead of immediately jumping to the new
world! I wish our Higher Education luck that it now retains a center
for astrophysics, while otherwise it would sink back into an outdated
position in this field.' It was a toss-up. All love for country and
people notwithstanding. But no sluggishness! His faculty, Ornstein,
and the College had made a combined effort. A few days after the letter
from Chicago, emergency meetings of the faculty board, its chair and
secretary with the College took place, which intervened in The Hague
within a few days. They reacted swiftly.

Two factors still played a role. The sun's installation on Bijlhouwerstraat
had almost become unusable due to city traffic vibrations. And Minnaert's
material demands were so modest that the curators could convince the
minister that the operation, on balance, amounted to a considerable
saving. It earned Minnaert a salary increase of 790 guilders on July
4, 1937, taking him from 4,788 guilders to 5,558 per year: this was
roughly what he had earned as a lecturer in 1933, because in 1933
and 1934, two absolute pay cuts of 10\% and 5\% for all government
personnel had been implemented.

\section*{Minnaert's inaugural lecture: The project of the sun atlas}

In 'The Importance of Solar Physics for Astrophysics,' his stardust
sprinkled inaugural lecture, Minnaert outlined his research program.
The chair had to be oriented within the national division of tasks
toward astrophysics, and thus the research had to be tailored accordingly.

Minnaert mentioned three general aspects of the relationship between
the sun and stars: first, the amount of radiation from the sun on
Earth is immensely greater than that from other stars. A wealth of
data about stellar atmospheres can be gleaned from solar spectra:
'When we have an Atlas on a large scale, detailing the light distribution
across the entire solar spectrum, astrophysics as a whole will certainly
benefit from it.'

Additionally, the sun is the only star that appears as a disk of finite
size, making it possible to observe segments of the solar disk, from
center to limb. The sun could give astrophysics a third dimension:
'It is remarkable how little the difference between center and limb
spectra is still understood from a modern perspective; I do not hesitate
to label the study of this as one of the most important, specifically
solar physics problems.' The samples of matter on the solar disk 'find
equivalents in the effective layers of all kinds of stars spread throughout
the universe.' Minnaert mentioned research on the corona and chromosphere,
sunspots, and prominences, and pointed to theoretical issues regarding
magnetic fields and the dynamics of solar material. Based on solar
research, a better understanding of the universe could emerge.

Finally, the study of the planets would provide information about
the history of the solar system, which would have general significance
for the history of stars: 'The Physics of the Sun relates to general
astrophysics as the study of the individual relates to that of the
community.' Minnaert extended this relationship into a whole network:
'Thus, the Physics of the Sun forms a link between Laboratory Physics,
Field Physics, and Earth Physics, and the Physics of the Cosmos.'
One of the mentioned physicists he had personally established.

Minnaert invited all his friends to the ceremony, : the activists
and Dietse Bond members, the contributors to Chreestarchia and the
Bilthoven Workshop, the Esperantists, and his colleagues from the
Physics Laboratory. An anonymous individual (X) 14 qualified his appointment
in Nieuw Vlaanderen as 'a justification after the fact for those who
appointed the 23-year-old lecturer at the Dutch-speaking university,
the noble revenge against those who deprived him, expelled him, condemned
him, and kept him in exile.' X knew Minnaert from the University:
'His slight forward inclination has become a symbol for one who was
fundamentally unyielding, yet bowed over his science, and the homo
humanior, full of kindness, listening to the needs and desires of
students and comrades.’

X. found a poem by the Afrikaner Cellier particularly applicable to
Minnaert:
\begin{verse}
‘I love a man who can stand like a man,

I love an arm that can strike a blow,

an eye that doesn’t flinch, that can give a fierce look,

and a will as firm as a rock!

I love a man who honors his mother...’
\end{verse}
X. admired the man who, with his telescope, daily studied the peculiarities
of the celestial body that his fellow townsman Gezelle had called
‘the heart of God,’ and hoped that for Minnaert, {*}vir justus et
nobilis{*}, the motto of Utrecht’s Alma Mater might be fulfilled:
{*}Sol Justitiae illustra nos{*}, freely translated as: ‘Sun of justice,
shine your full light - on him!’ That was a beautiful tribute from
this Catholic friend.

Minnaert distanced himself somewhat from the Flemish disputes. The
conflicts within his own ranks left him helpless. The truce with God
was not only violated by the NSB and VNV but also by his own fist.
The majority of his activist friends once again sided with powerful
Germany, from which they expected the liberation of Flanders: Thiry
and Kimpe, Domela and Picard, Speleers and Jacob, Borms, Van Genechten,
and Moens. And also the disabled Roza De Guchtenaere for whom a fund
was established in May 1939, to which Minnaert expressed written support.
They would make a terrible mistake.

\section*{The astronomical practicals}

The astronomy education was overhauled. Half of the professor-manager’s
residence had to be demolished for the spectrograph and the institute,
laboratory, library, and workshops. Instrument maker N. van Straten
was hired. As head assistant, Minnaert appointed his PhD student J.
Houtgast. The library became both a modern study center and a favorite
meeting place. Renovations lasted three years: demolitions and renovations;
only in February 1940, the relocation of the spectrograph finally
took place.

Naturally, Minnaert, just as in Ghent, wanted to put his educational
principles into practice undiluted. This brought him an immense amount
of work. In 1937, he immediately established a Astrophysics Practical
Course. Such a course didn't exist anywhere else in the world! That
was no obstacle for Minnaert: it became a mandatory part of the curriculum
for all students of mathematics, physics, and astronomy, a breeding
ground for astronomers. Minnaert believed that students should experience
the 'beauty of the sky' sensorily and emotionally.

The indoor practical course took up one evening a week. With the help
of books, 18 illustrations, and photographic plates, students carried
out an assignment: 'Often, later in the evening, Minnaert himself
would stop by, which further enhanced the special atmosphere in the
cozy library through his infectious enthusiasm. No one who experienced
such a practical course by the hissing coal stove in the pre-war years
will ever forget that atmosphere.' The exercises included determining
the growth curve of several lines in the solar spectrum, determining
the shape of the corona from photographic images, deriving the color
of stars from double star registrations, or drawing the orbit of a
visual double star.

The outdoor practical course on the roof had its own charm. Participants
used simple equipment, often made by the workshop: an inclination
meter (a small board with a string), a portable telescope, a comet
seeker, and a Jacob's staff. Volunteer assistant Aennie Elink Schuurman
had organized the collection of sextants. The tasks included drawing
a constellation, determining the extinction of the Earth's atmosphere,
estimating the brightness of a star with subsequent verification in
the catalog, and drawing the Milky Way. They aligned with Minnaert's
two main courses: one year on The Earth and the planets, the other
on Sun, stars, and the Universe. After passing their preliminary exams,
students could take elective courses with Minnaert.

Henk van de Hulst arrived in 1936 and had to wait a year for astronomy
classes: 'To our great joy, Minnaert was appointed. A fresh wind began
to blow. We really got astronomy with a physics background. After
class, we sometimes went up to the roof with the first-years to determine
the geographical latitude by measuring the sun's altitude. Later,
we were given assignments in pairs. I can still remember Minnaert's
disappointment when out of those six groups, only one had worked out
their results into a report! Dedde de Jong and I volunteered to check
how the other experiments had turned out in practice. My first article
in {*}Hemel en Dampkring{*} was a report on such an experiment. Minnaert's
enthusiasm for nature always stayed with you. He said, 'Really, you
should see the sunrise once a week. If you miss that, you're missing
so much...' In our third year, we walked to Den Dolder with Minnaert.
That’s when it became clear how much he knew about biology: not just
flowers, but the landscape as a whole. Excursions and practicals were
also ways to bring us closer together and share enthusiasm. We once
started an experiment around 5:30 PM and wondered if it still made
sense. Then Minnaert said, 'It could still be worthwhile.' He had
a different sense of time.'

Kees de Jager was also among the first group: 'My first day of lectures
began with astronomy. I arrived at the Observatory by bike and met
a tall, lanky man with a bushy mustache and tousled hair, whom I asked
if I could park my bike next to his in the hallway. The tousled-haired
man turned out to be the professor. That week, I attended my first
weekly evening astronomy practical session without realizing we were
attending the world’s first astronomy practical ever. A simple, effective,
and didactically excellent method for measuring the solar constant
made a big impression on me. I have the best memories of this practical
because of the elementary scientific actions and the insight into
the scientific method that I gained there.'

\section*{The atlas of the solar spectrum}

Minnaert, Houtgast, and Mulders quickly turned the first objective
of the research program into a standard work that attracted global
attention. Minnaert and Bannier had recorded the ultraviolet part
of the spectrum in 1936 using their own spectrograph. A subsidy enabled
Mulders to create plates using the best instrument, located at Mount
Wilson. Between September 1935 and March 1937, Mulders took plates
of the visible spectrum and part of the infrared spectrum. His correspondence
allows for a comparison between working conditions in the United States
and those in the Netherlands.

The American telescopes were the best in the world, but Mulders was
not impressed by how his colleagues operated them. Mulders used Minnaert's
step attenuator to determine the true intensities of the spectral
lines. His supervisor had expressed skepticism but secretly used the
platinum-coated glass slide for measuring their own recordings: 'He
then came to ask me about the calibration and became so enthusiastic
that he now wants one too. He asked if it was difficult to make himself.
Well, 'difficult' isn't the word, but it requires a tremendous number
of experiments, and it takes a long time before the process yields
a usable and reproducible result. Then he modestly asked if I would
write to Utrecht to see if you could make one for him and how much
it would cost.'

The astronomer Mitchell had him measure eclipse plates at Harvard.
Mulders wrote disapprovingly about his developer recipe: 'I must honestly
say that someone using this developer for intensity work doesn't have
much experience with such measurements. You can imagine how the films
look when treated with such an active developer in tropical heat:
they are brown and heavily fogged, also covered with dirty spots and
fingerprints.' Someone who had learned the trade from Ornstein and
Minnaert apparently had no reason to be ashamed in the United States.

Mulders passed on messages from Chicago: 'Chandrasekhar asked me why
we stopped sending him prints a few years ago. He really wants them...
Do you know what he thinks is the most beautiful piece you've ever
published? The article on resonance and water droplets in Philosophical
Magazine a few years ago! I remember it so well, that we did experiments,
usually on Saturdays at 12 o'clock, when we had time to play a bit
with tuning forks and buckets of water with tubes in them'.

In the spring of 1937, Mulders' plates were in Utrecht, and Minnaert
and Houtgast were in charge. The registration was laborious: after
all, the transmission profiles of the microphotometer had to be manually
converted into the true 24 intensity profiles. In 1938, Houtgast managed
to improve the microphotometer so that it directly provided profiles
with the true intensity. The plates of the spectrum that needed to
be recorded were a total of 13 meters long: the line profiles produced
by the photometer were 120 meters long. Together, they formed the
core of the Atlas. Minnaert spoke enthusiastically about this work:
'We usually worked at night because the microphotometer was free then.
You were alone in the building, in the silence of a dark room, under
the dull red light, developing. And there it slowly appeared, out
of nothing, as if by magic, the profile of the cyanide band on paper,
or the atmospheric oxygen lines; never before observed in their true,
quantitative form.'

The Atlas was completed in January 1940 due to their hectic work.
It was published in both English and Esperanto. The beautiful cover
with intaglio printing and gold-edged paper shows that the creators
were aware of the exceptional nature of their achievement: 'a magnificent
piece of work that for decades after the war contained the basic material
for the systematic study of the solar spectrum.' The Atlas would bring
project leader Minnaert the highest astronomical honors.

\section*{An unusual promotion}

Minnaert's career progressed extremely successfully in the late 1930s.
However, there were developments in his family and in the world that
contradicted this success. On December 12, 1938, Miep Coelingh graduated
as a doctor in physics and mathematics under the chemist Kruyt on
Optical investigations into the liquid-vapor equilibrium in capillary
systems. She had studied the structure of the drying agent silica
gel and conducted measurements on water absorption and release. The
chemist Van Bemmelen had preceded her by four decades in measuring
the water content of the gel under variable water vapor pressure.
One time, he started with completely dry material that he moistened
by increasing the water vapor pressure, and another time with fully
moistened material that he slowly dried. Sometimes it took weeks for
equilibrium to establish itself, which he could measure. Miep Coelingh
cleverly avoided this.

Her measurement method was highly unusual: 'The reason was a bottle
discovered by chance, which, probably due to weathering, was covered
on the inside with a beautifully uniform layer of a initially unknown
substance.' That substance remained unknown, by the way. When the
steamed bottle was almost completely filled with alcohol, the non-moistened
upper part of the bottle wall was covered with a layer of alcohol
vapor that displayed brilliant interference colors. She considered
the weathered layer to be an extremely thin, homogeneous layer of
gel. Attempts to produce a similar layer on other bottles failed.
Therefore, 'most experiments were done with this one bottle.'

She assumed that changes in water vapor pressure would affect the
gel layer; for example, part of the capillary-bound water would disappear
at a lower temperature and vapor pressure, and the layer would then
display a different interference color according to the ring pattern
described by Newton. In contrast to Van Bemmelen's measurements, she
used her quick color measurements. She found, unlike her colleague,
not only gradual transitions but also two sharp color changes. She
assumed that the pores of the layer were either fully or completely
drained. She did not have quantitative data on the water content of
the gel during these changes.

Miep Coelingh had built a device herself for the measurements. The
color changes of the layer in the particular bottle were reproducible
and sometimes preceded by 'speckles' which she interpreted as 'evaporation
nuclei,' whose existence she had 'demonstrated with certainty.' She
had photographed, traced, and colored several interference colors.

It was a curious investigation. She based herself quantitatively on
the theory of capillary forces and worked with a formula from 1871
by Thomson, with which they converted color changes into an estimate
of the diameters of the gel pores. The dissertation was not outstanding,
but it was patient and creative work. There were no acknowledgments
to Kruyt or to people in her immediate surroundings in the thesis.
However, friends of the Minnaerts reported that the couple had discussed
the thesis a lot and had also worked on it together.

Her friend Greet Miessen remembers that at Miep Coelingh's graduation
dinner, the menu cards were made in the colors she had observed on
the glass surface. Minnaert noted: 'Miep graduates, much acclaim and
interest. The boys are allowed to sit at the meal, which they remember
fondly.'

\section*{The psychological problems of Miep Coelingh}

Minnaert's appointment meant that they would move to Utrecht. It remained
unclear when the family would occupy the professor's house. In December
1938, the house on Parklaan was sold for fl. 8,000,- at a loss. The
move became a plaything of the curators' and beauty committees' approvals
regarding renovation, relocation, and reorganization. Cards from 1939
show the concern of friends and family members. From sister Wil and
Marius in Norway: 'How are you, Miep? Still so tired and exhausted?
You should come here to recover!' In August, friend Truus wrote from
Bergen (NH): 'What about the moving plans? Otherwise, I'll rent a
house for you in August! See you soon...'

Miep Coelingh was under pressure. She had difficulties with raising
the boys, to whom she could give little warmth, and the boys reacted
with cleanliness problems. She must have felt guilty. Her relationships
with her mother-in-law and her own mother had not improved. Her husband
continued to care for his own mother, who lived on the estate, and
was involved in various time-consuming projects by the late 1930s.
His lectures for amateur groups and folk universities continued as
usual. The nights were reserved for the Atlas and De Natuurkunde van
't Vrije Veld. He must have been largely absent for his wife. His
career was all-determining: it decided whether the family, including
her, would move to America or Utrecht.

Her obtaining a doctorate had been extremely progressive, but it was
unclear how things would proceed further. It is possible that Miep
clung to her degree for some time. However, obtaining a doctorate
was merely a reprieve. Legislation in conservative Netherlands prohibited
married women from paid government work after the age of 30. Her subject
was unconventional, so she had little chance with private institutions.
She was too proud to take on a role in her husband's shadow. The brave,
unconventional, and combative Miep Coelingh could no longer cope.

In 1939, Minnaert noted in his diary: 'Miep must be treated by Dr.
Rümke. Initially, two months of rest in Utrecht.' After her return
to Bilthoven: 'Miep must constantly write down her dreams, which takes
up a large part of her day; also read and work on psychological books.'
Rümke was a colleague of Minnaert and a pioneer of psychotherapy in
the Netherlands. Miep had to create a life story and describe her
dreams. It had struck Rümke that her mother was absent from her life
story. Miep's friend Greet Miessen, who sometimes babysat the boys
and was allowed to read her notes, remembered that they also dealt
with her husband and sexual issues. The approach used in Rümke's psychotherapy
makes this likely as well. Furthermore, it is questionable whether
the problems with her mother were not reactivated because Miep had
to confront her mother-in-law on Parklaan.

At the end of 1939, on December 18, Jozefina Minnaert-Van Overberge
passed away. The ban on visiting Belgium was lifted earlier because
he wrote: 'The children strongly feel the loss, even though we tried
to take away their fear of death. We are all going to Ghent for the
funeral; during the ceremony, the boys will stay with the Mahy family.'
His children shared his feelings; his wife was included in the word
'all.' His mother had reached the age of 83 and passed away 37 years
after her husband Jozef.

Suddenly, the decision was made. A week after the funeral, Minnaert
rented the property at Bijlhouwerstraat 1, directly across from the
Physics Laboratory. In January 1940, they moved. On February 1st,
the last moving truck arrived in temperatures of minus ten degrees.
The diary states: 'The Pegus heating in the new house is wonderful.
The boys are attending the Montessori school in Utrecht, which doesn’t
quite suit them. Now we can have lunch together every day; I save
a lot of time.' On that same day, the renovation and relocation of
the spectrograph to the future home on Bolwerk also began.

Miep could resume her tasks. Minnaert wrote: 'Koen is often very troublesome.
We are also having him treated by Dr. Rümke, which seems to help.
Especially his relationship with Bou is a weak point. We’re taking
Bou to see Dr. Carstens because he complains a lot about headaches
and 'stomach pain.' However, this doesn’t seem to have any particular
meaning. Since there are indications of general weakness, a metabolic
investigation is recommended.' The trust in medical expertise grew
in line with the disbelief in his own psychological abilities. Half
of the residential house was allocated to the Observatory: 'These
first few months are terribly chaotic and very makeshift, but gradually
one can see everything taking on the forms we had imagined for ourselves.'
That house had recently been the subject of a curious and ominous
poem by the poet Hendrik Marsman, {*}The Zodiac of the Dead{*}. He
described the memories of an adolescent about this bastion with an
ever-observant father and a mother in mortal distress.\textquotedbl{}
\begin{verse}
\textquotedbl He no longer strays. Without him knowing it,

A magnet has redirected his step

Toward the wavy bulwark

Where the old house lies on the modest hill.

The moat forms a lasso around the park,

The morning lingers under the elm's night

The swans drift asleep in the canal.

He walks around the house

And feels the wall;

Here was his room,

Here his mother lay in her agony;

Above that tall window on the bastion

His father every starry night

Would aim his binoculars from the beach of the horizon

Along the deserts of the firmament.\textquotedbl{}
\end{verse}
A year later, Marsman drowned when the ship carrying him to England
was torpedoed by the Germans. During those wartime days, Minnaert,
who had to guard the bastion with his family, wrote: 'We sleep together
in the Observatory, which is being renovated; the boys are a bit scared
in the strange environment, creaking doors. Evacuation is being prepared;
Miep would be an acting group leader. - A large cloud above burning
Rotterdam on Sunday; the sun has a blood-red color, soot particles
and pieces of burnt paper swirl through the streets. - The German
troops' entry takes a long time, along Ledig Erf.' War had broken
out.

\section*{Ornstein removed from his laboratory}

For Minnaert, it was a foregone conclusion that scientific work had
to be continued, especially during wartime. In an anecdote from the
first day of the war, instrument maker Van Straten arrives at work
around half-past eight. He sends one of the apprentices to Bijlhouwerstraat:
\textquotedbl Professor, what should we do?\textquotedbl{} Minnaert:
\textquotedbl What's wrong, Jan, have you run out of work?\textquotedbl{}
For Minnaert, science was the highest cultural asset and, as such,
a weapon against fascism. Maintaining it was an act of resistance
for him, a mission in the struggle for a better world. Yet it also
seemed as if he retreated into his scientific domain to defend himself
against the hostile outside world.

On September 1st, the house on Bolwerk became habitable, and they
moved. On December 7, the Observatory officially came into operation.
NRC provided an impression of the renovation ordeal. The spectrograph
tower was initially supposed to extend eight meters above the roof
but ultimately placed on the flat roof in a movable metal shed when
the plans finally met the requirements of the beautification committee
and the department found the costs too high.

Classes resumed at the beginning of the academic year. Various anti-Jewish
measures came into effect. Jews had to register separately. Jewish
butchers were barred from the slaughterhouse. The antisemitic film
{*}The Eternal Jew{*} was screened. Posters reading 'Jews not wanted'
appeared in the city. On November 25, 1940, Jewish professors were
formally banned from entering the buildings: at Minnaert's faculty,
this affected Ornstein and the mathematician Wolff.

At 10 a.m. that day, twenty first- and second-year students arrived
for class at Sonnenborgh 35. According to De Jager, the atmosphere
was tense: 'Minnaert opened the class with an unforgettable speech
in his magnificent Dutch. He expressed his deep indignation over the
recent measures against Jewish professors and shared his conviction
that in the end, the forces of darkness would be defeated by the positive
forces of progress, culture, justice, and civilization. The seemingly
weak forces of the researching scholar, seeking truth and accuracy,
would ultimately triumph over crude intimidation, brutal force, and
the racial delusions of the oppressor. He concluded with a reference
to his own professor MacLeod, who once addressed his Flemish students:
Students, work! In your study rooms, you are invincible!'

The speech did not gain the recognition of those by Leiden jurist
Cleveringa or Utrecht botanist Koningsberger, who addressed larger
audiences. Ornstein bitterly resigned as chairman of the Dutch Physics
Association and even terminated his membership. He withdrew and no
longer received visitors at home. The dynamic researcher and Zionist
was a broken man at 36. He passed away on May 20, 1941.

Minnaert had undertaken all sorts of things with Ornstein for twenty
years. Together, they had established the teacher training program
from scratch. Minnaert owed no small part of his fame to the 'world
center of photometry' that Ornstein had created. On the front page
of the Utrecht Faculty Gazette of May 23, Minnaert wrote an In Memoriam
for his Jewish colleague: 'A man of great stature has left us. (...)
Sorrow fills us when we recall all that he had to endure in recent
months. But as he lies on his deathbed, the machines hum and the electric
current flows through the thousands of wires in his laboratory. His
deeds live on. His work cannot perish. His memory will remain unforgettable
for us all.'

Minnaert was indirectly involved in Ornstein's succession. The faculty
board asked him if he aspired to the position of the physicist Ornstein,
as he could claim it as an extraordinary professor. Minnaert declined
the promotion. The faculty board was pleased with this and wrote a
letter on its own initiative to the curators, advocating for Minnaert's
appointment as a full professor. Following this, J. Milatz was appointed
as professor-director.

\section*{Conference and arrest}

In June 1941, Minnaert organized a three-day meeting of astronomers,
PhD students, teachers, and students. The first Dutch Astronomers
Conference (NAC) took place at the Maerten Maartens House in Doorn.
The goal was to promote a sense of shared fate and resilience: strengthening
the morale and cohesion of the astronomical community. Astronomist
De Jager recalls that Minnaert, during an evening conversation, proclaimed
the argument that when international interest conflicts with national
interest, the former should take precedence: 'A strange remark in
those bizarre times, when the world consisted of only two types of
people: ours and the enemy.' However, Minnaert's stance distanced
him from the second Flemish collaboration. This NAC proved to be a
viable institution.

The nocturnal observations of the moon were ideal above the darkened
city of Utrecht. During the war years, Minnaert would intensively
study the moon: the only celestial body where 'individual points'
could be considered. The brightness of the lunar surface during changing
phases and angles of observation indirectly reveals information about
the geometric and physical properties of the lunar soil. Minnaert
made original use of an optical theorem by the physicist H. von Helmholtz
from his {*}Theorie der Wärme{*}: the principle of reciprocity. Assuming
that significant parts of the lunar surface have the same composition
everywhere, he could predict points of equal brightness on various
locations of the lunar disk using this principle. He then checked
whether those places on the lunar surface, whose brightness he had
determined photometrically, were indeed covered with the same type
of material. Minnaert's argument aroused interest due to the broad
applicability of the principle he uncovered.

In 1941, his {*}The reciprocity principle in lunar photometry{*} appeared.
'A systematic investigation of all published data is now being undertaken
and will be published later.' With this, he had another extensive
project underway. On photographic plates of the lunar surface, he
selected several areas of the same type, either 'plains' or mariae,
or 'mountainous landscapes,' terrae, whose brightness he compared.
Subsequently, building on the work of many observers, he derived collections
of points with equal brightness: isophotes. He transformed these curves,
which largely follow the meridians, into hypotheses about the photometric
properties of different types of lunar surfaces. According to the
Utrechts Dagblad, during the lunar eclipse of March 3, 1942, Minnaert
made 30 photographs of the moon's surface in various stages of eclipse.
Testing Minnaert's hypotheses against such photo series resulted in
an extensive publication about the presumed condition of the moon's
surface.

Sometimes it seemed as if the war didn't concern Minnaert at all.
Utrecht, like Ghent, was far from the front. Scientific activities
completely occupied him. Yet he had spoken out publicly at least twice
against the injustice done to Jewish colleagues. It didn't seem to
matter much to the occupiers. Until they arrested him in Rotterdam
and took him away in a van to Brabant via the still-unopened Maas
tunnel.\\

Endnotes:

1 Notes from the Curators of October 27, 1936. Archief-Curatoren.
De Jager, 1993, 75.

2 Struve to Minnaert, March 6, 1937.

3 Kuiper to Minnaert, March 7, 1937.

4 An extraordinary professor received the salary of a lecturer.

5 Minnaert to Struve, undated.

6 Minnaert, handwritten to L. Rutten, undated. 7 Kuiper to Minnaert,
April 12, 1937.

8 Struve to Minnaert, April 20, 1937.

9 Kuiper to Minnaert, April 20, 1937. Bok is Prof. Dr. Bart Bok, who
worked at Harvard with Donald M. Menzel.

10 Utrechts Dagblad of June 17, 1937.

11 Pannekoek to Minnaert, June 20, 1937. 12 Het Utrechts Archief,
number 464, employee salary statements university.

13 Minnaert, 1937. The little blue book was sprinkled with gray drops,
so the cover seemed like a starry sky. This discovery was accidentally
repeated on the ceiling of the study room in the Minnaert building
(1998).

14 Nieuw Vlaanderen, November 13, 1937.

15 Dedeurwaerder, 2002, 573.

16 Hugo Claus' Het Verdriet van België deals with this, beautifully
depicting Flemish sorrow, and Erwin Mortier's ( nota bene! ) Marcel.

17 De Jager, 1993, 76.

18 De Jager, 1993, 78.

19 Minnaert's lecture notes: The Earth and the Planets and Sun, Stars,
and the Universe. These were very progressive at the time.

20 Interview with Henk van de Hulst.

21 Interview with Kees de Jager. Also {*}A Man After My Own Heart{*},
for a Studium Generale in Utrecht, in Haakma (1998).

22 Minnaert and Bannier, 1936e.

23 Mulders to Minnaert. File with letters from 1936-1937 in the astronomy
archives.

24 Minnaert and Houtgast (1938c).

25 Minnaert's retrospective on {*}Fourty Years{*} in: De Jager, 1965.

26 De Jager, 1993, 49.

27 Coelingh, 1938.

28 Wil Coelingh and her husband Marius, undated from 1939.

29 Truus Van Cittert-Eymers to Miep, August 3, 1939.

30 Her bosom friend Truus Eymers was dismissed in 1938 when she married
her colleague Van Cittert. M.I.C. Offereins: Johanna Geertruida van
Cittert-Eymers, NVOX 8, October 1997.

31 Diary, March 29, 1940.

32 Hendrik Marsman, {*}Tempel en Kruis{*}, 1939.

33 De Jager, 1993, 44. 34 NRC, December 8, 1940.

35 De Jager, 1993, 45.

36 Heijmans, 1994, 156. Van Walsum, 1995, 43, claims that Ornstein
committed suicide.

37 In the professional journals of 1941 and 1942, respectful biographies
of Ornstein by Kramers and Moll were published.

38 Signed by chairman W. Barrau and secretary V.J. Koningsberger.

39 De Jager, {*}A Man After My Own Heart{*}, in Haakma, 1998.

40 The NAC has become an institution; now together with Flemish colleagues.
The 1993 conference was dedicated to the 100th anniversary of Minnaert's
birth.

41 Helmholtz, {*}Theorie der Wärme{*}, 1844, reprinted in 1922. 42
Minnaert, (1941).

43 Utrechts Dagblad, March 4, 1942.

44 These must have been processed in a review article by Minnaert
(1960) in the astronomy series of his friend G.P. Kuiper: {*}The Photometry
of the Moon{*}.

\chapter{A Prisoner-of-War Camp as a People's University}
\begin{quote}
'Skilled engineers will find work in abundance: the world is waiting
to focus on peace and start rebuilding, and it will be wonderful to
be able to contribute to that.'
\end{quote}

\section*{Hostage and resistance}

On May 4, 1942, the Germans had arrested Minnaert. On May 8, Miep
wrote to Minnaert's niece Marie: 'At 12 o'clock, they took him out
of his final exams in Rotterdam and brought him together with 80 other
South Hollanders; at 5 o'clock, they were loaded into large trucks
and driven to North Brabant. He had nothing with him: no toothbrush,
no pajamas, no food.' She suspected that 'former friends' had put
him on a list: 'I am experiencing a lot of kindness, both from friends
and acquaintances and even from complete strangers. If anything can
teach us Dutch people a sense of solidarity, it’s these methods!'
She thought of collaborating nationalists who wanted to settle a score
and, when asked, said: 'I went to the Sicherheitsdienst in The Hague
and asked to speak to the head of the department for 2.50 guilders,
which I visibly stuffed into a small hand towel. I was allowed inside.
I provoked them: 'I don’t understand why you’ve arrested my husband.
He is not anti-German.' That was true, because he was anti-fascist.
Then the SD officer picked up a thick book, looked it up, and said:
'Your husband made a communist fist at De Clercq’s funeral.'

Minnaert’s arrest was no coincidence; neither was the entire operation.
The occupiers arrested 600 men who played a significant role in society
and brought them together at the Beekvliet seminary in Sint Michielsgestel.
Something similar had already happened shortly after the conquest
of the Netherlands. The first group of hostages, initially detained
in Buchenwald, was meant to prevent the Dutch government from taking
revenge on Germans in the East Indies. After Japan’s conquest of the
island kingdom, that motivation disappeared. Still, in the spring
of 1942, a new group was added to the 'Indische gijzelaars' (East
Indies hostages). For the Germans, both groups, who stayed together
in Beekvliet for several months, served as insurance against spectacular
resistance actions and a popular uprising. The Germans could as reprisal
measure execute an arbitrary number of hostages.

The hostages lived in groups in the rooms. Minnaert's group of four
called themselves the kongsi. The professor was extremely sociable,
shared the contents of the family and observatory packages, and cleaned
the toilet every morning before dawn. Roommate Roest once received
two large turnips, cut them into pieces, cooked them, and found them
inedible. The kongsi watched in amazement as the scholar devoured
the turnip with relish. A caricature of Miel Prager added to the fun
afterward. Camp life was sheltered and in no way comparable to the
situation in concentration camps.

Immediately upon his arrival, Minnaert had known what he had to do.
His mission was to forge the gathered group into a solidary community
through study. The Romanist Brugmans wrote: 'Minnaert had become accustomed
to the idea of a long stay and noted names in a small notebook with
very tiny letters. Names and subjects. For lectures, courses, and
lessons. He always had a new collaborator and his eyes sparkled when
he nodded vigorously and noted: \textquotedbl Yes! Yes! Very interesting,
really VERY interesting! So we'll arrange...\textquotedbl{} Some laughed
at him when he made plans for three months ahead. But if one is used
to thinking in light-years, oh...' His equal was Robert Baelde, a
descendant of a Rotterdam aristocratic family, former student of the
Erasmiaans Gymnasium, leader of the Volksuniversiteiten and the Nederlandse
Unie.

Minnaert became chairman and Baelde adjutant of the lectures and courses
committee. As early as May 9, the writer Anton van Duinkerken gave
a lecture on Brabantse humor. Brugmans: 'We laughed. Hearty and unselfconscious.
But also with something triumphant, something challenging, as if we
wanted to demonstrate that they couldn't break us.' The adage from
Julius Mac Leod, 'in your study space you are invincible,' it came
true: the entire camp became a people's university. The foundation
for the later 'spirit of Michielsgestel' was laid in those first weeks.
On May 11, registration began for the many dozens of lectures, courses,
and films. By late May, Minnaert asked the people from his workshop
to manufacture brackets with which he could mount blackboards on the
walls. He provided the instrument maker N. van Straten with precise
measurements and working drawings.

After the August 1942 bombing of a railway line in Rotterdam, the
Germans carried out the first executions. On August 14, this involved
the Rotterdammers Robert Baelde, Chris Bennekers, and Willem Ruys,
along with Otto Count of Limburg Stirum and Alexander Baron Schimmelpenninck
van der Oije. Minnaert had feared for his life and made this known
to the people at the Observatory: 'The tension here is naturally very
great, but people are keeping themselves dignified and composed. One
cannot know how this will end. My best wishes are with you; may your
activities at the Observatory bring you good fortune and much satisfaction.'
These words had caused great consternation.

Before the executions, Minnaert had spoken at length with the Flemish
photographer and writer Paul Guermonprez, who had become his best
friend in the camp: 'We have considered where we, atheists, must draw
our strength from. We sat together with a few people. When he spoke
about his wife, he could not control himself. We are mortal - our
actions live on, they are immortal, insignificant when viewed individually,
important within the complex of the great whole.' That had been Minnaert's
comfort. The death of Baelde had moved Minnaert deeply. Who would
be next in line for new executions? Minnaert had reason to suspect
that he was at increased risk: former associates like NSB leader Robert
Van Genechten, who had become an Officer of Justice in The Hague,
could easily take revenge on him.

\section*{Minnaert and the circle work}

The Utrecht journalist P.H. Ritter jr., like Brugmans a flamingant,
wrote that Minnaert's nature easily maintained itself in the camp:
'At Beekvliet, I met him reading, writing, noting, measuring on the
edge of sloppy tables, amidst deafening noise. Surrounded by gesticulating,
chattering, smoking, and billiard-playing people, this work saint
stayed as if in an invisible chapel of attention.' Minnaert always
moved at a trot, rushing from one task to another. In Ritter's loving
and slightly idolizing sketch, it is clear that Minnaert devoted much
energy to personal assistance, which he considered part of his duties
in Michielsgestel: he uplifted people and unwaveringly showed faith
in a better future.

He continued his moon research, which he could pursue thanks to the
many shipments from his assistant Houtgast, edited the 'Astronomy'
series from publisher Servire, and contributed to Oosthoek's encyclopedia.
He corresponded intensively with his wife, children, friends, and
colleagues. With his friend Jan Burgers, he began a philosophical
correspondence on 'free will.' As in Utrecht, he was short on time.

He gave a 45-lecture course on astronomy in the camp. According to
Brugmans, Minnaert became a defining figure in the camp also through
these lectures. He contributed to the course 'The Origin of Earth,'
where his colleagues J.J. Kloppert, V.J. Koningsberger, and N. Tinbergen
presented as well. The lectures on 'Modern Painting' by Dr. G. Knuttel
Wzn., director of The Hague Municipal Museum, inspired him to give
a lecture on 'Artificial Light and Architecture,' drawing from the
dissertation of his friend Truus Eymers. In addition to courses in
navigation, he taught Swedish to both beginners and advanced learners
and assisted with H. van der Heyden's carpentry course. Using slides,
he held talks on topics such as 'Light and Color in the Dutch Landscape,'
'The Moon in Folk Life and Science,' 'Balloon Flights in the Stratosphere,'
'Does Music Have Meaning?,' 'Selma Lagerlöf’s Gösta Berling,' 'The
National Anthem Through the Ages,' and 'Musical Instruments.'

On New Year's Eve 1942, the appreciation of this community was expressed
in play of Thomasvaer and Pieternel:

T: 'But aside from entertainment, as is said in the country, many
have dedicated themselves to study here. A chief study coordinator
will therefore certainly not be lacking, for they must have looked
out for a professor for that purpose.'

G: 'It is Minnaert, who is however called the course coordinator,
an astronomer, also appointed as a darkening specialist. Besides darkening
the windows of the lecture rooms, he works with Sanders on countless
courses together.'

T: 'Thus, many who previously knew neither God nor sour apples will
soon leave the camp as all-around encyclopedists.'

In the second camp year, the 'circles' were formed, which would map
out post-war Netherlands. Co-opted groups of politicians, industrialists,
and civil servants discussed party formation, education policy, and
political renewal here. The transitional Schermerhorn cabinet was
formed here. Minnaert participated in the circle on education policy
and was also a member of Knuttel's Arts Circle. This group consisted
of seventeen selected participants, including composer Hendrik Andriessen,
Simon Vestdijk, and classicist Onno Damsté. Minnaert had Ibsen's Norwegian
text of Peer Gynt brought over, made his own translation, introduced
Grieg's music on the piano, and led the piece with three lectures.
Afterwards, it was 'role-wise' performed on a Saturday evening, with
Minnaert connecting the roles through speech. Sanders wrote: 'Seldom
have we been brought so respectfully and closely to a work of art.'
Others spoke of a hilarious performance. Notably, Minnaert had a preference
for Peer and Gösta, the two great rogue figures from Scandinavian
literature who sell their souls to the devil and leave their loved
ones waiting in vain for a lifetime.

At Easter 1943, the carpentry group organized the exhibition What
Hostages Are Making Now. The croquet game, 'made by Prof. Minnaert
himself in the carpentry workshop,' was considered the highlight.
His children gratefully tried it out on the lawn of Sonnenborgh. He
was also a member of the exclusive Travel Club, whose members organized
lectures with slides: this was the prelude to the public Sunday afternoon
lecture Gestel on Travel.

\section*{Brugmans on Minnaert}

All of this can be inferred from publications and correspondences.
A more intimate glimpse was provided by a conversation with the Romanist
Henk Brugmans. He could vividly recall Minnaert at the camp and found
him to be lonely. Brugmans described him as somewhat naive, an independent
communist, for whom 'communism' meant that everything should become
collective and self-interest should disappear. The joy of labor was
lacking due to capitalism. If capitalism were gone, things would be
very different.

Minnaert had once approached Brugmans on his own initiative: 'You
are a social democrat\ldots ,' he began. He explained that a new way
of living together could only be achieved through revolution. Brugmans
responded by saying that a better society could not be realized with
the help of a dictatorship. Their opinions clashed, but Minnaert had
no problem with that. The conversation broke down on another point.
Minnaert had heard that Brugmans had 'become a Christian.' Brugmans
confirmed this. Minnaert then said: 'Is that your final word? Then
we can end the conversation.'

Brugmans added: 'Minnaert strongly believed that the Catholic Church
had condemned Galileo. As an institution, the Church was hostile to
intellectual development. The fact that Flemings were Catholic was
bad enough. That an intellectual like myself was not an atheist, he
found unbearable---I was putting reason aside. He professed atheism;
for that, he had faced the stake. As a conversational partner, you
were depersonalized. He didn’t say: 'Tell me, Brugmans, you’re not
that foolish, how do you see it?' Otherwise, he was a kind personality.'

Brugmans found Minnaert an excellent teacher: 'In workers' education,
you must reduce difficult issues to comprehensible conclusions without
simplifying them. Your listeners want to make an effort, but there
must be a result to match that. Minnaert was an excellent popularizer.
He had learned this in the Flemish Movement. He wasn’t the only one
working all day, though. I was working on my book about European culture,
which was meant to answer the question of how the German mind could
have been receptive to Hitler. The difference was that I engaged in
numerous discussions about it, while Minnaert worked in isolation
on his publications about astronomy.

According to Brugmans, Minnaert had a close friend, Paul Guermonprez,
a photography instructor at The Hague Academy of Arts. They were like
a clan of two. I remember a lecture about the Casablanca Conference
in January 1943, where the Allies decided to continue until the unconditional
surrender. It sounded appealing, but I thought it was a gift for Goebbels.
Any opposition would lose its footing. Guermonprez said that National
Socialism had to be eradicated root and branch. I can’t imagine Minnaert
thinking differently about it.

In 1936, Minnaert had described himself as a left-wing socialist;
according to Brugmans, he had since moved further along. While authoritarian
Flemish nationalism committed suicide by siding with Hitler in the
Soviet Union, Minnaert seemed drawn to the ideological principles
of communism. The fact that Stalin was engaged in a life-and-death
struggle with Hitler likely did not diminish his sympathy. His pursuit
of 'mutual service' appeared to align in philosophy, science, and
politics with the struggle for a socialist society.

\section*{Miep’s hospital admission and the family problems}

Minnaert would remain captive for two years. During that time, he
received three days of leave on three occasions. On February 12, 1944,
his 51st birthday, he was allowed to receive his wife and children.
His 50th birthday had been celebrated with the group. The occupiers
restricted the frequency and size of the censored letters. Minnaert
dedicated parts of the letters to his wife for Koen and Boudewijn.
His children preserved those strips. Minnaert's letters to Miep's
friend Greet Smit-Miessen were also kept. The little that has been
preserved provides an image of what happened at home.

The Minnaerts would have gone to Dalfsen that summer. Greet had invited
Miep to join her, but during the vacation, Miep had suffered a sciatica
attack and could barely move. On August 9, 1942, Greet received a
response to her letter 16 'from Marcel': 'I can so well understand
how you feel about psychoanalysis and dream interpretations; for more
than three years, I have experienced this myself, and while I highly
value the scientific insights of this field, I somewhat doubt its
value as a remedy. But ultimately, I don't know much about it.'

In October 1942, Rümke had decided to admit Miep Coelingh to a rest
home. Greet then offered, together with her husband, physicist Hans
Smit, to take care of the children and the observatory. On December
17, Minnaert wrote to Greet, who had sought his help: 'You ask how
we can best help Miep regain her will to live. Rest is first necessary;
afterward, she must feel that she can do something, develop self-respect
and self-worth. If she notices that others love her, it already helps:
When I am dear to someone else, then I am more dear to myself. But
something else is also needed to captivate her, something she can
do well and with which she can achieve something in not too long a
time. I see the most opportunity in the direction of interior architecture.'

At the end of December, Minnaert was granted three days of leave and
took stock. He thanked Greet in an open-hearted letter, providing
guidelines for upbringing: 'You know that Miep is the great love of
my life, which becomes ever more complete and pure. But you have naturally
long felt, with a woman's intuition, that I have loved you a little
since the first time I saw you. This means that in my friendship for
you, there is an added quality that always gives our conversations
and correspondence a special charm. Is it any wonder that I trust
you where it concerns the upbringing of our boys? Because you understand
me, I may occasionally write down what seems to me to be specially
considered: you will not regard this as criticism, but as comradely
consultation.

May I then mention two things? 1. Let us give Bou and Koen our full
understanding and love. Bou's childlike nature is more immediately
striking; I feel a deep connection with him; but my love goes just
as much to Koen. At the moment, he misses his father and mother in
a more conscious way than Bou; let us respect his first idealistic
thoughts and try to make them purer, more beautiful, and more substantial.
2. Let us always represent to them their mother's love as continuously
present. For that is how it is. Miep loves them, she loves them passionately,
even if you wouldn't say so, and even if she herself denies it. The
danger against which I am warning you I have felt threatening myself;
I found it so painful that the children had to miss expressions of
maternal love that I was in danger of displacing Miep. That must not
happen! It would be an irreparable loss both for Miep and for the
boys. Yet, do comfort them with all the warmth they have missed for
so long---please do that, let them feel plenty of sunshine around
them! But keep talking about Miep, about Miep who will soon return,
keep memories of her alive. I have not for one moment noticed that
you would act otherwise than in this spirit; I write this only because
I know the danger myself.’

Minnaert revealed to Greet the dynamics of his family: he denied any
shared responsibility for his wife's illness. About Christmas in the
camp, he wrote: ‘For us unbelievers, such days are always a bit harder
than for others: we cannot join in the great community, which we would
nevertheless like to do. I want to write especially now to the people
at home, read in {*}Pan{*} and work on a sketch about life's problems,
in which I try to outline modern man's faith in life.’

On January 6, 1943, Miep Coelingh wrote to Greet that resting was
not working out. She wasn't getting anything nice to eat: not even
a biscuit, even during Christmas. She made a wish list of food items.
In the seven-page letter, there were no questions about the boys:
only at the end 'many greetings for you four.' From a few letters,
it appears that the relationship between Miep Coelingh and her friend
Smit-Miessen became strained at one point.

On January 26, Minnaert wrote to Greet: 'Fortunately, I had already
written a note to Koen when a note from Miep arrived, which makes
me so gloomy that I need all my strength to endure it.' But, Minnaert
wrote, 'the plan of the friends coming to eat and stay is wonderful!!'
A birthday party for Koen: perhaps the children had never experienced
anything like it. On February 2, 1943, Marcel asked Greet for help
organizing his fiftieth birthday in the camp. He needed money to treat
the communal meal and added for the Sterrenwachters: 'If anyone wants
to give me some music, then: César Franck: Eros and Psyche (preferably).
Or works by Björnsen in Norwegian (Arne; Arnljot Gelline), R.M. Rilke:
Die Weise von Liebe und Tod des Cornets Christoph Rilke, A. Schweitzer:
J.S. Bach, Le musicien-poète. Or from the Palet series: Breitner (I
already have Scorel).' On February 3, Amice Minnaert received a note
from Rümke, in which he wrote that 'Miep' was doing well.

On March 20, Hans Smit wrote a long letter 'to Marcel,' complaining:
'I consider a longer stay of Koen and Greet unsuitable: 1. because
I (very simply) do not want Greet to be repeatedly insulted; 2. because
the situation that Koen can continue doing this without sharp reactions
is educationally incorrect.' Minnaert wrote back that it was useful
that he knew about the problems: 'I will do what I can.' He maintained
an intensive correspondence with his sons, after all. On March 22,
1943, Miep Coelingh could take charge of the family again. The Smits
had held out for five months; it seemed that it wouldn't have to last
much longer.\textquotedbl{}

\section*{Koen: the world is waiting for builders...}

In over 700 days, Minnaert sent Koen 111 strips of paper and letters.
He told him about what he was doing and things that seemed interesting
to him for his eldest son. He asked about the twelve-year-old's new
gymnasium and the names of his new friends. The children looked forward
to their father's heartfelt mail, which they promptly answered so
that he could connect with their experiences. Back then, the mail
took a day or half a day because there were once two or even three
postal deliveries a day. Minnaert reminisced about his own memories,
for example, that in Flanders, instead of a pencil case, he had a
small container when he went to the atheneum. He drew the equipment:
'It contained a pencil, eraser, penholder, etc. We also had a 50 cm
ruler, which was especially useful as a \textquotedbl sword\textquotedbl{}
in fights! And everything together was in a bag with a long strap
that we wore over our shoulder. That strap was also very useful; you
could swing the bag until it landed on someone's head with a thud!
- But today's youth is so much more peaceful, isn't it?' Of course,
he continued the relationship he already had with the children. Koen
was jealous, troublesome, sensitive, and stubborn but showed scientific
interest, which Minnaert skillfully encouraged. On August 31st, he
responded to Koen's first Latin sentences. The poor boy had written
in the best gymnasium tradition: 'presemus feminis': 'We have precedence
over women.' His father: 'That sounds so unpleasant that I can hardly
believe you mean it.' Minnaert told about the way he had shown the
solar eclipse of September 1942 to the hostages: a pasted-on eyeglass
with a focal length of two meters had neatly depicted the eclipse.

When Koen had written that he had borrowed 'Aktaion under the stars'
from the library, his father could answer: 'Vestdijk, the writer,
one of our best young literary figures, is also here in Gestel.' The
figure Cheiron inspired wisdom about life: 'Every human being is actually
more or less a centaur. One part of us is wild, wants to devour greedily,
fight, be rough; but another part is human, civilized, familiar with
art and science. The connection of those two groups of feelings in
humans creates all kinds of contradictions, just as the duality of
the centaur does.'

At the beginning of 1943, he had consulted his roommate, a Delft engineer:
'I thought you would feel most for mechanical engineering. The future
possibilities after the war will certainly be great. Capable engineers
will find work in abundance: the world is waiting for peace to start
rebuilding, and it will be wonderful to be able to contribute to that.
Especially important is that you grow up to be a decent person who
is happy and makes others around him happy, who dares to do something
and can do something, who has ideals and strives for them. I promise
you that I will do everything I can to help you develop a genuine
beautiful technical hobby, and Mother thinks the same way.' Here Minnaert
followed in the footsteps of both his parents and repeated patterns
from his childhood. For both boys, he wrote in May 1943 a 'history
of mechanics' in six installments with numerous drawings and portraits.
The story began on May 22, 1543, exactly 400 years ago, with Copernicus
and Rheticus, and continued through Galileo, Tycho Brahe, and Kepler
to Newton. He honored the commemoration of the Pole in the camp. These
strips were a rich possession, and it's no wonder that both boys preserved
them.

Minnaert wrote to Koen about the carpentry course that was organized.
The students had to saw a piece of wood perpendicularly, straighten
one side of a plank, mark and straighten a cross-section with a carpenter’s
square, make the surface of a plank smooth, and learn how to make
wooden joints. Then you could make a letter stand: 'Don’t you feel
like joining that course from afar?' Minnaert sent Koen the drawings
and told him stories about them. He also sent two caricatures of himself.

In June, exactly two years after the German invasion of the Soviet
Union, he replied: 'What do I hear? Have you read Max Havelaar? That
knocks me over! I think it’s a wonderful book, but I didn’t expect
you to appreciate it at your age. Wonderful---the story of Saïdja
and Adinda, the speech to the chiefs of Lebak, the chapter about the
buffalo, and Reverend Wawelaar. And then that brilliant ending! What
a writer, what a man! Do you know where he wrote the book? In the
cafes on the Grand Market in Brussels. Multatuli once gave a lecture
in Ghent. He was wearing a top hat when he entered the hall. Suddenly
he threw the hat on the ground and kicked it!' Then he wrote about
his friend, the journalist Philippus Roest, who had asked him to create
an ex libris with a Latin inscription. 'I came up with all sorts of
ideas. From my proposals, he chose this one: Paululum Ride (= laugh
a little from time to time). Roest is quite a joker, which seems suitable
for him. You see what Latin is good for! It’s really a language to
express something briefly and powerfully in two words.' After the
intervention of the classicist Damsté, it became paulum rideas.

From the vacation address in Hardenberg that Minnaert had been able
to arrange through a fellow inmate, Koen had written about his unfamiliarity
with 'praying before meals.’ Minnaert wrote that this was an ‘old
custom’: ‘Even here in the camp, my fellow prisoners always pray before
eating, and of course I wait until they are finished so we can start
eating together pleasantly. You must understand that in such an old
custom, especially among country people, there is something very human
and even poetic to be felt: the farmer imagined that the rain and
sunshine were controlled by God; he knew nothing of the natural laws
that determine the weather.’ Minnaert managed to continue the education
and give the children the warmth and attention they deserved. Intimacy
was part of this: ‘You would probably look up in surprise if you could
take a look into the little hut where we sleep and if you were to
open my cupboard. Your eye would immediately fall on your portrait
and Bou’s, and underneath lies a list of your classmates, family birthdays,
and Star Watch. In my wallet is Mother’s portrait - that’s what I
always want to carry with me; and along with those two photos from
when you were little. That way, I continue to live in thought with
you.’

On August 20, 1943, Koen received a model of the gearwork of Eise
Eisinga’s planetarium in Franeker, which Minnaert had sketched and
built for a lecture. This was a rich gift that the sensitive, technically
gifted boy must have appreciated. Koen also received a family tree
of the Minnaert family with the comment: ‘You know there are people
who attach great importance to their family tree and spend a lot of
money on genealogical research. We don’t feel that way about it. What
matters is whether a man is decent and good, not whether his ancestors
were rich or famous. Preserving those family memories is therefore
interesting in our eyes, but not in such a way that you could see
from it what we are worth. Similarly, we find all the pride of noble
families inappropriate: these people are no more valuable than others
just because their ancestors were knights.’ After Koen told about
the funeral of his rector, Minnaert wrote: ‘For a long time, people
there have seen something ghastly, gloomy, and mournful in funerals;
only in recent years has one begun to turn such gatherings into something
beautiful. Isn’t it truly beautiful to think together once more about
the one who has left us? To remember everything he gave us in terms
of kindness and camaraderie? And when you return, you feel the need
to enjoy the life you’ve been given: you want to set your hand to
work in the spirit of the man you just commemorated.\textquotedbl{}

Minnaert even once called Koen for help following something he had
heard ‘in our educational circle about Montessori schools. The advantages
and disadvantages were examined. There was something that struck me;
there were people who claimed that you don’t learn to listen at a
Montessori school. What do you think about that? Now that you’ve come
to a Gymnasium after a Montessori school, you can certainly notice
the difference. If it’s truly the case that Montessori students don’t
know the art of listening, then by all means make an effort to master
it soon! It’s essential to fully immerse yourself in what the teacher
is saying and explaining, and ensure that your thoughts don’t wander
off in the meantime! Especially at the University, this art of listening
is of the utmost importance.' It is remarkable that this objection
could surprise the educationalist Minnaert so much. Koen’s response
has not been preserved.

\section*{Cataloging Father's bookcase}

On November 30, 1943, Koen proposed to his father that he catalog
his bookcase, after Minnaert had earlier pointed out its contents
to him and Boudewijn. He responded hesitantly: 'My books are indeed
a precious possession for me; they are for me what the saw and the
chisel are for the carpenter: tools to work with, thinking tools!
I highly appreciate your willingness to take good care of this valuable
collection. You will receive further instructions about it.' A few
days later, Koen received a guide on how to catalog a field, such
as physics or astronomy: 'I imagine you leave all the books as they
are but write down all the authors' names on a scrap piece of paper.
Then you number them in alphabetical order. Then you take the script
and write the names with titles alphabetically according to the names.
If you prefer typing, you could take writing paper sheets that will
later be compiled together.' He asked Koen what he thought of the
proposal. Koen readily agreed.

With his father, Koen had ended up in a serious project. After several
letters, number 41 promptly stated: 'In your last letter, you write
that I misunderstood you, that you proposed organizing the catalog
alphabetically by authors. For curiosity's sake, I'm sending you back
your penultimate letter; you'll see that nowhere is there any mention
of a catalog organized alphabetically by authors. The etiquette---which
you so eagerly want to learn!---requires that in such a case, you
write: 'I fear I did not express myself clearly enough,' even if you
might think the other person didn't read carefully! Organizing alphabetically
seems fine to me. But then per subject. You really have no idea how
much work it is to create such a catalog. That would cost you so much
time that school, music, or hobbies would seriously suffer from it.
If you start with the whole thing and abandon it halfway, we won't
get anything out of it. However, if you do one subject, maybe a second
later, the catalog already has use. For expansion: a rule open after
each book. Is that right?---So everything is then well agreed upon.'
That was quite a reprimand for a 13-year-old. What did Minnaert actually
care if the boy stopped after a few attempts? The father seemed to
take it even more seriously than the son. Maybe Minnaert thought it
was good to teach Koen discipline. Anyway, Father Minnaert suddenly
forgot to approach Koen as a child, while he had done so perfectly
in his strips for a year and a half. Koen, however, was brave and
didn't give up. This cataloging became the main theme in the correspondence.

At the beginning of 1944, Koen received a thick insufficient grade
in Latin. Minnaert admonished: 'In these times, when the world is
upside down, it is absolutely necessary to use the period of quiet
learning and working 100\% efficiently, because you don't know what
the future will bring, and only if you are good at your work will
you stand strong in the world later. And now we shake hands, look
each other in the eye, and agree that the next report will be a good
one. Won't it? Warm regards from Father.'

The first round of Koen's titanic work was crowned around Minnaert's
birthday with a typed catalog. Father thanked him for the eight-page
letter: 'I've already enjoyed looking at the catalog; it's truly a
beautiful achievement. May I mention one small thing to pay attention
to in the future? Nouns in German should be capitalized! But that
doesn't hinder practical use, it's just a minor flaw. It's nice that
you managed to get a new sheet ready for Mother's birthday.' It's
remarkable that his mother didn't point out those German capital letters
either.

Koen's Easter report had no failures. He had three eights and good
grades 44 in the languages. There was a six for mathematics, which
needed to improve if Koen wanted to advance in technology: 'Don't
think there will be time later to work on it more. No subject shows
such strong internal consistency as mathematics; if you don't know
the basics very well, it will come back to haunt you again and again.'
45 Also, the last two letters to Koen were about the catalog. On April
20, Minnaert was unexpectedly released from the camp. With that, the
project was likely off the table. It's striking how the relaxed tone
of the correspondence with Koen became more businesslike when the
child entered the father's territory. Koen's intention was considered,
his effort incorporated into the 'program' of the father, and coolly
assessed for its practical use. It seemed as if Koen, who had turned
14 had accelerated into adulthood. How different the correspondence
with Boudewijn would develop!

\section*{Boudewijn: 'no need for earned conversations'}

Of the strips and letters to Boudewijn, 122 have been preserved. It
seems Minnaert approached his youngest, who was one year younger than
Koen, more openly and dared to be more vulnerable. He evoked more
memories and appealed to shared emotions: 'Do you remember our house
in Bilthoven, and Oma, and Miss Anna, and how we were treated on Sundays?
And our walks and bike rides, so many over all those years, making
all those roads and paths as familiar as dear friends? And our trips,
the highlights of the year: Zandvoort, Wageningen, and The Hague?
I also think of our little home celebrations, with St. Nicholas, New
Year's, and birthdays. Your first bike, swimming, ice skating. That
time you came to pick me up from the boat in Amsterdam after my trip
from France. Boy, did we have a good time.'

He also envisioned the future for Bou: 'More than once I dream that
you've become a geographer, wandering the world, enjoying nature everywhere,
observing and drawing with sketchbook and compass in hand. It's a
wonderful profession... But I can also imagine our Bou differently.
For instance, as a literary figure. Beautiful stories would emerge
from you without you knowing it; you'd see the heroes and heroines
of your tales come to life before your eyes. The more you think about
them, the clearer you'd hear what they say. And then you'd write it
down... There's so much beauty and wonder to be found for anyone willing
to roll up their sleeves!'

A few days later, he returned to 'something very close to both our
hearts: enjoying nature. If only I could explain how much happiness
I've had in my life thanks to the surrounding landscape, observing
plants and animals, traveling, and wandering! Could you but understand
how much happiness I have owed to the landscapes around me, to the
observation of plants and animals, to travel and roaming!'

The simplest way to enjoy nature is to go out into it without worries,
with sandwiches in your bag, wandering through rain and shine, across
hills and plains. Then there’s no need for learned conversations,
just observe, listen, smell, and feel how beautiful the landscape
is in all its forms, whether in the morning, evening, or afternoon.\textquotedbl{}
Minnaert shared his own experiences: \textquotedbl You start paying
closer attention to the plants and flowers, and then you discover
that each cluster of flowers is like a tiny world unto itself; each
flower is like a city, with streets, squares, and beetles walking
around as if they were inhabitants. All these flowers together pay
attention to one another, opening one after another, ensuring they
don’t take too much light from each other. Flora, like the works of
Heimans, Heinsius, and Thijsse, not only tells you the names of the
flowers but also encourages you to examine the entire plant thoroughly;
once you’ve carefully observed a plant, you’ll never forget it. I
could also tell you how much I love clouds, fossils, and erratic boulders,
and so on.\textquotedbl{}

After that, he switched to what Bou would probably love to hear: \textquotedbl The
most complete form of nature enjoyment for me is traveling. And on
such a journey, you’re not just looking at nature but also at what
people have created: their farms, churches, dikes. Also their dialects
or language. Gradually, you begin to feel how all these things belong
together. I could only dream of the Mediterranean coast with a French-
or rather Italian-speaking population; along with that come plants
that love sunshine and a mild climate, and villages on mountains with
narrow streets and picturesque church towers. And so every country
forms a whole. People often say, 'As many languages as you know, as
many times you are human.' I would say, 'As many countries as you
know, as many times you are human.' A journey is a highlight of life;
I’m certain that until my old age, I’ll enjoy the unforgettable memories
of the beautiful world, which holds new charm in every country.' This
is the third letter, and there’s a more intimate tone: it reveals
a romantic, wandering Minnaert who dares to allow himself and his
son desires. He wrote: \textquotedbl When I read your letter, I had
the greatest urge to grab you and give you a kiss.\textquotedbl{}
Then again, he asked Boudewijn to fetch the photo books: \textquotedbl There
are only a few pictures where you’ll see me in them. Judging by the
photos, you might almost think I wasn’t there at all. But you know
better. Later, especially, you will understand how happy I was when
we made our trips together, how much I enjoyed your childhood, and
how a little of my love for you is connected to each of those snapshots.\textquotedbl{}

His wife perhaps asked him to enlighten their twelve-year-old sexually.
Minnaert, 50, promised, \textquotedbl to tell you about such things
as love, marriage, having children, etc. Of course, you've already
heard some things about it, we've even talked about it together before,
but still, it might be good for you to read everything all at once.\textquotedbl{}
In three letters, he enlightened his son, starting with plants and
moving on to humans, ending with mammals and birds. An original order.
It was a clear biological story, written honestly and respectfully:
\textquotedbl That's how all of this works. There's much more to
tell about it. For now, don't discuss this with anyone other than
your mother. And you should already understand this: it is such a
beautiful thing to have a child that you are only allowed to do so
with a woman you care deeply for. It's incredibly wonderful how two
people together can create a new human being. How is it possible that
some people or schoolchildren laugh about these things or tell silly
stories about them!\textquotedbl{} If Bou didn't understand something,
he could write to his father or ask his mother: \textquotedbl You
know you can always count on an answer.\textquotedbl{} He was touched
because Bou had just written that his letters meant a lot to him.

Minnaert heard about Bou's choice to attend the gymnasium but thought
he would be happier at a school that demanded less: \textquotedbl Well,
I hope you'll do well there.\textquotedbl{} When school began: \textquotedbl I
hope you will make good friends, who will also be of value to you
later. Don't first look at whether a boy is a bit funny; don't just
look at whether he's particularly handsome; but look at whether he
has character: that's what determines someone's worth. I mean: become
friends with boys who are honest, cheerful, and brave.\textquotedbl{}
He casually suggested, \textquotedbl Do you already know your way
around the bookshelves? Ask your mother if you are allowed to sniff
around. Pay attention to how the books are arranged according to subjects.
Even within each individual section, there is a more or less fixed
order; so try to put the books you take out back in their proper places
each time. For example, you could look at Astronomy or Dutch, or Botany
and Zoology: just look at titles that interest you, check out the
pictures or the table of contents, so you know what you can find in
such a book.’ Unlike Koen, Boudewijn ignored this invitation. In the
fall of 1943, Minnaert wrote a series of letters about his memories
of his father and mother and his youth in Flanders. He had never told
about the outcome: 'We were sentenced to prison, but since we were
gone, this had no effect, only we could not return to the country.’

Bou had received a four in Dutch during the winter of 1943, probably
mainly due to spelling mistakes. His father commented: 'Your letters
are just as welcome to me with 100 spelling mistakes as if they were
perfect; perhaps even more so.’ Koen would not have gotten away with
that! He wished Bou for 1944: 'That there may be a good peace, leading
to a happier and safer world. That Mother would fully recover, as
she already almost is, and that she wouldn’t have too much trouble.’
While Minnaert’s correspondence with Koen became more impersonal,
the one with Boudewijn seemed to become increasingly confidential.
He shared a lot about himself in both correspondences: the children
could certainly count on his attention and care. Conversely, this
correspondence was a means for Minnaert to continue influencing and
directing the world of Sonnenborgh. He also remained very precisely
informed about the management of the Observatory while in the camp.

\section*{The Observatory and Michielsgestel}

At the Observatory in 1942, the staff included head assistant J. Houtgast,
secretary Ch. van Sminia, assistant-promovendus W.J. Claas, instrument
maker N. van Straten, two apprentices, and the cleaner. Minnaert maintained
close contact with Houtgast: they exchanged 158 letters and packages.
A few weeks after the hostage-taking, Houtgast graduated cum laude
under the theoretical physicist Leon Rosenfeld, a Waloon, and a former
colleague of Niels Bohr. After his promotion, Houtgast wrote: 'Your
absence has deeply moved everyone, professors and audience alike,
and no less those who congratulated me in writing, so you have been
even more in all our thoughts and on all lips than would otherwise
have been the case. All professors came to the reception in gowns
in your honor, bringing 57 congratulations.' His German friend Unsöld
complimented Minnaert on his dissertation.

Minnaert also arranged for his replacement as a lecturer in didactics
and methodology. For twelve years, he had taught this course unpaid,
wrote the faculty board to the curators: 'Professor Minnaert wished
not to hand over this task to anyone else until a suitable person
could be appointed for this purpose.' From Beekvliet, Minnaert announced
that W. Reindersma, former rector of the Nederlandsch Lyceum, wanted
to succeed him: 'Someone entirely suited for this task in all respects.'
Houtgast proposed to Minnaert that summer of 1942 that they would
start measuring the Atlas. With the help of the planimeter, they could
begin manually measuring the equivalent widths of all line profiles:
'Claas starts with the lines needed for his research. The position
of the continuum is indicated by a pencil line so that one can always
check later what has been measured.' Minnaert welcomed this initiative.

On February 6, 1943, German soldiers raided the University building
and took 116 students hostage. After several months, they were released.
Students had to sign a loyalty declaration. In Utrecht, only one in
eight students signed it. This was followed by mass hiding and cessation
of education. Students went into hiding at the Observatory, such as
Kees de Jager, Dedde de Jong, Hans Hubenet, and Wim Claas, while Henk
van de Hulst was appointed as an assistant on Minnaert's proposal.
The more people went into hiding, the faster the Sisyphean task progressed:
a 'work community for measuring the Atlas' emerged.

Sometimes it seemed as if the many activities made Minnaert forget
that there was a war raging. For instance, it became clear that he,
as chairman of the IAU spectrometry committee, had pressed for decision-making
regarding the international standardization of his equivalent width.
His astonished Swiss colleague W. Brunner responded to Prof. Dr Minnaert,
Block VII/1, Lager St Michielsgestel: 'I would like to inform the
reachable members of the committee about what you wrote regarding
Dunham's proposal for the unit of equivalent latitude. However, I
also fear that a definitive introduction of the proposed unit by the
end of the war will not be possible due to the current disorganization
of international scientific cooperation.' Minnaert had sent the letter
to Brunner during the week of the executions: the tendency to seek
refuge in science must have been great.

He arranged for Houtgast to send him materials for lectures and scientific
work and gradually took over part of the management of the Observatory.
This way, he wrote the Annual Report for 1942. On September 9, 1943,
after a year and a half of captivity, Houtgast wrote ironically: 'Your
last letter with guidelines for various activities reminded me of
the general discussions we used to have, where the result was that
I carried out my work more carefully and supported by your ideas.'
Minnaert replied: 'Our correspondence is always a great pleasure for
me and gives me the impression that I am still somewhat present at
the Observatory from time to time.' He maintained control over events,
which strengthened his morale. Even after Van der Bilt refused to
replace him, he continued to prepare and review all written exams.

Minnaert continued with the photometry of the moon, incorporating
data from his PhD student Claas. He owed this to the caricature and
poem by the classicist Onno Damsté:
\begin{verse}
'Vitia ne quaeras lunae, doctissime Minnaert,

Neve velis liberas vituperare artes!

Nonne vides noctis reginam dulce ridentem?

Quod, vir docte, putas saxum esse, illa ridet.'
\end{verse}
In 1943, he completed two extensive articles on the moon, which could
fill an entire issue of the irregularly published Recherches de l'Observatoire
d'Utrecht. He kept instrument maker Van Straten continuously working
on the photometry plates. He studied the latest literature on the
corona, calculated Venus's atmosphere, and determined the temperature
of Halley's Comet.

He converted these studies into articles for the magazine {*}Hemel
en Dampkring{*}. In the fall of 1943, Minnaert began to seriously
worry about the lack of international publications from the Observatory,
while material from himself, Houtgast, De Jong, and Van de Hulst was
indeed publishable. In February 1944, Minnaert turned to his French
colleague Chalonge with the question of whether he would like to print
their 63 treatises. The Frenchman informed Houtgast that he was pleased
that Minnaert could 'work toward satisfaction. Please tell him that
we would be very honored and happy to publish articles by him and
his collaborators in {*}Annales d'Astrophysique{*}. If these articles
are sent, he can be assured that they will be accepted without further
ado. We will even give them priority as proof of the special sympathy
we hold for him in the current situation and of the admiration he
inspires in us through his attitude.' Even this correspondence had
an air of detachment from reality; Minnaert acted as if scientific
channels with Paris could still function in a France where shortly
afterward, the second Allied front would be established.

In March 1944, a high barbed-wire fence was erected around the Observatory
for the construction of a massive bunker 100 meters from the Bolwerk:
the center for the German Nachrichtenstelle. Also, an air-raid shelter
was built in one of the casemates for German female military personnel
who were housed in building Hieronymus, across the singel. At that
moment, not only Minnaert but also his family were behind barbed wire.

On April 20, Hitler's birthday, Minnaert was suddenly released. Van
de Hulst wrote {*}Het Weerzien{*}, which began with:
\begin{verse}
'Two years father has been away

And suddenly: He is home again!

His daughter sings throughout the house:

It's a celebration, it's a celebration!'
\end{verse}
On April 24, there was a celebration at the Observatory. The joy over
Minnaert's return was immense.\\

Endnotes:

1 Miep Coelingh to Marie Minnaert, Marcel's niece, on May 8, 1942.

2 Interview with Miep Coelingh, 1989.

3 Philip Roest jr, 1946.

4 Memorial Book Beekvliet, 1946, H. Brugmans, Good morning, gentlemen!
Today the following will happen, 142-151.

5 Keizer, 1979, does not mention Minnaert.

6 Minnaert to Van Straten, May 1942.

7 Minnaert to Houtgast, August 14, 1942.

8 Memorial Book Beekvliet, 1946, Paul Guermonprez, The murder of five
hostages, 76-88. A memorial book about Baelde appeared in 1947 with
a selection of his articles: the contribution from Michielsgestel
was by Wim Banning. Guermonprez was released on July 30, 1943, and
executed in Amsterdam on June 5, 1944. Vitrine, 2000, 6, 14-19 writes
about Guermonprez as a photography teacher at The Hague Academy. The
article draws a parallel between Guermonprez's themes and the Flemish
painter James Ensor.

9 Minnaert, in his Diary of 1944, retrospectively.

10 Memorial Book Beekvliet, 1946, P.H. Ritter jr, Meetings at Beekvliet,
131.

11 For Servire, he edited works such as G.B. van Albada's The Construction
of the Interior of the Stars, P.Th. Oosterhoff's Star Clusters, J.
Houtgast's The Sun, and H. Groot's Cosmogony.

12 The Memorial Book Beekvliet provides an overview of lectures and
courses, 153, 173. Minnaert's correspondence also provides clarity.

13 Memorial Book Beekvliet, 1946, L.A.H. Albering and W. Kok, 128.

14 Memorial Book Beekvliet, 1946, P. Sanders, Relaxation in Beekvliet,
186.

15 Interview with Brugmans, 1989.

16 Minnaert to Smit-Miessen, August 9, 1942.

17 Minnaert to Smit-Miessen, December 17, 1942.

18 Minnaert to Smit-Miessen, December 26, 1942.

19 Coelingh to Smit-Miessen, January 6, 1943.

20 Minnaert to Smit-Miessen, January 26, 1943. 21 Minnaert to Smit-Miessen,
February 2, 1943.

22 Rümke to Minnaert, February 3, 1943.

23 Smit to Minnaert, March 20, 1943. Greet copied the letters she
sent and kept them.

24 Minnaert to Smit, April 1, 1943.

25 Minnaert to Koen, July 14, 1942.

26 Minnaert to Koen, August 31, 1942.

27 Minnaert to Koen, September 14, 1942.

28 Minnaert to Koen, September 26, 1942.

29 Minnaert to Koen, January 25, 1943.

30 Ten years later, he led a Copernicus meeting at the University
of Amsterdam.

31 Minnaert to Koen, May 4, 1943.

32 Minnaert to Koen, May 19, 1943.

33 Minnaert to Koen, June 22, 1943.

34 Minnaert to Koen, June 30, 1943.

35 Minnaert to Koen, July 1943.

36 Minnaert to Koen, August 3, 1943.

37 Minnaert to Koen, August 20, 1943.

38 Minnaert to Koen, fall 1943.

39 Minnaert to Koen, October 5, 1943.

40 Minnaert to Koen, November 17, 1943.

41 Minnaert to Koen, December 16, 1943.

42 Minnaert to Koen, January 27, 1944.

43 Minnaert to Koen, February 4 and 18, 1944.

44 Minnaert to Koen, April 9, 1944.

45 Minnaert to Koen, April 11 and 17, 1944.

46 Minnaert to Boudewijn, August 20, 1942.

47 Minnaert to Boudewijn, August 2, 1942.

48 Minnaert to Boudewijn, September 4, 1942.

49 Minnaert to Boudewijn, September 24, 1942.

50 Minnaert to Boudewijn, April 16, 1943.

51 Minnaert to Boudewijn, April 27, 1943.

52 Minnaert to Boudewijn, August 31, 1943.

53 Minnaert to Boudewijn, May 21, 1943.

54 From this series, use was made in chapter 1.

55 Minnaert to Boudewijn, December 22, 1943.

56 Minnaert to Boudewijn, December 29, 1943.

57 Unsöld to Minnaert, September 26, 1942.

58 Letter of June 11, 1942.

59 Minnaert to Brunner, August 17, 1942. The year before, he had managed
to reach his colleague Menzel at Harvard and was informed in a letter
dated June 24, 1941, that establishing the new unit did not need to
await the outcome of the war. After Pearl Harbor, the Americans also
set different priorities.

60 Houtgast to Minnaert, September 9, 1943.

61 Minnaert to Houtgast, January 27, 1944.

62 Onno Damsté, January 23, 1943. The classicist's own translation:
'Do not seek so, Minnaert, learned sir, / How faulty the artists render
the moon's image / Do you not see the smile on the face of the night
queen? / She takes pleasure in being mistaken for a stone!'

63 Chalonge (Paris) to Minnaert (via Houtgast), February 26, 1944.

64 The remains of the one-meter-thick concrete layers were cleared
in the winter of 2000-2001.\textquotedbl{}

65 Archive-Astrophysics. Greet Smit-Miessen, by the way, remembered
that Van de Hulst told her during a walk in the winter of 1942-1943
about his calculation, which indicated that there must be radio radiation
from hydrogen particles at a wavelength of 21 cm: this was the beginning
of radio astronomy.

\subsection*{Photosection}

\textbackslash figcaptions

Father Joseph Minnaert.

Mother Jozefina Van Overberge.

Uncle Gilles Desideer Minnaert.

Marcel around 1900

Domela preaches around 1910 in the Protestant church in Ghent.

Jet Mahy, the first female student of the Flemish University.

Julius Mac Leod, Marcel's promoter and teacher.

Marcel Minnaert around 1914.

National meeting of Jong-Vlaanderen circles, 1912. Minnaert in the
middle back.

Minnaert, middle back, as a teacher at the university opened by the
Germans, 1916.

Frederik Gerritson, alias Geerten Gossaert; Dutchman in German service.

Hippoliet Meert, teacher and activist after 1916.

Cesar De Bruyker, Mac Leod's right-hand man.

Roza De Guchtenaere, student of Jozefina Van Overberge.

The Utrecht Physics Laboratory in 1922. Notable personalities include
Lily Bleeker top left and from right to left at the front Minnaert,
Van Cittert, Julius, Ornstein, and Moll.

At the microphotometer, among others, Minnaert, Moll, Ornstein, and
far right Burger.\textquotedbl{}

Pannekoek and Minnaert during the successful eclipse expedition in
Lapland.

The Physics Laboratory in honor of the German quantum pioneer Arnold
Sommerfeld in 1923. Seated from left to right: Burger, Lily Bleeker,
Dr. Riwlin, Ans Huffnagel, Sommerfeld, Moll, Ornstein, Minnaert, and
H.B. Dorgelo.

Minnaert at work during the eclipse expedition in Sumatra in 1926.

Marcel Minnaert.

Miep Coelingh.

Minnaert with Koen (left) and Boudewijn (right), around 1937.

The graduation dinner on December 1, 1941, for Ornstein's student
R. Dorrestein (second from the left). At the back right, wearing glasses,
are Hans Smit and Greet Smit-Miessen.

The Sonnenborgh community in the library on July 31, 1944. Front row:
the Houtgast couple, Lady Van Sminia, and Minnaert. Second row: Dedde
de Jong, Hans Hubenet, the cleaning lady, Aennie Elink Schuurman,
E. Sijthof, Nico van Straten, and Mrs. F. Cruys-van der Kuip. Back
row: Joop van den Broek, Henk van de Hulst, Kees de Jager, Wim Claas,
apprentice Gijs, and Miss B. Braak.

Minnaert and Houtgast set up the coelostat of the solar spectrograph
on the roof.

Minnaert and his wife receive a Russian delegation in 1953, which
will visit a Groningen conference. Second from the left: Victor Ambartsumian.
Far right: Boudewijn Minnaert.

Minnaert in action in the lecture hall of the Observatory: 1942.

Minnaert in action: 1968.

Having tea at Sonnenborgh with Joop Damen Sterck, Minnaert, Hennie
Trappermann, P. Proisy, Kees de Jager, Jean-Claude Pecker, and Aennie
Elink Schuurman.

Minnaert's farewell symposium 'The Solar Spectrum' with three great
figures of half a century of solar physics (1918-1963): Charlotte
Moore-Sitterly, Minnaert, and Albrecht Unsöld.

Koen Minnaert, 1963.

Els Minnaert-Hondius, 1963.

Minnaert around 1966 in Eindhoven with his grandchildren.

Minnaert at his desk on Zuilenstraat.

Minnaert recovering from surgery, Maliesingel, around 1967.

Minnaert drawing with chalk in his sketchbook during one of his many
travels.

Minnaert at the Vietnam demonstration in December 1968.\textquotedbl{}

\chapter{Freud and Free Will}
\begin{quote}
'But almost always, upon closer analysis, such a leap of thought turns
out to be connected to processes in our subconscious.'
\end{quote}

\section*{Minnaert and citizens on Free Will}

Minnaert intensively engaged with philosophical questions about life
during his time in Michielsgestel. According to Guermonprez, he delved
into the comfort an atheist could derive from life. He told Greet
Smit that he was working on a publication about Life Problems. Ritter
Jr. recalled Minnaert's philosophical discussions with the Catholic
anatomist-biologist Barge, 'always circling the small pond.'

With his friend Jan Burgers, he began a debate on 'free will.' Burgers
had written an article on the concept of 'entropy,' as he found it
unsatisfying that in many philosophical publications, such as those
by P. Jordan, N. Bohr, and H. Bergson, 'life' was reduced to 'a series
of processes exclusively governed by statistical causality.' After
becoming acquainted with the work of British mathematician and philosopher
A.N. Whitehead, which opposed a positivist separation of facts and
value judgments, he thought he had found a solution to the problems
of science, morality, and society that had long 3 4 occupied him.
Minnaert read that article in Michielsgestel and was amazed that Burgers
invoked Heisenberg's uncertainty principle to introduce 'conceptual
elements' that could 'intervene' in protein formation.

Minnaert wrote to his friend that organisms are just as strictly determined
as inanimate systems. This view led Burgers to object: 'It leaves
out, so to speak, half of the world from the connection it establishes.'
In his opinion, Burgers had succeeded in placing responsibility for
the future of society within the physical worldview. Since his early
acquaintance with Spinoza, he had attempted this and believed that
he had now succeeded better than the many philosophers and biologists
he had consulted. Minnaert, who was surprised, decided to go through
the publications cited by Burgers and took a lead on the outcome:
'I have no need for \textquotedbl free will\textquotedbl ; it is,
in my view, the last thing a biologist may assume; all other possibilities
must first be conclusively excluded.'

After several months, Minnaert had familiarized himself with Jordan's
ideas on quantum biology and numerous publications on 'life'. This
is evidenced by a stack of notebooks containing notes and summaries.
They confirmed his preconceived opinion that 'the uncertainty relation
for living beings is no different than for inanimate ones' and that
it otherwise has nothing to do with the question of 'free will.' He
added a remarkable view to this: The theory of evolution can explain
why living systems are more complex than inanimate ones: 'It is inconceivable
that something fundamentally new could suddenly appear in this evolutionary
chain. Therefore, I must assume \textquotedbl consciousness\textquotedbl{}
in all kinds of degrees, descending to the most latent in the \textquotedbl inanimate\textquotedbl{}
atom. This consciousness undergoes various impressions that accompany
life phenomena but does not exert the slightest influence itself.'
The implication is that, in Minnaert's view, nature knows gradations
of consciousness, a stone just as much as a flower or an animal, a
star system just as much as an anthill. In retrospect, he had already
implicitly expressed this vision in {*}De Natuurkunde van 't Vrije
Veld.{*}

Burgers ignored this unexpected turn. He wanted to hold on to his
idea, derived from Bohr, that 'quantum reactions are channeled by
certain mechanisms, causing reactions of a size perceptible to us
to appear.' Under certain conditions, a reaction involving a single
atom can cause a result important for the entire organism: not all
uncertainties of atomic reactions are averaged out in living organisms.
Such 'thought flashes' could be caused by these quantum reactions.
Sudden transitions, such as those at the origin of language, convinced
him that when moving from non-living matter to organisms, 'something
fundamentally new' had indeed emerged. Physics overlooked processes
'directed by purpose.' Human mental activity embedded 'an awareness
of values' in every thought formation: 'My conviction is that these
value judgments do play a role in determining the outcome of an elementary
process, at least in those processes significant for real life.' The
issue of 'free will' was thus resolved for Burgers. After all, alongside
physics, relationships focused on the future play a role, presenting
themselves to us as value judgments. He had created space for human
'freedom' to make judgments, to distinguish between good and evil,
and to live responsibly.

\section*{Freud is indispensable for understanding 'free will.'}

Minnaert understood that Burgers' tinkering with physical indeterminacy
had led to far-reaching consequences. He set his continuity thesis
aside and began a polemic: 'The biological phenomena studied best
are becoming increasingly understandable; acausality seems to hide
precisely where we still know little. That is suspicious. I completely
reject your 'thought flashes.' That is an important example! If an
'acausal' atomic reaction were the cause of this, the mental leap
would have to be much crazier than it usually is. But almost always,
upon closer analysis, such a mental leap turns out to be connected
to processes in our subconscious. Freud has mentioned so many incredibly
beautiful cases of this kind that it seems impossible to me to say,
for a specific example: A causal psychological explanation is impossible
here.'

Burgers' concept of 'goal-orientedness' also came under fire: 'I would
strongly oppose the idea that life processes are determined by a goal,
especially if \textquotedbl goal\textquotedbl{} implies \textquotedbl future.\textquotedbl{}
When we act with a purpose, it means: we form some representation
(of an imaginary thing we call the future) and this representation
becomes the cause of our actions. Thus, it is always the past that
determines the future, never the other way around. In human actions,
I consider the entire concept of \textquotedbl goal\textquotedbl{}
to be something that exists only in our consciousness; it is a particular
way in which impressions and memories group together and combine before
leading to acts of will.' Minnaert also rejected Burgers' notion of
'responsibility': 'A person can no more be held responsible than an
animal. I see it this way: we create the myth \textquotedbl responsibility\textquotedbl{}
and all the emotions associated with it to prompt others, through
the utterance of that word, to reflect, exercise caution, compare
their actions to what has been instilled in them as an ideal, and
so on. To achieve this, we suggest to them that they are autonomous.'

He therefore dismissed Burgers' attempt to connect physical indeterminacy
with moral issues: 'You assume that all physical-chemical laws apply,
but you want to put free will into action within quantum uncertainty.
And this freedom is the ability \textquotedbl to distinguish between
good and evil.\textquotedbl{} Question: what is \textquotedbl good\textquotedbl{}
and \textquotedbl evil\textquotedbl ? My answer is: they are myths
imprinted in us during early childhood through education, yet they
correspond to no reality whatsoever. Certain actions promote the survival
and happiness of society. Over the centuries, people have made an
\textquotedbl ideal\textquotedbl{} out of this, which is presented
to every child until it becomes their ideal self (Freud). The so-called
voice of conscience is nothing but feelings of pleasure or displeasure
that we experience as a result of our upbringing, depending on whether
our actions align with these instilled norms. One can track the emergence
of conscience step by step. If we wish to explain how the evaluation
of good or evil influences our actions, then in principle, this is
physically and chemically explicable. Education has shaped our brains
in such a way that normal environmental influences guide us to socially
useful acts; however, opposing influences also work, and sometimes
they prevail.

The discussion went back and forth until the arguments began to repeat.
In a final attempt, Burgers appealed to Minnaert's own life stance:
'I believe that your entire life is at odds with the idea that your
responsibility would merely be a myth. You are a living person who
longs to do good things, who puts a lot of effort into helping others;
you do this naturally, partly as a result of a certain inclination,
but always with a purpose---to help someone, to be of service to
someone, and so on.' Burgers felt morally obliged to seek a worldview
in which physics was connected to values and obligations: 'If we do
not do this and instead present the world at this moment with the
idea that all sense of responsibility is merely a myth, then we will
never find the leverage needed to help it move forward right now.'
With this, Burgers revealed the core of his motivations: as a human
being and as an educator.

Minnaert refused to embrace a false belief afterward, even if it were
intended to inspire good deeds: 'No, we do not want to invent fairy
tales or resign ourselves to old delusions just because they might
be pedagogically useful. First and foremost, I want to determine what
the truth is; afterward, we can discuss how to popularize it. But
for my part, I am convinced that the truth is infinitely more beautiful
and life-affirming than the most beautiful fairy tales (certain philosophical
theories, religions) that one could devise with a moralizing purpose.'

He now also laid his cards on the table: 'We want to analyze what
goes on in someone who makes a decision. I claim that they exclusively
pursue their own happiness. This is a condensed expression for the
causal sequence that you yourself have worked out very precisely in
your letter. We form ideas about the possible consequences of our
actions; these ideas arise from past experiences! (1) The pursuit
of happiness and the avoidance of pain can be very simple; for example,
when I move my finger further away from the stove to prevent it from
burning. (2) In some other cases, we act 'good' because we fear that
our fellow humans would otherwise inflict suffering upon us, or because
we want to earn their respect and love. (3) In certain actions, we
feel such sharp pain in our 'conscience' that we would rather face
death than continue to endure that pain. These are three stages of
increasingly broad concepts. I now want to specifically examine the
last case, which is fundamentally the most important. For 'good' and
'evil,' I believe there is no better definition than being in agreement
with or in conflict with our conscience. And that helps us more for
understanding than one might say. Because that 'conscience,' we do
know it well, thanks to Freud's work! It is nothing other than the
Über-Ich, the Ich-Ideal, whose origin Freud has extensively investigated.
It turns out that it is almost entirely a product of upbringing, especially
at a very young age. This process is later forgotten, and people then
believe that conscience has been instilled in us by some mysterious
power! That 'must' of yours is what you have learned from your father,
mother, educators, perhaps also from reading.

In the past, I thought that there was also a strong hereditary component
here, developed through evolution. According to Freud, it seems that
this is much smaller than I once thought. Research on identical twins
can shed light on this matter. The fact that our educators present
us with certain concepts of 'good' is because they have inherited
these from their own educators; however, changes have also been superimposed
due to individual life experiences. In this way, the ideal of 'goodness'
gradually adapts to the demands of society. But even if a part of
conscience were hereditary, it is still purely causally determined.
How can you say: 'Natural selection is a misunderstanding, choice
implies purpose...' Isn't that just a critique of words? In the mechanism
discovered by Darwin, which is now irrefutably proven, everything
works causally; 'it is as if nature chooses,' but this happens just
as purposelessly as a chemical reaction. You may not say that there
is no reason for evolution: there is cause, but no purpose. Do you
find it troubling that life might not have a purpose? That seems to
me like someone asking, 'What is the Moon for?' Biologists haven't
asked about the purpose of a bodily structure for a long time; when
we say, 'The wings of a bird serve to fly,' we mean nothing other
than 'as a result of evolution, birds have developed wings that enable
them to fly. And so in every field. However, you cannot say: 'therefore,
every world order is equally good.' Beware! That’s where we go off
track. One world order will bring much happiness to humanity, another
much suffering. Good and bad, in the sense of bringing happiness or
unhappiness to humans, make complete sense; thus, the choice between
world orders also affects our personal happiness, insofar as we feel
connected to the happiness of all people. Therefore, it does matter
to us!'

Both friends had gone to great lengths to explain the backgrounds
of their positions. Minnaert grounded his choice for a better world
order through psychological means, as a choice for a societal system
that promised the most happiness. They would meet again in 1945 as
founders of the Union of Scientific Researchers, which sought a constructive
role for science in future society.\\

Endnotes:

1 Ritter jr, P.H., Beekvliet, 1946.

2 Burgers, VNAW, 1941.

3 Somsen, 2001, 218.

4 Minnaert to Burgers, May 23, 1943.

5 Burgers to Minnaert, May 26, 1943.

6 Minnaert to Burgers, January 17, 1944. Many notebooks contain excerpts
from publications by, among others, P. Jordan, N. Bohr, A.J. Kluyver,
P.I. Helwig, W.F. Wertheim, E. Bünning, F. Zernike, A. Frey-Wyssling,
B. Dürken, and H. Driesch. History Archive.

7 Burgers to Minnaert, January 21, 1944.

8 Burgers' side of the debate in Burgers, 1944, later elaborated in
Burgers, 1956.

9 Minnaert to Burgers, January 26, 1944.

10 Burgers to Minnaert, March 6, 1944.

11 Minnaert to Burgers, March 11, 1944.

\chapter{Astronomy and Humanity}
\begin{quote}
Without the resonance of great humanity, the melody of science would
sound poor and faint, and eventually die out.
\end{quote}

\section*{Miep Coelingh takes charge again}

Upon his return, Minnaert found a woman who had been running the household
for a year. They had maintained an intense correspondence together.
Miep Coelingh had undergone psychoanalysis for five years and was
hospitalized for nearly a year. This therapy had made her more independent
from her parents and her husband. She had converted the large bedroom
into a living space where she could shower and cook using a small
gas stove. Minnaert had to move into an unheated room with a Spartan
bed.

What must the 38-year-old Miep Coelingh have realized during the therapeutic
process? As a young woman, she had struggled in her parental home
and had jumped onto the luggage carrier of the older Minnaert, whom
she admired as a student. She had married a man who, like her father
and herself, was obsessed with science. In 1939, she had become overworked
and depressed. Nowadays, her 'neurotic decompensation' would be the
reason for an investigation into the system, involving Minnaert himself.
At the time, treatment was individual, and Miep Coelingh must have
harshly confronted him with the results of her awareness process.

Miep Coelingh's crisis must have been related to the symbiotic relationship
between Minnaert and his mother. When Jozefina Minnaert went on vacation
with her son and Miep stayed home with the babies, it became painfully
clear that Minnaert could not resist his mother. The fact that Minnaert
remained a child between two women must have been an immense disappointment
for Miep. Her depression in 1939 may have been a cry for attention.
Minnaert, however, was devoted to his work in the late 1930s. For
Miep Coelingh, other disappointments followed. She lacked the emotional
qualities for raising the boys and repeated the cool relationship
she had experienced with her own mother in her parenting. For a long
time, her dedication to science had kept her going. It is not unimaginable
that the termination of her promotion and the realization of being
excluded from science as a married mother contributed to the crisis.
The variety of physical ailments may have played an independent role.

In a 1939 letter to Hans Smit, she apologized for not visiting more
often. The 'returning to work' was disrupted by household chores,
childcare, stoking the stoves, and commuting: 'Starting work again
suddenly brought on a massive psychological depression, unexpected
for me but apparently not for Rümke. I’m somewhat over it now.' She
wrote that she had been in bed for ten days with a bladder infection
and a stomach bleeding: every sip of water caused her pain. According
to Minnaert's Diary, in the late 1930s, she was tormented by shingles,
facial conditions, sores, back pain, and sciatica.

Miep's frustration in her social roles could have turned inward as
'pent-up anger' and manifested as depression. If Rümke's therapy made
her aware of the course of her life, she would have longed for a more
independent position. She may have discussed with Rümke Minnaert's
intense correspondence with the children and how she felt pushed into
the background. She must have shared her anger with Minnaert. A common
result of psychotherapy is that old conventions no longer fit, relationships
become stuck, and need to be redefined. Minnaert agreed to separate
beds. It is not known whether he shared or understood her psychological
insights. The couple shared a strong desire for a better world with
greater social justice. They likely read the resistance newspaper
{*}De Waarheid{*} during that last pre-war year. They redefined their
relationship and became companions.

\section*{Astronomy and humanity}

In the summer of 1944, Minnaert wrote {*}De Sterrenkunde en de Mensheid{*}
({*}Astronomy and Humanity{*}), which he dedicated to his wife. He
began with a fascination from southern Canada in the 1930s: feeling
immersed in the marbled starry sky. 'The spaces and times that astronomy
has revealed to us bring us a first inkling of what infinity and eternity
mean. The irresistibility of the forces expressed here fills us with
awe for the lawfulness of nature. And the human condition, which otherwise
fills us with care, becomes insignificant in the face of this reality.'
For many, such radiant starry beauty was a religious experience, evoking
in all people a 'premonition of immortal Beauty.'

Minnaert began his book by describing the working conditions of the
astronomer: observing, calculating, making instruments, and theorizing.
The passion for the starry sky gave the astronomer the energy for
a life of intense work. Often, he had to work alone at night, after
which he spent months measuring and calculating the observations.
The observations in the domes take place at outdoor temperatures because
temperature differences before and behind the lens cause deviations
in image formation: it is often intensely cold during this work.

Afterward, he gave an impression of the place of astronomy within
science. It describes the geography of the universe with attention
to the individual forms of planets and stars but also considers the
physical-chemical structure of matter: 'It is precisely this mixture
of the general and the particular that gives astronomy great beauty.'
It allows us to enjoy the firmament as we would the flowers in a meadow.
By understanding the vastness of the universe, it teaches humanity
humility. Of all the sciences, it is the oldest, having been the first
to learn that the movement of celestial bodies obeys natural laws.
The romantic Minnaert wrote in that final year of the war: 'A science
that thus transcends the boundaries of time cannot be divided by the
demarcations between countries or peoples. Of all the sciences, astronomy
is the most universal; international cooperation is a necessity for
it.' Astronomers often think about nature in its grandest manifestations
and therefore naturally contribute to world peace and true brotherhood.

Minnaert conducted a debate in the manner of Galileo between an idealist
and a utilitarian view on 'the utility of astronomy.' Only the indirect
utility could be convincingly demonstrated. Finally, he himself spoke
as a philosopher, uttering the redeeming, dialectical words: 'If someone
discovers the rotation of the Milky Way system, then spiritual enjoyment
remains an egoistic privilege of the few professional astronomers
unless education and popularization make our beautiful science accessible
to the many.' Science should spread from the university to the entire
population, like mountain water flowing into the valleys. Minnaert
derived from this the obligation of the expert to proclaim the field,
support amateurs, and collaborate free of charge with the work of
people's universities and amateur observatories: 'Without the resonance
of humanity, the melody of science would sound poor and faint, and
ultimately die out. It is both our pride and joy to contribute to
a science that inspires all of humanity to reflect on the structure
of the universe, the place where we live, the origin, and the evolution
of all that is visible.'

He expressed his righteous indignation about the school subject of
cosmography, the one-hour class that was killing astronomy. Alongside
mechanics, cosmography had also fallen victim to the epistemological
approach: 'For a long time, the teaching of astronomy in high schools
has been a classic example of how one can ruin a beautiful science
through academic pedantry. The beauty of the starry sky was replaced
by a system of circles and coordinates; the radiant celestial bodies
became exercise objects for trigonometry.' Minnaert therefore advocated
for a new course where, step by step, the modern worldview would be
constructed: 'Modern astrophysics and the structure of the universe
must take center stage.' Naturally, it should be the physicist, not
the geographer or mathematician, who teaches this subject. In his
curriculum, 'astronomy' looked attractive and modern. He rightly pleaded
for an introduction to the subject in primary school and for a 'refreshing'
of the teacher training college curriculum.

Finally, Minnaert tackled astrology. The particularity of his polemic
against 'the horoscope drawers' was that he extensively addressed
their arguments before formulating his objections. A stumbling block
were the prophesied solutions to the World Riddles, of which he reviewed
about twenty. A special place was given to The Doctrine of the Great
Pyramid, which through the book {*}The Stones Speak{*}, had gained
popularity. He addressed his opponents: 'Astrologers, my friends,
do not think that I am insensitive to the courage and poetry of your
belief in the stars. It was a grand, dramatic theory, connecting humans
to the universe, attempting to create harmony in what seemed like
chance... But it was wrong. True beauty can never flourish on the
foundation of falsehood.' Astronomy indeed invited respect and emotion.
To achieve this, one did not need to resort to the pseudo-scholarship
of astrologers but should instead draw inspiration from great poets.
Therefore, the book concluded with references to the poets Ovidius,
Byron, Hugo, Verwey, and Gorter.

Unlike in the 1920s and '30s, Minnaert connected astronomy to society
during this time. It is an inspiring publication that exudes confidence
in the imminent liberation of humanity. That Minnaert would live to
see this liberation was not a given.

\section*{From the Hunger Winter to Liberation}

The final year of the war brought great hardship to the cities. There
were no more coal supplies for the central heating at the observatory:
a single stove burned in the workshop, with individual desks arranged
around it. Food shortages became dire. Like many others, Minnaert
had to go on hunger journeys by bicycle. On October 26, 1944, Greet
Smit-Miessen wrote about a potato expedition to Leerdam, across the
Lek River: 'Once I lay in a ditch with Minnaert for three quarters
of an hour while German cars were being set on fire. Our wooden tires
made a bothersome noise. The rowboat across the Lek had to be muffled
(our feet floated). A memory from Minnaert was also preserved: 'I
regularly have to undertake expeditions to try to buy all sorts of
things from farmers. First between Utrecht and Montfoort; later up
to the Lek; then, as the area becomes more grazed, to the Betuwe near
Leerdam (25 km); even later to Tiel (35 km). Sometimes this is done
with small rowboats at Vianen or Everdingen, other times via ferries
at Beusichem, Culemborg, or Wijk bij Duurstede. Finally, in December,
the Betuwe is blocked, and the trips must take place northward. As
a result, there are several multi-day trips to Weesp, Zwolle, Dalfsen,
and Hulshorst. In principle, these trips are straightforward; however,
almost always, major difficulties arise due to technical faults---especially
because my bicycle has solid tires and the spokes keep breaking, while
the road resistance is also significantly greater than with pneumatic
tires. Other significant difficulties arise from headwinds, rain,
later: frost, snow, thick fog. Lastly, all kinds of restrictions and
bans imposed by the Germans.\textquotedbl{}

The expeditions to Tiel in February 1945 were the last ones. A former
Ph.D. student had to leave his edible supplies at home: 'Because a
lot of snow has fallen, the trip becomes extraordinarily heavy. The
axle of one of the bicycle wheels under the cart breaks twice; with
difficulty, we manage to get the cart to the northern bank of the
Lek. A few days later, I try to retrieve it on a sled, but I’m forced
to limit myself to dragging half of the load. Once the snow has melted,
removing the cart becomes much easier. On the other hand, compared
to others, we have relatively many food supplies; we are among the
lucky ones. The meals taste like feasts, the boys are always hungry
and can eat endlessly. Koen looks sturdy and strong; Bou not so much,
but we do our best to feed him a bit more.

Almost all conversations revolve around food: 'Further on about the
war and reconstruction. The jealousy among the boys towards each other
is still a painful point that causes us a lot of worry. It manifests
itself in all sorts of trivial matters; they keep trying to treat
each other with indifference and hostility. It’s not possible to leave
them alone at home or send them both on an errand. We still hope that
this will change over the next few years, as seems to happen in many
other families. Apparently, the main thing is to make them feel that
our love flows abundantly towards them and that they don’t need to
fear competition.'

Minnaert noted that he was at the end of his strength: 'The trips
have demanded a lot from me; my resistance has diminished significantly,
probably also due to certain nutritional deficiencies. I can perform
much less than six months ago; my leg muscles in particular have no
strength; I am constantly cold; skin wounds do not heal. I am very
emaciated.\textquotedbl{}

On March 12, he took a new initiative. Because everyone had to provide
their own food and men under 50 were fair game, hiding was no longer
possible. Minnaert still wanted to provide his dozen loyal supporters
with weekly news and had \textquotedbl De Sterrenwacht-Post\textquotedbl{}
printed in multiple copies. In the first issue, Minnaert reported
that Pannekoek and he were working on problems related to the post-war
society, such as the reorganization of the university. The third issue
could not be published on time. On March 24, 1945, Wehrmacht soldiers
inspected the house. They used the discovery of a prohibited radio
set as an excuse to order the evacuation of the workshop, living quarters,
and instruments. According to a report by botanist Koningsberger,
Minnaert wanted to move the contents himself and put the equipment
in safety. This effort was too much for the malnourished Minnaert,
who had to be admitted to his colleague De Langen's clinic on Schoolstraat
on March 25. Minnaert's weight had dropped to 53 kilograms and his
body temperature to 35 degrees Celsius. The children helped move the
furniture and instruments to Koningsberger's Botanical Laboratory.
After 14 days of being fed, the danger to Minnaert's life had passed.
The family moved to an aunt of Miep Coelingh on Oudwijkerlaan until
after the liberation. The \textquotedbl Sterrenwacht-Post\textquotedbl{}
of April 17 reported on the seizure and Minnaert's recovery.

According to Minnaert, they were comfortable at 'Aunt Gerre', who
died on May 13. He wrote in his diary: 'The arrival of the Canadians
is unforgettable. Bou and Koen riding on Canadian vehicles with bunches
of children, driving through the city. Our house is vacated by the
Germans, now occupied by the English Red Cross. Two weeks later, they
leave, we can return. The famine is at its worst around these days.
Finally, planes appear, dropping food packages.'\textquotedbl{}

In {*}De Sterrenwacht-Post{*} of May 5, Minnaert wrote: 'Peace. -
An immense joy rises within us and wants to break through. But at
the same time, grief and sorrow emerge for all those who have fallen,
for the enormous cultural values that humanity has lost. And the question
arises as to how far the liberation will truly lead to the construction
of a better world order...... - But still, overwhelming everything,
overflowing, unstoppable: j o y !' He invited all staff members to
a pleasant gathering on Monday, May 7.

After the liberation, Minnaert pursued the confiscated goods like
a terrier. He managed to track them down in several garages and a
storage facility. With the help of the curators, the Rector, and the
Canadian Military Authority, he succeeded in retrieving them. On May
18, he invited the entire community to restore the equipment and bring
the Astronomical Institute back to life.

\section*{Snapshot 1945: Minnaert in the Netherlands}

The Minnaert of 1945 is different from the Minnaert of 1919. In the
first snapshot of {*}Minnaert in the Netherlands{*}, we see Minnaert
making a name for himself in the country. He focuses his immense energy
on a scientific career, which is impressive. He is extremely fortunate
to land in a laboratory where two professors have created an environment
tailor-made for him. It is to his credit that he quickly produces
creative results and proposals. His \textquotedbl conversion\textquotedbl{}
to Julius's theory of the sun characterizes him. In this, he continues
his uncritical veneration of his teachers and shows himself willing
to oppose everyone and everything to make {*}his master's voice{*}
heard. On scientific ground, he eventually overcomes that unproductive
\textquotedbl resistance attitude.\textquotedbl{} He is helped in
this by Pannekoek and Ornstein, who both become his friends. In the
1930s, he becomes a versatile scholar who, nevertheless, continues
to defend elements of Julius' theory. He proves capable of choosing
long-term goals to which he persistently dedicates himself for decades
which makes exceptional achievements possible, such as the Atlas and
De Natuurkunde van 't Vrije Veld.

This broadening of his scientific perspective does not directly translate
to other areas. In 1928, he strongly distances himself from Dijksterhuis'
views on physics education. During his lectures on didactics in the
1930s, he portrays sitting physics and mechanics teachers as an inert
mass and places his hope exclusively on newcomers. Psychological mechanisms
are at work here that led him to extreme activism in 1914.

On Flemish-national issues, his stance is that Flanders must become
independent and free, if necessary with outside help, and that people
and nation must necessarily coincide. At the same time, he proclaims
that nationalism coincides with pacifism and that all violence must
be rejected. Then comes September 19, 1936. He refuses to believe
that comrades, even if they have since become fascists and nationalists,
will disrupt the tribute to De Clercq. When this does happen, he impulsively
chooses a solitary anti-fascist protest. This tears apart the truce.

During the war, he publicly protests against the occupier's measures
against his Jewish colleagues. He seems sympathetic to the communist
worldview, which he partly bases on Freud's views on conscience. He
works on a worldview in which there is no room for 'free will,' but
where everyone's social and moral choices are of the utmost importance.
He succeeds in this, which is an impressive achievement.

His personal development appears to be largely determined by his upbringing.
He continues his 'resistance attitude' in the Netherlands, although
the sublimation in science and culture during the interwar period
is the most striking image. The lack of separation from his mother
casts a heavy mortgage on his relationship with his younger wife.
The children grow up in a rational environment. Minnaert loves his
children and goes on outings with them, but does not have the time
to accompany them intensively. In his Diary, unlike his father, he
keeps a great distance. This changes after his imprisonment. The boys
then deal with an absent father who is more than ever present. They
are 12 and 11 years old, the same age at which Marcel lost his father.

In his correspondence with his children, Jozef Minnaert's Diary subtly
shines through. He casts a web of future expectations over them, going
further than Jozef himself. He draws Koen into his scientific domain;
Boudewijn ignores the obvious invitation. Besides his disposition,
competition may also play a role for Koen: he then chooses the position
of 'chosen son' who follows in his father's footsteps. This correspondence
maneuvers his wife into the background: she actually disappears from
the children's lives for half a year. In 1944, Miep Coelingh presents
the bill for the lack of reflection on his own youth and emotional
life. They still share their progressive ideals.

The 52-year-old Minnaert, who participates in the liberation of the
Netherlands, finds himself in a very different situation than the
26-year-old Minnaert who fled from the liberation of Belgium. He has
gained great authority in science and is part of a small minority
of the professorial corps that has expressed a spirit of resistance,
democracy, and tolerance and also has ideas about the democratization
of the university.

\section*{The post-war ideals}

In the last issues of De Sterrenwacht-Post, dated May 1, 8, and 15,
Minnaert addressed the issue of Science and University after the war.
In compact form, he presented several program points for 'the reconstruction
of universities.' On May 1, for him, the restoration of international
contact between all astronomers was at stake. The situation after
the First World War, when the scientific world was divided by political
strife, must not repeat itself: 'We must place ourselves above the
fray, switch off feelings of revenge, and honor the great minds that
exist in every country. Precisely because we ourselves have suffered
in this war, our voice in this matter will carry some authority.'
Students were henceforth to gain international experience; after their
doctoral studies, this should even be compulsory.

He recommended his colleagues with the {*}Blätter für Hochschulpädagogik{*}
for the improvement of their study methods and the necessary shortening
of their examination material.

On May 8, he observed in {*}De toegang tot de universiteit{*} that
children from higher social classes were highly privileged. Due to
the extension of the study period for affluent students, access to
the university was being blocked. Stricter selection and shorter study
durations were supposed to provide relief. A modest allowance for
every student would be an excellent measure 'to provide sons and daughters'
of simple people with a modest livelihood.' Here and there, Minnaert
already heard his colleagues object that such simple young people
lacked lifestyle and civilization: however, such qualities were relative
and could be developed through a {*}studium generale{*} and courses
in philosophy and the history of science.

Minnaert optimistically and resolutely took an advance on a better
future. He had begun collecting poems about stars in Michielsgestel.
A select few he included in {*}De Sterrenkunde en de Mensheid{*}.
The first place he reserved for ten lines from Ovid's {*}Fasti I{*}.
Minnaert had translated and renamed them {*}Lof der Sterrenkundigen{*}
(In Praise of Astronomers). Huygens had engraved the ninth line, {*}admovere
oculis distantia sidera nostris{*}, on the lens with which he discovered
Saturn's ring.
\begin{verse}
'They did not know the thirst for great wealth and riches.

They brought the distant stars closer to our eyes,

encompassing the universe through the power of their genius.'
\end{verse}
The selfless dissemination of the blessings of science seemed to him
indeed the noblest and most important profession that existed. He
did not yet consider giving Walt Whitman's relativizing {*}The learn'd
astronomer{*} a place of honor. That would only happen in the late
1960s.\\

Endnotes:

1 Coelingh to Smit, October 23, 1939. She addressed Hans Smit as one
of the few who had shown interest in her dissertation.

2 Minnaert's book was not published until 1946, in his Servire series
on astronomy.

3 De Jager explained that Minnaert provided an account of the discussion
at the First Dutch Astronomers Conference (1941) between the Christian
student Van de Hulst (the 'idealist') and the Marxist student Walraven
(as the 'utilitarian'). The 'philosopher' was Minnaert.

4 Minnaert continued his crusade from 1928 against Dijksterhuis et
al.

5 Vecht, 1939.

6 Greet Smit-Miessen to C.H.M. Braat in Roosendaal, October 26, 1944.

7 Van der Meer, 1936.

8 The Starwatch-Post. Issues 1 through 7 in the Observatory Archives.

9 Minnaert's Diary.

10 Minnaert, 1946, 118-119. Minnaert, 1949, 150, 209. See the translation
of this 'Praise of Astronomers' at the beginning of Part II, before
Chapter 7.

11 Lines 8 to 10 are repeated here.

\part{(1945-1970) Man of the Cosmos\protect \\
Pioneer of 'Science and Society'}

The astronomical lecture
\begin{verse}
When I heard the learn'd astronomer,

When the proofs, the figures, were ranged in columns before me,

When I was shown the charts and diagrams, to add, divide, and measure
them,

When I sitting heard the astronomer

Where he lectured with much applause in the lecture room,

How soon unaccountable I became tired and sick,

Till rising and gliding out I wander'd off by myself,

In the mystical moist night-air, and from time to time,

Look'd up in perfect silence at the stars.
\end{verse}
Walt Whitman

\chapter{Science, Politics, and Society: Euphoria and Disillusionment}
\begin{quote}
'The only adequate solution is the realization of a new form of society
that automatically eliminates war. With this goal in mind, we must
find a modus vivendi for the transition period in the literal sense
of the word: atomic energy control can contribute to this.'
\end{quote}

\section*{From Flemish Nationalism to Internationalism}

In The Astronomy and Humanity, Minnaert had dealt with numerous points
of contact with society. Together with Pannekoek, he had delved into
the renewal of the university. During the war, he had not been involved
with the Flemish issue, although his captivity had been directly related
to it.

Most of Minnaert's activist friends in the Netherlands and Flanders
had collaborated with the German occupier for the sake of the Greater
Netherlands cause. Leader Adolf Hitler had made it possible on September
6, 1940, that under Borms' chairmanship, a 'commission for the implementation
of restoration ordinances' was established. This commission had reinstated
many former activists in their rights, as a result of which they received
significant amounts from the Belgian state funds in 1942.

Robert Van Genechten, as an NSB leader, had become a prosecutor at
The Hague court and committed suicide in 1946 with apologies to the
Dutch people. The image could arise that Flemish collaboration with
the Germans had taken terrible dimensions. In hindsight, it can be
soberly established that collaboration in Wallonia was just as extensive,
and in the Netherlands, it was much more widespread.

Belgium counted hundreds of summary executions in 1944 compared to
a single one in the Netherlands of 1945. Belgian courts sentenced
2,900 people to death: 242 were actually executed, including 105 Flemings.
In the Netherlands, 123 death sentences were pronounced, of which
38 were carried out. The Belgian state once again mixed punishments
for treason with the deliberate repression of Flemish desires. Just
as after 1918, the Flemish Movement temporarily disappeared from the
scene. Prominent individuals from the army and security services were
responsible for blowing up the IJzertoren twice: on March 16, 1946,
it lay in ruins. The death sentence against Borms, which had been
suspended in 1919, was finally carried out in 1946. This tempted Elsschot
to write his controversial poem, in which he expressed sympathy for
the elderly flamingant who had been enchanted by Hitler. In the 1950s,
another amnesty movement began.

After 1918, Minnaert had acted as a spokesperson for an anti-Belgian
faction of the Flemish-nationalist movement in exile. After 1945,
he remained silent in public about his Flemish-nationalist ideals
and avoided questions about his past. He was on the other side of
the political spectrum. He mourned his Jewish colleagues Ornstein
and Wolff and attended commemorations for fellow prisoners like Robert
Baelde and the Fleming Paul Guermonprez. Natural science would later
offer him, through the Bruges-born Simon Stevin, the opportunity to
continue expressing his love for Flanders.

Ideological socialism, mutual aid, had always been an essential aspect
of his philosophy and now became its core. As fervently as he had
fought for an independent Flanders, he now devoted himself unwaveringly
to peace and scientific cooperation. He connected these ideals with
a 'science and society' movement, with the self-organization of scientists.
This was a new phenomenon in the Netherlands. Minnaert felt called
by this movement, which aimed to eliminate war, poverty, and hunger
through science and a world government of the United Nations. In this
field, he developed significant activity.

\section*{The Union of Scientific Researchers (VWO)}

In England, the social interest of scientists after World War I led
to the establishment of an Association of Scientific Workers (ASW),
which had joined the Trade Union Congress and the Labour Party. In
the 1930s, the ASW had developed a blueprint for 'Science and its
Social Relations.' They criticized capitalism for frustrating much
research and imposing futile purposes, such as expertise for biological
and chemical warfare. With the Labour government at the helm in 1945,
this ASW called on colleagues in other countries to organize similarly.

During the Interwar period, Minnaert had not been involved with the
issue of science and society. However, the war had given an impetus
to the idea that science should be employed in constructing a better
world. In 1945, organs of the United Nations (UN) were established,
such as the World Health Organization (WHO) and the Food and Agricultural
Organization (FAO), where the application of scientific insights was
a starting point. UNESCO was meant to promote worldwide cooperation
among people in education, culture, and science.

In 1944, the Danish theoretical physicist Niels Bohr had argued to
British leader Churchill that the atomic bomb was a suicidal weapon,
obliging a 'qualitatively new way of thinking.' The use of the atomic
bomb against Japan in August 1945 dealt a blow to the post-war ideals
of scientists. On October 27, 1945, the Physics section of the Royal
Netherlands Academy of Arts and Sciences (KNAW) unanimously adopted
a resolution from physicist Burgers, who, following Hiroshima, called
on 'all sister institutions in other countries' to 'express their
desire to bear direct co-responsibility for the outcomes of science
and to make their knowledge and expertise available to develop the
applications of science into beneficial results for humanity.' At
the first FAO conference in November 1945, Scottish chairman J. Boyd
Orr declared: 'The fight against malnutrition worldwide must be our
answer to the atomic bomb!' In December, American scientists joined
forces in the Federation of Atomic Scientists (FAS). In their publication
{*}One World or None{*} (1946), the American-Hungarian physicist Leo
Szilard advocated for establishing UN world authority to ensure 'atomic
control.' To avoid a nuclear arms race, he urged 'giving up our own
atomic bombs and setting aside our production capabilities.'

Many scientists felt called to enter the political arena. The reform
of the university was seen as an urgent issue. From the lecterns of
the universities, moral resistance could have been preached, while
the actual defense usually came from students. At the 1946 Amsterdam
University Day, historian Jan Romein criticized the 'far-reaching
political naivety and social incompetence of the Dutch intelligentsia'
and pleaded for his Faculty of Political and Social Sciences. Others
had high expectations for a Studium Generale or mandatory philosophy
courses. The authoritarian management structures were also criticized.
Minnaert himself had addressed the one-sided composition of the student
population. In 1946, the Schermerhorn government established a commission
for the Reorganization of Higher Education. This commission was also
tasked with reviewing the scholarship system 'so that all Dutch people
have equal opportunities for university education.'

In this progressive climate, Dutch researchers responded to the call
from the ASW and established a sister organization called Verbond
van Wetenschappelijke Onderzoekers (VWO), a literal translation. The
initiative came from various quarters. A national group of university
'staff members' demanded more money for research and democratization
of the management structures of universities and research institutes.
Several dozen researchers from Philips' Natuurkundig Laboratorium
wanted to share their knowledge of the implications of technological
developments with a broad audience and sought support from intellectuals
in the humanities and churches. A third core consisted of established
scholars such as Burgers, Minnaert, and Leon Rosenfeld, who were concerned
about the issue of political control over atomic energy. The latter
was a Walloon, a close collaborator of Niels Bohr: he had taught theoretical
physics in Liège and was appointed in Utrecht in 1940.

Representatives from these groups discussed the organization's objectives
and methods. Who did the Verbond aim to reach: primarily natural science
researchers or should it address all scientists, including those in
the humanities? The British ASW and the American FAS both had a 'narrow'
base, which meant they could make statements with scientific authority.Minnaert's
aversion to the humanities explains why he and Rosenfeld opted for
a narrower approach. Burgers, on the other hand, found collaboration
with people from the humanities indispensable. In light of the defeat,
Minnaert proposed a federal association of 'sections' with their own
initiative rights, which could be more or less representative of the
profession. Of those present at the constituent meeting, two voted
for a 'narrow' approach, twelve for 'Minnaert,' and twenty-one for
a broader basis. The deadline was the establishment in London on July
19 and 20, 1946, of the World Federation of Scientific Workers (WFSW),
to which ASW had taken the initiative. The week before, the VWO was
founded, and the delegation was determined.

The theoretical physicist Rosenfeld had been elected chairman of the
Union. The staff members dominating this board saw him merely as a
figurehead. However, at the founding congress of the WFSW, Rosenfeld
was elected vice-chairman and member of the Executive Council. The
present staff members noted with surprise that the WFSW was dominated
by left-socialist and communist prominent figures such as French nuclear
physicist F. Joliot-Curie, British crystallographer J.D. Bernal, and
Canadian physicist N. Veall. Their pride was significantly tempered
by their reservations about the political course. In his book {*}The
Opening of the Atomic Nucleus{*}, Rosenfeld would write that year:
'The World Society will be a Socialist World Union or it will not
exist.' The WFSW received accommodation in Paris from British science
historian J. Needham, leader of ASW and UNESCO, at the headquarters
of that UN organization.

\section*{Minnaert's Board and the Cold War}

The Union aimed 'to strengthen the social position of the scientific
researcher, to achieve the greatest possible deployment of scientific
research, and to deepen the sense of social responsibility among researchers,
so that science will achieve its highest yield for humanity and society.'
However, people like Minnaert who wanted to advocate for UN control
over atomic energy did not receive support from the board, as their
efforts were thwarted by Cold War. Rosenfeld therefore created autonomous
work committees. He chaired the committee for foreign contacts and
established a subcommittee on atomic energy under Minnaert's leadership,
which organized dozens of lectures on atomic energy.

On April 13, 1947, they jointly provided instruction to a group of
introducers, distinguishing three aspects of the atomic question.
The economic potential and military danger made a monopoly by the
United Nations necessary. In the long run, the UN should develop into
a socialist world federation: 'The only effective solution is to realize
a new form of society that automatically eliminates war. With this
goal in mind, however, we must find a modus vivendi for the transition
period in the most literal sense of the word, and control over atomic
energy can contribute to this.' A second aspect was that both political
superpowers were unwilling: the United States wanted to maintain its
monopoly, and the Soviet Union abused its veto power in the UN. In
the common view, only the Soviet Union was obstructing. Moreover,
Minnaert and Rosenfeld held a special view on the position of the
Netherlands: it could 'pursue an independent policy together with
other small countries, and use its great intellectual power to foster
better understanding between the major nations.' They praised their
colleague Hans Kramers in this regard, who chaired the UN's technical
committee on the atomic question on behalf of the Netherlands. Both
instructors had previously publicly expressed support for the approval
of the Linggadjati Agreement between the Netherlands and the Republic
of Indonesia, which could prevent a colonial war.

After Rosenfeld was appointed in Manchester in 1947, Minnaert accepted
the chairmanship of the VWO. He immediately formed a more balanced
board. From the Central Planning Bureau, he brought in sociologist
F.L. Polak. The major departments such as Amsterdam, Eindhoven, Groningen,
Leiden, and Wageningen each elected a representative. The Amsterdam
jurist-sociologist W.F. Wertheim proved to be a kindred spirit of
Minnaert's. On November 22, 1947, the chair could host a prestigious
conference on the organization of pure scientific research in Amsterdam.
At this conference, VWO, through Polak, argued for a drastic increase
in the ZWO budget and structural funding for the humanities.

The colonial wars with Indonesia and Vietnam, as well as the now-erupted
Cold War, limited the flexibility of organizations such as VWO and
WFSW. This would ultimately lead to the dismissal of leftist dignitaries
from WFSW from their government positions in their respective countries;
they even subordinated themselves to the policies of the Soviet Union.
Organizations like the Federation that wanted to remain independent
were hampered.

The first test came with the coup in Prague on February 20, 1948.
Political parties expressed their disgust, except for the CPN. The
Dutch Student Council completely ignored the Netherlands' colonial
war but now collected signatures from professors protesting the violation
of academic freedom in Czechoslovakia. This provoked a reaction from
physicist J. de Boer and historians J. Romein and J. Presser, all
VWO members. They had not signed without hesitation: 'We may, with
appropriate modesty but with some emphasis, bring to your attention
that we have championed these principles even in a time when it involved
much greater risk than now and when many who now loudly blow the horn
of righteousness observed a safer, yet no less eloquent silence.'

These members undoubtedly expressed the opinion of chairman Minnaert.
A majority of his board had rejected making political statements about
the war with Indonesia and the British occupation of Greece. Why then
suddenly take a stance by the Federation while the country was filled
with indignation? His board members exerted pressure to issue a statement.
The chemist M.G.J. Beets wrote to Minnaert: 'Personally, I would have
liked to see VWO take a clearer position on Greece, Argentina, and
other countries at the time.' He proposed offering hospitality to
fleeing colleagues and formulated a compromise: 'The General Board
of the VWO, moved by the events in Czechoslovakia, Argentina, Spain,
and Greece, considering that there is a possibility that similar events
will occur in other countries, makes an urgent appeal to its members...'

At the board meeting on March 17, 1948, a motion drafted by Wertheim
in this spirit was indeed unanimously approved. The Federation also
'made an urgent appeal to scientific researchers in the Netherlands
to keep their heads cool and not be carried away by a sentiment in
which a new war is regarded as inevitable and even useful.' Finally,
it called 'on all scientific researchers around the world, both in
the East and the West, to join forces in the struggle for human values,
for the freedom of science, and for the preservation of world peace.'
Therefore, the Federation defended a cooperation that most politicians
and media had already written off.

This did not remove 'Prague' from the agenda. The WFSW board believed
it had to hold its General Assembly in Prague in September 1948. The
three chemists on the board, R. Schmidt, H.C.J. De Decker, and M.G.J.
Beets, concluded that the WFSW had become an instrument in irresponsible
hands. They desired consultations with the FAS to establish a new
International Federation. Such a split was being promoted at the time
in the field of trade unions, women's organizations, and even regarding
the United Nations. Moreover, the VWO had to declare, in addition
to the Prague motion, that it 'also considers Russia, Romania, Yugoslavia,
Estonia, Latvia, and Lithuania as countries where intellectual freedom
is restricted.'

Minnaert found these proposals unacceptable. The Leiden secretary
A.N. Gerritsen, an experimental physicist, wrote in retrospect: 'It
will be difficult to find anyone who has shown as much enthusiasm
for the Federation's objectives as he did, for whom it hseemed most
important that the Federation was active. He himself set an example
in this regard.' The Eindhoven members regarded Minnaert's resignation
as 'inevitable, given his overly naive political stance' and turned
to Wertheim with the retorical question: 'We can't just let VWO sink
back to the level of a debating union, can we?’

Ultimately, a coalition emerged that kept Minnaert in power, which
led Schmidt and De Decker to resign their memberships.

Meanwhile, the Union had been involved with the monthly magazine Atoom
22 (1946-1948) and from 1949 onwards had its own periodical, Wetenschap
en Samenleving. Despite political disagreements, it experienced significant
growth. Local working groups brought national attention to the professionalization
of disciplines focused on society. The Union had a mission. From Utrecht,
Minnaert contributed with his working group 'Social aspects of student
recruitment.'

\section*{'The social aspects of student recruitment'}

The 'external democratization' of the university was advocated from
many sides. However, in 1948, the conservative KVP minister Gielen
from the red-room cabinet-Beel proposed increasing tuition fees to
curb the influx of students. In response, the Utrecht VWO department
started a working group consisting of astronomers Minnaert and De
Jager, physicists H.A. Tolhoek and H.J. Groenewold, dentist Martha
de Boer, chemist W. Terwiel, and mathematician H. Freudenthal. The
working group argued in its publication 'The social aspects of student
recruitment' that 'all those who have the ability to study should
be given the opportunity.' Their report showed that 44\% of students
came from the 8\% highest-income groups, while only 1\% came from
the 23\% of working-class backgrounds. These were shocking figures:
'Students from higher social classes are thus proportionally 242 times
more likely to attend university than those from working-class circles.'

Minnaert took this as the starting point for his first contribution
to the anthology. He distinguished between 'intellectual factors'
and 'financial barriers.' He cited studies 'from all countries' that
proved that intelligence levels are linked to the social position
of the family and that intelligence arises from the interaction between
a child's innate abilities and their environment: 'Of course, there
is a strong variation in inherited traits among individuals; but there
is no research that demonstrates a clear difference in average ability
between large social groups; many of the best authorities are convinced
of the opposite. Of the 6,000 children with high IQs in primary school,
4,500 reached secondary education and 2,000 higher education. Two-thirds
of the highly gifted were lost to scientific work: 'The drop-out of
1,500 between primary and secondary education is certainly largely
a group of working-class children.' There was a considerable reserve
of intelligence, allowing the number of students to be doubled and
the level raised. Higher social classes produced many poor students
who, after years of delays, obtained their diplomas, while excellent
students from lower classes were excluded.

Minnaert assumed that a student cost their parents hfl 2,000 per year:
'How could such high sacrifices be made by 74\% of Dutch families,
whose incomes now amount to perhaps hfl 1,500 to hfl 3,000?' Financial
barriers must play a major role: 'Freedom is a concept that resonates
strongly in every Dutch heart; well, studying at the university is
free according to the letter of the law, but practically forbidden
for large social classes. This creates a very questionable contrast
between proletarians and scientists: a contrast that leads to misunderstanding,
distrust, or worse.'

What was to be done? In England, it was considered undesirable for
students to work alongside their studies: they would then do unskilled
work while needing relaxation. 'Interest-free advances' were also
unsuitable because academics had to repay them when they wanted to
start a family. In 1921, Wageningen professor A. Blaauw wrote {*}The
Socialization of Education as Duty and Solution{*}, in which he proposed
a 'study wage': the student 'receives what is necessary for modest
living, enough to dedicate themselves calmly and focused to their
studies and general development.' It made no sense to first have students
pay tuition fees and then provide them with the same amount as part
of the study wage.'

The 'study grant' could amount to between fl. 1,200 and fl. 1,600:
for 20,000 students, this cost the government 30 million per year.
'In return, the introduction of the study grant will result in a purer
selection, greater average talent, while the student can fully devote
themselves to their studies; the drop-out rate during the course of
study will therefore be smaller than it is now, which represents a
very real economic gain for society.'

Minnaert deliberately chose to provide study grants to all students.
There were well-off parents who refused to let their children study.
The study grant ensured economic independence and moral equality:
'The legislator has the ability to make the education tax progressively
arbitrary in the way that seems most effective to them, so that wealthy
families will have to repay a significant portion of what they received
for their studying children.' The study grant was a 'mutual study
insurance for the entire population, ensuring that all talented individuals
will have the opportunity to develop themselves, regardless of the
family into which they were born.' The University could certainly
benefit from an injection of talent from the working class: 'The tradition
of labor, mutual service, and solidarity is particularly strong in
their demographic group. Their healthy strength and life energy form
a beneficial counterbalance to decadent individualism. Can we not
hope that the best qualities of our people will develop when young
individuals from all backgrounds live and study together during the
most wonderful years of their youth?' Minnaert linked the ideal of
mutual service to a social class.

At the end of December 1949, Minnaert opened a VWO conference in Amsterdam
on 'Admission to Higher Education.' A Committee of Recommendation
united all Rectors Magnifici. The Secretary-General of the Ministry
of Education, Culture, and Sciences, H.J. Reinink, joined the problem
statement: if research in the U.S. showed that 32\% of all children
could pursue higher education, why, according to psychologist J. Luning
Prak, was this only 2\% in the Netherlands?

The Utrecht report was attacked by a sociologist from VWO ranks: Fred
Polak. He mocked the 'idealism' of the compilers who were squandering
hundreds of millions from the state treasury. Minnaert replied: 'Consider
that through the lenient use of the word \textquotedbl idealist,\textquotedbl{}
reforms are being opposed or no effort is made to change anything!”
He was so indignant that he directly confronted another participant,
asking, “Are you in favor of or against study grants?” When the bewildered
man said he didn’t know, Minnaert exclaimed that opponents ‘express
their criticism in this way and thus divert attention from our plan.’

The Labour Party and the Humanist League asked Minnaert for discussion
contributions. The student association Politeia produced the report
Democracy in Higher Education, in which they advocated for study grants.
Minnaert wrote a contribution for the 1951 Politeia Congress, which
focused on The Social Aspects of Student Recruitment.

In 'Socialism and Democracy', Minnaert emphasized that study grants
were necessary for scientific manpower. The obstacles of this class
system were ‘a major source of bitterness among workers.’ Through
the one-sided composition of the student body, influential positions
remained in the hands of the higher classes. It made a huge difference
whether such an issue was addressed by a conservative or a progressive
scientist: ‘The former always tends to argue that the current situation
is linked to unchangeable hereditary traits or the (in his view constant)
intelligence quotient. The latter starts from the conviction that
one can achieve immensely much by purposefully choosing education
and environment. There is no doubt that the second method is infinitely
more fruitful and inspiring. May the future show that the optimists
were right!’

Minnaert’s working group consisted of progressive natural scientists,
so that their unanimous report would shore up the dykes. He had never
had any need for hesitant psychologists and cynical sociologists.
Nevertheless, the working group referred to The Rise and Fall on the
Social Ladder by sociologist Van Heek (1945). His 'Hidden Talent'
would later empirically substantiate the assumptions of the working
group.

\section*{A Constructive Science for a Better World}

In the fall of 1948, Minnaert opened the congress The World's Food
Supply, organized by Wageningen VWO. For the Wageningers, it was a
first step toward technical agricultural assistance. Current issues
surrounding the UN and the Dutch contribution to the FAO were presented
to Minister S. Mansholt and FAO Director A.H. Boerma. The latter brought
up the cooperation in agriculture between Western and Eastern Europe,
but the former pointed out the final separation between East and West.

On May 26, 1950, The Hague VWO organized a congress on Scientific
and Technical Assistance to Underdeveloped Countries. Everyone who
mattered was present: agricultural economist E. de Vries, legal expert
H.G. Quik, Minister without Portfolio L. Götzen, and numerous representatives
from scientific associations. In the background, De Vries' attempt
to transform the Ministry of Colonies into a Ministry of Development
Aid was underway. The establishment of the later NUFFIC, the collaboration
organization of Dutch universities in the field of development aid,
was taking shape. According to Minnaert, the congress tied into an
initiative by the UN. The backwardness of a number of countries led
to an imbalance in world development, which would be accompanied by
great tensions: 'What is more reasonable than that scientifically
and technically advanced countries make their knowledge available
to less developed countries to accelerate their evolution?' He pointed
out 'how for a century it was said that one wanted to pursue the well-being
of colonial populations, while primarily exploiting these countries
for their own benefit. Science has brought significant technical progress
during this period, but on the other hand, it has also been used as
an instrument to maintain economic or political power over these regions.'
A poignant question came from Götzen to De Vries: why did the United
States offer their technical assistance through their own Point Four
program and ignore the UN? Even in this area, the schism between West
and East had occurred.

This series of VWO meetings, which also addressed the renewal of university
governance structures (Leiden), the quality of science journalism
(Groningen), or the potential avoidability of military research (Eindhoven),
promoted the role of scientists in a world dedicated to reconstruction
and development. That was what Minnaert wanted to focus on! He was
adamantly opposed to a new atomic weapons race and the war preparations
that would accompany it.

It is notable that there was no conference addressing the issues surrounding
atomic energy and armament. In early 1947, Rosenfeld had designed
a conference titled 'The Necessity of International Atomic Control,'
but his board rejected it out of fear of controversy. Minnaert provided
an overview in 1949 of forty lectures that were nonetheless organized.
The League screened the film {*}One World or None{*} at such gatherings,
which the American FAS had produced in 1946, before the Cold War.
Minnaert continued to emphasize what the Netherlands could do independently
and ignored the confrontation between East and West. At the end of
1950, he wrote to 'amice' Wertheim about the lack of response to President
Truman's decision of January 31 to produce the H-bomb: 'It is actually
unheard of that we have not made any statement during this turbulent
period; I had drafted a proposal some time ago but was hesitant due
to the ever-looming divisions. Maybe I was wrong!'

\section*{Minnaert in the Firing Line}

Minnaert's commitment to his League was uncontested. However, his
political preference seemed to lean toward the Soviet Union, making
him less suitable as chairman. His rapprochement with the CPN, in
particular, aroused suspicion. In April 1947, he became a board member
of the Netherlands-USSR Association, which was seen as a front organization
for the CPN.

In November 1947, Minnaert attended the 'Science and Society' conference
organized by the Church and World group, where PvdA politician W.
Banning, former minister G.H. Slotemaker de Bruïne, and he himself
delivered introductory speeches. 'It was a closed conference where
about fifty prominent Dutch individuals from various religious, political,
and scientific backgrounds openly discussed the issues,' wrote H.J.
Vink, a chemist from Eindhoven and a board member of the VWO, in appreciation.
Minnaert spoke about Humanisme and pleaded for a worldview based on
reason. When asked if it could exist 'without God or religion,' Minnaert
replied, 'It works even better without religion.' Minnaert advocated
for mutual service: make people happy, love people; do unto others
'as you would have them do unto you.' From this, civic duty and all
refined feelings between people could be derived, which were simply
the product of humanity's history. He continued his exploration of
'free will' and found a platform for it within the Humanist League.

The marginal extension of his presidency had not made Minnaert more
cautious. He took the liberty of appearing publicly as he saw fit.
In Prague, the World Federation of Scientific Workers had decided
to hold a World Peace Congress, in close consultation with policy
circles in the Soviet Union. The leaders of the WFSW became the leaders
of the World Peace Council, with F. Joliot-Curie and J.D. Bernal at
the forefront. In 1949, under the editorship of Marcus Bakker, the
young secretary of the Dutch Peace Council (NVR) and later a CPN member
of parliament, the first issue of the periodical {*}Vrede{*} (Peace)
appeared in 500,000 copies. In it was a handwritten appeal by Minnaert:
'I value expressing my support for the purpose and organization of
the World Peace Congress. I am pleased that you have made contact
with broad layers of our population, and I conclude from this that
there is a strong and general desire for peace.' On page 33, a thick
peace dove painted by Picasso appears next to Theun de Vries' {*}The
Dove Takes Flight{*}. The poem asked the Americans, on behalf of all
inhabitants of the Earth, for peace. Exclusive accusations against
the West would characterize this 'peace movement.'

The board had to deal with external reactions to this public appearance.
For example, they had asked the Second Chamber for a quality seat
in the ZWO administration. The PvdA member J. de Kadt had casually
remarked in the Second Chamber on October 7, 1949: 'If the VWO is
an organization (...) that also concerns itself with organizing 'peace
congresses,' it seems less desirable to me.' The year before, he had
called VWO board member Wertheim a Stalin lickspittle. On November
12 and 13, Minnaert delivered an introduction at the Dutch follow-up
to the World Peace Congress. Other speakers included Utrecht school
director H.J. Jordan, his Amsterdam colleague C.P. Gunning, and Groningen
romanist J. Engels. Minnaert advocated for the establishment of a
'peace scientific institute.' The PvdA had called for a boycott because
'the communists are the driving forces behind the scenes.' Dutch newspapers,
with the exception of the communist daily De Waarheid, were unanimous
in their scorn for this 'peace activity.'

The WFSW, which was supposed to provide researchers with a window
on world politics, fell into disrepute in the West. Partly under the
influence of the American CIA, a pro-Western International Confederation
of Free Trade Unions (ICFTU) had previously split from the World Federation
of Trade Unions (WFTU). In 1950, the WFSW decided to join this WFTU.
The Confederation then decided to withdraw: the cancellation letter
of July 1950 was signed by 34 Minnaert and secretary Gerritsen: 'Your
recent decision to make a contract with the World Federation of Trade
Unions (...) is not an impartial act.'

Meanwhile, the general board and the chairman were under fire at the
Union's member meetings. There was a flood of motions of disapproval.
For example, a motion from the Eindhoven branch regretted 'that the
general board has adopted a wavering attitude toward communism.' A
new course would be set at a January 1951 member meeting. Minnaert
observed that most motions favored Western Europe: 'It is said that
science in the Soviet lands is not free enough. But the Catholic Church
and orthodox Protestantism also impose dogmas; and in Western countries,
the social structure exercises great power over science.' According
to him, the two groups into which the world threatened to split were
not related as good and evil: 'A Federation of Scientific Researchers
should not align itself on one side, not even on the side that seems
generally more favorable. It can take a position on any specific issue,
but it must not a priori bind itself to one of the two parties.'

Minnaert thought that against those who regarded Russia as a threat
to humanity and science, there were others, 'equally selfless, skilled,
and idealistic, who believed they saw progress in that direction.'
His considerations written for the annual meeting could not convince
those present. He therefore decided to relinquish a second term as
chairman.

An absolute majority was required for an election. The members elected
only four out of twelve candidates: Minnaert received the most votes
with 43 out of 65. Clearly, the members did not doubt his integrity
and must have appreciated that he made his chairmanship available.
Sociologist Polak, by then a professor, was non-removable and succeeded
Minnaert. Wertheim disappeared from the board with 27 votes.

In his Diary, Minnaert wrote in early 1951: 'I stepped down as chairman
in January. There was continuous opposition to \textquotedbl communist
tendencies,\textquotedbl{} which strongly hindered effectiveness.
As a reward for much work and concern, I received suspicion.'

\section*{The BVD and the Man from Moscow}

The Alliance had become a subject of investigation for the new Domestic
Security Service (BVD) in 1949. After all, it acted against the secrecy
of scientific work and in favor of keeping open the channels between
East and West. The BVD saw this as a declaration of intent to unilaterally
transfer secrets. Agent D wrote on December 21, 1949, about the VWO
members: 'Whether they are all aware that a group of the most prominent
members, with every activity, have their eye on a refined form of
propaganda, opening themselves up to abuse of scientific data exchange
and creating opportunities for espionage or its preparation, is very
much in question.' The Service could hardly see the natural tendency
of scientists to share information as anything other than propaganda
for the other side.

A BVD informant and VWO member concluded after the aforementioned
members' meeting: 'In my opinion, even after the heated debate at
this last general meeting, it has once again failed to steer the course
of the VWO from neutral to positively anti-communist': only Polak
succeeding Minnaert was 'a major improvement'. The BVD appeared to
promote 'professional bans' for members of the Union: together with
other agencies, they 'constantly tried to keep unreliable persons---to
the extent known to the Service---out of the most important laboratories
and institutions.' The Service had employed agent provocateurs who
attempted to persuade VWO members to engage in scientific espionage:
so far without success. The informant considered the Union an important
target for observation because intellectuals, after the Warsaw World
Peace Congress, 'became actively involved in the Soviet offensive
against the non-communist world.' He believed that the world peace
movement, originally perhaps intended as a protest against the atomic
bomb, should now be regarded as having a complete mission: neutralizing
the active resistance forces of the West as much as possible. They
would try to achieve this with regard to intellectuals in the Netherlands,
among other things, within and through the VWO. The word 'with' in
the last sentence implies that the Union had become an instrument
of the opposing side. For Minnaert, this was the second time he, the
man of the cosmos, had become the Man of Moscow.

Agent C5 characterized Minnaert on June 15, 1950, as a 'fellow traveler':
someone who followed the policy of the Soviet Union. The BVD agent
found the Union a creepy affair: 'Although by no means all VWO members
can be directly classified as part of the communist camp, attention
should be paid to the fact that the Union is represented in the Netherlands
in every laboratory, university, etc., by one or more VWO members.'
After this, C5 listed the names of dozens of VWO members who deserved
the Service's attention.

With the 1951 Korean crisis and the push for rearmament, professors
Minnaert, Wertheim, Freudenthal, and Burgers were responsible for
extensive files. The mathematician Freudenthal said when asked: 'Everyone
around Minnaert was being watched. I had a friend, a chaplain, who
told me that someone had come to him in Maastricht. That BVD agent
said: \textquotedbl We consider everyone who receives study pay to
be a communist.\textquotedbl{} Then the chaplain said: \textquotedbl We've
had that in the Catholic Church for a long time: our studies are paid
by the Church.\textquotedbl ' Such an incident is typical of those
years of cryptocommunism, an important term. The physicist Casimir,
director of Philips' Natuurkundig Laboratorium, was not a member of
the VWO: 'Someone who sought scientific contacts with Eastern Europe
was already considered 'fellow traveller' by the BVD'.

In 1951, the American embassy denied Minnaert a visa for the US, considering
him an individual associated with Eastern Europe. In his diary, he
wrote: 'A painful story that really puts one's nerves to the test.'
Nevertheless, in 1952, Minnaert welcomed a delegation of Russian astronomers
at Schiphol Airport. The BVD archives on individuals are currently
closed to researchers.

The agents in the 1950s likely judged Minnaert, who they saw as unchangeable,
very harshly. Gerritsen, secretary of the VWO and later professor
of experimental physics in the US, said afterward: 'Minnaert was one
of the best people I've ever met in my life. But he couldn't think
socially the way he could about solar physics. Every now and then,
I would say to him: 'Minnaert, you know that the Earth is flat.' He
often interpreted things subjectively and used phrases full of judgments
like 'wrong,' 'bad,' and 'incorrect.' Still, Minnaert could be convinced
if you could explain to him rationally where his stance would lead.
I considered it a privilege to have been one of his friends.'\\

Endnotes:

1 Minnaert, Poets on Stars, 70.

2 Dedeurwaerder, 2002, 718.

3 De Jong, Part 12, Epilogue, 545. Van Genechten wrote that he had
offended the Dutch people as a result of his 'bitter intellectual
arrogance.'

4 Blom, J.C.H., The Netherlands under German occupation, 70; Eenoo,
R. Van, Belgium in international politics 1940-1944, 43-46; General
History of the Netherlands, Volume 15, Haarlem 1982.

5 Luykx, 1969, 405.

6 Geyl, P., Why I spoke at the Yser Pilgrimage, August 19, 1962, in:
Figures and Problems 1, Amsterdam-Antwerp 1964. 7 Museum Guide, Diksmuide
2001, 8. 8 Willem Elsschot, Borms, 1946.

9 Wolff died in 1944 in Bergen-Belsen.

10 Beekvliet, 76, and De Jong, Part 7, 948.

11 Molenaar, 1994, 23, 29.

12 Molenaar, 1994, 47. 13 Molenaar, 1994, 39.

14 Given the publication by Zondergeld in 2002, an exception must
be made for VU. 15 Molenaar, 1994, 47.

16 An early formulation of the ideology behind the 'Hollanditis' of
the seventies and eighties.

17 Declaration of December 3, 1946.

18 Minnaert was approached at the end of March 1947 by C.J. Gorter,
the new director of the Leiden Kamerlingh Onnes Laboratory. For a
biographical sketch of Gorter, see Molenaar, 1994, 75.

19 From Wageningen, it was the agricultural expert S.J. Wellensiek,
and from Groningen, the physiologist M.N.J. Dirken. Eindhoven provided
the chemists C.J. Dippel and H.J. Vink. 20 Molenaar, 1994, 97.

21 Molenaar, 1994, 99. The young politician Joop den Uyl advocated
in 1948 for the dissolution of the United Nations and the establishment
of a Western splinter group.

22 Even Zeno, then Science, Technology, and Society, and after half
a century, disbanded at the end of 2002.

23 VWO-Utrecht, The Social Aspects of Student Recruitment, Utrecht
1949.

24 Minnaert, (1949a).

25 Minnaert, (1949b).

26 In post-war France, the student union UNEF called it a présalaire.
Minnaert had access to that information.

27 This sentence has been retained because that absurd practice is
now common. 28 VWO Congress on Admission to Higher Education, Amsterdam,
December 9 and 10, 1949. Utrecht, 1950. 29 Minnaert, (1949c).

30 VWO Congress Wageningen, The World Food Supply, October 8 and 9,
1948, Wageningen 1949.

31 Minnaert to Wertheim, March 28, 1950. Molenaar, 1994. 84-94.

32 The date is April 5, 1948.

33 Picasso's later doves were elegant. The first dove didn't look
good: it resembled a bomber.

34 Minnaert remained a personal member of the WFSW, as did Wertheim
and the physician L.H. van der Tweel.

35 Minnaert, Consideration, January 10, 1951.

36 Minnaert, Diary, 1951.

37 Interview with the mathematician H. Freudenthal.

38 Interview with the theoretical physicist H.B.G. Casimir.

39 The author viewed the BVD reports on VWO and its leaders from the
early 1990s in connection with his dissertation on the history of
the Verbond. Efforts from the Tweede Kamer to make the archives more
publicly accessible have resulted in the archive now being hermetically
sealed for researchers. A regrettable development.

40 Interview with experimental physicist A.N. Gerritsen.

\chapter{A Passion for Traveling}
\begin{quote}
'I see the human spirit, where life reaches a pinnacle, creating art
and science, reflecting on itself.'
\end{quote}

\section*{The family at Sonnenborgh}

After the war, the Minnaerts could finally enjoy their home to the
fullest. It towers like a fortress above the singel. On the ground
floor, the kitchen, bay window, and study opened into a living room
with a large side window. Miep Coelingh brought in chrome steel chairs
and tables. She chose square seaweed matting for the floor, which
allowed dust to fall through and eliminated the need for a vacuum
cleaner. She had three cabinets built next to the front door, each
accessible by the milkman, baker, and greengrocer with their own keys.
She placed an order note inside. She could collect her groceries from
these cabinets. The kitchen, where Minnaert would never set foot,
had a hardwood counter and electric stove. It housed one of the first
dishwashers in the Netherlands. The bay window was Miep's domain:
it held her birchwood Rietveld desk with chairs by Mies van der Rohe.
She looked out onto a small garden with pear and plum trees and currant
bushes.

In Minnaert's study, there was a black piano and a desk: the walls
were filled with books. He played classical works, particularly by
Beethoven, Schubert, Brahms, Rimsky-Korsakov, and other Russian composers.
Boudewijn noted that the piano could be heard throughout the house:
'When Mother was late with dinner, Father would start playing the
piano.' They often went to concerts together. In the living room on
page 341, the children played and guests were received. The bedrooms
were on the first floor. One room was reserved for musical instruments,
which Minnaert occasionally demonstrated to visitors; the rest of
the collection was in the attic.

The Flemish Rudolf Mahy sometimes stayed there: 'My own parents remained
in love - arm in arm. The Minnaerts seemed like two good acquaintances.
The interior was Neue Sachlichkeit. She cooked so deliciously that
I forgot the soyballs in the soup were meatless. Koen and Bou were
immediately drawn into their conversations. I remember a remark directed
at me: 'You should read P.J. Schmidt's 'Geld en krediet' once.' Minnaert
explained the tube system of the sun viewer to me: it was as if you
already knew it. When we went for walks, he kept asking 'why...'

Also, Hanneke van Konijnenburg, the daughter of Truus van Cittert-Eymers,
often came by. She had to say 'Marcel' and 'Miep,' even though she
didn't particularly like that: 'Minnaert didn't mind me walking around.
Miep was reserved at first. She was stubborn and had to completely
trust someone before she opened up. Then the warmth would come too.
On the other hand, I've never seen a milder person than Marcel! Their
relationship wasn't good or bad: they led a completely independent
life.'

According to Boudewijn, his parents didn't attach much importance
to appearances: 'Clothes weren't important. Appearance wasn't important.
They taught us table manners and talked about the Bible. We had to
know what it was about. We learned bridge as a social grace.' He quickly
noticed that his parents had a pronounced political preference: 'A
few days after the liberation, I said, 'We'll take a subscription
to Het Parool.' Then Mother said, 'No, we're subscribing to De Waarheid.'
During the war, you were either good or bad. I learned that there
were apparently nuances within 'good.' They explained to us why they
took that attitude. They weren’t strict, and we never got a slap.
They exercised psychological pressure: this is what you ÓUGHT to do
and that is what you SHOULDN’T. They gave clear messages.’

Miep Coelingh often felt disgusted and tired. When her helper Corrie
got married, she started doing the household herself again. She received
very few visitors. In the summer of 1946, her father passed away,
and she was fully involved in family visits and aftercare. In 1949,
she arranged for her mother to be admitted to a nursing home.

\section*{Koen’s love for studying and Bou’s wandering blood}

In 1945, Koen had become a member of the Dutch Youth Association for
Nature Study (NJN). He went out every weekend, gave lectures, and
organized summer camps. In 1947, he took an excellent final exam.
The rector asked him to give the annual address, but he skipped it
because of the NJN. At the observatory, he set up a ‘chemical room.’
Minnaert observed: ‘It’s nice how he quickly becomes more handy and
develops a style in experimenting.’ In the summer of 1948, he cycled
to Nos oiseaux in Evolène, Switzerland.

He went to study in Utrecht and lived at home for the first few years.
A library card shows that at the age of nineteen, he immersed himself
in spectral analysis, his father’s specialty. He chose chemistry:
that must have been safer than physics. He got a girlfriend in the
NJN, Corrie, but it suddenly ended between them in 1951. Minnaert
wrote: ‘What went wrong, we don’t know’ and ‘he leaves around Easter.’
He later turned up on the Riviera. That summer, Koen had to go to
Malmédy for the NJN camps and obtained his cum laude degree.

His friend Theo Quené, later an urban planner and chairman of the
SER, said when asked: 'Koen wore a khaki military uniform and thick
glasses. He had an expensive pair of binoculars, which he used to
watch birds. Those binoculars and he were inseparable. We felt like
the elite of the NJN and came together on the board: him as secretary
of summer camps and me as chairman. Koen believed that the real work
lay in organizing the eighteen camps. Once, he had rented cooking
pots from a rental company. He noticed that the thickness of the pans
was 18 millimeters instead of the agreed 16 and mercilessly stood
his ground. That summer, he stood with a sealed shipment at customs
in Eijsden and refused inspection. I handled it over the phone. Later,
he said: 'That Theo is a fixer.' There was something admiring and
something condescending about it: you were 'just' a diplomat.' Quené
found Sonnenborgh a deserted place: 'You never saw father and mother.
The kitchen was spotlessly clean. Koen made an omelet, and then we
went to his room. I exchanged no more than three sentences with his
mother. She didn't know who his friends were. His father knew a bit
better but seemed distracted. When we returned from Belgium, the light
in his study was on. Koen gave him a handshake: 'Father, you need
to go to bed. It's four o'clock in the morning.' Minnaert gave the
impression of being a scholar: distant but friendly. Where a woman
often brings warmth to a household, that wasn't the case at Koen's
home. He told me his father had received the 'Nobel Prize' for astronomy.
An extraordinarily clever father can become an idol.' Koen was sexually
uninhibited: 'The NJN, as a mixed movement, could only be chaste.
We saw his girlfriends from outside the NJN as delayed puberty. He
was an attractive man, engaging and sharp-witted, rivaling and challenging,
erudite when he talked about Menno ter Braak.' He could talk friendly
and engagingly with young people for hours. That was his hidden side.

In 1945, Boudewijn had become a sea scout: that summer he went to
a Scout camp. At sixteen, he was allowed to go alone to acquaintances
at Lake Walen. Minnaert wrote: \textquotedbl He helps pick berries,
herd goats, and makes trips as well. It’s nice how well he has studied
Switzerland in the Baedeker and how much he loves it.\textquotedbl{}
In 1949, he passed his gymnasium exams with seven straight sixes,
six minuses, and a five for Greek. He chose to work in the travel
industry: \textquotedbl My first job was at the Dutch Travel Association.
I rented a room in The Hague and that worked out well. I received
an allowance from home. My desire to travel was also a result of the
lack of freedom during the war. Father had told stories about foreign
countries, which sparked my interest. Through Father, I became a tour
guide in Lugano, later in Lausanne. Every weekend we went into the
mountains. That was an eye-opener: I was independent, away from parental
supervision. I got a job with the Royal Java-China Packet Shipping
Lines. First, I trained at a stevedoring company because I knew nothing
about shipping. After that, I traveled the world.\textquotedbl{}

The choice of his youngest son also encouraged the father. In the
summer of 1948, Minnaert arranged with Boudewijn to meet after an
IAU congress in Zurich. They walked together---17 and 55 years old---over
the Furka, Grimsel, and Scheidegg passes. Minnaert noted: \textquotedbl Everything
by hitchhiking or walking, sleeping in stables or barns, on hay or
straw. We enjoyed it immensely. In many ways, we share the same taste
and way of traveling.\textquotedbl{} The following summer, Minnaert
repeated the joint hiking trip before a hydrodynamics colloquium in
Paris. They made trips near Montana, slept in the Cabane des Violettes,
and near Zermatt in the Garnergrat and in Saas-Fee in the Britannia
refugre. 'A trip we will never forget. We also grew closer than ever
before. The colloquium in Paris was exhausting; I felt only half at
ease there. {]} A scene from a poem by the American romantic Walt
Whitman, which Minnaert himself published in {*}Poets on Stars{*},
irresistibly comes to mind.

\section*{Whitman’s {*}The Astronomical Lecture{*}.}

In Sint Michielsgestel, Minnaert contributed to the {*}Friend’s Book{*}
of a fellow camper. He was asked to express what he loved and found
most beautiful. At the time, he created a ‘festive symphony,’ an incomparable
celebration of the ‘unity of the universe’:

‘In my mind, I see the utmost refinement of atoms, clouds of electric
charges swarming around at incredible speeds. In the dark vastness
of space, I see the waving veils of spiral nebulae, with their millions
of silver-shining stars. I see life blooming as part of this nature:
colorful beauty of forms and graceful, flowing changes. I see the
human spirit, where life reaches its peak, creating art and science,
reflecting on itself.’

Here, Minnaert revealed himself as a pure romantic. In the camp, he
had begun collecting poems about stars and planets. Many friends helped
him. He gathered three hundred titles, translated most of them, and
selected 94 poems from all over the world. The anthology {*}Poets
on Stars{*} was published in 1949 by the literary publisher Van Loghum
Slaterus. It began with twenty Dutch contributions: the first by a
man from Ghent. There were more subtle nods throughout. He translated
a poem by the Frenchman Louis Aragon about an August night full of
‘falling stars.’ A footnote explained that the surrealist Aragon was
a communist resistor, a poet of ‘young France.’ The meteor shower
was a metaphor for ‘the memory of the fallen, the announcement of
the great turning point.’ He could have added that this poet was one
of the leaders of the World Peace Movement.

The German and English poems remained untranslated; the 1948 reader
knew those languages. For the French poems, Minnaert's translation
was added. The others Minnaert had translated himself from Latin,
Greek, Swedish, Italian, Spanish, Portuguese, Danish, and Russian,
because 'where no translator is mentioned, I am responsible for it
myself.' Minnaert loved languages: the anthology must have been a
delightful project. His retired friend Anton Pannekoek wrote from
10 Wageningen: 'Heard with pleasure that you sent Jet Holst your Poets.'

However, a poem chosen by Minnaert from the American romantic Walt
Whitman gives rise to speculations about Minnaert's own state of mind.
It deals with the kind of lectures that Minnaert himself had given
so many of:
\begin{verse}
'The astronomical lecture 

When I heard the learn'd astronomer,

When the proofs, the figures, were ranged in columns before me,

When I was shown the charts and diagrams, to add, divide, and measure
them,

When I sitting heard the astronomer

Where he lectured with much applause in the lecture room,

How soon unaccountable I became tired and sick,

Till rising and gliding out I wander'd off by myself,

In the mystical moist night-air, and from time to time,

Look'd up in perfect silence at the stars.'
\end{verse}
It seems that Minnaert identifies with both figures here. He was the
learned astronomer, recounting his stories everywhere to grateful
applause. He was also the bored and almost nauseous listener who,
in the damp night air, surrendered to the mystical experience of the
starry sky. Did these two figures merge for the 55-year-old? Had he
gradually distanced himself from his earlier fixation on astrophysics?
His note at the Paris colloquium makes that likely.

Following his Charles Darwin lecture in the summer of 1947, Minnaert
embarked on a 12-day trek through Scotland. Writing from Oban, he
described 'a landscape already possessing the northern purity of color
and line, with the sea penetrating everywhere between the mountains
and forming long lochs.' In 1950, he took advantage of a conference
in Zurich to make a hiking trip over the Corsican Alps. Reporting
from Ajaccio: 'Monday morning, the driver and his assistant swore
ten times that I would have my backpack in Piana by evening. By Wednesday,
it still hadn’t arrived. Unreachable by phone. So I returned 75 km
to pick up the backpack: they had forgotten it. Voilà, c'est tout.
First, you get angry, then you understand it’s part of the country.
And never have I traveled as free and joyful as during those three
days without a jacket and with a thick beard. I sleep wonderfully,
without headaches, eat almost nothing, am healthy as a fish, and jump
from one adventure to another. And warm! And beautiful! Those colors
of water and rocks.' He felt perfectly happy with himself in nature.

\section*{The boys moving out}

Within his biochemistry specialization, Koen devoted himself to spectrometry.
In 1953, he interned at the Vlaardingen site of BPM. He wrote: 'This
week I had night duty; it’s wonderfully quiet. You only hear the hum
and hiss of the pipes and a faint sound of pumps. Around four o’clock,
you climb to the top of the distillation column (35 meters high) and
slowly watch the day break; you see how the water changes from a dark
strip to a light one, and then in the distance, you see the Waterweg
flowing together with the Oude Maas, and far away -- Maassluis and
Hoek van Holland. And so the day begins.'

He had moved into rooms. He lived at various addresses, including
a Corpshuis on Springweg 53. He enjoyed photography, pottery, and
singing in the Student Choir. He behaved like a bohemian and, influenced
by Du Perron, played the role of an anti-bourgeois, erudite intellectual.
Yet he was shy. He needed friendship, but people could be put off
by his clumsy and merciless honesty.

In the summer of 1956, he graduated cum laude. He could pursue his
Ph.D. under biochemist Slater and moved to Amsterdam. He was athletic,
went on winter sports at the age of 14, and walked a lot. Minnaert
wrote to his girlfriend Jet Mahy: 'Koen works hard but believes little
in life and ideals. The young generation has noticed how many of our
\textquotedbl ideals\textquotedbl{} were empty words and they want
no illusions; I trust that soon they will realize that not all principles
and ideals were meaningless, and then their skepticism will have had
a healthy and purifying effect.'

Koen's rebellious period ended when he fell in love with Els Hondius,
a psychology student and, according to Minnaert, a gifted violinist.
She was not very approachable, and his NJN friends sometimes wondered
if it clicked between the sensual Koen and his bride-to-be. Theo Quené
spoke of a 'young, quiet, stately, reserved woman of cool beauty.'
They married in Overveen on July 1, 1959. A year later, he earned
his Ph.D. on the respiratory enzyme Cytochrome C oxidase. He prepared
the enzyme from horse hearts obtained from the slaughterhouse, cut
them into pieces, ground them in a meat grinder, centrifuged them,
and collected the fluid. He exposed it to light of variable wavelengths
and recorded the spectrum with a spectrophotometer. His promoter thought
Koen had made an original proposal to explain the bizarre reaction
mechanism: he belonged 'to the small select group of people who truly
belong in a university environment.' Koen's solution would even endure
as Minnaert-IV in biochemical reaction kinetics.

Koen applied for a NATO scholarship to further qualify himself. On
the recommendations of professors Slater and Westenbrink, he eventually
received a ZWO grant, allowing him to work for a year in the U.S.
under Professor Lucile Smith at Dartmouth Medical School in Hanover,
New Hampshire. The following summer, Els became pregnant, and they
returned to the Netherlands. Koen could not start working with Slater
and took a position at Philips' Physics Laboratory. He joined the
'photosynthesis' subgroup of the biology department, which was led
by J. Voogd and consisted of 16 people. At that time, they were already
considering developing a biological basis for process technologies.
He continued his university research there. Els and he moved into
an apartment in Eindhoven. They named their child Paul Alexander Marcel.
Two years later, in the summer of 1963, Els was expecting her second
child.

Boudewijn left in 1954 as a trainee helmsman on the 'Straat Bali'.
In Hong Kong, he became the second-in-command in the 'passages' department
of his shipping company. In 1956, he used his vacation to travel through
Indonesia. He was transferred to Japan, where he lived in a traditional
house in the imperial city of Kobe. After four years, he received
six months of paid leave and briefly visited the Netherlands via America.
He spent two weeks with his parents in Spain and returned to Hong
Kong via Athens, Egypt, New Delhi, and the Himalayas. There, he was
instructed to depart for Sydney to prepare a new liner service.

By the end of 1958, he had joined forces with an architect at Contemporary
Constructions Ltd. His father wrote prudently: 'It is indeed risky
to give up such a secure, permanent position.' The project ultimately
failed. He became a seller of cigarette machines---one of those blue
moon jobs---and ended up at a travel agency in Sydney in 1961, which
was a subsidiary of a company in Melbourne. Minnaert noted: 'Modest,
but he's the boss. Meanwhile, he had met Noortje Steenbeeke; they
loved each other but could not marry because Noortje's husband refused
to divorce. In July 1961, they finally decided to live together.'
Years passed without the parents seeing their youngest child. They
only knew their daughter-in-law from letters.

In the 1950s, cats came into the house, which became especially important
for Miep. Minnaert: 'One evening in July 1953, a small black cat lies
in the kitchen. Presumably pushed inside through the open window.
We decide to keep him. He grows quickly, becomes a center of attention
in the family; especially Miep is very attached to him.' In Minnaert's
Diary, there are mentions of the favorites: 'Our cat had four kittens
in the spring: three black and one striped. The three are given away,
the striped one we keep; it’s Pietje. In the fall, another four come,
which, on the advice of the veterinarian, must be put down. The 'cat'
and Pietje remain and are a source of concern and interest.'

Minnaert, during this period, went to play music on Saturday evenings
with Gré Westenbrink, the wife of his chemistry colleague, and improved
his Russian in weekly lessons with Mrs. Kortsjagina.

\section*{Minnaert's passion for walking and traveling}

His wife did not like the hours-long walks and was no more suited
for it. Minnaert therefore traveled alone, sometimes with his son
or a friend. He likely avoided company unless he had to guide them.
At Easter 1954, he went to Sicily with Marieke van der Meer, the widow
of a PhD student. In 1956, he undertook a tour of Norway with her,
including day trips, 18 walks, much by bus and boat. Minnaert: 'One
very beautiful trip. Lots of clouds and quite a bit of rain. Marieke
gets tired quickly and isn't entirely resistant to the trips.' At
Easter 1959, they went to Greece: 'Small disagreements with Marieke
about hotel price classes, etc.' Minnaert did it as cheaply and simply
as possible. On the boat, he studied Modern Greek to manage. They
saw a lunar eclipse at the Acropolis and visited Crete.

In 1960, he traveled to Yugoslavia with Kee Proost-Thoden van Velzen,
a graduated astronomer, his companion on the Sumatra expedition of
1926 and by now the widow of Karel Proost, the leader of the Rotterdam-based
Ons Huis. She found the trip, which you completely devised and prepared,
unforgettable. Minnaert noted something about 'glowing glowworms';
this time, criticism of the partner was absent. She once asked him
to look at the slides together: in the Netherlands, it seemed that
20 of them were no longer there. Kee addressed him in her correspondence
as 'dearest Marcel.'

Most of his travels he undertook alone. Sometimes he visited friends
or went on a day trip with conference attendees. In 1959, he spent
more than two months abroad: after 1961, that became the minimum.

In 1955, he combined a lecture in Helsinki with a tour of Scandinavia.
On October 15, he wrote to Koen from Stockholm: 'Of everything I've
seen so far, certainly the most beautiful is the sculpture by Milles,
a modern Swede who died two weeks ago. His house, magnificently situated,
he has turned into a real museum where one can admire about a hundred
of his works. It's truly impressive. Particularly striking is, among
other things, the statue of your colleague Scheele.' On October 18,
he wrote to Bou from Finland: 'The biggest surprise for me were the
brilliant autumn colors of the birches against the dark spruces.'
A few weeks earlier, he had combined an IAU congress in Dublin with
a hiking trip on the west side of the island. On September 8, he wrote
to his wife from Caskel: 'The food is primitive: every day large potatoes
in their skins and one-centimeter peas, without salt or sauce. But
it tastes good.'

In 1957, he made a long journey through the United States, which he
had prepared with a Baedeker travel guide from 1900. He visited many
astronomers and emigrated Dutch and Flemish people. On August 16,
1957, he wrote to his wife from Ann Arbor: 'The colleagues are everywhere
equally cordial and almost every evening I'm invited somewhere.' His
PhD student Mulders showed him the nature parks. In Mount Parks in
Denver, he saw a performance of Wagner's {*}Die Walküre{*} amidst
impressive rock formations. He visited his childhood friend Jet Mahy:
'That sensation that Jet opened the door and that I saw her face and
eyes again!'

The following year, a trip was scheduled in connection with the IAU
congress in Moscow. This time, his wife was joining him. On August
18, Minnaert wrote to Koen: 'The congress is going wonderfully. We
all feel that we are experiencing a unique event, a true demonstration
of the unity of science (and humanity) above all borders. - There
is so much to see; we get lost in the immense distances and buildings.
Here you see the university, with 22,000 rooms. Have a good trip and
success in Vienna, and greetings to Els. Father.' After visiting Samarkand,
Tashkent, Kiev, Vladimir, and Gorki: 'A strange feeling, traveling
completely alone through this country, which is so ancient and yet
remarkably rejuvenated.' He visited the observatory of Ulugh Beg,
the Uzbek astronomer who, centuries before Copernicus, had concluded
from his observations that the Earth revolves around the Sun.

In the fall of 1958, {*}De Natuurkunde van 't Vrije Veld{*} appeared
in a Russian translation, soon followed by a Romanian one. Instead
of royalties, he spent a month traveling through Romania. On a trip
to an IAU congress in the United States, he had decided to stop in
Iceland and wander around for a week. The plane was filled with Swedish
conference attendees who, as he wrote to his wife in 1961, 'fortunately
flew directly to the US.' He took the bus and enjoyed the treeless
landscape. In Palo Alto, he visited Jet Mahy again. Later, he wrote
to her: 'If you were closer, I would love to talk to you about life's
problems and gatherings. I am certain that you would let me benefit
from your feminine wisdom and understanding. Even at this great distance,
I feel safe because we will not forget each other and we will help
each other by wishing the best for one another.'

Traveling became a passion to which Minnaert dedicated himself, alongside
many other activities, all pursued with equal energy.\\

1 Interview with Rudolf Mahy.

2 Interview with Hanneke van Konijnenburg-van Cittert Eymers.

3 Acknowledgment to D. Coelingh, secretary of the Dutch Natural and
Medical Congresses from 1905 to 1941, who passed away on September
4, 1946; Yearbook NNG 1947, 53 Minnaert was in the United States at
the time of the funeral.

4 Quotes from Minnaert's Diary.

5 In 1949, Koen borrowed, among others, the book by Ornstein and Moll:
Objective Spectrophotometry.

6 Interview with Prof. Dr. Th. Quené.

7 Interview with Boudewijn Minnaert.

8 Minnaert, Diary.

9 A letter from P.J. Waardenburg, son of Minnaert's fellow sufferer,
provided this quote. Minnaert wrote this in April 1943.

10 Pannekoek to Minnaert, June 7, 1950.

11 Minnaert, 1949, 70.

12 Postcard from Minnaert in the Hondius archive.

14 Minnaert to Jet Mahy (USA), December 24, 1956.

15 Letter from E.C. Slater.

16 The socialist J. Voogd was a prominent member of the Union of Scientific
Researchers.

17 Interview with Boudewijn Minnaert.

18 Hondius archive.

19 Kee Thoden van Velzen, Liber Amicorum for Minnaert.

20 Hondius archive.

21 Minnaert to Jet Mahy, November 2, 1957. Mahy Archive.

22 Minnaert to Jet Mahy, January 27, 1962.

\chapter{The Highest Astronomical Honors}
\begin{quote}
'There is hope that we will discover complete harmony in Nature, which
begins with the simplest atom or the far more complex star, and finds
its completion in us, mankind, humanity.'
\end{quote}

\section*{Minnaert as a Phenomenologist and Determinist}

Minnaert had written in The Astronomy and Mankind how a 'theoretical'
astronomer works and must have given an image of himself: 'From a
processing of measurements that represent the brightness distribution
across the solar disk, the theoretical astrophysicist has deduced
that an unexplained difference with existing theory remains. Cautiously,
he probes how our understanding of the sun's atmosphere should be
adjusted to achieve a better connection; his imagination moves back
and forth, he sees difficulties on all sides, one possibility seems
still open. He now calculates quantitatively what the consequences
would be of the new assumption. Such calculations are usually done
with letters, and only at the end are these replaced by known numbers
to see if the result meets expectations. Often, attempts must be repeated
many times before a satisfactory result is achieved. Our theorist
must continually follow the development of modern physics as well
as astronomy; mathematics is the tool he learns to handle fluently.'

Minnaert, who here describes the process leading to the explanation
of his growth curve, was close to the phenomena and experimental data.
The astronomer H.C. van de Hulst believed that Minnaert had a tendency
to understand a problem 'as elementarily as possible.' In this context,
he referred to a meeting in Chicago in 1946: 'I still vividly remember
how Chandrasekhar and Minnaert met at Yerkes. I was in the library.
Minnaert had just arrived to give a summer course and was browsing
through the shelves of books. Then Chandra came in and soon their
conversation turned to Minnaert's lectures: the spectrophotometry
of planets. Minnaert had discovered a method to determine the homogeneity
of planetary atmospheres by comparing photometric data from two points
on the planet's surface using the reciprocity principle. Chandra had
also encountered reciprocity in recent work on radiation transport,
and thus a lively discussion ensued.

Then came the moment I remember sharply. Minnaert held his left hand
as if it were the planet's surface, pointed to it with a finger of
his right hand, which was supposed to represent an incoming light
ray, and said, 'Let's consider it in the simplest way.' Chandra looked
at that finger, but his brain worked differently. A minute later,
I saw him pointing to the table as if it were filled with mathematical
formulas. I heard him say, 'It’s quite simple. This matrix is symmetrical.’
Both experts referred to 'simplicity' as a bridge to mutual understanding,
but they had different notions of simplicity. Minnaert always tried
to make the problem intuitive; his conversational partner operated
within a framework where such intuitiveness played no role.

The astronomer H.G. van Bueren also noted, when asked, that Minnaert's
descriptions align with Ornstein's tradition: 'Minnaert worked semi-quantitatively.
He was a phenomenologist; he wanted to explain phenomena in terms
of processes. That was the Utrecht tradition: line intensities were
measured experimentally and not theoretically predicted. Minnaert
also wanted to map the solar atmosphere using 'models' that increasingly
approximated the phenomena.'

In the 1950s, Minnaert had a correspondence with the theoretical physicist
H.J. Groenewold, an employee of his study grant project, who had been
appointed as a lecturer in Groningen. Groenewold had noted in his
inaugural lecture that over the centuries, there had generally been
a shift in physics from a more intuitive image to increasingly abstract
formalism. Everything indicated that in the future, there would be
further detachment from intuitiveness. Minnaert congratulated him
on his 'very interesting lecture' and praised him. However, Groenewold
had provoked him: 'You present it as if physics is continually becoming
more abstract, more formalistic, less intuitive. This may be true
for the last 50 years. But over longer periods, one sees a back-and-forth
swing from the concrete to the abstract and back. It is certainly
true that the introduction of kinetic heat theory meant a strong increase
in intuitiveness. - So I do have some hope that we will now move toward
intuitive forms again.' Minnaert did agree with Groenewold that one
had to continue working with formalism, which had already yielded
so much positive results: 'The possible development must occur from
within; emotional preferences do not help here.'

In the same letter, Minnaert pointed out that intuitiveness was important,
but rather that the question of whether phenomena are determined or
not was even more crucial. Groenewold replied that he believed these
two aspects were inseparable for Minnaert. In the late 1950s, Minnaert
gave a lecture on this topic under the title 'Prediction.' He had
taken the stance that 'principle predictable' was a challenge for
every natural science. The meteorologist W. Bleeker opposed this view,
summarizing Minnaert's position as follows: 'If we know the initial
state of the atmosphere with sufficient accuracy, and we understand
the laws of nature, then we can calculate the state at any arbitrary
future moment. We do not need to consider uncertainty relations, because
it is a large system. And after all, there are electronic computers!'
Bleeker recognized in this vision the 19th-century deterministic worldview
of Laplace. On the other hand, Bleeker believed that Minnaert's conditions
could never be met in principle: 'The 100\% reliable weather forecast
will never arrive.'

In smaller circles, Minnaert thus revealed himself as a phenomenologist,
but even more so as a committed determinist. The latter he would still
publicly uphold.

\section*{The Gold Medal of the Royal Astronomical Society (1947)}

In the late 1940s, Minnaert no longer directly contributed to groundbreaking
work. According to astronomer De Jager, he never fully regained his
old scientific fervor: 'He no longer had the patience or inner peace
to immerse himself in a long-term program.' De Jager also believed
that observational instruments and computational techniques were evolving
in ways that required a new generation. Minnaert still occupied himself
with a few astronomical projects, such as measuring the Atlas of the
solar spectrum and the photometry of Venus and the moon. His doctoral
students, such as Van de Hulst with his radio radiation and De Jager
with his refined solar models, drove groundbreaking developments.

However, in the 1930s, Minnaert had sown abundantly and could look
forward to a rich harvest after the war. In 1946, he was appointed
as a full professor and as a member of the Royal Netherlands Academy
of Arts and Sciences (KNAW). He became an advisor to Keesing's Historical
Archive, an editorial board member of the First Dutch Systematically
Arranged Encyclopedia (ENSIE) and a board member of the Utrecht Studium
Generale and the Volk Universiteit. At the request of his colleague
Oort, he gave a teaching assignment in astrophysics in Leiden. He
became a member of the Hollandse Maatschappij der Wetenschappen and
the Advisory Board of the Utrecht community of the Humanistisch Verbond.
His international reputation kept pace with his involvement in global
cooperation.

On March 10, 1946, he attended a 'nuclear congress' of the International
Astronomical Union (IAU) in Copenhagen, which succeeded in restoring
international collaboration: 'All participants were so happy to see
each other again after those difficult years. Even the Americans and
Russians came, totaling 25 people. I’m busy translating.' He must
have been one of the few who could effortlessly switch from English
to French and also spoke some Russian. In the summer of 1946, he spent
several months in the United States, where he gave guest lectures
in Chicago and made photographic plates. Until 1948, he chaired the
IAU Commission 36 on The Theory of Stellar Atmospheres and from 1948
to 1951, Commission 12 on Solar Radiation and Solar Structure.

British astronomer Eva Gorst thanked him in a postcard for 'the wonderful
and interesting evening in Utrecht' and added: 'We want to congratulate
you on the Gold Medal awarded to you by England, the highest honor
in the astronomical world.' Indeed, he received this medal in 1947.
On May 9th, he delivered the Charles Darwin Lecture to the Royal Astronomic
Society on The Fraunhofer Lines of the Solar Spectrum. It was a historical
summary of advances in solar research, revisiting the asymmetric redshift
in line profiles. Later, he had received plates from American astronomer
Evershed showing a shift toward violet: 'A true asymmetry of the Fraunhofer
lines, sometimes mentioned in literature, cannot be accepted unless
confirmed by very precise experiments.' He himself had worked for
years on the quantitative determination of this alleged redshift.
Minnaert missed the opportunity to revisit this topic after twenty-five
years with a witty remark: self-deprecation was not his strongest
suit.

Of course, he discussed his 'equivalent width,' which the IAU had
just elevated to an international standard. The Utrecht team was working
on a Table of measured values of the equivalent widths of 20,000 line
profiles: 'I can inform you that the infrared spectrum, in the first
order, is practically ready for publication.' His colleague Babcock
had told him that the Americans also wanted to create such a Table.
The efforts could be combined: 'The IAU is considering publishing
a table of transition probabilities based on the Fraunhofer lines,
using data from experimental physics and theoretical calculations.
This would be of the utmost importance for physics and astrophysics.'

Indeed, this project got off the ground under the leadership of Minnaert
and Charlotte Moore-Sitterly from the National Bureau of Standards
in Washington. The Fraunhofer lines thus became Minnaert's life's
work: his equivalent width dated back to 1923, the growth curve to
1929, the Atlas was from 1940, and now the Table followed, which could
occupy them for another ten years. He thereby followed in the footsteps
of Kirchhoff and Rowland. He was not named after either the 'equivalent
width' or the 'growth curve.' That Minnaert had designed these everyday
tools is eventually forgotten in a dynamic science. By 1947, that
was not yet the case, and it had earned him the Gold Medal.

The conclusion of his speech was triumphant: 'The main goal of all
our work must be to reduce a great diversity of seemingly unrelated
facts to the logical consequences of a few fundamental laws, with
the ultimate result being: the relative concentrations of the individual
elements. At this stage in our science, we recall a statement by August
Comte in his {*}Cours de Philosophie Positive{*} from 1835: \textquotedbl There
are questions that will forever remain hidden from the human mind,
such as, for example, the composition of celestial bodies.\textquotedbl{}
Time and again, we see that no one should attempt to limit the great
sweep of science or human progress.' Minnaert had no need for philosophers
who a priori defined the limits of what is knowable. He would use
his next major prize to venture into the philosophical domain himself.

\section*{The Catherine Bruce Medal}

In 1951 Minnaert received the second main prize of astronomy: the
Catherine Bruce Medal from the American Astronomical Society of the
Pacific. The winner is nominated by the directors of six leading observatories,
three of which are in the United States. Minnaert could not collect
the prize because the U.S. government, in the heat of the Cold War,
denied him a visa as a 'fellow traveler.' His American colleague Struve
heard his speech in Utrecht, presented the medal, and published the
remarkable text.

Minnaert wanted to refute philosophical statements that hindered the
development of biology in {*}The Significance of Astronomy for Biology{*}.
Considerations about stars and other celestial bodies were usually
not included in philosophical debates about 'vitalism.' Minnaert compiled
a list of characteristics typical of living organisms based on biological
textbooks. It turned out that 'living' systems have many properties
also found in 'lifeless' systems like a star. Therefore, Minnaert
suspected that a living organism could be understood as a physical
system of very great complexity.

Every star has a striking 'individuality,' just like an organism.
The hierarchy in the organs of living beings finds its counterpart
in the hierarchy of the astronomical sequence: satellite, planet,
star, stellar system, and cluster of systems. A star has an individual
'past': its mass and chemical composition are the result of its history
of origin and determine its development. A star does not rotate too
quickly, as this would eject too much matter; if it pulsates, a damping
mechanism automatically occurs. A star is 'a whole' with interactions
between the individual parts, which, when damaged, can provide an
impulse for recovery. If part of the material is removed, a star can
restore its spherical shape. All of this seems 'intended' to allow
the star to survive. Yet no astronomer would claim that there is an
intention behind it. So why do many biologists do so?

Some might object that a star does not possess individual consciousness
or 'soul.' The question of the star's soul raised the issue of the
soul of other organisms, comparable to those of a fungus or a mussel:
'There is indeed an important difference in this regard between the
organic and inorganic nature, but this difference is not fundamental,
just as differences in form, organization, and hierarchy are not.
These are differences that develop over the course of evolution and
gradually become more significant. To claim that consciousness begins
at some point in the chain of evolution contradicts, in my view, all
principles of continuity. If we know from direct experience that we
possess consciousness, then we must conclude that even the simplest
organisms possess this consciousness; and not just them, but also
individual atoms, albeit in the most primitive and latent way.'

Minnaert repeated here what he had earlier written to Burgers. People
had to free themselves from the idea that there is a fundamental difference
between living and non-living nature: 'There exists the hope that
we will discover complete harmony in Nature, which begins with the
simplest atom or the far more complex star, and finds its fulfillment
in us, humanity.' This deterministic view also implied that if knowledge
and instruments were sufficiently developed, a living being could
be made step by step.

Actually, Minnaert was not fighting against vitalism here but against
dialectics. The philosopher and dialectician Hegel had provided numerous
examples in his {*}Wissenschaft der Logik{*} of quantity transforming
into quality, and vice versa. His followers Marx and Engels had followed
him in this regard. In the Netherlands, few scientists engaged with
dialectics. Struik and Pannekoek, personal friends of Minnaert, were
exceptions to this rule. Pannekoek had responded extensively in a
1950 letter to this continuity thesis. When moving from atoms to things,
his mentor Minnaert had warned, it involves a difference of the order
of 10 to the power of 20: 'Quantitative differences of this magnitude
are unimaginable to us, manifesting as the most fundamental qualitative
differences, which appear to us as entirely incomparable worlds.'

Minnaert's continuity thesis implies that physical concepts expressing
'intensity,' such as pressure, temperature, or viscosity, do not exist
because they have no meaning at the atomic level. The reverse reasoning,
concluding from the 'fact' of human consciousness that atoms possess
a primitive form of consciousness is curious. Chlorine gas is yellow;
a chlorine molecule is not. If so many physical 'qualities' disappear
in the transition from object to molecule, why would a more complex
quality like 'consciousness' remain?

Minnaert gained significant international prestige in astronomical
circles after the war, even beyond the scientific domain. In the 1950s,
Minnaert became vice-chairman of IAU Commission Physical Studies of
Planets and Satellites and chairman of Subcommission 14a on Intensity
Tables. In 1951, he was appointed as a respondent at the Portuguese
University of Coimbra and as a member of the Royal Flemish Academy.
He was pleased with his appointment to this Dutch-speaking academy:
'I consider it a great honor to be one of you. May your work thrive
and benefit our beloved Flanders and humanity!'

He spoke in Brussels about The Surface of the Sun. He discussed Rupert
Wild's physical discovery (1939), which had led to a paradigm shift
in the interpretation of the continuous solar spectrum:

'Relatively recently, it has been shown that the radiating gas is
a very special negative ion, consisting of an H-atom that has bound
an extra electron. Such a negative H-ion forms, for example, during
the electrolysis of NaH; it has not yet been possible to produce it
in gaseous form, but its properties can be calculated. It is then
found that it must emit virtually uniform continuous light across
the entire visible solar spectrum and into the near infrared.' The
mystery of how a radiating gas ball could produce a continuous spectrum
with Fraunhofer lines was finally solved!

\section*{Minnaert's standing in Science}

Minnaert often had the ability to accurately sense which direction
astronomical research was taking. He published compilations, such
as his Charles Darwin Lecture, which opened up the field for the new
generation. In his 1952 lecture Astronomy to the Dutch Society of
Sciences, he championed a historical view of the universe and the
Milky Way system, the sun and the planets. If a geologist wondered
how the Earth's crust was composed, he inevitably had to consider
the forces that had worked in the past.

Therefore, the theoretical method of astronomy itself had to be extended
so that it could look back into the past: 'If the celestial body had
a simple structure at a certain moment in prehistoric times, we must
go back to that moment and, following all the behaviors of matter
from there, rediscover the entire complexity of the current state.
This detour of the theoretical method is thus a valuable tool for
us to solve the most inaccessible problems of the present, while also
bringing us to a piece of science about the past that is highly important
in itself.' The first results of radio astronomy had opened up a view
of the universe and removed this creation history from the realm of
Science Fiction and placed it on the scientific agenda.

Time and again, one had to ask: 'How did all this come about?' The
biologist answered with the concept of evolution, the astronomer with
the concept of cosmogony: 'In the last ten years, a remarkable boom
in this branch of science can be observed. It is my personal conviction
that it will soon become the central astronomical theory, in which
many now separately developed parts of our science will find their
place - just as the theory of evolution has been, in a certain sense,
a synthesis of paleontology, comparative anatomy, embryology, systematics,
and biogeography.'

It was now established that the stars were not all created once and
for all in a distant past, but that new stars are continuously being
formed, almost before our eyes, most of them as double stars. There
might have been a pre-stellar stage of the universe, in which matter
was extremely densely packed at a very high temperature. Opinions
were divided, but optimist Minnaert stated, 'Simply asking the question
about these such fundamental things testifies to the audacity of modern
astronomy and our confidence in the scientific method.'

Minnaert was convinced that there are countless 'habitable' worlds
in the Milky Way system, let alone beyond it: 'The biologist will
probably not doubt that many of those habitable planets are indeed
inhabited.' Interplanetary travel was technically conceivable, although
'military purposes had already been associated with such experiments
beforehand, which are a contamination of the exalted, benevolently
pure essence that has always been inherent in astronomy.' Therefore,
he emphasized the need for international cooperation between East
and West. With a nod to current panic stories about mysterious space
radiation, he suggested 'that this truly international orientation
of astronomers might well be due to a very special, beneficial radiation
from outer space.’ It was, in fact, a well-known fact that many leading
astronomers in the international community had a left-wing or far-left
leaning: Pannekoek, Minnaert, and De Jager formed the guarantee for
a respectable branch of this camaraderie in the Netherlands.

Minnaert was involved in his friend Gerard P. Kuiper’s prestigious
series on the solar system. In 1953, Part I about The Sun appeared
in Chicago with an article by Minnaert titled The Photosphere. In
such an article, he overviewed the field and sought to do justice
to his own contributions as well as those of his students. An important
role in this article was reserved for the discussion of recent solar
models by De Jager.

Work on the tables for the Utrecht-Washington project continued steadily.
Minnaert, Houtgast, and other staff members proceeded with a pre-publication
in 1960 because astrophysics urgently needed these values to compare
the intensities of stellar spectra with those of the sun. This Preliminary
Photometric Catalogue on Fraunhofer Lines contained values with wavelengths
ranging from 3614 to 8770 Å, covering ultraviolet to infrared. The
three columns provided, for all lines, the wavelengths, equivalent
widths, and ‘reduced widths,’ which are the equivalent widths divided
by the local wavelength.

In 1960, Minnaert contributed to Kuiper’s Part IV on The Planets with
the original article The Photometry of the Moon. No celestial body
offered such an excellent opportunity to utilize the range of photometric
methods. His argumentation notably featured a practical approach,
which had also characterized his choice of solar research. He included
space for his own contributions and the recent dissertation by his
doctoral student J. van Diggelen. Together, they had wondered how
the specific scattering and absorption of sunlight on the moon’s surface
could be explained. Numerous common formulas had not led them further.
This had tempted Minnaert to choose a new starting point: 'The photometric
properties of the moon are primarily determined by the shadows of
millions of irregularities on its surface.'

Van Diggelen had tested numerous surface models: 'The moon's surface
is apparently an assembly of tightly packed holes of various sizes,
next to and on top of each other, carved out in dark material. These
model experiments confirm the results obtained through theoretical
calculations.' They had compared the degree of porosity of the lunar
surface with biological material such as {*}Cladonia rangiferina{*},
reindeer moss. The photometric properties of this moss almost matched
the observations: 'As a result of the weakness of the moon's gravitational
force, very loose surface formations can be created, with properties
close to our model.' His fascination with the geophysical structure
of the lunar surface was striking. The celestial body exerted a special
attraction on the romantic Minnaert, who could speak about the beauty
of the moon and become enraptured by it.

\section*{Against divining rods and flying saucers}

Minnaert's enthusiasm for spreading knowledge and science found its
counter in the fanaticism with which he opposed 'pseudo-science.'
In {*}Sterrenkunde en de Mensheid{*} ({*}Astronomy and Humanity{*}),
his declaration of war against astrology had also been included. Minnaert
had argued for a structural responsibility of the Physics department
of the KNAW in this regard and had found support.

When a hype arose in the late 1940s about the religion of the Great
Pyramid, {*}De Stenen Spreken{*} ({*}The Stones Speak{*}), Minnaert
had responded on AVRO radio with {*}Spreken de 25 Stenen?{*} ({*}Do
the 25 Stones Speak?{*}). He repeated what he had earlier explained:
'The dimensions, interpreted in the right way, would indicate all
sorts of numbers from astronomy, numbers that the ancient Egyptians
could never have known. The pyramid would have been built under the
direct inspiration of God. If you convert those measurements into
time spans of years and days, he claims that the entire history of
humanity can be read from the pyramid, including predictions for the
future and all kinds of wise advice.' Minnaert amusingly demonstrated
how numbers can endlessly be manipulated. This criticism by the atheist
Minnaert was warmly supported this time by religious authorities.

When new fads emerged in the 1950s surrounding earth rays and dowsing
rods, the Academy, and Minnaert himself, were ready. For instance,
J.G. Mieremet had established the First Dutch Bureau for Dowsing Rod
Research against Health-Damaging Soil Influences and designed a portable
device with which one could register 'earth ray tracks.' The Physics
Department of the KNAW formed a committee under the leadership of
physicist J. Clay, with members S.T. Bok, G.G.J. Rademaker, and Minnaert.
This committee reported in 1954:

- That the dowsing rod had not proven its reliability as a detection
tool for either known or unknown phenomena in any of the cases examined;

- That the existence of so-called earth rays had not been demonstrated
or made plausible in any case;

- That convincing evidence had been provided of the worthlessness
of the examined devices for 'destroying earth rays or neutralizing
their effects';

- That it was desirable for the government to provide protection against
the activities of manufacturers of anti-earth ray devices, especially
when such activities operated in the medical field.

A working committee under the leadership of medical physicist L.H.
van der Tweel had conducted 28 experiments, during which, apparently,
there was quite a bit of laughter. Minnaert, the only one who took
the work very seriously, believed that Mieremet's challenge to test
his devices should be accepted. The working committee outright refused.
The ink of the KNAW report was barely dry when Mieremet promised new
29 wonders. He used the report as propaganda: 'For nineteen years,
I have had to fight, and the devastating statement from the KNAW remains
fresh in everyone's memory.' He had designed a better device: 'It
will no longer be possible to avoid the necessity of protecting hospitals,
sanatoriums, dormitories, schools, offices, many homes, barracks,
and apartment buildings from these overdoses. Stables and farms as
well, because with the aforementioned instrument, we can now demonstrate
that horses, cows, pigs, and poultry become restless and sick in certain
places. But especially for cancer treatment, our work will prove to
be of great value.’ These resounding reports could count on pages
of text in regional newspapers.

After the Report, the committee was dissolved and no new one was established.
More useful was probably the action in Het Vrije Volk from November
1954, where editor L. Velleman, together with professors Oort, Minnaert,
and W. Bleeker, tackled the 30 Flying Saucers: ‘The article aims to
deal with Flying Saucer madness before it spreads to our country.’
The astronomers provided sensational descriptions of observers with
commentary and pointed out natural phenomena such as meteors, fireballs,
comets, auroras, perspective distortion, sun dogs, moons, and mirages.
In a TV broadcast from December 1954, Minnaert himself critically
questioned a KLM pilot who had observed an opaque ‘light ball of orange
or pink color.’ These debates were likely more effective than the
KNAW committee’s report.

\section*{May 5th and the Humanist Youth}

Halfway through the 1950s, Minnaert gave a welcome speech at a meeting
of Humanistische Jongeren (HJG) on May 5th. He urged them to commit
themselves to the free, unbiased ‘growth of truth, unencumbered by
preconceived opinions.’ The authorities tried to obscure the commemoration
of May 5th because the Federal Republic had to be able to rearm. Young
people should not take the path of least resistance: there was no
reason to look at the behavior of the majority!

Other youth groups had religion as their central idea. In the program
of the Humanistische Jongeren, it was only stated that they did not
want to commit themselves to religion. Minnaert found this too negative.
It was about seeking truth honestly. Some suggested that the highest
wisdom lies in showing that we know nothing for certain. The young
people had to resist that. It was about 'scientific truth,' which
largely stood firm, even though science continued to evolve. In this
growth, it was hindered by religion and the church. When that institution
finally had to give in, it claimed that science 'has nothing to do
with faith.' The young people had to liberate themselves from the
arbitrary commands of the church. The great societal truths, the laws
and commandments of society, and the notions of 'good' and 'evil'
were not given by God but originated in society.

The humanistic youth had to act on principle and radiate certainty
where appropriate. That was not fanaticism; rather, it was a conscious
resistance against weakness and indifference: 'What now seems hopeless
will soon become achievable and self-evident. The conditions in the
world are changing at great speed.' Peace had to be preserved. In
their later lives, they had to remain loyal to these fresh convictions.
They had to give meaning to their lives and 'not live in vain': 'Only
those who do something for it deserve happiness, freedom, and life.'

Minnaert also continued to play a role with his stance on the 'social
aspects of student recruitment' in the discussion among progressive
youth about university reform. In the mid-1950s, he led a working
group on behalf of the board of the Humanist League, which produced
a policy document advocating for 'study grants.' By 1963, the torch
was finally taken up by the student movement itself.

\section*{The First Public Observatory in Oudenbosch}

In {*}Sterrenkunde en Mensheid{*}, Minnaert also argued that professional
astronomers should serve the work of amateurs. They should not ask
for compensation for this. He therefore gave numerous current lectures
for the Association of Weather and Astronomy. These stories often
ended up in magazines such as {*}Hemel en Dampkring{*} and the {*}Nederlands
Tijdschrift voor Natuurkunde{*}. His lectures were accompanied by
beautiful slides, which he naturally called 'lantern slides.' Entire
folders are filled with the lectures, which every time, were updated
with recent events and developments.

The stories alternated between The Sun, The photosphere, Sunspots,
Contemporary issues concerning Fraunhofer lines, The Corona, Clouds,
Halos, The origin of Earth, Aurora borealis, Disturbances in the ionosphere,
The ozone layer (since 1938!), Cosmic phenomena in the atmosphere,
The moon in folklore and science, Jupiter, Sputnik, Is atomic energy
a curse or a blessing?, The physics of comets, The evolution of stars,
Meteors and meteorites, Stellar radiation, Binary stars, Nebulae of
gas and dust, Magnetic fields in the Milky Way, The structure of the
Universe, Radio waves from space, and much more. New topics were constantly
added, such as quasars and results from satellite images in the 1960s;
old ones were updated. If he held two lectures of this nature per
week for forty years, he must have spoken at several thousand gatherings
with dozens of interested individuals about weather, natural history,
and astronomy. He was a propagandist proclaiming the salvation of
science.

As a popularizer and board member of the Utrecht People's University,
Minnaert became involved in an initiative by J.A.F. de Rijk, alias
Bruno Ernst. In 1959, De Rijk had visited a friend in Switzerland.
That evening, in the garden, his friend’s wife enthusiastically told
him all about the stars. De Rijk had just published The Atlas of the
Universe, with a foreword by Minnaert. The curious guest learned that
the hostess's interest had been sparked by a visit to the Zurich People's
Observatory and concluded: 'Then it would be worth starting a similar
initiative in the Netherlands.'

Shortly before, inspired by the attention surrounding Sputnik, Minnaert
had created the booklet Getting Acquainted with the Starry Sky for
the Current Topics series, which aligned with De Rijk's plan. He was
enthusiastic and encouraged the idea wherever he could. De Rijk wanted
to name the observatory in Brabant’s Oudenbosch, just a stone's throw
from the scale model of the Roman St. Peter's Basilica, after Professor
Minnaert, but Minnaert declined. Instead, Minnaert suggested naming
the institute after Simon Stevin, his fellow citizen from Bruges with
whom he interacted daily at the time. And so it happened.

On January 14, 1961, Minnaert opened the People's Observatory Simon
Stevin with a glowing speech in which he outlined the joys of personally
observing the starry sky. Indeed, 'amidst all the tensions and threats
in human society, the perfect beauty and regularity of the great Universe
stand as a comfort, an encouragement, a timeless harmony.' Since there
had been much fuss around the opening, De Rijk felt a sense of anxiety
when climbing the monastery tower that Minnaert would be disappointed
if he saw the observation platform measuring four by four meters with
its meager main instrument being a 10 cm diameter refracting telescope
and then the small room for slide presentations with its sixteen chairs.
When he addressed Minnaert in that vein, Minnaert had paused on the
stairs and said: 'That is precisely the way to begin something great.
A mighty oak also starts as a small sprout.'

The People's Observatory relied on volunteer work and funds collected
through De Rijk's talent for fundraising during its first years. He
sought subsidies for a paid staff member. Minnaert accompanied him
to the ministry to convince the officials of the importance of this
observatory for public education. The lobbying was successful. Later,
De Rijk received the Silver Carnation from the government for founding
the People's Observatory, with Minnaert acting as his paranimf: 'When
I saw how such a mark of honor was being awarded, it gave me the idea
to establish my own token of merit for the popularization of astronomy
and space exploration. It became the Simon Stevin Telescope.' De Rijk
asked Minnaert if he would accept this, and Minnaert assured him it
would be an honor. De Rijk garnered much publicity when Minnaert accepted
the second Simon Stevin Telescope; a gilded Dutch telescope that provided
double magnification.

Initiatives from dedicated amateurs could count on Minnaert's support.
He exemplified this attitude for his students, PhD candidates, and
staff. This helped ensure that the Utrecht Observatory remained the
center of the Dutch amateur astronomy and meteorology community.\\

1 Minnaert, 1946, 11.

2 Interview with astronomer HC van der Hulst. Also found in: Meeting
Chandra, Leiden Observatory, undated.

3 Interview with theoretical physicist H.G. van Bueren.

4 Congratulations from Minnaert on February 25, 1954; response from
Groenewold on March 1, 1954. Provided by Aart Groenewold. For a biographical
sketch of H.J. (Hip) Groenewold, see Molenaar, 1994, p. 200.

5 His opponent referred to this lecture in Bleeker, 1962.

6 De Jager in the SG lecture, 'A Man of My Taste,' in Haakma, 1998.

7 Many diplomas and honors in the Minnaert archive at the Utrecht
University Museum.

8 Postcard from Minnaert. Hondius Archive.

9 Postcard from E. Gorst in Castle Combe to Minnaert, February 9,
1947.

10 Minnaert, (1951).

11 He repeated the core of his argument in lectures on Astronomy and
Biology for the Humanistisch Verbond and Volksuniversiteiten. Summary
statements from a lecture on April 23, 1953, for 'Pyramide' have been
preserved. History Archive.

12 Astronomer E.J. Öpik labeled Minnaert's philosophical position
as 'animistic' in {*}Life and its evolution from an astronomical viewpoint{*},
which he sent to Minnaert with compliments: {*}The Irish Astronomical
Journal{*}, 2, 1952, 9. Such controversies did not hinder friendships;
on the contrary. History Archive.

13 Minnaert often mocked Hegel, whom he, like Comte, dismissed as
a dogmatic thinker.

14 Hegel, G.W.F., {*}Wissenschaft der Logik{*}, 1832-1845, Frankfurt
1969. Marx, {*}Das Kapital I{*}, 1867. Engels, {*}Dialektik der Natur{*},
1883.

15 Pannekoek to Minnaert, April 19 and 20, 1950.

16 Commentary on this passage from science philosopher Dr. E. Glas
of TU Delft was utilized.

17 From 1952-1955 and 1958-1964.

18 From 1955-1958.

19 Minnaert (1952), with a speech from November 9, 1951.

20 Kuiper's work on {*}The Sun{*} in 1953,

21 De Jager in Drummen, 2001, p. 38.

22 Minnaert and J. Houtgast, Preliminary Photometric Catalogue on
Fraunhofer Lines, Recherches de l' Observatoire d'Utrecht XV, 1960.
Contributions came from Claas, Hubenet, Van de Hulst, De Jager, and
the assistants J. Blom and M. Dekkers.

23 Kuiper's work on {*}The Planets{*} and the Moon's photometry in
Chicago, 1961.\textquotedbl{}

24 Van Diggelen, 1959. Minnaert, (1941).

25 Minnaert for AVRO, August 17, 1950: Do the Stones Speak?

26 The KNAW committee was officially called the Investigation of the
Divining Rod and Earth Rays Problem.

27 Report of the KNAW Committee on the Investigation of the Divining
Rod and Earth Rays Problem.

28 History Archive, Dowsing Rods and Earth Rays folder.

29 History Archive. Mieremet clipping.

30 L. Velleman, Het Vrije Volk, November 1954. The TV broadcast with
pilot H. Dill from Blaricum took place on December 1, 1954. There
were hardly any television programs yet: Minnaert was an early adopter
of the new medium.

31 Note of the welcome speech at a meeting of the HJG. Undated. History
Archive.

32 Humanist Alliance, Report from the Study Grant Committee, 1957.

33 Astronomy Archive.

34 Letter from J.A.F. de Rijk, alias Bruno Ernst, alias Brother Erich,
dated January 6, 1995.

35 B. Ernst, Atlas of the Universe, 1958. Ernst also wrote a well-known
publication about the graphic artist M.C. Escher.

36 Manuscript opening of the Public Observatory in Oudenbosch. Astronomy
Archive.

37 The installation later moved to Hoeven as the 'National Public
Observatory.'

38 Drummen, 2001, chapters 13 and 14.

\chapter{The Solar Wind of the Cold War}
\begin{quote}
'Therefore, you are already introducing McCarthy's methods (that's
what it is!) and thereby perpetuating the evil that one seeks to combat.
The witch hunt has begun.
\end{quote}

\section*{Minnaert and Coelingh as Peace Activists}

The sixty-year-old Minnaert, meanwhile, supported the work of World
Peace Congresses and the related Dutch Peace Council (NVR). This peace
movement opposed the atomic weapons race, against the American atomic
bomb and German rearmament. The 'activists' saw this as a struggle
against resurgent fascism and militarism. However, the movement was
hierarchically structured along the lines of national communist parties,
with the Soviet Union determining the course. Minnaert was aware of
this and apparently considered it the lesser evil. He remained in
resistance.

He believed it necessary for a peace movement to be fueled by popular
activities. The World Peace Congresses had produced the Stockholm
Appeal of March 1950, which under the leadership of his French colleague
F. Joliot-Curie had grown into a worldwide campaign against atomic
weapons. The wording played on the threat of an atomic war over Korea:
'We will regard as a war criminal any government that first uses the
atomic weapon, against any country whatsoever.' The campaign could
count on support from painters Picasso and Chagall, the Russian writer
Ehrenburg and the American singer Paul Robeson, West German Thomas
Mann and East German Anna Seghers, architect Le Corbusier, and Finnish
President Kekkonen. In the Netherlands, the NVR, with logistical support
from the CPN, collected more than a million signatures of support.
The campaign was fiercely opposed by other political parties, particularly
the PvdA.

In 1951, action was taken against the planned rearmament of West Germany.
This aroused aversion in the country itself, in neighboring countries
like France, and in the Soviet Union. NATO considered this rearmament
a necessary defense against communism. The Stalinist executions in
the 'People's Republics' were the perfect catalyst for this arms race
after Korea. The World Peace Congresses ignored the show trials and
repression in Eastern Europe and thereby took sides.

Minnaert supported the initiative For a peaceful solution to the German
question. He was present at an International Conference on this topic
in Paris. The secretary of the Dutch group, communist artist Hermine
Lataster-van Hall, came to Sonnenborgh to discuss it with him. Miep
Coelingh later said: 'They were trying to win my husband over when
I entered with coffee. Someone asked, ‘Don’t you have any interest?’
They invited me to speak at the World Peace Congress in Vienna. I
became a member of the Dutch Peace Council. Whenever Irène Joliot-Curie
needed to be somewhere in the Netherlands, she stayed with us.'

She attended colloquia on The German Question, where Germans from
West and East began discussions on preventing rearmament. In the Netherlands,
a petition campaign was started, supported by the magazine {*}Vrede{*}.
On August 22, 1952, the front page featured ‘Miep Minnaert-Coelingh,’
who argued that it must be prevented that a German army would rise
again under the leadership of ‘Nazi generals.’ The arming of the West
would provoke the arming of the East. Only a peace policy based on
reason and justice had a future.

The first Conference had to be postponed until late 1952 due to the
denial of visas for East Germans, moving it from Paris to Odense,
then to Stockholm and Basel, until it could finally take place in
East Berlin. Out of 200 ‘delegates,’ 46 came from West Germany, 26
from East Germany, and 35 from ‘Greater Berlin.’ Minnaert-Coelingh,
along with Lataster-van Hall and Sita Anderson-Cochius, became the
leader of the Dutch campaign against the European Defense Community.
She gave speeches everywhere, wrote for {*}Vrede{*} and other magazines,
and lent a hand or provided support. She found herself in a whirlpool
of contacts and foreign trips. She participated passionately and disciplined
in the activities: she had a new calling. Both spouses eventually
had an equal relationship: not in science, but in the peace movement.

\section*{The Minnaerts in the Soviet Union}

In March 1952, Minnaert and Coelingh together received an invitation
for a scientific trip to the Soviet Union. This was likely connected
to their role in the peace movement. They visited Prague, Minsk, Moscow,
Leningrad, and Crimea. Amidst the ruins of Leningrad, Minnaert gave
a lecture at the Poelkowo Observatory. They visited the Academy of
Sciences, the Lenin Library, the House of Scholars, and the Planetarium.
His lecture at the Sternberg Institute in Moscow attracted a hundred
interested attendees. In Armenia, he met astronomers like Ambartsumian
who wanted to exchange publications and participate more actively
in the work of the IAU. He noted: 'insufficiently reversed.'

The discussion contributions after Minnaert's lectures consistently
started and ended with quotes from Lenin, Stalin, Engels, or Marx.
He wrote: 'uncertainty.' In his view, they constructed an artificial
opposition between 'Soviet science' and 'Western science.' He decided
to point out the ideological nature of their interventions: perhaps
it was because they maintained insufficient contact with the scientific
outside world. At the time, it must have been partly uncertainty and
partly self-preservation, as the KGB always and everywhere made notes.

He observed that his conversation partners feared scientific results
that seemed to contradict communist ideology. In the fields of heredity,
cell formation, and cosmogony, scientific theories were rejected a
priori. Everyone should be able to agree with some minimal starting
points, Minnaert thought:

- every theory that can be scientifically justified has the right
to exist and must not be attacked for ideological reasons;

- experience, observations, are the foundation of science;

- one must not attack Western science (if it is serious), but rather
certain philosophical, erroneous conclusions;

- many Westerners present all sorts of things as established in their
popular works, which is more like 'self-promotion.'

He chose a position that clashed with the 'theory of two sciences'
that party ideologist Zhdanov had decreed in 1948. Following this,
there was the canonization of the biologist T.D. Lysenko, who had
placed the primacy of all science in 'dialectical materialism.' Incidentally,
the views of Lysenko's astronomical counterpart, Ter-Oganesov, had
dominated in the 1930s had lead to the death of his Russian correspondent
at Pulkovo, E. Perepelkin, and the latter's boss Gerasimovitch. Minnaert
seemed unaware of this.

According to his hosts, there was an insurmountable opposition between
'materialists' and 'idealists.' The use of this terminology was new
to him, although he had encountered it before with De Jager. Minnaert
had upheld his 'idealism' in the matter of study grants against Polak.
In the Soviet Union, the term had a negative connotation. Minnaert
found it surprising and 'extremely interesting.' The couple made a
wonderful trip and gained a positive impression of the country and
its people.

\section*{A Utrecht polemic about the GDR}

Minnaert sometimes became involved in polemics in the Utrecht university
magazine {*}Sol Iustitiae{*}, such as the one about study grants with
his colleague Hijmans van den Bergh, who accused him of state dirigism.
Equally fierce was a 1953 polemic over an action by the University
Asylum Fund for East German students, involving Minnaert, Miep Minnaert-Coelingh,
and student A.W. Burger.

Burger had criticized the lack of intellectual freedom at the East
Berlin university: the state scholarships for workers' children, the
compulsory Marxist philosophy curricula, and the mandatory membership
in the Freie Deutsche Jugend (FDJ): 'Every lecture must be submitted
in writing beforehand and presented to the university leadership for
approval, to ensure it aligns with Marxist-Leninist doctrine and communist
self-satisfaction.' He also reported that 30\% of all students suffered
from serious illnesses, particularly tuberculosis, and that 25 girls
had recently been deported to Siberia.

Minnaert thought Burger made no effort to understand the complicated
situation. Most students in the GDR lived on scholarships. The young
state wanted to quickly establish an intellectual cadre. That deserved
appreciation rather than contempt. Burger seemed to reject all forms
of study financing, whereas Minnaert was in favor of it. Additionally,
he questioned whether the alleged abuses were truly as widespread
as claimed.

Two weeks later, again across the entire front page, Burger responded.
The study included political exams that one had to pass. The police
surveilled students: 'Children from bourgeois, educated families are
barred from the university unless they have clearly demonstrated their
political zeal.' The East German student was not given time to think
and judge: 'This is how one would create an intellectual proletariat
willing to follow the regime in everything.'Ironically, fifteen years
later, 'activist' students would make the same claims about studies
in Utrecht, Paris, and West Berlin.

Minnaert found Burger's vision exaggerated. Since his wife was going
to Berlin, he asked her to present Burger's criticism to people she
could meet at the university. On June 12, he approached the editorial
board with a letter from his wife containing 'some data proving how
cautious one must be with 11 one-sided information.' That letter also
adorned the front page. She had presented healthcare statistics to
a West Berlin doctor. Tuberculosis control was more effective in East
Germany than in West Germany. At the Peace Congress, she met the Dutch
filmmaker Joris Ivens, who had been expelled from the Netherlands
because of his film Indonesia calling and found that the freedom to
work in the East was greater than in the West. She spoke with people
involved in student housing who provided an explanation for collective
facilities that didn't even exist in the Netherlands. A conversation
with two non-Marxist philosophers revealed that it wasn't entirely
unreasonable to engage with Marxist ideas. 374 This polemic illustrates
how things were done at the time. Burger's criticism covered both
East Germany's efforts to create healthcare and education as well
as the state's ideological terror. Miep Coelingh praised the facilities
and, considering the circumstances, had understanding for the ideological
pressure. Minnaert had publicly criticized that pressure during their
visit to the Soviet Union. He would continue to do so, as in his speech
commemorating the Polish astronomer Kopernikus, better known as Copernicus.

\section*{The commemoration of Copernicus (1953)}

After Stalin's death and Khrushchev's rise to power, a brief 'thaw'
occurred in 1953. The World Peace Council had drawn attention to Copernicus
(1473-1543). Indeed, the war had prevented the commemoration of his
death. Minnaert was the speaker at a gathering in Amsterdam where
representatives of the diplomatic corps from Poland and the Soviet
Union sat fraternally next to the Rector Magnificus and representatives
from politics and science.

Minnaert condemned the Roman Inquisitors who had embittered the lives
of Copernicus and Galileo. Copernicus had distanced himself from the
Earth as the center of the universe and replaced it with the heliocentric
worldview. Galileo had defended his views. According to the spiritual
authorities, this was heresy: 'Galileo was forced to retract his statements
and was broken for the rest of his life. Copernicus's book was placed
on the Index of forbidden books in 1616, which could not be read until
they were amended.' Science had forged its own path. When the Church
officially lifted the ban in 1822, it no longer mattered to anyone.

Minnaert praised the courage of this 'free spirit' and advocated for
the freedom of research, which also includes the freedom to hesitate.
In the usual biographies of such scientists, it was as if they knew
exactly what they were doing, as if they had discovered the truth
through a series of systematic investigations. In reality, science
was a process of wonder and searching, with concentration and dedication
being necessary conditions.

In recent years, that heliocentric worldview had, in turn, been brought
down to Earth. In the cosmos of Kapteyn and Oort, the Sun was an insignificant
star: 'We are part of an enormous lens-shaped swarm, a kind of flat
disk consisting of about 100 billion stars. One of these Milky Way
stars is the Sun, around which the Earth orbits. It is relatively
far out in the system; closer to the center, the stars are packed
more densely, but there is no 'central star.'

Thinking of Copernicus, and likely also of Flanders and socialist
Poland, Minnaert sighed: 'How few have the imagination needed to fully
think through a new idea, the goodwill to embrace it, the courage
not to shrink from the new!' This is indeed also the reason why various
social reforms take so long to come about. And for this reason, one
can also understand that practitioners of the exact natural sciences,
trained in the pursuit of truth, are often socially progressive, just
like poets or politicians. Imagination, a good will to examine new
ideas, and the courage to accept them when they prove to be good---these
are the characteristics of the truly progressive person, both in science
and in societal life.'

Minnaert now also theoretically linked the practice of exact natural
sciences with social progressivism, although on this occasion he had
also included politicians and artists. He called Copernicus a symbol
of 'struggling and victorious science' and thus shook off the Catholic
{*}ecclesia militans et triumphans{*}.

His diary mentions a trip to Poland, where he visited his niece Helena
and astronomical institutes and observatories. Among his colleagues,
in addition to Russian reference works, he saw publications from all
parts of Europe: \textquotedbl While they energetically build on
the development of their own Polish science, they can also play an
important connecting role in bringing East and West into contact with
each other. We think with great sympathy of that reborn country, which
has directed itself toward a new future through energetic work, and
we send our warmest wishes to the scientific workers there.\textquotedbl{}

Just as he assigned a peace role to the Netherlands as a small country
within the Western camp, so too did he propagate such a role for Poland
within the Eastern Bloc. His view of the role of the Netherlands had
earned him a file with the BVD (the Dutch security service): would
his view on Poland be less controversial? This message was between
the lines and will have completely escaped most people. Attentive
listeners, however, could interpret these elegant sentences as an
invitation to Polish politicians to fulfill a bridging function between
East and West during this time of détente.

\section*{A hidden polemic?}

Minnaert received a letter from sculptor J.W. Havermans, a board member
of the CPN (Communist Party of the Netherlands), following his lecture
on Copernicus. His speech had dealt with 'the struggle for the freedom
of science against the dogma (and thus the compulsion to limit) of
the Church.' Havermans believed that Minnaert had suggested that society
would always cause resistance. As a result, he had created a basis
for attacks on the Soviet Union, 'where the Central Committee of the
communist party expresses its preference for a particular interpretation
or interpretations in a branch of science on points that are of major
importance to society, based on the circumstance that science and
society form an unbreakable unity.'

He interspersed his views with a selection of quotes about 'historical
materialism.' One of them came from F. Engels' {*}Anti-Dühring{*}:
'It is not nature and the realm of humans that align with principles,
but rather the principles are correct only insofar as they correspond
with nature and history.' Havermans feared accusations that could
be made possible through Minnaert's speech. He had consulted with
the prominent communist Sebald Rutgers, a pioneer of heavy industry
in the Soviet Union, who shared his concerns.

Minnaert acknowledged that societal conditions direct researchers'
interest toward certain subjects, encourage or hinder boldness: they
influence the way humans engage in science. However, science {*}itself{*}
must free itself from this: 'Everything that does not align with experimentation
is simply dismissed by science, entirely independent of societal wishes
or tendencies. It is therefore wise that historical materialism opens
our eyes to many things that were previously accepted unconsciously
as self-evident, while in fact societal influence was at play; in
this way, a salutary critique is exercised, and we are able to free
ourselves more swiftly from these prejudices.'

However, Minnaert had observed that communists believed that 'historical
materialism' provided them with a magic formula to dictate scientific
guidelines, {*}even for natural science{*}: 'There is no such thing.
The statements of historical materialism in their classical form are
completely outdated and incorrect.' The Russian Communist Party had
'committed the greatest errors and mistakes in this regard. Worse
than these errors is the fact that ignorant individuals are granted
the right to judge matters they know nothing about, and whose judgments
are clouded by misunderstood societal views and emotional prejudices.'
Nature would soon call Russia to order: 'I can do no better service
to Russia than by honestly stating where they are mistaken.' Minnaert
referred to the history of Lysenko as a mistake that would heavily
exact revenge in agriculture. Similar issues existed in chemistry
and physics: 'This drive does not stem from serious scholars, but
from individuals of limited expertise that have not been properly
schooled in the scientific research method.’

Havermans' quotes from Marx, Engels, and Lenin made no impression
on Minnaert. One could extract whatever one wanted from their work:
‘Wouldn’t we rather rely on our common sense?’ These men had lived
long ago, and since then much had changed in philosophy! They too
could be mistaken: ‘However, I gladly agree with your quote from Engels,
which states that principles are not the starting point of research,
but that they must adapt to nature. That speaks to my heart.’ If scientists
in freedom could conduct their work, the chance was greatest that
free minds would find a direction that at that moment would be considered
anti-social and yet prove to point the way to a new society: ‘Thanks
to Copernicus!’

Minnaert felt connected to the artist’s line of reasoning but considered
his positions and those of the CPN {[}Communist Party of the Netherlands{]}
fatal. Havermans was right: Minnaert had indeed sent a signal that
would not have gone unnoticed by Polish and Russian authorities and
the leaders of the CPN.

\section*{Resurrection of the Union}

Minnaert, as a board member of the Union of Scientific Researchers,
continued to follow its activities. Chairman Fred Polak set a different
course. The arms race around Korea, in his view, had made cuts in
education and research inevitable. In an NRC series from April 1951
on Culture and Defense, he had come up with the egg of Columbus. He
wanted to allocate part of those defense expenditures to ‘culture,’
thereby connecting cultural and military resilience. This trick was
intended to secure additional funds for scientific research.

Minnaert felt compelled to respond to this cuckoo's egg. Totalitarian
aspects lurked: collaboration between defense and education had characterized
Nazi Germany. American colleagues, on the other hand, wanted to take
the funds out of the hands of the military and had established the
National Science Foundation for that purpose: ‘How can someone lull
themselves into such illusions at a time when our defense has simply
become a small part of the Atlantic Pact, where political and economic
forces play an enormous role!\textquotedbl{} Was that really the stance
of the chairman of a Federation that had wanted to engage in battle
with the powers seeking rearmament?

A few members within his Federation, such as Beets, Wertheim, the
philosopher H.J. Pos, and the ethnologist P.J. Meertens, were at that
time involved in the establishment of the peace movement {*}The Third
Way{*}, but Minnaert did not join them. Their manifesto rejected both
the Atlantic Pact dominated by America and the Kominform controlled
by Russia. Minnaert would have placed the primary responsibility for
the Cold War on the West. This reorganization nevertheless served
as an anchor for the revival of the Federation, within which the left-wing
sociologist Wertheim could evolve from a tuned administrator to chairman
within three years.

The international thaw created space for criticism of the arms race.
Minnaert joined the opposition within the Federation that manifested
itself with the Amsterdam report {*}The Frustration of Science{*}.
The conclusions of that report were presented by Wertheim and Pos
at the 1954 conference on {*}Freedom and Bondage of Science{*} in
The Hague, attended by Queen Juliana. Minnaert had covered the event:
'If scientific freedom in the Soviet Union is in danger, we must oppose
it, just as when it is threatened by fascism, Catholicism, Calvinism,
or American witch-hunts. But it is improper to exploit certain mistakes
of the Soviets regarding science to simultaneously combat a social
system that has brought new happiness to millions of \textquotedbl the
wretched of the earth\textquotedbl{} and which, in principle, conflicts
no more with scientific freedom than \textquotedbl true\textquotedbl{}
religion does with true science. Scientific freedom must not be used
as a weapon in the disastrous Cold War \textquotedbl to defend a
social system that is faltering\textquotedbl{} (Pos).'

He participated in the revived criticism within the Federation of
the nuclear arms race. Two American hydrogen bomb tests on Bikini
Atoll had led to the contamination with radioactive fallout of the
inhabitants of the Marshall Islands and Japanese fishermen. Wetenschap
\& Samenleving issued a call to 'the most prominent scholars' to share
their insights regarding nuclear weapon tests. Minnaert sounded the
alarm; only the geneticist M.J. Sirks had preceded him. He ultimately
read the American Bulletin of the Atomic Scientists, which had raised
the storm flag a few months earlier. With symposia like Danger to
Mankind and actions against Civil Defense---the government's plan
for an unprecedented building program of atomic shelters---philosophers
and scientists such as B. Russell, H. Bethe, H.C. Urey, L. Mumford,
J.R. Oppenheimer, L. Pauling, and A. Schweitzer engaged in the public
debate. In April 1955, American nuclear physicists led by the Dutchman
S. Goudsmit launched an attack against McCarthy's accusations and
the government's visa refusals. They established a Scientists' Committee
on Loyalty and Security, which declared that fear could not be a good
advisor in defending against any 'ism.'

The Dutch initiatives ran parallel. Minnaert called for 'the establishment
of groups of scientific researchers in many countries around the world,
with the aim of warning humanity about the immense danger threatening
us.' He pointed out that his Verbond had idly watched while the Federation
of Atomic Scientists remained steadfast. Albert Schweitzer had appealed
to the researchers: 'They alone oversee the dangers; they alone have
sufficient authority to make it clear to humanity that such tests
must stop.'

Minnaert pleaded for the establishment of a movement of researchers
and experts who would strive to halt the arms race and initiate mutual
disarmament of weapon capabilities. The Verbond organized a meeting
on April 2, 1955, for two hundred members and interested parties about
The dangers of current atomic weapons. The speakers were the theoretical
physicist Tolhoek, the biologist De Haan, and Minnaert.

\section*{The societal consequences of nuclear weapons}

Speeches were compiled in the book {*}Scientific Researchers Warn
Against the Use of Nuclear Weapons{*}. Out of a print run of six thousand
copies, three-quarters were pre-ordered by the Dutch Peace Council.
Minnaert's activities within the League and those of Miep Coelingh
in the NVR aligned with the same goals. In the foreword by Wertheim,
it was stated that 'scientific researchers cannot better fulfill their
shared responsibility for global events---at this decisive moment
in world history---than by helping to prevent the one great catastrophe
that could render all further human activities meaningless.' Tolhoek
described the physical effects of a hydrogen bomb and the genetic
consequences of radioactive fallout. De Haan expressed concerns about
the waste from future 'nuclear reactors.'

Minnaert's speech on {*}The Societal Consequences of Nuclear Weapons{*}
was comparable only to that of theoretical physicist Leo Szilard in
the United States. Science and technology were harnessed in the nuclear
arms race: 'The joy of discovery is soured by the awareness that every
advance will soon be used for greater destruction.' The arms race
created an iron necessity to anticipate the progress of the opposing
side. Every party wanted to possess more and better bombs and superior
defense systems. Civil defense exercises created a sense of security,
making 'civilian defense' part of the race.

The American slogan {*}Peace through Strength{*} suggested a feeling
of safety, while humanity approached the abyss with eyes closed: 'The
fact that such a large portion of humanity watches and waits relatively
passively and indifferently is, in my opinion, a reason for the greatest
concern. Apparently, the threat is so great that people feel completely
powerless against it and try to make their lives bearable while disengaging
from responsibility for the future.'

Nuclear scientists and biologists failed to enlighten the public adequately:
'Acquiring and disseminating knowledge of the truth must always be
the foundation of our actions. The citizen of a democratic society
should not passively endure what happens to them; they must be able
to form a conscious judgment about important matters that deeply affect
their life and that of the community.' He called exploiting ignorance
'the worst form of public deception.' The international politics had
to seek solutions 'that are in the interest of both parties' and gradually
dispel the mutual distrust.

Minnaert discussed work of that nature already being carried out by
a few scientists: 'But where is the drive of the masses? Why don’t
intellectuals from all countries unite to take the lead in the popular
movement?' A revision of the media's approach was needed: 'From the
stream of facts, everything that could contribute to creating a favorable
impression of the East is systematically filtered out; a single newspaper
taking the opposing view does the same against America. How unwise,
how foolish, how dangerous! While before our eyes a grand experiment
is being conducted with two social systems, each trying to fully develop
their potential, we can do nothing but carp and sow hatred in hearts.'
He was apparently referring to the communist daily De Waarheid.

Minnaert envisioned a 'convergence theory,' also propagated by an
authoritative economist like J. Tinbergen. The society in the East
was one-sidedly based on the principle of mutual cooperation, while
in the West it was one-sidedly based on competition: 'I am personally
convinced that the flaws on both sides of the world will naturally
correct themselves through natural development; and that those two
halves will inevitably converge toward a common societal form---at
least, if no war occurs to disrupt this entire development.' Minnaert
still adhered to the view that 'mutual service' formed the foundation
of the relationship between the Soviet Union and the 'People's Republics.'

People should not look for what divides us, what is ugly, small, and
bad, but for what unites us, what is beautiful and good. No contribution
could be missed: 'One person feels more drawn to direct action, another
to participating in a new mindset. As long as many, as long as everyone
participates, while there is still time. And as long as everywhere,
in all hearts, the awareness awakens that life is sacred, and that,
ultimately, it is equally terrible if the hydrogen bomb strikes people
in the East or in the West. For here as well as there, joy and suffering
are known, here as well as there, people work and love, and mothers
and children exist everywhere in the world.' Renewal of the Peace
Council In 1955, the World Peace Council, under the leadership of
F., began...\textquotedbl{}

\section*{Renewal of the Peace Council}

In 1955 Joliot-Curie launched a campaign for a summit of The Five---namely,
the United States, the Soviet Union, France, England, and China---to
peacefully resolve mutual disputes. Within the Dutch Peace Council,
the 'thaw' created space for reflection on its sectarianism and the
democratization of the organization. Secretary Nico Luirink wrote
to Minnaert: 'Many of us believe that your insights, elaborated in
the VWO magazine, are correct and a guide for our work.' Minnaert's
joining could be significant: many people were waiting for each other,
and a breakthrough would strengthen the peace movement. Minnaert noted
in the margin: 'Yes.'

Luirink asked whom he would like to have by his side in that case.
Minnaert specified: 'Three professors, several doctors, pastors, jurists,
teachers, union representatives, workers, etc.' At the same time,
Minnaert-Coelingh was in Helsinki, where the leadership of the Dutch
delegation at the World Peace Congress had been entrusted to non-communists
like Reverend Lazonder. The independent space offered to her in the
peace movement since 1951 was partly occupied again by her husband.
The movement seemed to replicate daily family relationships, with
Minnaert in political leadership and Coelingh in the working committee.

The Peace Council achieved success in those years through an activity
focused on a common goal, to which the organization was subordinate.
At the end of 1955, Elske de Smit-Kruyt initiated a petition among
doctors calling for 'a general ban on experimenting with and using
atomic weapons.' It was a great success: 524 doctors signed it. Subsequently,
the NVR designed an Appeal requesting the government and parliament
to urge the United Nations 'to stop the arms race; cease nuclear weapons
tests, condemn their use, and limit armament.'

On January 11, 1956, the Committee for the Cessation of Atomic Bomb
Tests (CSA) was formed 'at the initiative of the NVR, which, however,
completely withdrew afterward and left decision-making to the meeting.'
All present agreed that 'NVR board members should preferably not serve
as chairman or secretary of the Committee.' Wim van Dobben, a Wageningen
biologist and treasurer of the NVR, could participate, but the then-secretary,
theologian Feitse Boerwinkel, could not. The Committee received support
from several members of the PvdA, Rabbi Soetendorp, theoretical physicists
like Tolhoek and Nijboer, and liberals such as Kappeyne van de Coppello
were involved. Much of the executive work was handled by the NVR.
The CSA made its public debut on May 15, 1956, with a publication
involving 384 prominent individuals, including Minnaert, who presented
it to Prime Minister Drees.

In April 1956, the famous congress of the Communist Party of the Soviet
Union took place. Khrushchev had condemned Stalin's crimes in a secret
session, paving the way for a relaxation of internal repression and
his own 'collective leadership.' The CPN wanted to initiate criticism
of the personality cult within its ranks, which affected the core
of its functioning. In this situation, the Daily Board turned to several
intellectuals, asking them to respond to their new assessment of the
international situation. Minnaert responded positively.

\section*{Minnaert to the Daily Board of the CPN}

He began his letter by explicitly stating that it should not be published,
'neither with nor without a signature.' However, the Daily Board's
sudden change in insight prompted serious criticism from Minnaert:
'Firstly, it is striking that your change in perspective precisely
followed the changed insights in the USSR. This reflects a very limited
independence of thought and speech; it is remarkable to hear you,
at this juncture, claim how independent the CPN is from foreign influences.
You may mean that no one dictates to you---this is possible---but
then you are, at least spiritually, unconditional followers of the
Soviet Union. Clearly, many among you have long held the thoughts
now being expressed; but they have spoken and written differently
than they thought. Who assures us that you truly mean what you say
now?

You have consistently approved everything that happened in the USSR.
I challenge you to show me a single article from De Waarheid in recent
years where you have fundamentally criticized any action of the USSR.

Even now, you are continuing entirely in the same direction; for example,
you have not considered critically discussing the latest decision
of the CPSU and pointing out its weak points. Is there really no one
among you who wishes to defend Stalin against what he is accused of?
You no longer want to take a stand on internal events in other countries
'if they escape your judgment.' The misfortune is that one can never
determine when an event escapes judgment and when it does not. In
this way, strictly speaking, one would never be able to publish foreign
news. Therefore, you must be guided by the best possible assessment
of available reports and an impartial weighing of the facts. And this
is precisely where you have consistently failed. Out of a desire to
provide a counterbalance to the bourgeois press, you have presented
a distorted image of events in the opposite direction. It now becomes
clear how you have blindly approved and defended all sorts of things
that happened in the Soviet Union, while everyone could plainly see
that they were wrong. I remind you of the history of Lysenko and certain
trials.

The principle of exercising caution in judging foreign affairs is
not sufficient. A global opinion must form about events in other countries,
whether they bring humanity more happiness and prosperity or violate
human rights.

Taken altogether, my criticism of the CPN amounts to: a lack of independent
thinking and a lack of love for truth, both toward yourselves and
in your statements. And now I fully understand how you arrive at these
actions. Under the pressure of conservative groups, you feel compelled
to close ranks more tightly. In the face of threats from capitalist
countries, you feel compelled to support the USSR, the bastion of
socialism. These considerations are correct and good. However, they
must never result in distorting the truth. For ultimately, truth prevails.
Then you must revoke your previous statements and disavow what you
have maintained for years. Moreover, people no longer believe that
you yourself think what you say. I have the impression that you would
be stronger if you were to think and speak more purely and honestly.
After all, there are significant differences in the positions of the
social democratic parties of individual countries. It seems to me
that there should be more freedom of personal opinion among your party
members: why shouldn't someone be allowed to think differently from
the party leadership about Stalin, about birth control, about emigration,
etc., without being immediately regarded as a traitor?\textquotedbl{}

Minnaert showed himself at the end of his argument to be wary of provocations,
precisely because the new course was a step in the right direction.
The Russians had finally made heroic efforts toward 'true democracy':
'Hands off the Soviet Union at a moment when it is making a turn in
a direction we all welcome!'

The revelations about the crimes of the Stalin regime, the extermination
of the party cadre in the 1930s, and the millions of victims of hunger
and terror must surely inspire the peoples in the brutally subjugated
satellite states to rise up, even if rational Minnaert considered
it inopportune in light of global relations. After the bloody November
of 1956, the renewal of the CPN was temporarily off the table. Not
because it was 'historically necessary,' but due to factors Minnaert
had pointed out in his analysis.

\section*{The witch hunt has begun!}

The thaw came to an end when the Soviet Union suppressed the Hungarian
uprising. At the same time, the Suez Canal, nationalized by Egyptian
President Nasser, was seized by the British and French with the help
of Israel. These bloody events put the 'Spirit of Geneva' back in
the bottle. Minnaert wrote in his Diary: 'The events of November 4,
1956, in Hungary bring confusion and concern. It is difficult to hold
the Peace Council together. Whoever thinks differently from the masses
is suspect. The idea of coexistence is being abandoned by many. Miep
and I are mainly in agreement, although I go further and disapprove
of the Russian action. But our great goal remains: to preserve peace;
it was wrong to start an armed uprising at the time when there were
real opportunities for improvement. We supported each other a lot.\textquotedbl{}

Two anecdotes may further illustrate Minnaert's response. The first
comes from the communist Luirink shortly after a crowd in Amsterdam
stormed the Felix Meritis building, where at the time the CPN leadership
and the daily newspaper De Waarheid were housed. In Utrecht, too,
homes of communists had been attacked. He wrote: \textquotedbl Miep
Minnaert and her husband, both at the door of their house by the Observatory.
Concerned, very concerned about friends. How are you? Did they harm
you? And your family? How are they, with her, with him. Do you need
food? And then, after finally believing, or so it seemed, that we
were really managing reasonably well, the question: What are we going
to do? Not a word about their doubts. Whether they would continue.
Questions that many others had asked themselves. What are we going
to do? Should we perhaps issue a statement or something similar? It
is after all a serious situation. And then - somewhat shyly, perhaps
hesitantly - we have already made something, if you agree with it...\textquotedbl{}

The second anecdote concerns the working atmosphere at the Observatory.
His then Ph.D. student Aert Schadee: \textquotedbl I experienced
the Hungarian uprising as a brutality and found it terrifying. A mood
arose that we didn't dare look each other straight in the eye. Houtgast
asked directly: 'Professor, what do you think of the events in Hungary?'
Minnaert said that the events were taking a course he could not approve
of. That took the pressure off.

For Minnaert, a delayed reaction to Hungary emerged in the privacy
of the Utrecht Senate Hall. Unlike Houtgast, his colleague Koningsberger
dared not ask him how he stood on the developments. Minnaert would
become Secretary of the Senate in 1957 due to seniority and Rector
Magnificus the following year. Koningsberger had previously opposed
Minnaert, as in the late 1940s when he himself called for the preservation
of the Indies for the Netherlands. He believed he had to protect Minnaert
from himself. If the Russians were to invade the Netherlands, Minnaert
might collaborate. He contacted several colleagues who would follow
Minnaert in line. Just before the Senate meeting with the decisive
vote on the appointment, Koningsberger informed Minnaert. His defense
gives an idea of the havoc caused by Koningsberger's action. Minnaert
turned to the rector:

'You have informed me that the five colleagues who will next come
up in sequence for the rectorship do not wish to collaborate with
me as Secretary; they do not want to see me become Rector because
they object to my political views. Under these circumstances, I do
not intend to make myself available for the rectorship. To fulfill
my task properly, the Rector must feel supported by the trust of his
colleagues. If a group under You is unwilling to give me that trust,
then I do not covet the rectorship. But I do have something to say
to these colleagues and to those who might agree with them. What they
reproach me for is mainly that I continue to cooperate with groups
whom people fear would suppress free speech if they came to power.
Therefore, you are already introducing McCarthy's methods (for that
is what it is!) and thereby perpetuating the very evil one seeks to
combat. The witch hunt has begun.

Well, I stand on the ground of coexistence, the necessity of dialogue
and negotiations, in the conviction that by doing so I promote the
cause of freedom more than You do. Breaking off contacts, which You
would want from me and which You wish to impose on me with boycott
methods, do You know what it is, do you recognize this? It is part
of the Cold War.

You who do not wish to engage in joint consultation have no other
alternative to offer than this Cold War and the accompanying arms
race, which drains all the strength of nations and leads us to the
abyss. If a third world war were to break out and atomic bombs were
to fall, ask your conscience whether you have done all you could to
avoid this war, or whether you may have contributed to it in some
small part.

I oppose the fact that the university is being misused by you here
to push through your political views. Do not tell me that it is about
universal human values. Human values are as dear to me as they are
to you; we only differ in the way we seek to realize them. You push
your own approach and, in doing so, enter a slippery slope. For what
is now being done to me will soon be repeated against other minorities
in the Senate. All the less do I think of breaking off the contacts
that you reproach me for, because it is about cooperation to preserve
peace and oppose atomic weapons between people who are each among
the warmest advocates of peace within their own groups. It will not
happen again that peace movements fall apart at the very moment when
the danger of war looms!

Dear colleagues, I sincerely wish those who will later become Rectors
may be able to do fruitful work that brings satisfaction and promotes
the flourishing of our university. As for me, I have stability and
peace in the belief that the world is progressing and that new forms
of society are emerging, where war will be a thing of the past, where
everyone will share in life's happiness, and where freedom, which
is as dear to you as it is to me, will also be realized.'

Minnaert's strong words were listened to in deathly silence. His friend
Freudenthal felt guilty: 'What I regret afterward is that I did not
request a recess at that moment. After Minnaert's words, spoken in
his beautiful voice, everyone was impressed. If only I had asked for
a recess after that and then asked Koningsberger whether those who
had refused to collaborate with Minnaert would be willing to stand
up. At least one of the first five assured me he had not said 'yes.'
These are moments where you fail.'

The professors Van der Blij, Endt, Van Hove, Nijboer, and Smit, together
with Freudenthal, established at the following Senate meeting that,
'knowing Minnaert's qualities as a person, colleague, and official,'
they deeply regretted the course of events. They refrained from a
revision, 'because we do not wish to question the good faith and sincere
intentions of those involved in these events.' These supporters also
chose not to address the issue in a principled or public manner. At
the Observatory, during his lifetime, no one was aware of what had
happened: it remained 'among professors.'

\section*{With back against the wall}

In the late 1950s, several scientists took responsibility for issues
related to peace and disarmament. The Dutch section of Pugwash, led
by theoretical physicists like Groenewold, Nijboer, and Tolhoek, exposed
the government's information on hydrogen bombs and civil defense,
which was called Bescherming Bevolking here. With a special informational
issue of the VWO magazine, of which tens of thousands of copies were
distributed, a connection was made between natural science and the
peace movement in 1962, just as in 1956. A new phenomenon was that
many church discussion groups placed orders for this issue.

Minnaert-Coelingh remained the spokesperson for the Dutch Peace Council
during those years at international consultations. She was part of
the presidium of the 1958 World Peace Congress in Stockholm. Chair
F. Joliot-Curie cited a quote there from American Henry Kissinger:
'Since the peace movement began with the Stockholm Appeal for Peace
in 1950, for which it collected more than 500 million signatures worldwide,
it has conducted a well-organized campaign to organize massive protests
and actions against the use of nuclear weapons.' The work of the World
Peace Council and national peace councils could not be dismissed as
propaganda, warned the future minister.

At that congress, Minnaert-Coelingh spoke during the plenary session
about the conflict between the Netherlands and Indonesia over Papua
New Guinea: 'I myself, like several members, believe that immediate
sovereignty should be granted to West Irian. We deeply regret that
we must decide that our difference of opinion will be expressed in
our voting behavior: some will vote for the resolution, others will
abstain. I am deeply sorry because I know that progress does not come
from abstaining from voting. I wish our Indonesian friends great success
in their future struggle for independence, which is in the interest
of both our countries and world peace.' In another version, possibly
the spoken version, this passage was omitted.

In the Netherlands, she was involved with action committees advocating
for an immediate withdrawal of troops from New Guinea and direct negotiations
with Indonesia. She spoke at a National Peace Congress on January
27, 1961. The monthly magazine {*}Vrede{*} published her speech to
1,100 delegates on the front page: 'Our country must therefore disengage
from all military treaties that create international dangers, foreign
aircraft bases and nuclear weapons must be removed from our country,
our own army must come under Dutch command, and the unilateral ties
to West Germany must be replaced by the old, trusted policy of neutrality,
which finds expression in a peace treaty with both Germanys.' The
Dutch government thought it necessary to fight the conflict with Indonesia
militarily again and was once more called to order by its allies.
In August 1962, the transfer to Indonesia took place.

\section*{Andries Mac Leod: Minnaert Not a 'Pacifist'}

Minnaert considered himself a pacifist. Many people in his environment
found this a questionable political stance, but the seriousness of
his pacifism was not doubted. With the founding of the Pacifist Socialist
Party (PSP) in 1957, as an extension of {*}De Derde Weg{*}, a political
translation of pacifist ideals emerged in the Netherlands.

Among Minnaert's manuscripts is a speech on War and Peace from 1958
that he delivered at the first action against the establishment of
American missile bases in the Netherlands. He began with a quote from
Bertha von Suttner's {*}Lay Down Your Arms!{*} and outlined a course
of gradual steps toward complete disarmament. He explicitly counted
the PSP as part of the peace movement, which was previously unthinkable
within the CPN. He believed that the PSP and the CPN should work closely
together, first and foremost in the field of peace.

Minnaert had maintained contact with his childhood friend Andries
Mac Leod, the son of his teacher. Occasionally, he visited him in
Uppsala, Sweden. MacLeod 37 contradicted the idea that Minnaert would
naturally be an opponent of violence. In their discussions, he defended
the position that the use of nuclear weapons is criminal and that
there are no circumstances that justify such use: 'Therefore, even
if the United States refuses to relinquish nuclear weapons, it is
a crime for the Russian rulers to retain their nuclear weapons. It
is their duty to declare that they will not use nuclear weapons under
any circumstances, even if they were to be attacked with nuclear weapons
by the US.' The reverse applies equally.

Minnaert disagreed with this. He believed that if the United States
refused to conclude a treaty with Russia banning the use and possession
of nuclear weapons, Russia had the right to retain its nuclear weapons
and defend itself against a nuclear attack by the US: 'To this, I
responded as follows, addressing Marcel: you do not have the right
to claim that you are an opponent of nuclear weapons; both parties,
the US and Russia, are willing to relinquish nuclear weapons if the
other party does so too; your views on nuclear weapons are therefore
shared by everyone; you are no more opposed to nuclear weapons than
anyone else. This accusation I made was indignantly dismissed by Marcel.'

MacLeod contrasted his principled pacifism with Minnaert's pragmatic
pacifism; or rather, the PSP's stance with that of the CPN and the
Peace Council. Minnaert did not know how to deal with political reality
well and remained too much of a romanticist and a 'theoretical' pacifist.
He refused to seriously address his friend's objections.

This observation by his comrade from the Ghent Bollandkring may be
helpful in analyzing Minnaert's stance. For his own feeling, he was
always a 'pacifist.' Yet his resistance attitude and the identification
with the struggle of the 'Boers' and 'poor Flanders' that followed
from it seemed to outweigh his pacifism in his early publications.
It seems as if after World War II, 'poor Flanders' in Minnaert's worldview
was replaced by 'poor humanity,' which he identified with the struggling
and suffering Soviet Union. MacLeod's testimony suggests that it was
not his 'pacifism' but his militant resistance was the constant factor
in his thinking and actions.

\section*{To make an omelet...}

In 1957, Minnaert's American colleague Otto Struve, who came from
a family that had produced generations of astronomers in tsarist Russia,
wrote a shocking article in Sky and Telescope about his former colleague
B.P. Gerasimovich (1889-1937). This leader among Russian astronomers
had vanished from the face of the earth: the astronomical community
of the Soviet Union had likely gone through the same ordeal as the
biological one. The older generation of geneticists, led by S.I. Vavilov,
had been eliminated by younger scientists who, under the leadership
of T.D. Lysenko, invoked 'dialectical materialism' and state favor.
The same seemed to have happened among the astronomers under the leadership
of V.T. Ter-Oganezov.

Minnaert had developed a friendly relationship with this new generation
of astronomers in the 1950s. For him, at that time, continuing the
collaboration between East and West was central. He must have been
an eager reader of Struve's journals and thus informed about his revelations.
However, there is nothing to suggest that Minnaert fully grasped the
implications of this article. He had a tendency to close his eyes
to cruelties committed in the service of a greater cause. During World
War I, he had ignored the war crimes of the Germans, as they would
bring the liberation of Flanders closer. The ultimate goal, the elevation
of humanity, was sacred.

Minnaert, as a scientist, had a 'theoretical' disposition. He loved
regularity, the harmony of a coherent worldview. This explained why
Domela's vision from autumn 1914 and Julius' solar theory had exerted
such great attraction on him. As a lover of the big picture, he would
not be the first to call out that something was wrong upon the first
deviation. He had a tendency to go to the bitter end when he believed
a cause was morally justified. He clung to principles when he thought
they were correct, paying no attention to what it would cost him or
others in practice. The harmonious goal of Soviet society would have
sanctified chaotic actions for him. To make an omelet, you need to
break an egg. At least eggs were broken.

In the early 1960s, the United States and the Soviet Union had already
silently maintained an impending treaty on atmospheric nuclear tests
for several years. Unexpectedly, the Khrushchev government conducted
a series of tests that caused explosions with the strength of hundreds
of Hiroshima bombs. This greatly upset the Dutch Peace Council. Minnaert
proposed to the board to formulate it as follows: 'We must regretfully
acknowledge that the Soviet Union was the first to be forced to resume
nuclear testing.'

In the 1950s, Minnaert had joined forces with medical physicist Van
der Tweel in the fight against dowsing rods and earth rays. When asked,
Van der Tweel said: 'Minnaert was a great man, but still a fanatic.
He was deeply impressed by what was happening in Russia. I knew him
well: as a physicist, on the KNAW committee, and also here at home.
He brought mushrooms when he came to eat: that was never oatmeal.
He was a good interpreter of Bach, poetically gifted, and The Physics
of the Free Field is a brilliant book. But fundamentally, he was not
open to arguments: he couldn't stand it if you opposed him on that
point. I see a pattern there. He had spent his right-wing youth in
Flanders and turned around. But I'm no psychologist.'\\

Endnotes:

1 From May 12 to 14, 1951.

2 Interview with Mrs. Minnaert-Coelingh.

3 Personal notes. History Archive.

4 Molenaar, 1981.

5 Eremeeva, 1996, 297.

6 De Jager, K., {*}De bouw en ontwikkeling van het heelal I, II en
III{*}, Politiek en Cultuur, 1950, 5369, 586; 1951, 37. De Jager stated
that Minnaert had reviewed these articles at his request and warned
him about passages where he made philosophical statements or even
referred to 'dialectical materialism.' Minnaert advised him to remove
them, but a few remained. Minnaert played for De Jager the role that
his own teachers could not provide for him.

7 Hijmans van den Bergh, {*}Sol Iustitiae{*}, 1955.

8 Burger, {*}SI{*}, April 25, 1953.

9 Minnaert, {*}SI{*}, May 2, 1953.

10 Burger, {*}SI{*}, May 16, 1953. 11 Minnaert-Coelingh, {*}SI{*},
July 4, 1953.

12 Minnaert, 1953. The meeting took place on May 22, 1953. The Poland-Netherlands
Association published a brochure titled {*}Kopernik - Copernicus{*}
(number 8), which included two articles by Minnaert. It was introduced
by the humanist leader S. van Praag. The first article, a lecture,
also appeared in four installments in the weekly magazine of the NVR,
{*}Vrede{*}, under the title {*}Nicolaus Coppernicus{*} on June 12,
19, 26, and July 3. The publication of the series coincided with reactions
to the execution of the Jewish couple Ethel and Julius Rosenberg,
accused of atomic espionage, in the United States. A lecture by Minnaert
about the Rosenbergs is present in the History archive.

13 Minnaert, M.G.J., {*}De beoefening van de astronomie in het Polen
van heden{*}, 44. 14 J.W. Havermans to Minnaert on June 16, 1953.
Minnaert to Havermans on June 19, 1953.

15 And in astronomy. The article that revealed this was by his Russian-American
friend Otto Struve and appeared in {*}Sky and Telescope{*} in 1957,
titled {*}About a Russian astronomer{*}, XVI, 379. It concerned their
mutual colleague Gerasimovitch, a victim of Stalin's repression.

16 For a review of this Marxist classic: Molenaar, L., {*}Politiek
\& Cultuur{*}, 1979, 263.

17 Molenaar, 1994, 135. 18 Minnaert, review in {*}W\&S{*}, December
1955.

19 The philosopher Henk Pos took a very critical stance toward NATO
and the reasons for the deterioration of international relations at
that congress. Sociologist W.F. Wertheim told the author that Minnaert
had also gotten to know the young Pos well during his time in Leiden
(1915-1916). Interviews with Wertheim. 20 J.C.G. Nottrot, {*}W\&S{*},
September 1954. Molenaar, 1994, 155.

21 The biologists Sirks and Minnaert had, before 1914, together in
the same volume of the Ghent Botanical Yearbook!

22 Minnaert, W\&S, February 1955. 23 In the mid-1950s, the prefix
'atoom' was gradually replaced by 'kern,' and an atomic bomb became
a nuclear bomb, atomic physics became nuclear physics, and an atomic
reactor became a nuclear reactor. This change is maintained here.

24 Molenaar, Szilard, the first polemologist, Nature \& Technology,
September 1989. Fifty years earlier, Szilard and Einstein had drawn
the attention of the American president to the necessity of producing
an atomic bomb as a preventive weapon against German attempts to produce
the weapon. Also on the CD-ROM Chemistry and Society, Molenaar, 1998.

25 Two years later, Minister Luns would give a positive answer and
then do nothing.

26 Minnaert, letter to DB of CPN from April 1956. Jan Willem Stutje,
the biographer of Paul de Groot, came across the letter. CPN Archive.

27 Nico Luirink, Liber Amicorum for Miep Minnaert-Coelingh on her
80th birthday (1986). Harry Mulisch described these events in The
Assault and also referred, which cannot be a coincidence, to Minnaert's
Physics of the Free Field.

28 Interview with Aert Schadee. 29 From February 9, 1957.

30 Minnaert's defense in the Senate. Freudenthal to the author.

31 Interview with Freudenthal. This story was already made public
by C. Bol in The Restorative Facade: the years 1946-1966 in Von der
Dunk's history of Utrecht University, 1986, 75.

32 Statement from six professors of mathematics and physics at the
Senate meeting of June 15. Copy provided by Freudenthal.

33 Molenaar, 1994, 171, 202.

34 Introduction by F. Joliot-Curie at the World Peace Congress in
Stockholm in 1958. Minnaert-Coelingh Archive.

35 Introduction on New Guinea. Minnaert-Coelingh Archive.

36 Minnaert-Coelingh at the National Peace Congress of January 27,
1961.

37 Andries Mac Leod to L. Buning, November 5, 1970, a few days after
Minnaert's death. Buning Archive.

38 After Struve's article, there was deafening silence on this subject,
which was only broken by the Moscow astronomer A.I. Eremeeva in 1995.
The new generation of astronomers who had occupied the state-run posts
was also led by the brilliant Viktor A. Ambartsoemian, the later winner
of the Bruce Medal, who in 1938 had turned sharply against his superiors
with fierce accusations. Lysenko ultimately proved to be a charlatan;
this was by no means true for the Armenian. For the sake of a lightning-fast
career, he was willing to step over corpses. Ambartsoemian became
personally acquainted with Minnaert, who likely was not aware of his
activities.

39 Regarding the generational aspect of Soviet terror within the scientific
domain, see Molenaar, 1981.

40 Eremeeva's article makes it clear that a Russian correspondent
of Minnaert, the solar physicist E.I. Perepelkin, was killed by state
terror in the late 1930s.

41 These formulations are partly the result of a discussion with Boudewijn
Minnaert on October 10, 2001.

42 This Jesuit formulation impressed E. de Smit-Kruyt, who still remembered
it word for word.

43 Interview with Prof. Dr. L.H. van der Tweel: one of the handful
of members who, like Minnaert and Wertheim, had maintained their personal
membership in the World Federation of Scientific Workers.

\chapter{An inspirer of young astronomers}
\begin{quote}
'Therefore, theory must always be the final stage of research. It
must fully absorb, process, and explain the observation so that the
observation itself can become obsolete.'
\end{quote}

\section*{An inviting attitude}

There are numerous anecdotes about the inspiration emanating from
Minnaert as a teacher. Van de Hulst and De Jager, later leading figures
in astronomy in Leiden and Utrecht, testified to this at the beginning
of his professorship. Many post-war students remember Minnaert as
an enthusiastic, gifted, and even-tempered teacher. He could occasionally
get angry on rare occasions.

The radio astronomer A.D. Fokker Jr recalled: 'He gave guest lectures
in the 1947-1948 course in Leiden. In one lecture, he explained how
transitions between energy levels in atoms work, along with the associated
nomenclature.' He expected the students to internalize this and have
it ready for the next lecture. It turned out the following time that
almost no one had bothered to process the material. That really made
Minnaert angry! '

For the physician P. Boer, Minnaert's welcome speech to the first-year
students was guiding for his teaching practice: 'In 1956, I started
my chemistry studies in Utrecht. After a career as an analyst at Philips,
I decided to start a university study after all. I will never forget
how we gathered with the incoming students of the Faculty of Mathematics
and Physics in the lecture hall of the Physics Laboratory on Bijlhouwerstraat
and were addressed by Minnaert. 'Ladies and gentlemen,' he said, 'your
school time with its classroom teaching is now behind you. Ahead of
you lies the University with its wide range of sciences in which you
can immerse yourself. I strongly advise you, in addition to your chosen
field of study, to orient yourselves very broadly and to enjoy this
fully.' It sounded like music to my ears. I felt as though he was
speaking directly to me personally, and my hesitation about 'the University'
seemed to fade away. Someone warmly invited me to join that academic
community!'

Someone who graduated under the older Minnaert is the astronomer E.P.J.
van den Heuvel, now a leading figure in Dutch astrophysics. For him,
the astronomy practicals were a stimulus, just as they were for Van
de Hulst and De Jager: 'During the instruction of an experiment about
the constellations, the assignment was: 'enjoy the beauty of the starry
sky'---typical Minnaert. We laughed to each other: have you already
completed that assignment? I remember those winter evenings in my
first year, on that cold roof, when you were busy with an experiment
together with your partner, and Minnaert suddenly appeared out of
the darkness next to you, interested in how it was going.'

Van den Heuvel found Minnaert's lectures fantastic: 'I came to Utrecht
at the age of sixteen. I had never heard of Minnaert before. His lectures
were a pleasure. So clear, presented with restrained enthusiasm, perfectly
prepared, and with an excellent lecture script. When I arrived in
1957, he was already 64 years old but in very good condition: a somewhat
stooped, tall man with thick, short, upright gray hair. A handsome
man to look at. Everything about him was harmonious and genuine---that’s
what you felt as a student. He had beautiful slides for every lecture.
He spoke from memory and never used notes. What amazed us were the
perfect circles he drew on the board in one swift motion. When he
drew a circle during one of the first lectures, you could hear a collective
sigh of amazement.

You had to study hard for the exams. He conducted all exams orally,
one hour per student per year of coursework. That meant he quickly
had around 120 hours of exams to administer each year, plus, of course,
the exams for the doctoral courses. Including preparation and administration,
that’s a month of continuous work. He didn’t give it to you easily;
he asked tough questions.

In his 'Planetary Systems' lecture, there was a large section on atmospheric
optics and about Earth in general: how they had determined the correct
shape of the Earth in the 18th century and how this confirmed Newton’s
mechanics. It always struck me how far ahead of his time Minnaert
was: geologists didn’t believe in Wegener’s hypothesis of continental
drift before the 1960s---it was a groundbreaking revolution when
they finally accepted it. But with Minnaert, you got that in his lectures.
A striking example of how original Minnaert was as a thinker can be
found in Otto Struve’s 1962 book {*}The Universe{*}. Struve writes
that around 1950, he had asked about fifty-five prominent astronomers
what they considered the most important developments in their field
for the coming decade. He writes: there was one who mentioned the
possibility of launching a rocket to the Moon, and that was Minnaert!

Minnaert’s personality also intrigued and stimulated his audience:
'My fellow students and I would talk about Minnaert as a person during
our early study years. It fascinated us that he had started as a biologist,
later earned a doctorate in physics, he had achieved great things
and was now a professor of astronomy. Or that he spoke a dozen languages
and, in addition, Esperanto: why Esperanto if you already speak twelve
languages?

His breadth --- his music and instruments, his languages alongside
his natural sciences --- served as an example for us. It gave us
the feeling that even if you studied mathematics or physics, you could
still explore many areas; you weren’t trapped in a narrow discipline.
Among the professors of experimental physics in Utrecht, there were
some intimidating examples: colorless, boring superspecialists who
annoyed us endlessly with their nitpicking during lectures. We heard
about Minnaert’s pacifism, his vegetarianism---he was clearly a very
emotional man. That such a great scientist and fantastic teacher also
had so many other aspects to him was important for us. It deeply impressed
me that when we, the Anti-Militaristic Students' Working Group (WAS),
organized an exhibition in 1960-1961 about the dangers of nuclear
weapons---it was during the Cuban crisis, when Khrushchev detonated
a series of hydrogen bombs measuring many megatons over Nova Zembla---Minnaert
was the first professor to visit our exhibition.\textquotedbl{}

Minnaert, like De Jager and Van de Hulst a generation earlier, also
determined Van den Heuvel's career: \textquotedbl The perspective
you gained from Minnaert’s example---that a physicist can also be
an artist, a universal person---was enormously important for me and
determined my final choice to pursue astronomy. You knew that if you
chose astronomy as your major, your only professional prospect was
becoming a mathematics and physics teacher: something that hadn’t
been my first choice. But I decided I would rather spend my study
years in an inspiring, interesting environment and then become a teacher
than ruin my studies for the sake of broader career prospects by enduring
those intolerably mediocre experimental physicists. So I definitively
switched to astronomy with theoretical physics as a minor. Minnaert
immediately had an interesting research project for me, which even
resulted in a nice publication. I have never regretted this decision
in my life.\textquotedbl{}

Van den Heuvel would later take over Pannekoek’s position in Amsterdam
and collaborate on the first biographical publication about this astronomer.

\section*{The first post-war PhD students}

In 1949, Minnaert outlined his views on the relationship between theory
and experiment, the foundation for guiding his PhD students. Observations
generate the inspiration to discover a coherence that 'we with our
limited imagination would never have conceived.' Observation and theory
are inseparably connected but not equal: 'Observation is essentially
the establishment of a fact. Theory is the establishment of a connection.
Therefore, theory must always be the final stage of research. It must
fully incorporate, process, and explain the observation so that the
observation itself can become redundant.' The seemingly erratic must
give way to regularity, the apparently manifold to unity, and the
seemingly confused to harmony. He himself, together with Mulders,
had struggled to physically explain the 'growth curve.' He had experienced
the triumph of the alignment of observation and interpretation. In
a dissertation, theory must always explain and underpin empirical
evidence: 'There is a compelling beauty in this struggle of the human
mind to reduce the infinite multiplicity and richness of the universe
to the simplest natural laws.' If the theory to explain the phenomena
did not yet exist, then the PhD student had a problem!

Two of his PhD students would leave their mark on astronomical developments
in Leiden and Utrecht. Henk van de Hulst's dissertation on matter
in interstellar space (1946), which ended up in Oort's research area
6 during the war, was formally completed by Minnaert. In 1942, at
Minnaert's urging, Van de Hulst had participated in a competition
on the formation and growth of solid particles in interstellar matter:
'These were related to the age of the Milky Way system. Both Minnaert
and Oort had an attitude of 'it’s all ours.' There was no envy; they
complemented each other and stimulated each other wherever possible.
Any secrecy was alien to them.' The 21-cm hydrogen line in the spectrum
of stars, predicted by Van de Hulst in 1943, was discovered in 1951
and became the starting point for radio astronomy. In 1947, Minnaert
had built an improvised dish antenna on the roof of the Observatory
with Houtgast to detect the sun's radio radiation. However, the noise
level was too high and the equipment was too insensitive. In 1949,
through the efforts of Oort, Minnaert, and J.H. Bannier, the Foundation
for Radio Radiation from the Sun and Milky Way (SRZM) was established
as one of the first activities of the Pure Scientific Research (ZWO)
foundation. The PTT, which dealt with the influence of the ionosphere
on radio traffic, was a partner. This was the first step toward the
enormous radio telescopes that were constructed in the Netherlands
for this research. Van de Hulst became Oort's successor in Leiden.

Kees de Jager's dissertation provided a detailed model of the solar
atmosphere, which was used worldwide for about ten years. He recorded
the required hydrogen spectra at the center and edge of the sun using
the Utrecht spectrograph. When he wanted to build a new photometer
in 1946, Minnaert made him an unusual proposal: 'He said, \textquotedbl It
would be good for you if you could make instruments yourself.\textquotedbl{}
So, I reported to our instrument maker N. van Straten every morning
at 9 o'clock for two years. This later benefited me greatly when setting
up space research. I am the only one Minnaert ever suggested this
to.' De Jager became Minnaert's successor in Utrecht and the founder
of the ZWO foundation, Space Research Netherlands.

As one of Minnaert's most loyal and talented students, he also experienced
the patriarchal side of Minnaert: 'After the war, I got a position
\textquotedbl without objection from the treasury\textquotedbl{}
with Minnaert. Rosenfeld later offered me an assistantship of 100
guilders a month. So, I went to Minnaert and told him I could get
another position. He said, \textquotedbl But Mr. De Jager, life is
not about money, is it?\textquotedbl{} I replied, \textquotedbl I
will make sure you don't suffer any inconvenience.\textquotedbl{}
I worked for Rosenfeld from 9 to 5 and for Minnaert from 5 to 2 in
the morning, sleeping at the observatory. I hardly saw my parents
during that time.' In 1950, De Jager placed his nearly completed dissertation
on Minnaert's desk. After eleven months, it was still there: 'I was
then working on sun measurements at Pic du Midi and wrote him an angry
letter.' I immediately received the response: 'It is always a special
experience to receive a letter from you.' After that, we intensely
focused on the dissertation. Good writing and explanation were of
utmost importance to him.

De Jager was deeply aware of his teacher's pioneering role: \textquotedbl We
can now determine the temperature, pressure, and their progression
with depth in a star atmosphere by studying the star's spectrum. The
diagnostics for this purpose were developed in the 1930s by Minnaert
and his collaborators.\textquotedbl{}

De Jager and Van de Hulst built remarkable international careers as
astronomers and science managers. Wim Claas' dissertation dealt with
the analysis of the chemical composition of the solar atmosphere based
on weak line profiles in the Atlas. From Sint Michielsgestel, Minnaert
urged action on 11 fronts: \textquotedbl Would it not be good to
temporarily attempt to determine the concentration ratios for a few
elements?\textquotedbl{}

During and after the war, his colleagues B. Strömgren in Copenhagen
and A. Unsöld in Kiel indeed achieved the first results. This relieved
the pressure. Claas, meanwhile, became a physics teacher and had to
complete his promotion in his free time. There was usually no more
than two days between the concepts of Claas' chapters and Minnaert's
detailed responses. The instructions also concerned how to deal with
the great predecessors. In April 1948, for example, Minnaert wrote:
\textquotedbl You must be careful here and there not to belittle
Unsöld too much, constantly investigating whether he deviates from
your results, which are taken as the standard. It is rather up to
you to compare your findings with his. He has historical priority
and is a great master. But of course, this is only a matter of phrasing;
logically and scientifically, you have every right to regard your
results as better than his.\textquotedbl{}

Claas was in 1951 the first -teacher- to officially graduate under
Minnaert. The teachers Wanders and Van der Meer had formally been
Ornstein's PhD students in the 1930s: many teachers would follow.

The differences in supervision are striking. One had to wait at most
a day for a response; another had to wait a year. One was guided step
by step; the other works independently. De Jager is the only one offered
a specialized course. Minnaert paid close attention to formulating
and embedding the experiments in adequate theory for all three of
them.

\section*{The Minnaert school (1955-1964)}

Before his emeritus status, Minnaert supervised six more PhD students.
Piet Gathier's dissertation deals with stars of the same temperature
as the sun but with differences in gravitational acceleration: 'In
1946, Minnaert had recorded spectra in the US. He said, ‘That’s something
nice for you to figure out.’ After a few months, I let him know I
hadn’t made much progress. Minnaert said, ‘It’s also difficult. I’ll
think of something else for you and assign it to someone else.’ I
ran back: that wouldn’t happen to me! Eventually, I figured it out.
At one point, I had a quantified elaboration. He wanted an 'explanation'
added. I had to promise him that after my promotion, I would take
the necessary spectra in Italy and complete the theoretical part.
And I did.' Minnaert carefully read everything: ‘He always kept his
distance, although there was mutual appreciation. Looking back, despite
my criticism, I still think he was a fantastic educator. I was an
outsider, but my research was still connected to his solar program.
Actually, he wanted a sun with variable gravity.’ Gathier graduated
in 1955, worked as a teacher in Bussum, and together with the West
German company Phywe, designed a 'physics school practical.' He later
became a science manager at the Ministry of Education, Culture, and
Science, where he advocated for, among other things, the modernization
of mathematics education.

The dissertation of Joop van Diggelen focused on the photometry of
the moon. In 1946, Minnaert had recorded unique plate material in
the US: ‘He came up with an original idea: the sun shines on the slopes
of the lunar surface; then the progression of brightness in the photos
should reveal information about the slope. What we developed together
as first was clinophotometry.’ Van Diggelen’s texts were meticulously
reviewed: ‘You learn criticism for the rest of your life. Every sentence
was weighed.’ His 1959 dissertation was reissued by NASA in 1964 with
the moon landing in mind: {*}Photometric Properties of Lunar Crater
Floors{*}. According to him, Minnaert as an educator insisted on publication
in an official journal before graduation. Van Diggelen became a teacher
and produced many popular-scientific publications about the moon and
the 13-planet system.

The dissertation of Frans van 't Veer in 1960 dealt with the weakening
of radiation brightness at the edge of the stars: ‘Minnaert said:
“And do you specifically study the results that can be obtained from
the light curves of eclipsing binary stars.” He didn’t know much about
that. He chose topics he expected would open up new fields. Thanks
to him, I could go to work in Paris at the Institut d'Astrophysique.’
For Van 't Veer, this change of position was a liberation: he could
break free from his admiration for Minnaert and his paternalism. Minnaert
analyzed each chapter of his dissertation by letter, shared pleasant
messages about life at the Observatory: ‘He enjoyed acting as a witness
at my wedding to a French astronomer, but I never managed to penetrate
the intimacy of this extraordinary man.’

Hans Hubenet’s dissertation was initially going to focus on the ‘transition
probabilities’ that mediate between spectra and particle concentrations.
That required a quantum mechanical approach that was too ambitious:
‘Minnaert said: “You should try to get that from the solar spectrum.”
I then started working on models of the solar atmosphere. My thesis
dealt with the influence of uncertainties about the model on the accuracy
of determining the chemical composition of the solar atmosphere.’
Hubenet graduated in 1961, became a staff member of the Observatory,
and took over Minnaert’s {*}Astronomical Practicum{*}.

Aert Schadee’s dissertation focused on the concentration of fluorine
atoms in the solar atmosphere. Fluorine did not appear in atomic spectral
lines. However, fluorine atoms could be bound in molecules such as
barium and strontium fluoride. The lower temperature in sunspots could
facilitate the formation of these molecules. Schadee could not find
these molecular lines in the spectra of sunspots: 'Minnaert could
unravel a problem and had good intuition about the steps leading to
a solution. My dissertation was about the interpretation of this negative
result. It was an original topic. The California Institute of Technology
asked me for a year.' Schadee admired his supervisor's 'chilling'
erudition: 'He gave monumental standard lectures and then suddenly
came up with Extragalactic systems. I still hear him say, 'Sometimes
I feel the urge to do something completely different from what I'm
used to.' Then he would give a brilliant summary of the state of affairs
in 1961. A topic in an unfamiliar field for him, about which he lectured
non-stop for 60 hours! You have to be an incredible generalist.' Schadee
earned his PhD in 1964 and also became a staff member at the Observatory.

Jacques Beckers' dissertation was a study of the fine structures of
the sun's chromosphere: 'Because one lecture was canceled, I went
to see Minnaert. He became the leading figure in my academic career.
After my doctorate, I went to Australia for optical solar research.
This had been arranged independently of him, but I wanted to earn
my PhD under his supervision. He examined my concepts in detail and
provided quick responses and suggestions. The guidance was fantastic.'
Beckers earned his PhD in 1964 and became an astronomer.

The PhD students were involved in aspects of Minnaert's research program.
He assigned them topics he felt could become important. Minnaert's
supervision, who turned seventy in 1963, was of a high level. A 'theoretical'
framework remained a requirement. Over time, there was more work for
professional astronomers, also at the Utrecht Observatory. His students
carried on the standards and values of his 'school': they strongly
committed themselves to teaching physics practicum and consistently
engaged in popularizing astronomy and natural science in society.

\section*{The last PhD students}

After Minnaert's emeritus status, the majority of three more promotions
took place. Hans Heintze's dissertation focused on the flash spectra
of the outermost layer of the sun: the escaping rays that appear just
before a total solar eclipse due to the valleys on the moon's surface:
'In 1954, I participated in the eclipse expedition to Gotland. I helped
Minnaert align the equipment. It happened in a consecrated atmosphere:
you appreciate the ingenuity in which you skill yourself and at the
same time realize the futility of the endeavor against the boundlessness
of the cosmos. Using the recordings, I calculated the temperature
profile in the outer layers of the sun.' Heintze began teaching physics
from his first year of study. His school was close to the Observatory,
so he could work there before and after school: 'When I photometered
those flash spectra, I arrived at the Observatory every morning at
four o'clock. I had the key. When I temporarily worked in Brussels,
we met every other weekend: 'You make an effort, and so do I.' I haven't
experienced anything better than with Minnaert.'

Cees Zwaan's thesis, who later became a professor of solar physics,
dealt with models of sunspots: 'Minnaert handed me a box: 'Mr. Zwaan,
here are some spectral plates of a sunspot that Dr. Mulders recorded
at Mount Wilson in 1937. See what you can do with them?' He apparently
expected me to find my own way. He would occasionally drop by to hear
about the problems. I discovered that stray light in the recordings
played a major role, so all spots, even the smaller ones, were almost
equally dark. He wanted to be thoroughly convinced, but afterwards,
those results pleased him a lot.' It was also Zwaan's experience that
he came into his element as soon as text appeared on paper: 'Mr. Zwaan,
far too verbose, it sounds like a 1900 physics book. Explain it in
simple words: what do you mean?' When I think of a draft text, I still
wonder: 'What would Minnaert say about this?'' Zwaan found the supervision
good.

That was not the opinion of his last PhD student, the electronics
expert Tom de Groot: 'Electronics were new to Minnaert, who had thought
optically all his life. My research was about short-lived eruptions,
so-called noise storms in the radio radiation of the sun, which we
detected using a multi-channel receiver. We thought we were capturing
a fundamental phenomenon that lent itself to interpretation. However,
the noise storms exhibited an unclassifiable unpredictability. Plasma
physics provided insufficient support. My dissertation limited itself
to a report on the measurement methods and a discussion of the results.
I had the idea that he was troubled by the thesis. He found those
radio bursts beautiful but believed they needed to be 'understood.'
The lack of explanations hindered development, not just for me but
in general. He returned my texts weeks later with little commentary.
Then he asked for an expansion and went on a trip. Finally, during
the literature colloquium, which he had established himself after
his emeritus status, he fell asleep. He always had that hurried nature,
which made it impossible to gauge his age. He was a vital man! I still
have enormous appreciation for him.'

From the memories of his PhD students, Minnaert emerges as a gifted
educator who adapted to what they expected from him. He placed his
students in situations at the forefront of science. If someone wanted
to work independently, that was possible. If someone wanted to be
guided, that happened too. The appreciation of his PhD students is
great, especially for the guidance during writing. Some immersed themselves
in this microcosm; others ensured they quickly forged their own paths.\\

Endnotes:

1 Letter from radio astronomer A.D. Fokker dated June 24, 1999.

2 Letter from physician P. Boer dated June 8, 1998.

3 Letter from astronomer E.P.J. van den Heuvel dated December 24,
1997.

4 Sijes, B.A., and E.P.J. van den Heuvel, 1976, 198.

5 Minnaert, (1949), 26. Lecture for the Dutch Society of Sciences.

6 Interview with the astronomer H.C. van de Hulst.

7 Bannier was a staff member of Minnaert in the 1930s and became the
first director of ZWO in 1949.

8 De Jager, 1993: Van Bueren, H.G., Radio Astronomy of the Sun, p.
114.

9 Interview with the astronomer H.C. van de Hulst. Hulst, 1946.

10 Interviews with the astronomer C. de Jager. De Jager (1952). De
Jager, in: Haakma, 1998.

11 Interviews with the astronomer and mathematics teacher W. Claas.
Claas, 1951.

12 Interview with the astronomer and science manager P.J. Gathier.
Gathier, 1955.

13 Interview with the astronomer and physics teacher J. van Diggelen.
Diggelen, 1959.

14 Letter from the astronomer F. van 't Veer of December 16, 1997.
Veer, 1960.

15 Interview with the astronomer H. Hubenet. Hubenet, 1960.

16 Interview with the astronomer A. Schadee. Schadee, 1964.

17 Telephone conversation with the astronomer J. Beckers. Beckers,
1964.

18 Interview with the astronomer and physics teacher H. Heintze. Heintze,
1965.

19 Interview with the astronomer C. Zwaan. Zwaan, 1965.

29 Interview with the astronomer T. de Groot. De Groot, 1966. Page
406 Chapter 22.\textquotedbl{}

Chapter 22

\chapter{The microcosm at Sonnenborgh}
\begin{quote}
'You shouldn't send people home filled with uncertainty.'
\end{quote}

\section*{A warm work climate}

Minnaert developed his skills in the social domain after the war,
within the Verbond, within the many boards and working communities
he was part of, and at the observatory. He created a warm work climate
where people could thrive. It was about 'mutual service': creating
that atmosphere was his contribution to a better world of labor.

Henny Tappermann was his secretary and can testify to the 'order'
at the observatory: 'I came straight after the war and, because not
everywhere could be heated in the winter, I got a spot by the window
in the workshop. Professor Minnaert pointed out to me how neatly everything
was tidied up on Saturday mornings. That's how I learned to work carefully,
clean up thoroughly, and archive things. On his desk lay stacks of
cards. He made summaries of the contents of articles on cards with
the author's name. On my desk, a shelf had been installed, neatly
varnished, with drawers underneath and small racks on top. He conducted
inspections over the weekends. If he found any irregularities, there
would be a note on the relevant desk Monday morning. I felt unhappy
when such a note was there.

He was attentive and kind to women. When I got a new typewriter, I
could choose between a sturdy black machine and a beautiful green
Olivetti. I thought I should go for the robust one, but I got the
green one. He said it was much nicer to look at. He bought curtains
for my room made of Ploegstof fabric, which he had personally selected:
it was the only thing not supplied by the Rijksinkoopbureau. If there
was an interesting exhibition at the Centraal Museum, all staff members
would be given a evening tour by the directress. We even went to the
Rijksmuseum for the Rembrandt exhibition. One evening, we biked to
a cherry orchard. If he had to attend a lecture, he only wanted his
train costs reimbursed, possibly with a cup of coffee and a sandwich.
He was very approachable. One time, he had to take some measurements
somewhere. Someone from the workshop rode a motorcycle, and Minnaert
sat behind him. Shortly after the war, gray dust jackets arrived from
the United States. They were as happy as a group of children. Along
with his wife, he went for a walk wearing them. Another time, the
hallways were dirty after the coal merchant’s visit. Then they took
off their shoes and socks, rolled up their pants, and the professor
and De Jager made the hallway presentable.

Foreigners often worked there, such as the Frenchman Jean-Claude Pecker,
who had exchanged his home with Houtgast for a year. He recalled the
daily zenith of this microcosm in {*}Le thé d'Utrecht{*}: 'On nice
weather days, people gathered on the bastion’s embankment. Tea with
milk, cookies... Minnaert, the kind host, gave each person a turn
to speak. They talked about the day’s work, the rain and good weather,
politics, or painting. For me, it was something completely new---these
daily communal gatherings that united people of very different backgrounds
and interests as if in a family. Later, I attended tea more often
at observatories in Copenhagen, Harvard, Boulder, and Hawaii. But
nowhere did I encounter the friendly, human warmth that I experienced
in Utrecht, which influenced my later attitude in the astronomical
world: a striving for group awareness based on a sense of community.'

His PhD student Piet Gathier, when asked, said: 'Around 1950, Minnaert
announced that a new staff member would join: Joop Damen Sterck. Joop
was homosexual and openly acknowledged it. For Minnaert, that was
no problem. Damen’s appointment had dragged on a lot at the University
Board. During tea, conversations aside from everyday work revolved
around concerts and social poverty in the city. Even then, he remained
convinced of his beliefs. He never called for anyone to take a pot
of soup into neighborhood C, butw ould have been different. His commitment,
especially in the first years after the war, was remarkable. He believed
deeply in it and made you think.\textquotedbl{}

\section*{A priestly attitude}

Minnaert was, according to his colleagues, a dominant man who was
always on the go and preoccupied with a thousand things at once. Yet
he didn’t seem tense or rushed and was always punctual---never 15
seconds early or 5 seconds late. He worked late into the night, mailing
his letters with the 3 AM post and chatting with staff members who
were observing in the towers. He managed fine on four hours of sleep.

Regarding the colloquia, he established a structural collaboration
with Pannekoek’s Amsterdam Astronomical Institute. The Physical Laboratory
staff celebrated Sinterklaas lavishly every year. Over time, the 10:30
coffee break replaced the tea ceremony. Minnaert was deeply attached
to community rituals.

To his PhD student J. van Diggelen, Minnaert stood out as a man of
high caliber: \textquotedbl Simple, warm, and kind---a sort of father
figure. There were professors who didn’t even notice you. In church
circles, I’ve met many people, but never anyone as convinced of their
principles or as ready to stand by others based on them.\textquotedbl{}

Due to housing shortages, Minnaert had accommodated student Corrie
Knoppers in the basement of Sonnenborgh for several months: \textquotedbl In
the evenings, I heard piano playing. The first time, I thought it
was a radio concert. Sometimes I’d sit at the top of the stairs to
listen. After a cold night of observing, he sometimes placed a warm
hot water bottle in bed.\textquotedbl{} She also came indoors and
was even allowed to address Marcel and Miep informally: \textquotedbl He
lived for science but showed definite affection. That’s why I find
his name so characteristic. He read NRC but didn’t know who Tom Poes
or Heer Bommel were! Corrie once quoted a piece from Utrecht...\textquotedbl{}

\textquotedbl A newspaper clipping about the first Sputnik. Miep
had said, 'Please put that away.' The journalist had written about
a 'tall old gentleman' coming down the stairs. Minnaert would get
angry if he read it: he didn't want to be called an 'old gentleman'!
Corrie had found Miep very kind: 'Do you know what I do when there's
too much dust? I take off my glasses, then I don't see it,' she said.

When Minnaert performed in the evenings, Corrie would sometimes go
along to enjoy his lectures. In Amsterdam, he showed her the Wallen
'where ladies sit in shop windows.' She felt that the charming Minnaert
was fond of women. At Sinterklaas, Houtgast had once asked her for
a strapless black bra as a surprise: 'Minnaert stretched the thing
out completely and said, \textquotedbl This must have been lost by
the constellation Virgo.\textquotedbl{} He was quite to the point.
He was a great scientist and yet remained ordinary and lovable.'

His PhD student J. Beckers told: 'My mother-in-law complained in the
hospital that her daughter had married an astronomer: what prospects
did that offer? Her fellow patient was the wife of a colleague, and
so this complaint reached Minnaert. He invited my in-laws and explained
to them that astronomy is a forward-looking field.'

His PhD student Heintze had been invited by Minnaert even as a schoolboy:
'Minnaert taught me how to use the Merz telescope. He knew how to
inspire young guests! I have taught with pleasure for forty years;
that enthusiasm I owe to him. During coffee, the cleaners would sit
in---he wanted to put communism into practice and expressed himself
in a way they could follow. I never experienced a cleaner who dared
not speak.'

This paternalistic atmosphere didn't appeal to Gathier: 'You look
behind the scenes of that microcosm. Those colloquia, where you're
on duty again too quickly. I didn't get the idea that my future lay
there.' In the early 1950s, his wife was still alive:\textquotedbl{}

\textquotedbl Both were at my promotion dinner. I didn’t have much
money for my dissertation. Minnaert pointed me to a cheap printing
company. However, the proofs were full of mistakes. I had to correct
them like crazy. In one of the propositions, the word ‘not’ had been
omitted. During dinner, Minnaert said: ‘You can defend everything.’
I replied: ‘That’s true, just look at that proposition: the ‘not’
has been left out.’ Minnaert was shocked: that couldn’t be true. His
wife was laughing so hard she was bent over the table.\textquotedbl{}

His PhD student C. Zwaan had experienced firsthand how Minnaert upheld
standards and values: \textquotedbl I once gave a popular lecture
on how scientific insight is established and posed several ‘open questions.’
He approached me and said: ‘But Mr. Zwaan... You must not send people
home with uncertainty.’ He had an almost priestly view of education:
it had to bring the salvation of scientific insight. That explains
his drive regarding teaching and didactics: the message had to be
spread. The people had to become participants in science.\textquotedbl{}

His last PhD student, T. de Groot, recounted: \textquotedbl Minnaert
demanded the same commitment from his colleagues, even though he eventually
understood that there were people with different attitudes. He always
talked about ‘labor.’ That word had a sacred aura for many socialists.
‘Diligent labor’ is characteristic of Minnaert. Take the way he compiled
the Atlas, with Houtgast in his wake: that enormous energy in the
prime of their lives. Then again, that labor on the Table. I sometimes
thought: it’s a shame that someone as intelligent as you keeps doing
such simple work! The unshakability of his leftist principles could
be frustrating at times. But then you would hear him playing Chopin
in the dead of night. Fortunately.’

\section*{The man from the small institute}

In the 1950s, the work at Sonnenborgh was expanded and the staff increased.
In 1953, Houtgast became a lecturer and in 1957 De Jager. The latter
received 'stellar physics' as his teaching assignment. Radio astronomy
required electronic expertise, for which people were recruited. Sputnik
proved in 1957 that the West was lagging behind by several months
on a militarily sensitive point and created a political climate in
which initiatives in the fields of astronomy, geophysics, and space
research could count on more than benevolent attention from politicians
and funding agencies. The same rocket technology that could bring
observation equipment into the stratosphere was suitable for launching
spy satellites and nuclear weapons. Minnaert rightly worried about
this.

In the 1950s, physics and mathematics institutes sprouted up. Minnaert
behaved like a thrifty housefather and let financial opportunities
pass by. De Jager: 'You couldn't hammer a nail into the wall without
him passing judgment on it. Houtgast once came up during coffee with
a rusty wood screw from an eclipse box: 'Professor, do you know where
this screw comes from?' It was a joke, but Minnaert knew. He interfered
with everything and therefore remained the man of the small institute.
We complained that everything his colleague Oort asked for was promised.
When I got the chance to start space research, I wanted to grow quickly!'

The mathematician F. van der Blij attended the annual discussions
about the applications submitted by Freudenthal, the physicist J.
Milatz, and Minnaert on behalf of the faculty to the State: 'I'm talking
about the golden years. The number of students increased and so did
the financial resources. Milatz would then claim several dozen places.
Freudenthal always needed a lot of staff for his labor-intensive practical
work. Minnaert then requested a half-time position; once he asked
for two positions. The department reduced the requests for natural
sciences and mathematics but nonetheless granted many positions, enabling
rapid growth. Minnaert simply got what he had asked for: in business
matters, he was not at the forefront.'

Radio astronomy advanced. To interpret the observations, knowledge
of plasma physics and magnetohydrodynamics was needed. De Jager provided
an impetus for this in his lecture {*}The Tenuous Stellar Plasma{*}.
This attracted students and opened up new perspectives. The KNAW established
a Committee for Geophysics and Space Research in 1959 under the leadership
of astronomer Van de Hulst. One of its goals was to study the corona,
which, with its temperature of one million degrees, proved to be a
source of X-rays. In 1960, De Jager became a professor and was allowed
to address Professor Minnaert informally as 'Minnaert.' On October
1, 1961, he launched the ZWO-sponsored Working Group for Space Research
on the Sun and Stars. Ten years later, this group had its own accommodation
with a staff of 13 out of 100 employees.

In 1963, De Jager succeeded Minnaert as director. In 1962, Canadian
astronomer A. Underhill started a research group for stellar research,
and theoretical physicist H.G. van Bueren was appointed in 1964 to
work on the physical foundations of astrophysics. The 'astronomical
practical' had been taken over by Hubenet in the 1950s. The solar
physics work was crowned when De Jager, Hubenet, and Heintze presented
an internationally accepted model of the solar atmosphere in 1963:
the Utrecht Reference Model of the Solar Atmosphere.

Minnaert's optical research program from 1937 was largely completed,
although some parts would continue into the 1970s. Upon his departure,
according to the prominent solar theorist Albrecht Unsöld, 'a small
institute had grown, based in part on Minnaert's impulses, into an
institution with world fame.’

De Jager later wrote ambiguously: ‘In the 1950s, the foundation was
laid for growth in the following decade. The increasing prosperity
of the country, which had repaired the damage of the war in just over
ten years and the growth of the West European economy made it possible
to take bigger steps in the 1960s.’ De Jager gave Minnaert all the
credit and left it unclear who was responsible for the growth of the
1960s.\\

Endnotes:

1 Interview with Henny Tappermann.

2 De Jager, 1993, 53, mentions Pecker’s quote from 1992.

3 Interview with P.J. Gathier.

4 Interview with J. van Diggelen.

5 Interview with Corrie Sanders-Knoppers.

6 The Utrecht Observatory detected the Artificial Moon of October
12, 1957. A respectful interview with Minnaert about an entire page
was published in the Algemeen Handelsblad on October 22, 1957: The
artificial moon is a ‘vessel full of contradictions’: What perspectives
are opening up now?

7 Telephone conversation with astronomer J. Beckers.

8 Interview with astronomer H. Heintze.

9 Interview with astronomer C. Zwaan.

10 Interview with astronomer T. de Groot, June.

11 Interview with astronomer C. de Jager.

12 Interview with mathematician F. van der Blij.

13 De Jager, 1993, 57, 125.

14 Unsöld dedicated his astrophysical synthesis of 1967, Der neue
Kosmos, to his friend Minnaert.

15 De Jager, 1993, 57.

\chapter{Farewell with 'The Unity of the Universe'}
\begin{quote}
‘I am deeply grateful for your kindness and friendship, often lifelong,
which added human warmth to the seriousness of scientific work.’
\end{quote}

\section*{The sixties and the renewal of the university.}

Minnaert had made use of the then-current right of a professor to
retire at the age of 70. His views had not become outdated. Thus,
he publicly denounced the hazing scandals within the Amsterdam Student
Corps and the Utrecht society Tres. Unlike his colleagues, he exposed
the psychological mechanisms behind hazing: 'What is it that incoming
students learn? To submit without protest to degrading, sometimes
humiliating treatment; to abandon their personal dignity; to view
sexual matters in their coarsest and most disgusting forms; to drink.
And perhaps worst of all: they see with their own eyes how those in
power can degrade the weaker. The result of this 'education' can be
clearly observed a year later: the victims of then are now the ones
in power, tormentors, offenders.' Where people were degraded, they
could still count on his solidarity, just as Evariste had been able
to do half a century earlier.

In 1963, the Dutch student corps world with its rituals and traditions
was shocked by the establishment of the Student Union Movement (SVB).
Finally, there emerged a movement close to Minnaert's heart, which
sought to advocate for housing, external democratization, and shared
decision-making. The SVB endorsed the 'study grant,' which in 1964
became a top priority on their political agenda under the name Integrale
Studiekostenvergoeding. Their Democratic Manifesto linked rights regarding
material and social provisions to fulfilling obligations: 'Students
must be willing to give up part of their freedom; specifically, the
freedom not to study, or to study poorly or halfway.' Board member
Kees Kolthoff explained that 'the SVB does not regard wasting time
at state expense as true academic freedom and considers it an unstable
basis for negotiation.'

In 1960, the forward-thinking Minnaert, with his Utrecht branch of
the Union, took an initiative to raise the level of educators. Minnaert
set up a working group of nine teachers who wanted to reflect on the
'further education' of academics. Every month, people from various
departments explained the demands of the future: 'It was encouraging
to note that people everywhere are beginning to engage with Post-Academic
Education (PAO); the issue is in the air,' Minnaert wrote.

It turned out that two 'groups' were directly eligible for this type
of education: teachers and academics in professions or companies outside
the university. A separate group consisted of academically educated,
married women who 'have devoted ten or fifteen years of their lives
to their families but now have their hands free since their children
are grown and require less care. They yearn to be re-engaged in societal
work. There is, moreover, a great need for their labor force, as many
teachers are lacking in higher secondary education; they could also
be useful as doctors, dentists, civil servants in libraries, and in
many other professions. However, the difficulty arises that they have
forgotten much and have not kept up with scientific progress.'

In countries such as the United States and England, much attention
was paid to these women, attempting to channel them back iner. Place
a cover on top, cook for 1 hr 20 mins until the duck is really tender,
adding extra water if you need. Lift off the cover and cook 10 mins
more until the sauce is thickened, glossy and slightly sticky. Serve
with couscous or a rice pilau and a simple green salad.to universities
and imparting knowledge and self-confidence so they could return to
work. In conservative Netherlands, no such effort was noticeable.

Minnaert championed central didactic institutes per discipline: 'Courses
could be organized there where new techniques can be learned; the
initiative would be taken to regularly organize courses for teachers;
the latest tested instruments and teaching aids could also be made
known. It is a form of PAO when teachers work as assistants in scientific
laboratories for part of their time, while their number of teaching
hours is correspondingly reduced.' In this way, 'at least in a limited
field, the teacher remains in touch with their science in its ongoing
development.' Of course, the didactics of this education could be
based on self-activity.

In professions such as those in the socio-medical sector, post-academic
education would become lively if participants could exchange experiences
with colleagues. In practice, much resistance had to be overcome:
'Trying to create a tradition: an academic should keep up with progress.'
Minnaert saw little benefit in coercion: 'Rather grant a small salary
increase to those who regularly attend the courses, also to show that
their efforts are appreciated.'

Much resistance would disappear if PAO were incorporated as a normal
part of education and free enterprise: 'The participants should be
freed from their usual daily tasks so that real mental refreshment
can be spoken of. The possibility that many would have the opportunity
once again, at a more mature age, to fully dedicate themselves to
study is not as fantastic as it may seem. The institution of the sabbatical
year, already fairly normal for professors in the US, completely meets
what we were seeking. Perhaps obtaining a half-year period would be
easier and yet still be very effective.\textquotedbl{} He cited the
example of Danish classical language teachers who, every other year
while retaining their salary, were allowed to spend a semester at
the Danish Institute in Rome.

At the 1962 annual meeting, Minnaert emphasized that the Association
needed to develop visions for the future. Postgraduate education could
increase the efficiency of universities and elevate the scientific
life of the community: \textquotedbl Imagine that an academic, on
average, still has 40 years of work left after leaving university;
and suppose that during this time they learn nothing new and forget
nothing. Then, as a result (again, on average), scientific activity
in our country would operate at a level 20 years behind that of science.\textquotedbl{}
If postgraduate education could reduce this gap to 10, perhaps even
5 years, it would mean achieving the same as if we had advanced our
country's progress by 10 or 15 years! The Ministry of Education wanted
to provide official support for the working group. Pioneer Minnaert
was nearly 70 years old and had to prepare for his inevitable farewell.
His colleagues at the Observatory were preparing for a grand farewell.

\section*{Bibeb interviews Minnaert}

The weekly magazine Vrij Nederland featured interviews by a writer
using the pseudonym Bibeb. Every politician, sportswoman, or author
considered it an honor to be selected by her. Natural scientists rarely
became her subject, but in late summer 1962, Minnaert was indeed featured
under the title 'Astronomer and Pacifist.' She found Minnaert's voice
soft, calm, and melodious. He moved 'long-legged, quickly, yet thoughtfully,
his head with white brush-like hair bent forward.' The conversation
began with a paternal reprimand because she had immediately wanted
to use her notebook at the beginning of his explanation: 'You must
not take notes before understanding what you hear.' Advice that was
urgently repeated when I tried my luck: 'No, don't write. If you jot
things down mechanically, I feel disconnected from you, and moreover...
then I could just as well write it myself, and perhaps it would even
a bit better.\textquotedbl{} A remark that

was softened by his (blushing) bursting into laughter, but which his
gaze did not retract. It was mentioned in passing that on the initiative
of Leiden resident Oort, several European countries had begun to establish
an observatory in the southern hemisphere. Minnaert further praised
De Jager's initiatives in space research: \textquotedbl Very important
because it offers new opportunities to learn all sorts of things about
the Earth itself, where we live, its enviroer. Place a cover on top,
cook for 1 hr 20 mins until the duck is really tender, adding extra
water if you need. Lift off the cover and cook 10 mins more until
the sauce is thickened, glossy and slightly sticky. Serve with couscous
or a rice pilau and a simple green salad.nment, and celestial bodies.
Rockets and artificial satellites can explore the universe without
hindrance once they have risen above the atmosphere. The practical
certainty that we will land on the moon, with the possibility of making
observations there, is incredibly important for astronomy. The rapid
development of science gives the conviction that this will be achieved.\textquotedbl{}

Bibeb noted in surprise that according to Minnaert, the moon landing
was a matter of years, not centuries. She asked: \textquotedbl Did
you read Jules Verne in the past?\textquotedbl{} and received the
reply: \textquotedbl Not once (eagerly and laughing) but 50 times.
I have the original editions in French. There you have someone with
great confidence in progress and enormous imagination. I don't mean
that the astronomer should fantasize blindly. It's about imagination
controlled by science. But you must start by seeing. Jules Verne is
very close to space travel in {*}From the Earth to the Moon{*}.\textquotedbl{}

She asked him about the influence of astronomy on religion: \textquotedbl Astronomy
has freed humanity from a lot of prejudices and all sorts of superstitions.
We know how small the Earth is; before, it was considered central.
Now we also know with certainty how old the Earth is. There are people
who say: the Bible is symbolic. That way you can still justify anything.
I don't believe it myself. I find what's in the Bible very poetic
and beautiful, but it can no longer be reconciled with modern science.\textquotedbl{}
Forcefully, \textquotedbl It's the task of science to show that there
are no miracles. I mean: no things happen that break the laws of nature.
However, if I say (looking up, hand above the eyes), 'how wonderful,
those stars,' that’s something different. You must distinguish between
these things, you must be careful.'

Of course, she asked him about his role in the peace movement: 'I
believe every scientist should strive for peace; otherwise, he might
wonder: should I still engage in science? Every new discovery is soon
used for war purposes afterward---that’s the terrible part. We must
try to prevent this. We all must be members of one of the many associations
that combat war. Those who study natural sciences must provide information
about the dangers of war that are still insufficiently known. Sociologists
must seek the causes of war and help avoid them.' He caller. for action:
'We must not cross our arms. You won’t achieve anything by waiting.
People who oppose peace movements should ask themselves what they’re
doing---they’re doing nothing. The mentality of the Cold War must
change: a magazine like {*}Het Beste{*}, which appears in many languages,
doesn’t go by without agitation against the Eastern Bloc. Whoever
wants to improve the world must start with themselves.'

Her interview also confirmed a shift in public opinion. Fear of communism
was decreasing in the early 1960s. {*}Bibeb{*} and {*}Vrij Nederland{*}
took Minnaert’s pacifism seriously.

\section*{Farewell to the Academy: 'the unity of the universe'}

Both the Academy and its Astronomical Institute celebrated his emeritus
status in 1963. Minnaert had continued to ponder the consequences
of cosmogony for his views on humanity. Occasionally, he had shared
fragments of his thought process. In late 1957, he had delivered a
radio lecture for VARA as a humanist, titled {*}Human and Cosmos{*}.
He had presented his vision of the universe, where everything---from
the infinitely large to the infinitely small---unfolds in ‘a beautiful
harmony’ according to the great laws of nature.

The assumption of supernatural powers undermines this harmony: 'There
is also no reason to seek a purpose for the Universe. Nature knows
causes and effects; natural laws determine which changes will occur
in the future, but one may not call that a purpose. A human can have
a purpose, meaning they imagine something they wish to achieve and
strive for it. But the stars do not think and truly do not exert themselves
to reach somewhere or to give much light.'

The universe exists and evolves in all its beauty and order, and that
is enough: 'The great laws of nature, which guide the stars in their
orbits, also apply to living beings: we do not stand apart from nature
but amidst it; we are part of it.' Humanity is the product of a long
evolution, and all mental qualities are already present in germinal
form in animals. Yet, even though humanity is part of nature, 'this
does not prevent us from recognizing that thinking humanity has a
life purpose, one that inspires and empowers us with immense strength.
The purpose of our lives is to make humanity happy.'

Contributing to this, Minnaert called it the purpose of our lives:
'We do not count on a heaven beyond the stars; we strive to realize
heaven here on earth. Much goes awry in this endeavor, there is truly
enough to do! And immortality exists for us in this sense, that every
deed a human performs always has consequences, for better or for worse,
which can never be erased.' The insight that our words and actions
are immortal gives us the awareness of our responsibility for the
future: 'The future of the world will be what we make it, and each
person contributes to it in their own way. This is the most beautiful
and noble feeling that can inspire a human to lead a good life.' Here,
in a nutshell, is Minnaert's humanistic morality for a non-religious
future.

He elaborated on this in 1963 in his speech for the United Assembly
of the KNAW. His introduction was truly visionary: 'Before my eyes
rises the image by Rodin: L'âge d'airain. A human begins to stand
upright, laboriously raising their gaze toward their surroundings
and the heavens above them. From the mists of unconscious impressions,
feelings, and drives, concepts gradually take shape; they begin to
discern the relationships between these things, they are on the path
to becoming human. In a certain sense, we too still experience this
grand awakening to the knowledge of the Universe in which we live.
Driven by an irresistible, inner urge to understand, we strive to
free ourselves from human limitations and prejudices.'

Minnaert provided examples of the 'grand coherence' between all parts
of nature that we 'in principle' oversee, 'but on the outskirts of
the already explored territories, hazy distances stretch out in all
directions.' Nevertheless, Minnaert perceived everywhere 'the insatiable
desire for the coherence of science, unity of the Universe, harmony.'
This unity explicitly existed at the level of atoms and other particles
that were the same everywhere in the cosmos. Or at the level of the
connections between matter: \textquotedbl Everything in this great
Universe happens according to simple, great laws of physics, both
the movements of matter and the deformations of electromagnetic fields.
All of astrophysics is grand evidence for the statement that our fundamental
equations apply everywhere in the same way as they reveal themselves
here on Earth.\textquotedbl{} With those 'simple' laws and their combinations,
'we command an enormous and truly boundless terrain of phenomena.'

He made a leap from celestial bodies to everyday objects. The surface
of the Earth is covered with a lush carpet of green plants, clumps
of cytoplasm, living and moving beings. Together, these cover a billionth
part of the Earth's volume. It seemed unlikely to Minnaert that what
applies to the rest of the universe would not apply to these objects.
Yet, a special concept was traditionally declared applicable to these
objects: 'life.' However, the biological investigations of recent
decades had yielded two main results. First, matter behaves no differently
in a living being than outside of it, provided it is surrounded by
the same conditions and circumstances. Also, the elementary laws concerning
the behavior of living matter are the same as those that apply elsewhere:
\textquotedbl Every day, new victories are achieved in reducing life
processes to physical and chemical forces.\textquotedbl{}

There could be no room for old and new forms of 'vitalism': the whole
was nothing other than the sum of its parts, if the interactions were
considered. If one marvels at the wonderful structure and functioning
of living organisms, the prehistory is responsible, which has come
about through an evolution spanning billions of years and whose result
is embedded in the chromosomes. The line of evolution continues in
the body structure and equally mental properties: 'It is impossible
to escape the fact that even human life processes are determined by
the laws of physics; thus, also his movements, actions, and the still
largely mysterious brain processes that form the material substrate
for his thoughts. This therefore implies that there can be no such
thing as so-called 'free will' that would exist independently of the
history embedded in our organism and the influence of all factors
currently at work.' One cannot escape this by appealing to the experience
of 'subjective consciousness of freedom': 'It seems to me that there
would be every reason to carefully examine the feeling of apparent
freedom using the methods of modern psychology.'

Minnaert rightly argued that this view must have consequences for
the humanities, ethics, and criminal law: 'Concepts such as good and
evil, guilt and merit, freedom, originality, responsibility, naturally
continue to exist, but they are given a different interpretation.
By way of example, I provide an attempt at defining the new concept
of \textquotedbl freedom\textquotedbl : a person is free when their
actions are determined solely by their mental convictions; they are
also mentally free if they have knowledgeably considered highly diverse
perspectives in a balanced manner.' He also explained his view to
astonished colleagues on 'consciousness' that he believed must already
be present even in the lowest material forms. He posited his deterministic
view that brain content, reconstructed atom by atom with the exact
replacement of 10 billion neurons by conductors and circuits, would
think and act 'like us, but would it also feel?'

He concluded with a few romantic sentences: 'Thus we see how the unity
of the universe is reflected in the unity of science. Restlessly,
each of us works in our field to reduce the fleeting forms of appearance
to simplicity and firm lines. And as we progress further, all these
lines prove to run from one domain to another and merge together.
From my childhood, I remember the lace-makers in the streets of old
Bruges, sitting in front of their doors, bent over their lace pillows.
The few threads that form the starting point of their work they intertwine
while fantastic flowers, vines, and garlands of endless variety arise
under their nimble fingers. It is our task to trace the threads from
which this infinitely beautiful Cosmos is essentially constructed
in the wonderfully intricate lacework of nature. Perhaps it is just
a single thread.'

Minnaert was undoubtedly proud of the content of this lecture. He
had found a solution to the problem of 'free will' that, on one hand,
rested on the 'truth' of scientific views about the universe and humanity,
and on the other hand, on Freud's psycho-sociological views regarding
conscience and human history. Even if his continuity thesis is set
aside---a consequence of his aversion to dialectics, which he had
once been introduced to by Bolland---an impressive achievement remains.

In the late 1950s, he had addressed what students could additionally
expect from universities; in another article, he raised that question
for pre-university students as well. Neither attending a Studium Generale
nor a philosophy lecture resolved this dilemma. With {*}De Eenheid
van het Heelal{*} (The Unity of the Universe), he offered future scientists
a humanistic, science-based worldview.

\section*{A tribute to the Utrecht phenomenon}

The staff of the observatory had organized a symposium on the solar
spectrum in honor of Minnaert, attended by all the major figures in
solar physics. The mayor of Utrecht, De Ranitz, had called him 'the
Utrecht phenomenon.' In his opening speech, Albrecht Unsöld referred
to their joint work on the Fraunhofer lines and praised Minnaert's
commitment to international cooperation and peace: 'The extremely
bad experiences he endured in two world wars would justify any degree
of mistrust in reason and human moral qualities, but Minnaert continued
to work for the cause of reason, love, and peace.'

Minnaert had put a lot of effort into his lecture on {*}Forty Years
of Solar Spectroscopy{*}. He attributed the quantitative analysis
of the Fraunhofer lines to Unsöld. The discovery of 'growth curves'
was his own achievement, and he spoke about it with relish. The simple
model from the 1930s, which assumed a single-layer photosphere where
the Fraunhofer lines would form, had yielded much but later proved
to be an absurd simplification. Naturally, the properties of solar
gases varied with depth: 'The pioneer in this direction was Anton
Pannekoek, my revered older colleague and friend, who as early as
the 1930s---and particularly in his groundbreaking article from 1936---emphasized
the necessity of such an analysis and laid the foundation for its
application using his developed sense of numerical analysis.\textquotedbl{}
Later, those models, on which his PhD students De Jager, Hubenet,
and Heintze had worked, had become increasingly complicated. He also
looked back on the problem of the Fraunhofer lines. Unsöld had ultimately
been proven right: the explanation for the line profiles had lain
in absorption rather than scattering.

He thanked the colleagues who had honored him by presenting their
recent work. They had always inspired him: 'In moments of melancholy
and doubt, which I have known like anyone else, I withdrew into the
silence of our library and began to read the recent literature. You
would then address me; new suggestions, new research possibilities
came to mind. I was captivated by the beauty of nature, by the logic
and beauty of science. I regained confidence in myself, new hope,
and optimism. I am deeply indebted to you for your kindness and friendship,
often lifelong, which added human warmth to the seriousness of scientific
work.'

Minnaert wanted to share with his audience a general theoretical insight
he had acquired after a lifetime of research: 'New improvements and
refinements in concepts should not be proposed too early or too late.
If they come too early, they will hinder progress because they cause
unnecessary complications at a time when important results can still
be achieved with existing methods. If they come too late, time is
lost because one has to work with insufficiently detailed analyses,
while greater accuracy would have been desirable. It requires a great
deal of intuition and wisdom to apply this trivial moral correctly!'

The foreign colleagues offered him an abstract painting by the Japanese
artist Tanaka. No one thought he would leave science alone. He received
a handful of standing invitations, such as that from S.F. Smerd, one
of the world's most prominent radio astronomers, from the Commonwealth
Scientific \& Industrial Research Organization in Sydney: 'I am certain
that I speak on behalf of all scientists in Australia when I say that
we hope to meet you in the near future.' That would happen. According
to an article by Hubenet, Minnaert had accepted invitation for next
November.

\section*{Liber Amicorum}

His colleagues from the Observatory offered him, among other things,
a Liber Amicorum, in which many praised him in personal memories.
For instance, Jesse L. Greenstein, director of Mount Wilson Observatory,
wrote: 'In astrophysics, the methods you have developed now lead a
long life. The Photometric Atlas of the Solar Spectrum is the standard
by which we compare all sun-type stars, and the data collected there
are fundamental to our understanding of normal and special stars.
The theoretical techniques for analyzing the \textquotedbl growth
curve\textquotedbl{} owe much to your pioneering work.' Greenstein
also praised his stance during McCarthyism: 'What seems important
to me is your intellectual influence on young people. Through the
dark days, you maintained a commitment to science without forgetting
that we remain human beings first and foremost, and that preserving
the human spirit is our primary goal. Your idealism in troubled times
and your optimism have been an example for me personally, and for
many others.'

Swiss astronomer Edith Müller praised his dedication to youth: 'The
first time I saw you was when I was just a baby astronomer. It was
in Zurich at the 1948 IAU meeting. Ten years earlier had been the
last meeting, and there was much to discuss. Old friendships had to
be renewed and new contacts established. Nevertheless, you took ample
time to seek us out, young astronomers at the start of our careers.
You encouraged us enthusiastically; you stimulated us; I was deeply
impressed. In the past 15 years, this has not diminished. I think
there is no astronomer in the world who has had more influence on
the careers of young scientists, both directly and indirectly.'

His Leiden colleague Oort praised his cooperative spirit: 'For you,
it is self-evident that one can achieve the best by helping each other
as much as possible, and your whole life has adapted to this ideal
so thoroughly that those of us living around you almost forgot that
this kind of cooperation is something special. The beautiful atmosphere
you created with this has permeated everywhere in the Netherlands.
Your students have had an invaluable privilege in spending their study
period in such a harmonious and stimulating 'family' as you have formed
in Utrecht.\textquotedbl{}

His friend from Liège, Pol Swings, an astronomer of world class like
Oort, added: \textquotedbl Looking closely, we find that among those
we admire in our profession, there are not many who are also beloved
by everyone. You are the rare and admirable example of a scholar who
excelled in research, made essential contributions and even brilliant
innovations, created a 'School' of high international reputation,
nurtured Dutch and foreign students who prove themselves worthy of
their Master; and yet, you have won the affection of all your colleagues.\textquotedbl{}

Amateur observer P. Kuipers thanked him: \textquotedbl Your simplicity
and genuine interest were a great encouragement to us, such small
cogs in the vast machinery of science, all the more needed because
the material we worked with often went beyond our capabilities or
threatened to do so. And when some success was achieved, your pat
on the back was an unforgettable reward.\textquotedbl{} Meteorologist
W. Bleeker recalled the eclipse expedition to Lapland (1928): \textquotedbl There
I came to know and appreciate your extraordinary versatility. Besides
discussing the architectural styles of Lübeck and Stralsund, Swedish
nature and people (Skansen!), we talked about many other things, even
meteorology. I remember your piano playing in our hotel in Gellivare;
it was the first time I heard Dvořák's Humoresque!\textquotedbl{}

His socialist colleague Engels entrusted him with a line from Pascal's
Pensées: \textquotedbl When one sees the natural style, one is amazed
and delighted, for one expects to see a scholar and finds a human
being.\textquotedbl{} Mary Reule, Julius' daughter, needed only a
line from Ovid: \textquotedbl While other creatures have a bowed
posture and gaze toward the earth, nature gave humanity an upright
stance, to behold the heavens and turn their faces toward the stars.\textquotedbl{}

Musicologist Reeser was pleased with the donation of the largest part
of Minnaert's collection of musical instruments: \textquotedbl What
strikes me particularly about this gift is not only the inquisitive
spirit you demonstrated in assembling these many and rather rare instruments,
but above all the love for music that clearly inspired you and made
you willing to sacrifice much time and money. It is because of this
love for an art that is dearer to me than anything else that I wish
to express my deeply felt sympathy.\textquotedbl{} Of course, many
Utrecht colleagues recalled how he made an impression on them through
his voice, posture, gestures, expressions of empathy, and demeanor.
The theologian Van Unnik paid attention to his administrative work
for the Utrecht Volksuniversiteit.

His mathematics colleague Hans Freudenthal had experienced Minnaert
in a jungle of administrative roles: 'How familiar I was, over many
years, with the path to Zonnenburg, the winding road to a low building
with an astronomical clock in the hallway, where you had to know the
right door to reach the old library, which, with its old table and
many old printed works on astronomy, was the setting for the most
beautiful meetings I have ever known---meetings where, carried along
by Minnaert, everyone did their best to work hard.' Freudenthal also
remembered the early years of their association: 'It was after a VWO
congress when we, as the Board, dined festively with the speakers.
Jan Romein seized the opportunity to address Minnaert in an informal
table speech. He told a story from their time as students in Leiden,
when they were gathered in a cheerful company discussing, while Minnaert
sat at the piano playing classical music. Suddenly, while playing,
he raised his head and voice and called out to them through the murmur:
'Friends, just listen how beautiful it is!' Everyone who had collaborated
with Minnaert knew they were caught up in the special atmosphere he
created.'

His former secretary Hennie Tappermann thanked him warmly: 'No matter
how difficult things were privately, because of the warm atmosphere
that prevailed at the Observatory, it was possible to stay courageous.
That was entirely and solely your merit. You were the confidant for
your entire environment, always full of understanding for others,
and you have proven that it is possible to lead and yet only be good.
Through your example, everyone who worked under you was influenced.
And what is possible in a small community must also apply on a larger
scale. Personally, you convinced me that one of your greatest ideals
must be achievable: that there can be peace in this world when all
those in power indeed keep the interests of those who have entrusted
themselves to their leadership in mind.'

This Liber, with its 120 contributions, gives an impressive picture
of what Minnaert meant, both scientifically and personally, to the
people around him.

\section*{The professor's portrait upon his farewell}

Even the Sterrenwakers wanted to present Minnaert with a portrait.
It couldn't be a traditional professor's portrait. It had to depict
Minnaert as everyone knew him: active, enthusiastic, teaching, and
dedicated. Minnaert was persuaded to cooperate, 'because the community
wants it.' A committee, consisting of radio astronomer Fokker and
the PhD candidates Heintze and Zwaan, was tasked with finding a suitable
painter. Zwaan said, 'We had to suspect, for example, that Professor
Minnaert - an art connoisseur who enjoyed drawing himself - would
react not only courteously but also very distinctly to intermediate
results, and the artist would have to handle that.'

A meeting with the Haarlem painter Kees Verwey turned out to be a
complete failure: 'Then we were made aware of Pieter Defesche: a Limburg
artist who painted large non-figurative, strongly expressionistic
canvases but had, according to our sources, made remarkable portraits
during his academy days.' The artist responded enthusiastically to
the challenge. He proposed making sketches while Minnaert was busy
in a meeting or giving a lecture. Afterwards, all parties could freely
decide based on the sketches how the portrait should look.

After a few months, Defesche had a clear vision of the portrait. He
chose a life-size standing Minnaert, full of action, teaching, with
a book in his left hand. To make his striking 'head' stand out well,
the portrait had to be cut off above the knees. Zwaan had to explain
this to Minnaert; he could live with it: 'The active and enthusiastic
is not so much expressed through the pose of the portrayed man, who
is serious and restrained, but through the bold colors, especially
the orange-red part near the head and the deep blue additions to the
coat, which seem to leap off the canvas. The gray mustache and white
chest hair contrast with those contours, and the unusual canvas creates
the impression as if Minnaert could step out of the life-size frame
at any moment.'

Defesche remembered his subject well: 'Minnaert had something endearing
in his appearance. In any case, something of a man who possessed wisdom.
His statements were imbued with the seriousness of life. He had a
wide range of interests: current events, scientific, and artistic.
He was fascinated by what I did: 'It’s starting to resemble Stijn
Streuvels.’ He could create witty remarks and put things into perspective.
He was lively, charming, and presented ideas in a simple way. He sympathized
with communism; so did I, no less, as well as my friend Ger Lataster
and many others. That was common at the time. I had not painted figuratively
for ten years, but it felt fine to me. I fully supported the portrait,
though of course it couldn’t be revolutionary.

The committee, Minnaert, and the painter were very pleased with the
canvas, but its modernity caused a stir among colleagues. That was
then in line with the style of the person portrayed. His daughter-in-law
later said: 'It’s him, but he looks different than usual, sad.' The
painting was completed in October 1963, and the Liber Amicorum was
also presented to him at the end of October.

By then, a dark shadow had fallen over Minnaert’s life.\\

Footnotes:

1 Minnaert, text on hazing. Undated. History Archive.

2 Molenaar, 1994, 190. Minnaert presented the Report of VWO-Utrecht:
A program with wishes for the future. The members' meeting of the
Union on March 17, 1962, was about The necessity and need for Post-Academic
Education (PAO).

3 General members' meeting of VWO on March 17, 1962.

4 Bibeb, Vrij Nederland, September 15, 1962. 5 Minnaert, Vara-radio,
November 24, 1957: Man and cosmos, seen through a humanist lens. History
Archive.

6 Minnaert, 1963, KNAW lecture. 7 The Bronze Age. 8 Minnaert, 1951,
the inaugural speech upon receiving the Bruce Medal.

9 Minnaert, on what the university should offer students and what
VHMO should offer pupils in addition. History Archive.

10 De Jager, 1965, with all contributions to the Symposium The Solar
Spectrum.

11 This was also a tribute to Pannekoek, his twenty-years-older friend
with whom he had undertaken many things after the war, such as the
publication of the Principal Works of Simon Stevin. Pannekoek had
passed away in April 1960; Minnaert had delivered a beautiful eulogy
for his 'teacher' and honored him with a contribution to the Dictionary
of Scientific Biography. Archive-History.

12 Hubenet, H., Please, remember me... Prof. Minnaert loves Life,
Algemeen Handelsblad, September 12, 1963.

13 Liber Amicorum for Minnaert. Boudewijn Minnaert Archive.

14 Prof. dr C. Zwaan on the portrait in Fylakra, April 1982.

15 A dozen sketches can be found in Molenaar, 1998.

16 Interview with the painter Pieter Defesche.

17 Els Hondius to Kees de Jager, around 1971.

18 The original now hangs in the Minnaert Building at the end of a
corridor that runs from the canteen via the scullery to the hall.
One could imagine a better place for it, but it is at least 'in the
flow.' Minnaert might have found that quite amusing himself.

\chapter{An Irreparable Loss}
\begin{quote}
'Let us help each other to bravely face the future.'
\end{quote}

\section*{The move to Zuilenstraat}

The open community of the Observatory had contrasted with the hermetically
sealed professor's house. Zonnenburg 1 had become Miep Coelingh's
domain. According to her daughter-in-law, even Koen Minnaert's family
didn't visit because the cats were disturbed by their grandson Paultje.
Family visits were limited to birthdays. Miep's camaraderie primarily
went to her colleagues from the Peace Council, for whom she was a
support and confidante. Among them, she found the solidarity community
that she couldn't establish in her personal surroundings.

After Minnaert's emeritus status, they bought a top-floor house together
at Zuilenstraat 25bis, on the corner of Nieuwe Gracht. The ground
floor was already occupied, and the basement by the canal was rented
out cheaply to a visual artist. Minnaert lived on the first floor,
Miep took the attic. Her peace movement friend Sita Anderson-Cochius
remembered: 'On Zuilenstraat, Miep saw her chance for a small apartment
of her own. There was a spacious attic with rotten support beams and
cracks and gaps in the walls that let through wind and moisture. Hesitantly,
Miep asked: could something be made of this? Load-bearing beams, lightweight
walls, windows, warmth, light, and habitable? Drawings were made,
a thousand wishes expressed. What an exciting time. It succeeded beyond
expectations after many long discussions in Stockholm with the architect.'
That was Henry Anderson, Sita's husband. Marcel and Miep lived completely
separately.

Minnaert had two rooms that served as his living and workspace, a
small kitchen, and a bedroom. In the adjacent library stood a worktable;
above the piano hung an abstract Tanaka and a semi-figurative canvas
by the Dutch artist De Winter, reminiscent of birds and flowers: a
gift from the Dutch astronomers. There was a cast of the Greek goddess
Pallas Athena.

It was a good 400 steps from his home to the observatory. There, Minnaert
got a study room and continued his research, especially his experiments
on surrogates of the lunar surface. After the Solar Symposium on September
9, he had traveled to Italy, participated in a conference in Rome
from September 12 to 18 where he was in the spotlight, and went hiking
alone along Trieste, Rijeka, and the Yugoslavian coast.

\section*{The disappearance of Koen}

On Wednesday, September 18, 1963, Koen Minnaert did not return home
from work. His wife knew that he had had a meeting with the management
that day. Koen had been suspicious in recent weeks, thinking his colleagues
were out to get him, and no longer ate lunch at work. Els raised the
alarm that evening. The next morning, she investigated what had happened.
It turned out that director E.J.W. Verwey had told him that it seemed
as if he didn’t feel at home. His colleagues thought Koen was being
egocentric with the equipment. It seemed to Verwey that it would be
best for him to look for another position: at a university or in education.
He would still receive a year's salary from Philips. His colleague
Hein van den Berg was the only one with whom Koen had spoken afterward.
Looking back, he said: 'Koen was driven, had a kind of territorial
drive, collaboration was difficult, the relationship with the biology
group was strained, quarrels arose. He was awkward, many people found
him annoying. He talked about the Nobel Prize. Harassments occurred,
such as locking up the equipment for each other, Koen's evening work
was sabotaged, the assistants were in tears, the atmosphere was ruined.
Incidentally, not everyone was anti-Koen!' Group leader Voogd had
not been able to quell the conflicts. Therefore, Verwey thought it
necessary to make a decision.

The coincidence of his disappearance and the conversation with the
management suggested that this event had been the decisive factor.
According to his wife, the conversation must have been a heavy blow
to Koen's ego: 'Koen was in a hurry. He had passed thirty and needed
to achieve something big quickly. Science was everything to him. He
did his best to become the equal of his father in his research or
to surpass him. It was certain that he would become a professor. He
thought he would only deserve his father's love if he succeeded in
science.' Koen's friend Theo Quené also saw a direct connection: 'Koen
thought the boss was an errand boy. When the boss turned out not to
be one, it must have hit him hard.'

The next day, Koen's Volkswagen was found near Leende, three kilometers
from the Achelse Kluis. The police towed the car away, but Els immediately
had it put back with some food inside and a letter for Koen. She must
have felt that he was confused and might need something to eat. The
forests around Leende were searched by forty officers and soldiers
with the help of police dogs, but Koen remained untraceable. It had
happened before that Koen would disappear for a while during difficult
situations. Els, too, had occasionally lost track of him during vacations,
after which he would reappear without further explanation. She hoped
that an overwrought Koen had taken a few days off after the loss of
face and would show up again. M

innaert was unreachable during those first days. He went for walks,
visited cave formations, arrived in Pula on September 27, and traveled
back to Milan. That’s where he heard about Koen’s disappearance. He
returned immediately and joined the search. With his assistant Hubenet,
he visited monasteries in the area because Koen might have retreated
there. At the gate, Minnaert introduced himself as a ‘professor’ to
encourage breaking any potential vow of silence.

The Eindhoven press spoke of a ‘mystery-Minnaert.’ On September 29,
a newspaper reported: ‘It is not ruled out that Dr. Minnaert may currently
be in a “shadowy state,” as a result of which he is not aware of the
situation he is in.’ All sorts of speculations were circulating. Both
the national and Belgian press were occupied with the enigmatic disappearance
of the son of the renowned professor.

Els gave birth on October 20, a month after Koen’s disappearance,
to a second son. The magazine Life wanted to publish a two-page spread
of the mother and child through friends of Minnaert. However, the
question was: how would Koen react?

After five weeks of desperate searching, Minnaert made up his mind
for himself. He decided to keep his promise to the Australian astronomers
and attend several meetings in November where he would be the guest
of honor. He would combine this with a visit to his son Boudewijn
and get to meet his daughter-in-law Noortje. A visit to some islands
in the South Seas was also planned, followed by getting acquainted
with several countries in Southeast Asia.

Perhaps he found the situation so terrible that, as a man of action,
he couldn’t wait until others would inform him about the outcome.
Or perhaps the desire to be reunited with his youngest son tipped
the scales. Perhaps he simply wanted to keep his promises and felt
that there was nothing more he could do for Koen and Els. Likely,
he closed himself off from a reality he dare not confront---a pattern
that had occurred before.

On October 31, he took care of writing a thank-you letter to the contributors
to the Liber Amicorum he had received the day before. On the same
day, his successors De Jager and Underhill informed the astronomical
community that Professor Minnaert's farewell lecture, scheduled for
October 31, had been postponed indefinitely 'due to sad circumstances
in the family.' The Liber Amicorum and the oil painting would be presented
at a more appropriate time, perhaps in a very small circle. Minnaert
packed his suitcase that day and left the Netherlands on November
1.

He flew via Los Angeles and Tahiti to Samoa. He visited several South
Pacific islands and continued to Sydney, where Boudewijn and Noortje
picked him up from the airport. From November 12 to 15, he attended
the Solar Physics symposium there. From November 19 to 29, he visited
radio astronomers in Canberra and spent the month of December with
Boudewijn.

Letters to Miep may have been lost, and phone conversations cannot
be reconstructed. However, a postcard addressed to Els on November
10 survived: 'Leaving for Sydney tomorrow morning! I still hold the
quiet hope of finding news about Koen there. Father (spent two days
in Tahiti, three days in Fiji).' On the same day, he wrote to his
mother-in-law Coelingh: 'I arrive in Sydney tomorrow morning and will
be picked up by Bou. It almost feels like a dream.'

On January 1, he traveled back via Cambodia, where he visited Angkor
Wat, then continued to Bangkok, Rangoon, Calcutta, and Darjeeling.
He sketched the sunrise on Tiger Hill, visited Ghoom Monastery, drove
to Bagdogra by car, took a plane to Banarasi, drew the sunrise on
the Ganges, visited Agra and the Taj Mahal, traveled by bus to Fatipur
Sikir, and continued by car to New Delhi. On January 9, 1964, he wrote
from Bangkok: 'Dear Mother, this is one of the many colorful scenes
that flash by on this return trip. Wouldn’t you feel like going to
the market in such a boat? Lots of love, Marcel (How about Koen?).'

\section*{The death of Koen}

A few days later, he got clarity. On January 18, while in India, he
learned from pathologist J. Zeldenrust that his son’s remains, 'nearly
fully skeletonized,' had been found in a terrain fold. After fifteen
hours of travel, he arrived at Schiphol Airport. The body was discovered
by a forester who had participated in the September search efforts:
'At first, I thought it was a jacket lying there, but as I got closer,
I saw it was a man lying face down. By his shoes, watch, and glasses,
I knew it was the missing man's body.' The body was difficult to see
from a distance, but that wouldn’t have been a problem for the dogs.
Koen’s remains were found after four months, less than 400 meters
from his car. His glasses were unharmed; his wedding ring and wallet
were present. It was an ending that once again gave rise to speculation.
How could a man meet his end in a forest frequented by day-trippers
and thoroughly searched by the police? Was it suicide or perhaps murder?
When asked, Els said: 'Almost everyone said Koen had committed suicide.
I know he was almost paranoid in his last weeks. I discussed this
with a psychiatrist friend. If you don’t eat or drink in his mental
state, you end up in a vicious cycle: everyone is against me, I’ve
lost face. It was very hot weather. He didn’t come to his senses.
He must have wandered around for about six days before succumbing.'

During his trip, Minnaert had not sent any letters to his daughter-in-law.
He arrived just in time for the well-attended cremation in Dieren.
In his speech, he repeatedly addressed Koen: 'Young man, who could
have ever imagined that you would go before me?' At the end, he had
remarked 'that life goes on.' This had disturbed some people. Grandfather
Hondius thanked the attendees on behalf of the families. His boss
Voogd mentioned in an In Memoriam that Koen 'worked hard, very hard
and actually wanted to work even harder.' Koen's actions had been
permeated with 'overwhelming seriousness.' He described the two years
of collaboration as difficult and heavy, but thought that Koen's criticism
'of some habits and working methods of our laboratory had been useful
to us in various ways.'

That evening, Minnaert wrote to Els: 'I forgot to thank you for the
beautiful forsythias you sent me as a welcome. They look magnificent
in a large vase on my desk, and I am enjoying them while writing this
note. You have truly made me very happy by sending those flowers.
Count on my unconditional cooperation for anything I can help with.
I hope to visit often; only in these first days do I need to exert
all my strength to support Mother, as there is still an enormous amount
to arrange and organize before our lives return to normal. It was
a difficult day for you and for us. Let’s help each other to bravely
face the future. Farewell, dear Els. Greetings from Mother. Father.'

In a postscript, he mentioned practically: 'We have received large
numbers of letters from people expressing their condolences. We will
have a card printed to convey our gratitude. Probably, you too have
received more letters than you can answer. Have you considered how
to solve this? Could we possibly do it with a standard letter, identical
for all writers? Or do you believe that the text should differ for
you and for us? Let us know your thoughts with a single word.’ Minnaert
tried to regain control. Why couldn’t these people simply receive
a personal letter from his daughter-in-law? His departure for Australia
and the associated world trip continued to amaze. His flight from
the fate of his eldest son had brought him closer to his youngest
son, whose family life he had been able to observe in its intimacy.

\section*{Going forward and not forget}

In the {*}Liber Amicorum{*}, many had noted that they could rely on
Minnaert’s support during personal difficulties. For them, he was
a ‘father,’ a source of advice, a willing listener, and a confidant.
People thanked him for ‘45 years of warm interest,’ for ‘charity’
and ‘helpfulness.’ Bram van Heel, the optics professor from Delft,
wrote: ‘After all, you’ve given so much to so many!’ It had become
a moving book: all those pats on the back for his care and empathy
at the very moment when he was anxious about his own son!

Why hadn’t his son come to him? He kept others at a safe, functional
distance. From his Olympian heights, he could serve them kindly, generously,
and mercifully. His children were entitled to his love and care. The
question was whether Koen believed his father would stand by him,
and given Koen’s rivalry, his father might have been the last person
he could ask for help.

Those around Minnaert observed that Koen’s death had deeply shaken
him. With even greater passion, he threw himself into his work: apparently,
this was his way of coping. He maintained silence about Koen, broken
only in rare exceptions. He once told his secretary, Rie Hubenet-Bergman,
how terrible he had found it that he had supported so many people
while his own son had not asked for help. The population of the Observatory,
even if only out of pity for the old man, had to create their own
stories. His PhD student De Groot observed that Minnaert continued
to display iron discipline: 'He could look terribly gloomy and yet
be very kind. He could also explode after Koen's death. I experienced
that twice. Once he walked away furiously during coffee. He couldn't
handle our camaraderie anymore.'

Minnaert communicated a lot with his daughter-in-law and grandchildren.
Els wrote to her brother-in-law Boudewijn, looking back: 'He was so
involved with us. I sometimes thought it might not be very pleasant
for him to always be reminded here that Koen is no longer around and
to see the boys without their father. But if that was the case, he
never let it stop him from coming here, and he never showed any sign
of it. Except when he slipped up and called Paul 'Koen.' He was still
a piece of Koen for me, somewhat because of the physical resemblance,
and he represented the world Koen came from and where he was shaped.
When I first got to know your father, I was really a bit afraid of
his cleverness. But gradually I noticed that he could also appreciate
me without all that knowledge, and then my awe for him faded, and
we became more equal, while my respect remained. I have always been
proud to have such an exceptional man as a father-in-law, but actually
only after Koen's death, because Koen had so many mixed feelings towards
Father. De Jager recently said to me: 'Your father-in-law has been
like a second father to me.' And I thought of Koen (especially) and
thought, you might better see this man as a second rather than a first
father. I practically never spoke about Koen with Father again, at
least not about his death.'

Minnaert must have had a disagreement with his wife about Els's position.
In the first year, the contact between mother-in-law and daughter-in-law
had been relatively 'normal.' On the anniversary of his disappearance,
they had all looked at slides of Koen together. But over time, things
gradually deteriorated. In the background, the question of guilt played
a role. Was it the 'fault' of his cold mother, or had his wife shown
too little love? With this question, in which Minnaert and Koen were
spared, Miep Coelingh burdened the relationship with her daughter-in-law.
Her demeanor made it clear that she held her daughter-in-law responsible
for what happened. When Paultje was in play therapy in Utrecht for
a year and a half, and Els had to be at the Minnaerts' every fourteen
days, she dared not visit Zuilenstraat. Miep Coelingh also did not
visit her daughter-in-law in Eindhoven.

Minnaert signed his letters with 'Father and mother' and effectively
tried to be the guardian he had become in 1964. Both spouses probably
could not discuss Koen's death with each other. Minnaert could hardly
follow his wife in rejecting Koen's legacy: his wife and their two
grandchildren. In fact, Miep renounced her grandmotherhood. She had
a conversation partner: her friend Truus van Cittert-Eymers. Her daughter
recalls: 'She talked a lot with my mother about Koen's death. They
called each other twice a day, certainly for at least half an hour.'
For her acquaintances, Miep had a plausible explanation. One of her
pacifist friends, Elske de Smit-Kruyt, said: 'Miep said that Philips
wanted Koen to work in war preparations. He couldn't go on. Then he
took his own life.' His mother elevated Koen among her leftist friends
to the status of a victim of capitalism.

Minnaert must have written to his childhood friend Andries MacLeod
about Koen. MacLeod reported: 'A tragic event deeply affected him.
The loss of one of his sons. That was certainly suicide. From what
Marcel told me, I believe I can conclude that this son was overly
sensitive.' This is a rare testimony to Minnaert's perspective on
the events.

Koen had put all his eggs in one basket: he was determined to become
an outstanding researcher. In hindsight, his superiors found that
ambition unrealistic. Verwey had not considered Koen a top researcher,
but rather an excellent educator. His promoter Slater had described
him as a precise worker whose strength lay in thinking through explanations
and finding mathematical connections. In his recommendation, he wrote
that Koen Minnaert 'possesses very deeply founded and extensive knowledge
of the various fields of chemistry and has the patience and technical
expertise to work in areas requiring this level of precision.' Slater
deliberately did not award him a cum laude for his promotion.

Koen had set himself impossible standards. His idealized self-image
was far removed from reality: the short circuit and ultimate blow
to his fragile ego occurred during the conversation with the director.

Did Minnaert ever wonder if he had given Koen too little affirming
attention? Was it meaningful to ask such questions? Was it possible
to ignore them? Minnaert prominently placed Koen's photo on his desk.
For the first time in his life, he became seriously ill. In April
1964, he spent one and a half months in the hospital for prostate
surgery. By the end of 1967, he underwent another operation due to
metastasis, during which part of his intestines was also removed,
requiring him to stay in the hospital for rehabilitation for an additional
month. Minnaert had to accept that he was a cancer patient.

That after Koen's death, he resolutely wanted to resume his restless
life may be evident from a letter to Els, two weeks after the funeral.
In it, he promised to visit with a projector and slides of Koen: 'We
are not having an easy time and are overwhelmed with work. Tomorrow,
Tuesday, I will go to Deventer to discuss the Stevin publication with
the printer; at 4:30 PM, I need to be in Groningen for a lecture;
that evening, I fly to Paris for a meeting starting at 10:00 AM on
Wednesday. The same evening, I return to Utrecht. On Saturday, a lecture
in Leeuwarden. Last Sunday (yesterday) I visited our niece Marie van
Zadelhoff, who complained and vented for more than two hours (not
directed at me!). - Still no phone! We're doing our best and defending
ourselves as well as we can. Everything will gradually fall into place.’

In the same letter, he mentioned that Koens' promoter Slater wanted
to publish the research conducted at Philips, along with a chapter
from Koens' thesis, in Biochimica et Biophysica Acta. The following
year, this plan came to fruition. Slater added an editorial note:
‘Let what is good in it be a monument to this talented and meticulous
biochemist, whose early and tragic death is mourned by the colleagues
of the two laboratories where he carried out his work.’

The 1960s also brought joyful events. Son Boudewijn and girlfriend
Noortje got married in Sydney and came to the Netherlands in 1964.
The Minnaerts organized a reception at Hotel De Donderberg near Doorn,
attended by over a hundred family members, friends, and acquaintances.
Miep saw Noortje for the first time: the women developed a good relationship.
Boudewijn met his nephews there.

In 1968, a surprise emerged: Noortje had been pregnant in 1948. Fearful
of her strict father's anger, she, assisted by her mother, had given
up custody of the child and moved to Australia. That year, 20-year-old
Peter Kruiper tore open a distant letter addressed to his foster mother
and discovered who his natural mother was. Noortje and Peter accepted
each other, and a cordial relationship also developed between photographer
Peter and stepfather Boudewijn. Peter visited Zuilenstraat, where
he was welcomed as a full-fledged grandchild.

Endnotes:

1 Liber Amicorum for Miep Minnaert-Coelingh (1986).

2 Hein van den Berg.

3 Interview with Els Hondius.

4 Interview with planner Th. Quené.

5 Minnaert, Reizen. Archive-Boudewijn Minnaert.

6 Interview with the astronomer Hubenet.

7 Press releases.

8 As in the autumn of 1918 when activism suffered a defeat, or in
September 1936 in De Lage Vuursche.

9 Letter of thanks from Minnaert in archive-Minnaert. Farewell lecture
and farewell gift for Professor Minnaert, written by A.B. Underhill
and C. de Jager in archive-History.

10 Archive-Hondius.

11 Minnaert, Travels.

13 Minnaert to Els Hondius, January 20, 1964. Archive-Hondius.

14 Interview with Els Hondius.

15 Interview with astronomer T. de Groot.

16 Letter from Els Hondius to Boudewijn Minnaert, October 30, 1970.

17 One inevitably thinks of the poor relationship between mother-in-law
Jozefina Minnaert and daughter-in-law Miep, which is replicated in
the next generation.

18 Interview with Els Hondius.

19 Interview with Hanneke van Cittert-Eymers.

20 Interview with activist E. de Smit-Kruyt. 21 Letters from Andries
Mac Leod to L. Buning, November 5 and 16, 1970. ARA-Den Bosch.

23 Minnaert to Els Hondius, February 3, 1964. Archive-Hondius.

24 Minnaert, K., E.C. Slater (1965).

25 Interview with photographer Peter Kruiper.

\chapter{Tribute to Simon Stevin}
\begin{quote}
'My love for Flanders and the Netherlands has remained equally strong,
but embedded in the international brotherhood of nations and in social
reform.'
\end{quote}

\section*{The principal works of Simon Stevin}

Minnaert had become a member of the Royal Flemish Academy in 1951
and could naturally move freely in Belgium. In 1953, he was to speak
at a peace meeting in Antwerp and yet was expelled from the country.
His friend Leo Picard: 'He was particularly cordial---I don’t know
a better word---and brought up all sorts of old stories during dinner
at our place about the good old days in Ghent, which he had experienced
with me and my wife. The next morning---we didn’t have a proper guest
room at home and had therefore rented him a room at Hotel de Londres---I
went to visit him. He had already been taken away for a long time
by the police in connection with an old ban on residence. A second
lecture in Bruges was not allowed to take place.”

The incident formed a sad series together with the refusal of a visa
for the United States (1951) and the Utrecht rectorship (1957). The
conversation with his childhood friends suggests that Minnaert did
not shy away from discussing his Flemish years in personal settings.
His lifelong loyalty was the reason he spoke at the funeral of his
'erring' friend Domela Nieuwenhuis Nyegaard in 1955. For him, it was
about a comrade dedicated to Flanders with whom he felt connected.
However, when Dutch journalists asked him about his Flemish past,
he could sharply add that they were meddling in matters they had no
business with. It stood to reason that Minnaert would seek other ways
to express his bond with Flanders. He found it in the form of a comprehensive
tribute to his fellow townsman, the Bruges native Simon Stevin. He
even showed concern about the pronunciation of his name: 'Remember:
You pronounce Stevin like the Dutch given name Stéven; never pronounce
it in the French way with the emphasis on that nasal final syllable;
Stevin would turn in his grave!'

In 1938, Annie Verschoor had written her biographical sketch of Simon
Stevin in Erflaters van onze beschaving. In Michielsgestel, Minnaert
had thoroughly enjoyed Dijksterhuis' monograph on Simon Stevin. He
had called his book a 'preliminary study' and argued for redeeming
an honorary debt: 'This debt can be described in a few words: we must
erect the one monument for Stevin, by which one can truly and enduringly
honor someone who has laid down the results of their work in writing:
we must bring about a complete edition of his works.\textquotedbl{}
Minnaert had become a loyal visitor to the Physics department of the
Royal Netherlands Academy of Arts and Sciences (KNAW) after his membership
began in 1946. From this position, he could turn the proposal of mathematics
teacher and science historian Dijksterhuis into a grand project. He
later wrote: 'My love for Flanders led me to strive for a reprint
of Simon Stevin's works. It was a great task.'

Stevin (1548-1620) had been a great scientist who wandered around
Northern Europe for several years before settling in Leiden in 1581.
He had offered his services to Prince Maurice in the struggle against
the Spanish oppressors of the Netherlands. Stevin had been a major
advocate for Dutch as a scientific language: he designed mathematical
terms such as 'everedich' and 'evewijdigh', 'stomphouck' and 'raecklijn',
'driehouck' and 'regthouck', 'noemer' and 'omtreck', 'meetconst',
'wisconst', and 'snijlijn', 'thienich' (decimal) and 'worteltrecking',
'houckmaet' (sine) and 'brantsne' (parabola), 'keghelsne' and 'lanckrondt'
(ellipse). The word 'wiskunde' (mathematics) itself originated from
him. As a native of Bruges, Minnaert could identify with Stevin in
many ways. He himself had enriched scientific Dutch with terms like
'zonnevlam', 'vlamtong', 'groeikromme', and 'equivalente breedte'.
De Jager: 'When new scientific concepts from foreign languages were
introduced, he would consult with us about what the best Dutch translation
could be.'

A Stevin project could contribute culturally and historically to the
elevation of Flanders. Therefore, Minnaert and Pannekoek established
a committee to realize the publication of Simon Stevin's Works. This
happened in 1948: the year of Stevin's centenary. They invited the
science historian Dijksterhuis, whose conduct during the war had given
cause for further investigation, to become chairman of that KNAW committee.
They proposed him as a KNAW member based on his work {*}The Mechanization
of the Worldview{*}. Minnaert also played a key role in realizing
a professorship for Dijksterhuis in the history of natural sciences.
Minnaert therefore came to the defense of someone who had been an
outspoken opponent in both his political and didactic views.

In this way, the astronomy duo paved the way for Dijksterhuis' chairmanship
of the Stevin committee. Alongside this trio, the science historian
R.J. Forbes and the maritime expert E. Crone, director of the Maritime
Museum, became members of the committee. Minnaert would permanently
hold the position of secretary. The KNAW decided in 1950 to follow
the committee's proposal to publish Stevin's Works.

Upon closer inspection, the committee limited itself to reprinting
publications in which Stevin developed original ideas: {*}The Principal
Works of Simon Stevin{*}. Secretary Minnaert handled fundraising,
translation, publishing, and printing. For part I of the series, Minnaert
was able to secure sufficient financial support in the early 1950s.

In a final attempt to obtain a contribution from Flanders, if necessary
symbolic, he and Dijksterhuis turned to the board of the Association
for Science in the person of his flamingant friend from youth, J.
Goossenaerts: 'In addition to the work of Huygens, Beeckman, and Van
Leeuwenhoek, this work by Simon Stevin must also be considered a monument
of Dutch civilization.' They suggested an advance that would be repaid
'as the sale of the 500 copies takes place.' Minnaert separately explained:
'It would certainly be extremely regrettable if Flanders did not contribute
in some way. We have so far written to: the Royal Flemish Academy,
the Flemish Engineers Association, Noordstar Boerhave, the City of
Bruges, Professor Peters, the Académie de Marine of Belgium, and we
have received nothing from any of them. Now I have some hope that
the Association for Science might be able to achieve something. It
should be clear that even a small contribution would be valuable,
if only as a symbolic token of interest from Flanders.'

A year later, Minnaert wrote to him that the publication of part I
was secured, 'but from Flanders we have received nothing but promises.
It was a great pleasure for me to be in Ghent and to see you again.'
It remained a North Dutch project.

\section*{Enthusiast in the history of science}

In the early 1950s, Minnaert began exploring the history of science,
which must have been encouraged by his frequent interactions with
Dijksterhuis and Forbes. Additionally, his friend Dirk Jan Struik
had written one of the first 14 historical publications on the history
of mathematics. In 1953, in addition to his articles on Copernicus,
Minnaert also wrote the brochure {*}Sonnenborgh; The Utrecht Observatory
and its history; 1642 - 1853 - 1953{*}.

The Utrecht Observatory proved to be one of the oldest observatories
in the Western world. In anticipation of the Peace of Münster (1648),
the city council had already converted the Smeetoren into a scientific
institution. 'This transformation from a military stronghold into
a scientific institution must have made an impression.' In 1853, Buys
Ballot established a new observatory on the Sonnenborgh bastion, also
incorporating a meteorological institute. Minnaert's residence housed
a laboratory for meteorological observations that was moved to De
Bilt in 1897.

Minnaert, alongside Dijksterhuis, was the central figure of the Stevin
project, which in the 1950s delivered three volumes of the {*}Principal
Works{*}. On the left page was the photographically reproduced original
Dutch text; on the right page was the English translation. In 1955,
part I on Mechanics was published by Swets \& Zeitlinger with a General
Introduction by Dijksterhuis. Stevin appeared here as the founder
of the parallelogram of forces and a pioneer of falling experiments.
The 'klootkrans' (a decorative chain or garland) with the motto 'Wonder
en is gheen Wonder' ('A wonder is no wonder'), which could just as
well have been the motto for Minnaert's {*}Natuurkunde van 't Vrije
Veld{*}, was discussed here, as well as the experiments with lead
balls that Stevin and Mayor Jan de Groot of Delft dropped from the
tower of the Oude Kerk. Stevin had been a predecessor to Newton and
Galileo. Would he have become a greater international figure if he
had been able to publish his work during his lifetime? His son Hendrick
had been able to publish most work only thirty to fifty years after
his death.

Parts IIA and IIB on Mathematics began with an introduction by Struik,
who provided the state of affairs in 1600 and analyzed what Stevin
had added. Here, Stevin advocated, among other things, for the introduction
of a decimal system for measurements and weights and introduced the
principle of decimal fractions for the first time. When multiplying
by 1,325, he performed the operation with 1325, finally moving the
comma three places back. Multiplications with fractions became operations
with whole numbers: he was the pioneer. He made an adjustable glass
panel for Prince Maurits that converted knowledge of perspective into
an estimation of the distance of remote objects. In algebra, he designed
a method for the numerical solution of equations of any desired degree.

In the late 1950s, Dijksterhuis suffered a stroke that paralyzed the
right side of his body. The committee continued to meet at his home
in Bilthoven, but Minnaert had to take over the management. In 1961,
part III on Navigation by Crone and Astronomy by Pannekoek appeared.

The Flemish cartographer Mercator had designed maps with fixed courses,
aimed at the same compass point, which he had laid out as ‘loxodromes’
on his sea charts. Stevin explained these methods so that sailors
could work with them in practice. In another booklet, Stevin had written
about finding harbors, improving a method for determining longitude
by his fellow countryman Plancius.

In {*}De Hemelloop{*}, Stevin had shown himself in writing to be a
convinced follower of Copernicus as early as 1608, even before Galileo
(1613). The elderly Pannekoek had just managed to complete this astronomical
work. The ‘moving earth’ was called the ‘roe-445 rende eertcloot’
in Stevin’s ‘Duytsch’ or Diets; he rightly named the planets ‘dwaelders’
and used the word ‘cortbegrip’ instead of ‘argument.’

In part III, there was also an excerpt from {*}De Wijzentijd{*}. Minnaert
explained: ‘This is what Stevin called the long-past period in history
when humanity would have possessed much more complete knowledge than
ever since. While he indicating how a renewed 'Wijzentijd' could be
achieved, he provides insight into his ideas about the practice of
science, the relationship between theory and practice, and the beautiful
properties of the Dutch language, which is so well-suited for scientific
endeavors.' Stevin believed, long before Bolland, that Dutch, due
to its richness in monosyllabic core words, was exceptionally suitable
for expressing the laws of nature: in that 'Wijzentijd,' humanity
might have spoken Dutch!

Minnaert also involved himself in the Dijksterhuis vacancy and invited
his friend Struik to fill in during the 1963-1964 academic year. Struik,
who had written {*}The Land of Stevin and Huygens{*} (1957), indicated
that he wanted to accept the appointment, 'but the gentlemen should
be aware that I have not abandoned my old Marxist convictions.' 'Doesn't
matter,' Minnaert wrote, 'come along.' Struik had been a target of
McCarthy's campaign in the early 1950s as a professor at the Massachusetts
Institute of Technology and had to agree to retire at 65. The appointment
in the Netherlands came at the right time for him. Minnaert referred
Annie Verschoor to Struik when she asked for advice on finding an
author to write the scientific chapter of {*}On the Fault Line of
Two Centuries{*}, the magnum opus of her late husband Jan Romein.

In 1963, part IV was published under the editorship of Colonel W.H.
Schukking, a member of KIVI, about {*}The Art of War{*}. Stevin had
been Prince Maurits's bookkeeper, quartermaster, and military advisor
and had written books on fortifications and the logistics of warfare,
such as {*}De legermeting{*}. The permanent state of emergency in
the Dutch Republic likely hindered Stevin in publishing his manuscripts!

\section*{The Bourgeois Life}

In the early 1960s, international recognition of Minnaert's work accelerated.
In 1961, Minnaert became a member of the Kungliga Vetenskapssamhället
at Uppsala University and of the Société Royale des Sciences in Liège.
In 1964, he represented the KNAW at commemorations in Italy honoring
the memory of Galileo Galilei. In 1966, he was invited to join the
Academia Nazionale dei Lincei in Rome. In the United States, Minnaert
was offered memberships in the Academy of Arts and Sciences (1959)
and the National Academy of Sciences (1964). Some of these honors
were based on his work in the history of science. He wrote a series
of biographical sketches of astronomers for, among others, the Enciclopedia
Biografica and the Dictionary of Scientific Biography.

In 1965, part V of the Stevin Werken was finally published. It included
Stevin's contributions to Engineering, introduced by Forbes, as well
as Stevin's treatise {*}Van de Spiegheling der singconst{*}. The latter
is a pioneering work: a remarkably modern discussion of tone frequency
relationships, introduced by theoretical physicist A.D. Fokker. This
part concluded with {*}The life of the Citizen: de Vita Politica{*}
or {*}Het Burgherlijck Leven{*}.

For Minnaert, this marked the end of twenty years of shared responsibility
and nearly ten years of full responsibility for the Stevin project.
He could report with relief on the successful completion of the task.
Dijksterhuis and Pannekoek had not lived to see this. The publication
would undoubtedly stimulate interest in 'the fascinating personality'
of Stevin and contribute to knowledge of 17th-century science. Minnaert
had remained somewhat in the background for outsiders throughout the
project. With his article {*}Simon Stevin: A Wonder is No Wonder{*}
in the English-language Delta of spring 1968, Minnaert reviewed the
entire project once more.

He paid much attention to Stevin's {*}Het Burgherlijck Leven{*}, which
argues that an individual should adapt to the power and authority
relationships within the state. If they do not please him, he should
move to another country. Claiming political power based on history,
Stevin considered a futile endeavor. If one went back to the earliest
times of Rome or Gaul, any claim to power could be justified. An opponent
of authority would then have to invade the country as a declared enemy,
but should abstain from internal undermining of the state.

Stevin's view of religion was equally pragmatic. Religion is simply
necessary to instill virtue in children. Parents can punish their
children for misbehavior, but as soon as they are absent, the children
may revert to unwanted behavior. However, if they learn that God is
watching them and will later punish or reward them, this problem of
order is solved. Therefore, Stevin recommends that parents instill
belief in God in their children, even if the parents themselves are
completely irreligious.

Minnaert explained these positions by considering that Stevin lived
in a 'post-revolutionary period' of the Dutch Republic, where after
decades of war and revolution, it was necessary to consolidate and
stabilize the new power structures. Only a central authority could
create the conditions for the flourishing trade and prosperity of
the young Republic. This is a textbook example of 'historical materialism,'
25 Marxists would say: Stevin's post-revolutionary 'being' determined,
indeed dictated, his 'consciousness.'

\section*{It all comes down to truth}

After World War II, the issue of Flanders had taken on a different
dimension for Minnaert. He had begun to look more at the international
context. 26 Some statements give an idea of how Minnaert looked back
on his 'activism.' Struik: 'When I asked him if he still stood behind
the position he had taken in 1915-1918, he said: 'Yes: during the
First World War, we had to choose between two imperialist alliances
with little difference in principle; the Flemings were entirely justified
in using the war to their advantage. But during the Second World War,
it was humanity's task to defeat fascism, and this required supporting
the Allies, especially since the Soviet Union was part of them.''

In an interview with VRT, Minnaert confirmed that the activism of
Jong-Vlaanderen had been beneficial: 'The administrative separation
was already a shock, the state of Flanders an even greater shock.
The idea was to achieve something with this shock therapy, and that
happened. We wanted to make Ghent University Dutch-speaking, and that
Dutch university has come into existence.'

To Picard Minnaert acknowledged that Domela was a vain man, sometimes
a dangerous enthusiast, but he didn’t draw any consequences for his
position as his adjutant.

In the late sixties, he developed an intensive correspondence with
Lammert Buning from Drenthe who wanted to write a biography of Domela
and therefore was interested in the origins of Jong-Vlaanderen (Young
Flanders). After reviewing some conceptual chapters, Minnaert warned
the potential biographer: \textquotedbl Don't let your sympathy for
your hero carry you away too much; he also had weak points, which
you've almost entirely overlooked. After all, it's about the truth---the
only thing that will ultimately remain. Buning remained stuck in the
hagiography of a party member.

In the late sixties, Minnaert wrote letters to comrades like Picard
and even worried about the archiving of Jong-Vlaanderen. With the
Dutchification of Flanders and the federalization of Belgium, with
the emergence of Flemish dominance in the federal state, Minnaert
foresaw that his ideals would largely be realized. He wrote to Picard:
'What one sees happening at this moment is nothing more nor less than
an administrative separation in education. If Brussels were not there,
everything would proceed much faster.' The Flemish emancipation remained
close to his heart. But as he wrote to Buning in 1968: 'Since my stay
in North Netherlands, I have increasingly come to understand the importance
and beauty of socialism and the dangers of rising fascism. My love
for Flanders and the Netherlands has remained just as great, but incorporated
into the international brotherhood of nations and societal reform.'

This phrasing indicates that Minnaert consciously transformed his
Flemish nationalism into an internationalist conviction, in which
Flemish emancipation remained a crucial point of consideration.\\

Footnotes:

1 Picard to L. Buning on June 13, 1971, a year after Minnaert's death.
Minnaert had written on this to Buning himself on December 31, 1968.
Archive Buning.

2 Picard was married to Martha van Vlaenderen with whom Minnaert had
been in the management of the Gent ANV youth movement.

3 Announcement by L.R. van Dullemen.

4 Heard from, among others, prof. dr C. Zwaan. 5 De Jager, C., in:
Haakma, 1998.

6 Romein-Verschoor, A., Simon Stevin, in: Erflaters van onze beschaving,
1938.

7 Dijksterhuis, 1943.

8 Minnaert to Buning, letter of December 31, 1968. Buning Archives.

9 Van Berkel, 1996, 314, 381, 423,

10 In Stevin's time, the physicist Snel, also known as Snellius, provided
several Latin translations, then the scientific lingua franca.

11 From the Holland Society of Sciences, ZWO, the KNAW, the Bataafs
Genootschap, the Hendrik Muller Vaderlands Fonds, the Prins Bernhard
Fonds, the Royal Institute of Engineers, and the Provincial Utrecht
Society.

12 Dijksterhuis and Minnaert to J. Goossenaerts, July 10, 1951.

13 Minnaert to Goossenaerts, May 29, 1952.

14 Struik, The concise history of mathematics, New York 1948. And
Yankee Science in the Making, Boston 1948. 15 Minnaert, 1953.

16 The Principal Works of Simon Stevin, Part I (1955), IIa, IIb (1958),
III (1961), IV (1964), and V (1966).

17 Indeed, the father of Hugo de Groot.

18 Van Berkel, 1996, 501.

19 Minnaert, M.G.J., The astronomical and nautical works of Simon
Stevin, Lecture for KNAW, report Natuurkunde section, 1, 1961, 11.

20 Struik, D.J., De ‘zaak-Struik’ (1951-1955), Politiek \& Cultuur,
4, 1992. Chapter from unpublished Memories, translated by Leo Molenaar.

21 For Struik, forty more active years after his 65th were to come.

22 A. Romein-Verschoor, Omzien in verwondering II, 275.

23 At least the biographical sketches of Hortensius, Hoek, Kaiser,
Pannekoek, and Stevin for the Dictionary and those of Stevin, W. de
Sitter, and J.C. Kapteyn for the Enciclopedía. He also delivered a
historical lecture for the engineers of KIVI in 1970 on the occasion
of 150 years of the metric system, which appeared in both De Ingenieur
of July 3, 1970, and the magazine Delta.

24 Minnaert, 1968d.

25 The author joins them.

26 Struik, Memories, unpublished chapter about his student days.

27 Minnaert, tv-script J. Florquin, Ten huize van..., 1970, 9. The
VRT was then known as the BRT.

28 Minnaert to Picard, March 3, 1969, AMVC Archive.

29 Buning Archive. From August 25, 1968, until his death, Minnaert
wrote 21 letters.

30 Minnaert to Buning, March 15, 1970.

31 Buning, 1975.

32 Minnaert to Picard, February 20, 1969. He thought the Antwerp AMVC
would be suitable.

33 Minnaert to Picard, March 3, 1969, AMVC Archive.

34 Minnaert to Buning, December 22, 1968. Buning Archive.

\chapter{A Crumb on the Cloth of the Universe}
\begin{quote}
'The country urgently needs teachers in the natural sciences. A wonderful
career of great social significance lies open before you. May we expect
that you will feel drawn to it?'
\end{quote}

\section*{The Working Group on Science Didactics}

World War II once again stimulated the realization that everything
had to be fundamentally different. The New Education Fellowship had
become the leading UNESCO body in the field of education. The first
conference in 1947, held in Cirencester, UK, was dedicated to The
Promotion of Peace through Education. At the time of the first post-war
conference of the Dutch section in 1949, the Working Community for
Educational Renewal (WVO), with educators Kees Boeke and mathematician
Hans Freudenthal as initiators, politics had already put this effort
on hold again. Many educators had little inclination to follow the
government's lead: weren't two world wars enough? The issues faced
by Minnaert's Federation of Scientific Researchers repeated themselves.

Minnaert was involved in both the post-war reform of mathematics education
and that of natural and astronomical sciences. On October 6, 1948,
he sounded the alarm among his colleagues for the first time: 'The
lack of good teachers is becoming a disaster for secondary education
and indirectly also for universities. To address this, measures of
various kinds must be taken. Our task is to ensure that at least the
training of future teachers at the University is properly conducted.\textquotedbl{}
This marked the beginning of a period of continuous attention under
Minnaert's guidance for the training of science teachers in Utrecht.

In December 1949, he opened an exhibition for the Utrecht branch of
the then MULO schools, showcasing 'little devices with which physics
experiments are conducted.' The driving force behind this was G.H.
Frederik, a physics teacher who had written to Minnaert about the
two-year teacher training program organized by the MULO associations.
They hoped it could lay the foundation for teachers' qualifications:
the course spanned two years of 35 afternoons, alternating between
physics practicals and theory in biology and physics.

Minnaert made a strong impact at the opening. Everywhere, the natural
sciences were recognized as the foundation for technology and agriculture,
and for an industrialization that required a higher development of
workers: 'If we want to maintain our place in world science and the
global economy, this foundation must be laid for Dutch youth. And
not just physics education that is pushed into an exam and forgotten
the next year...' Dutch education was purely verbal: 'It is the natural
sciences that can free our education from such one-sided wordiness.
They bring gleaming copper and splashing water into the classroom,
the warm colors of the spectrum; the delightful stench of hydrochloric
acid or ammonia, or the loud explosion; the tender green shoots of
sprouting plants and the wonderful beauty of fish in the aquarium.'

He praised the young people presenting the instruments and commended
the cooperation between neutral, Christian, and Catholic education:
'The teacher training colleges seem to see no opportunity to teach
their students how to experiment, and a special certificate for natural
sciences does not exist - but from all corners of the country, they
have set up their own courses. There is no free time available for
MULO teachers to practice; well, these young people have sacrificed
their Saturdays for two years out of passion for their teaching task
and because they are captivated by the idea of bringing a piece of
nature into the classroom.' The authorities exploited the idealism
of these teachers. A quarter-century after the voluntary work of Soester
teachers for his 'physics in student practical' nothing seemed to
have changed!

On December 22, 1950, a day of historical significance, the Physics
Didactics Working Group (WND) of the WVO was established at the Observatory.
The mathematics group of the WVO, led by physicist Greet Smit-Miessen,
had deemed the time ripe. As their first study subject, the group
chose ‘the purpose and function of the practicum’: only six(!) schools
in the country were found to have a practicum room. The practicum
had to be defended against those who saw it as mere tinkering and
a waste of time. The members of the working group gathered counterarguments,
developed homework tests, and thought about a new curriculum. Minnaert
was one of the most loyal participants: ‘He believed he should decline
the chairmanship but made himself available as an advisor.’

The group shared the opinion that physical concepts must first grow
intuitively: only gradually should they be named and defined. This
stood in stark contrast to textbooks that took definitions as their
starting point. Additionally, the WND advocated for the practicum
to become part of the final exams. At a joint conference with the
mathematics working group in November 1952 on General Natural Sciences,
Minnaert discussed a new curriculum for astronomy, while Tatiana Ehrenfest-Afanasjeva
and Freudenthal addressed the principles of geometry and mechanics
in schools.

In the 1950s, the physics working group gained a foundation of one
hundred active members, about thirty of whom attended the monthly
meetings.

\section*{The Practicals and Teacher Training}

In the early 1950s, G.H. Frederik and A.W. Middeldorp's Test Book
with the motto ‘from doing to understanding’ became popular. According
to Frederik, the approach followed was ‘inspired by Minnaert’s ideals:
treatment took place on the basis of self-conducted experiments, and
in processing the reports, students had to pay attention to the interplay
between experiment and theory.’ Handy storage boxes with apparatus
were marketed by the firm Luctor. A tripod with a large base plate,
which Minnaert had recommended back in 1924, made the setup stable
and clear. Frederik succeeded Reindersma in Utrecht as a university
lecturer in 'didactics and methodology of physics,' which made the
setup stable and clear. In the late 1950s, textbooks aimed at upper-level
education also emerged, such as the three-volume {*}Handleiding voor
het natuurkundepracticum{*} by Bulthuis and Gathier, and the popular
{*}Doen en denken{*} by Kelder, Steller, and Zweers.

In addition to practical training, teacher education in the 1950s
received more government attention. A Royal Decree of August 28, 1952,
confirmed in 1955, mandated 'a university-level pedagogical-didactic
education for teachers following the approach of Utrecht.' This meant
that future teachers had to complete a school internship. This increased
requirement frustrated many students who viewed teaching merely as
a fallback option; they believed minimal effort for teacher training
was more than sufficient. Minnaert addressed these students with a
strict 'personal message', emphasizing that science and experience
play a crucial role in societal life: 'This is also true in education:
the individual must take into account what has been experienced, tried,
and thought by many others. Education fulfills such a central and
vital function in our society that we can no longer afford to pursue
it without systematic preparation. This applies all the more to the
natural sciences and mathematics, which are developing so rapidly
that continuous revision and modernization of teaching materials are
absolutely necessary for scientific progress to continue.'

There was much discussion about the university's social role. For
years, students had noted that 'internships,' or gaining experience
in schools, was one of the most useful parts of teacher training.
Now, however, students were objecting to this component of their education:
'Students of the Faculty of Mathematics and Natural Sciences! The
country urgently needs teachers in the natural sciences. A wonderful
career of great social significance lies open before you. May we expect
that you will feel drawn to it? Do you intend to prepare yourselves
for it?'

In 1961, Minnaert delivered a lecture titled {*}University and Didactics,{*}
in which he looked back on his own Aha-Erlebnis in Mrs. Ehrenfest-Afanasjeva's
discussion group, he said: 'I believe we then understood that an original
thought in the field of didactics can be just as important as a scientific
discovery.' On June 28, 1963, Minnaert chaired the committee for teacher
education in the natural sciences faculties for the last time. He
noted that the four didacticians of the exact sciences at the time
were part of the Institute of Pedagogy. Under their leadership, topics
being addressed in secondary education were tested. The Physics Laboratory
had established a 'didactics' department, and the number of students
aiming to obtain a teaching qualification rose sharply. His colleague
Freudenthal concluded that Utrecht's central role in the didactics
of pre-university education had begun with what Minnaert had achieved
for teacher education in the natural sciences.

In the 1960s, the implementation of physics practical work suddenly
gained momentum. In 1961, practical work was done in 25\% of lower-level
HBS and gymnasium classes; by 1965, this had risen to 50\%. With new
school constructions, practical work laboratories became standard
facilities, and funds were allocated for equipment. Commercial companies
responded to the growing market for educational materials.

J.Ph. Steller, a Utrecht HBS teacher, approached Minnaert to pursue
a Ph.D. in physics didactics. Steller, later a professor of physics
didactics in Eindhoven, wrote his 1966 dissertation titled {*}Handigheid
of inzicht?{*} ('Skill or Insight?'), focusing on 'the value of the
physics practical work.' He had surveyed 400 students who had two
years of experience with demonstration-based teaching and individual
practical work. The majority preferred practical work but believed
it made the subject matter more difficult. Steller concluded that
practical work needed improvement while demonstration experiments
remained important. Unlike Minnaert's astronomy Ph.D. candidates,
this new didactician was not impressed by his promoter's textual criticism.
He thanked him ironically: 'Your meticulous care for the smallest
detail, which I sometimes found not excessive but still annoying,
taught me a lot\textquotedbl . Steller's promotion was nonetheless
Minnaert's latest achievement in the field of education.

However, the success of the practical course led to the decline of
the WND, as the effort for the practical course had long been its
most important raison d'être. The group initiated the so-called Woudschoten
conferences in the 1960s, which were intended to stimulate the didactic
enthusiasm of physics teachers. Minnaert attended the first conferences.

His involvement with the subject 'astronomy' was at least as large,
but it ended in a huge disappointment.

\section*{A curriculum for astronomy}

Minnaert had already been convinced in the 1930s that the school subject
'cosmography' did not do justice to young people's fascination with
the starry sky. Just as mechanics had been colonized by mathematicians,
cosmography had been taken over by mathematicians and geographers.
He therefore designed an alternative curriculum in {*}The Astronomy
and Humanity{*} (1946). After the war, the Education Committee of
the Dutch Astronomers Club (NAC) was established. It consisted of
J. Raimond Jr., chairman of the Netherlands Weather and Astronomy
Association (VWS), A.J.M. Wanders, a physics teacher, and Minnaert.
In the background, there was the threat of abolishing this 'physical
geography'. The committee wrote a Report on this in 1946.

On June 6, they spoke with Minister Van der Leeuw. Wanders had not
even been granted leave from his Catholic school to attend the meeting
with the PvdA minister! They tried to enthuse him about a subject
called 'astronomy' with new content: the historical development of
ideas about the structure of the solar system, supported by numerous
observations to be carried out by students, followed by the main features
of the structure of the universe. It was to be entirely separate from
physics: the goal was to decipher the structure of the universe. The
committee wanted more than the two-hour course of 1920. There should
be university lecturers in 'didactics of astronomy' who would cultivate
qualified teachers. They were also supposed to be competent in natural
sciences and mathematics. In 1946, the committee sent a circular outlining
its principles to all rectors. Shortly after this meeting, the Schermerhorn
government came to an end, and the issue was temporarily set aside.

At the beginning of 1950, Minnaert warned that the Ministry wanted
to design a new program for Secondary Education. He wrote {*}The Astronomy
in Higher Secondary Education is in Danger{*}, as cuts were being
made to smaller subjects. He argued for retaining that one hour: 'Astronomy
is a wonderful school to teach us to distrust our naive impressions
and, with the help of clear reasoning, build a less obvious but more
harmonious worldview. The greatness of this worldview lies both in
its immense dimensions, which suddenly make us realize the insignificance
of humans and Earth compared to the universe, and in its simplicity
and regularity. It is a classic example of a subject not directly
tied to practical utility, yet everyone is impressed by its significance
for human thought. There is no other subject where such impressive
results can be achieved in thirty lessons.'

Finally, Minnaert advocated for a three-hour course over a quarter.
This was so utopian at the time that Raimond admonished him. Minnaert
also believed that the astronomy subject might have a better chance
if it were part of a new general science subject. This required intensive
collaboration with other scientific disciplines.

In 1953, Minnaert placed the anniversary of the Utrecht Observatory
under the banner of 'the social task' of astronomers. In August, he
organized a summer course for teachers in collaboration with teacher
associations and the NAC. More than 200 participants attended the
lectures and demonstrations. Thus, the anniversary served Minnaert's
lobbying efforts for astronomy in secondary education.

\section*{The mammoth under a dark star}

In the 1960s, the Dutch Mammoth Act would ultimately bring about the
abolition of the subject 'cosmography.' This realization gradually
dawned on the Education Committee, or at least on Minnaert. The subject
of mechanics also fell victim, but it was placed under physics. Minnaert
offered his chairman Raimond to write again about the significance
of astronomy. The Response Memorandum from February 1961 confirmed
his worst suspicions: astronomy had been added to the non-compulsory
subjects and thus excluded. Because the HBS (Higher Civil School)
was abolished, and therefore pre-university education (VWO) for many
students was extended from five to six years, this elimination was
even more regrettable.

Minnaert wrote a protest to Secretary of State Stubenrouch and demanded
a hearing: 'The subject of Astronomy has an unparalleled ideological
and broadening significance, making it indispensable for all who follow
the atheneum and who will later, at university, inevitably have to
specialize.' The audience took place four months later and resulted
in a referral to the Departmental Working Group and the Inspection.
As the meeting dates stalled and the months passed, faits acompli
were created.

Wanders suggested: 'Wouldn't it be time---third time!---to provide
some 'information' about the subject of Astronomy in the teachers'
journals? I remember your convincing articles from 1946/47 and later
once more, I think around '53.' Minnaert immediately wrote an apology:
'Astronomy is primarily aimed at answering the big questions: Where
are we? Where do we come from? Where are we going? It's so rewarding
that the most important general insights about this can be shared
with students in a limited number of lessons, in such a way that this
education becomes a true experience for them.'

The subject could show students through their own observations and
considerations that the Earth floats freely in space and moves around
the Sun: 'This requires a mental liberation from naive concepts and
prejudices, an awakening of the critical spirit, a first unfolding
of scientific imagination. The breakthrough of the Copernican worldview
is also connected in many ways with the colorful civilization history
of the 16th and 17th centuries.' For Minnaert, 'astronomy' had become
a philosophical issue: the subject should acquaint prospective students
with the 'wonderful structure of the universe.' It was to no avail.
The House of Representatives passed the Mammoth Act and eliminated
the mandatory subject of cosmography. Astronomy disappeared from the
regular curriculum for thirty years, although 'astrophysics' occasionally
worked on its own as a special topic within physics.

An observer involuntarily looks back for hidden agendas. Minnaert
wanted to dedicate the subject to the new scientific worldview, to
the cosmogony of a universe that took billions of years to form, with
humans as a crumb on the skirt of the universe. When 'cosmography'
still dealt with coordinate systems, zenith and azimuth, Jacob's staff
and sextant, angles and trigonometric distance measurements, inclinations
and declinations, it was a harmless subject. How eager was one for
the new worldview in the Catholic stronghold that the Ministry of
Education had become since 1946? Was there a wait there for a subject
based on the truths of Galileo and Copernicus, Darwin and Einstein?
Within biology, fundamentalist schools still ignore Darwin's truth
to this day, and therefore 'evolution' is excluded from the National
Examination in the third millennium after Christ. Minnaert's astronomy
placed evolution and cosmogony at its core and wanted to proclaim
the 'fresh' results of modern science.

However, Minnaert's involvement in astronomy education was not limited
to the Netherlands alone. The Groningen astronomer A. Blaauw: 'When
a commission for astronomy education was established within the International
Astronomical Union in the 1960s, Minnaert was asked to chair it. Whenever
it came to the human aspects of international astronomical endeavors,
Minnaert was always the first to come to mind.' In 1961, Minnaert
became president of 'Commission 38' regarding the 'exchange of astronomers,'
which paid special attention to colleagues from developing countries.

During the 1960s, he would participate in discussions on astronomy
and science education. On behalf of the IAU, he sat on an international
consultation body (CIES), through which he attended conferences such
as in the African Dakar of 1964. His colleague Freudenthal recalled
the CIES Congress of 1968 in Varna, Bulgaria, 'where Minnaert passionately
pleaded during the closing meeting that astronomical education not
be forgotten in the congress resolutions---and it was indeed included---
I don’t know anyone else who would have been listened to as seriously
as Minnaert when advocating for his own field, because everyone knew
his field was the entire world.'

That was the world: in the Netherlands, he had lost the battle, although
he loyally contributed to the development of a curriculum for schools
that would independently choose to offer the wonderful subject of
'astronomy' to their students. Fortunately, there were such schools.\\

Endnotes:

1 Morsch, the historian of this WVO, wrote in 1984: 'A Boeke redivivus
would not feel at home in the current peace conference between the
major powers, which essentially only deals with more or less arms
control.' How much more would these educators have dismissed the call
to start this arms race!

2 Van Dalen, 2001, 201.

3 On that day, Minnaert condemned in another writing the 'shameful'
conduct of the curators regarding the appointment of a teacher for
mathematics didactics and, with this letter, took the initiative to
convene a meeting of the 'members of the committee for teacher training'
on December 16, 1948, at the Observatory library. This relatively
quickly led to the appointments of teachers Krans (physics) and Van
der Steen (biology) for a sum of fl. 250,- per year, almost unpaid.
Two years later, Minnaert proposed doubling this amount while also
trying to reduce their 26 teaching hours by 6. The latter finally
happened. Minnaert would occupy himself with such matters throughout
his life: hence the title of Molenaar, 1998. Astronomy Archive.

4 G.H. Frederik to Minnaert, November 14, 1949. 5 Instead of 'woordeneenzijdigheid,'
it originally read 'vloek van de woorden.'

6 Hooymayers, H.P., and W.P.J. Lignac, 25 years of Working Group Physics-Didactics,
Report WND, 10th Conference, December 19 and 20, 1975.

7 Frederik quote in Van Genderen, 1994.

8 Gathier was a Ph.D. student (1955) of Minnaert; Steller would become
it (1966). Steller wrote in 1958 'Why Physics Practicum?', after which
he must have joined Minnaert.

9 Minnaert, 'Teacher Training is Not Popular Among Students,' lecture,
1956.

10 Minnaert, 'University and Didactics,' lecture, October 14, 1961.

11 Freudenthal, W\&S, 1971, 1-3.

12 Steller, 1966. When Minnaert emerged in the 1920s as the animator
of the school practicum, his graduating colleague Lili Bleeker (1928)
challenged him with a proposition (8): 'If the plan is implemented
to teach secondary school students the principles of physics through
self-experimentation, it will probably not achieve the expected results.'
Perhaps Minnaert hoped Steller could refute her proposition. See Ginkel,
1997, and Offereins, 1998, for more on Lili Bleeker.

13 Wanders was also a PhD student under Minnaert. Wanders, part II,
1933.

14 Map-Education Committee NAC. Astronomy Archives.

15 Minnaert's letter to the State Secretary of March 31, 1961. The
memorandum was titled 'The Importance of Astronomy for VHMO.' Astronomy
Archives.

16 Wanders to Minnaert, February 10, 1961.

17 Minnaert, (1961).

18 Meanwhile, the Second Phase offers, for now, within the subject
General Natural Sciences, a perspective on the one-hour course that
Minnaert minimally envisioned in 1950 and which he wanted to include
at the time in the context of, indeed, a new subject called General
Natural Sciences.

19 Lucebert, see the cover page and title of the biography.

20 Interview with A. Blaauw.

21 Freudenthal, W\&S, 1971, 1.

\chapter{I Am Lucky in Myself}
\begin{quote}
'We strengthen ourselves when we watch Sophocles' Antigone being performed,
or when we read Gorter's Pan or when Beethoven's Ninth Symphony resonates.'
\end{quote}

\section*{Bridge Builder Between Generations}

Minnaert became a professor when the university was still an ivory
tower, where an elite from the higher middle classes dedicated themselves
to science and education. {[}1{]} After the war, the university underwent
a series of revolutionary developments. The expenditures for scientific
research in the Netherlands increased from 0.26\% of the gross national
product in 1948 to 1.05\% in 1963. For Utrecht, they rose from 16
million in 1954 to 37 million in 1960; the annual increase in personnel
costs was 8\%. The Utrecht staff of scientific employees grew from
60 to 900 in fifteen years, but nothing was regulated regarding their
rights and duties. Managing the expanding institutions became a problem
because professors often were not suitable as managers. Minnaert had
also not been a high flyer.

The student population grew from 5,000 in 1946 to 14,000 in 1969,
although the social background hardly changed. Parts of the university
were relocated from the city center to new buildings in De Uithof.
Within the administration, the rector’s secretary’s apparatus began
to play a key role. A creeping reorganization took place: the result
was an idiosyncratic large enterprise with amateurish management.

What burst forth typhoon-like in the second half of the 1960s was
so surprising for the university administrators because they thought
they were fully taking into account the changing society and the challenges
of their fields and education. Beneath the surface of reverence for
the academic stars and the untouchable administrators, much resistance
and irritation simmered. The minutes of the Curators or Senate do
not mention anything about the Student Trade Union Movement (SVB).
On the other hand, the student magazine {*}Trophonios{*} attacked
the Utrecht authorities. A layer of students radicalized quickly;
the anti-authoritarian cohort of 1968 no longer believed in the consultation
model, unlike the SVB members of the early years, but instead opted
for the latest fashion of direct democracy.

A mild Van der Heiden wrote in the university’s history book: ‘In
the midst of the rapid secularization of those years, younger generations
reached for new support, for slogans and patterns, often moved and
sincerely striving for a better society, but just as often unable
to convey ideas to large groups that always failed to show up unless
their educational provisions were at stake.’ The historian H.W. von
der Dunk delivered a harsher judgment: ‘The radical innovators, skilled
in neomarxist terminology, were even more paternalistic than the elites
they opposed. Considering this, the protest can primarily be regarded
as an elite youth movement within the higher strata of society. Psychologically,
the deepest impulse was to force a change of guard because the old
guard clung too much to their positions. This always goes hand in
hand with the introduction of a new philosophy and sometimes even
a new style.'

The rebellious youth not only attacked hierarchical relationships
but also the nature of science itself, which they believed should
serve political and social goals. The agitation led to occupations
of administrative buildings and occasionally turned lecture halls
into political arenas. In the late 1960s, plans for administrative
reforms were circulating, and new legislation broke the professors'
monopoly on power. According to former student activist K.-J. Snijders,
the Senate responded very authoritatively to the self-willed students,
though he named Minnaert and theologian De Graaf as exceptions.

During the controversy over an issue of {*}Trophonios{*}, in which
illustrator Arne Zuidhoek depicted 'Schmeltzer's night' and the fall
of Cals as Judas' betrayal of Christ, Minnaert had responded coolly:
when asked, he found it in poor taste and urged the editorial board
to maintain unity instead. When the student movement caused intense
upheaval among colleagues, he tried to reconcile them, as in his 1968
Dies celebration speech: 'With joy but also with fear in our hearts,
we follow the student movements across the world expressing a longing
for a better, freer, happier future: in Rome and Madrid, Barcelona
and Prague, Warsaw, Berkeley, Jakarta, Frankfurt, Leuven, and Berlin;
they are active everywhere, sometimes going off track but mostly driven
by great idealism, a certain sense of right and wrong that no diplomacy
can withstand in the long run.' Even though some attendees interrupted
him with shouts of 'Prague! Prague!'---an echo of 1957---he showed
an understanding of the youth's idealism and positioned himself as
a bridge-builder between quarreling generations.

He demonstrated an interest in new developments. In early 1966, Minnaert
and Miep Minnaert-Coelingh had added a protest to their post-war political
statements against the 'state of affairs in Indonesia over recent
months': after the mass murders of fall 1965, the communist trade
union leader Njono had been arrested by the Soeharto government. The
77-year-old was also present at the key conference of his Union on
'The Ideology of the West' in February 1970, where Western capitalism
was put in the dock and examples of Marxist economic theory and ideological
criticism were presented. Minnaert's conviction had never been Marxist.
However, he engaged in the discussion, which most colleagues feared
due to the verbal violence of the students.

\section*{The responsibility of the researcher: books for Hanoi}

On the eve of these developments, Minnaert was invited to meetings
on 'responsibility.' On March 15, 1966, he spoke at the Delft Studium
Generale. He often emphasized 'the responsibility of the researcher.'
He focused specifically on collaboration in the design of weapons.
At a time of chemical and biological warfare in Vietnam, this was
a relevant approach. Minnaert believed that everything depended on
the researchers: if they refuse to cooperate, there will be no war.
Therefore, they bear full responsibility. On the other hand, the researcher
has the same responsibility as every citizen: society decides what
it wants to do with their work. As an ordinary citizen, he can voice
his opinion here. Both considerations lead to the conclusion that
the scientific researcher and engineer bear more responsibility than
other citizens. For example, researchers have a duty to inform elected
representatives about the dangers of biological and chemical warfare.
They are best informed about the confidentiality within the company
or the motives of the clients. If the latter are not decent, they
should not cooperate.

He addressed the students directly: 'I would so much like you to be
happy in your life, in your future. The feeling of building something,
contributing to the happiness of humanity. That feeling you will never
have if you dedicate yourself to devising weapons. Do you want to
give your beautiful, flourishing young life to that? Minnaert believed
that the university should not maintain ties with institutions conducting
military research, accept defense subsidies, participate in 11 NATO
seminars, or keep research results secret. The nobility of the engineering
profession required upholding its standards! According to him, the
engineer was naturally a 'whistleblower.' Physicians had the Hippocratic
oath: 'Why shouldn't engineers make a similar promise?'

In the summer of that year, an appeal to university communities in
Western Europe appeared in W\&S and other journals. A total of 83
professors from nine European countries addressed their colleagues
regarding Vietnam: 'We feel responsible for everything that can be
done in favor of peace. We stand in solidarity with the very active
movement in the United States, where many professors and students
are opposing the war waged by their country.' The appeal mentioned
the establishment of an international secretariat in Paris to coordinate
initiatives: 'The moment has come to give more strength and coherence
to the action against the war in Vietnam and to prepare a large-scale
movement in favor of peace, based on the Geneva Agreements (1954)
and their fundamental provisions: the principle of the withdrawal
of foreign soldiers and war materials from Vietnam, the principle
of non-participation of Vietnam in military alliances, and the principle
of long-term respect for Vietnamese unity.'

Second among the French signatories was Minnaert's pupil and friend
from the Collège de France, Jean-Claude Pecker. The 'great names'
were limited to Ernst Bloch, Gunnar Myrdal, Jean Piaget, Joseph Needham,
Joan Robinson, and Eric Hobsbawm. From the Netherlands, thirteen professors
participated, most of whom taught in Utrecht: Minnaert, De Jager,
Nijboer, and De Graaf. The first appeal from 1966 by a Dutch committee
pointed out that colleagues in Vietnam were continuing their scientific
work and youth education: 'French universities have taken the initiative
for an international action to raise funds intended for the purchase
of standard works, major textbooks, tables of constants, which primarily
appear to be needed at the research level. Netherlands will particularly
contribute to civil engineering and tropical agriculture.'

The necessary bibliographic data had already been collected: for the
books in these fields, an initial amount of fl. 5,000,- was needed.
The 73-year-old Minnaert took charge: Zuilenstraat 25bis became the
correspondence address. A year later, a letter stated that fl. 9,000,-
had been raised: the shipments were on their way. Education in North
Vietnam had been continued and expanded with 'incredible energy.'
The number of students at universities and colleges had risen from
21,700 in 1964 to 46,400 in 1966: 'One of our French colleagues, returning
from a short stay in Vietnam, visited the University of Hanoi in the
jungle and found that there was a great lack of modern Western books,
while there were quite a few scientific journals available.' Minnaert
wanted to travel to North Vietnam himself for a similar visit, but
the floods in the country prevented this.

In a TV interview, Minnaert explained: 'However the war unfolds, the
best way to help them is to ensure they have intellectuals, that engineers
and doctors are trained. They have little or no contact with the Western
world, and therefore we must assist them. The initiative originated
from the French universities and then spread to other countries, to
Sweden and the Netherlands, and more countries will join. Already
now, books worth half a million guilders from European countries have
been sent there. These are books of the highest scientific level,
because Vietnam is not a backward country.' For Minnaert, the political
situation was clear and, as always, black-and-white: 'In Vietnam,
justice is entirely on Vietnam's side, and injustice on the side of
the Americans.' De Jager wrote later: 'The fact that one could fight
a war with books had deep significance for him, and the great resonance
his action found among surprisingly large parts of our people did
him much good.'

Criticism of NATO and the United States was growing everywhere due
to the Vietnam War, the colonial war of NATO member Portugal in Angola,
and the colonels' coup in Athens. Meanwhile, Miep Minnaert-Coelingh's
activities within the Dutch Peace Council (NVR) came to a standstill.
Those years were led by communist and resistance fighter Gerard Maas.
Just for him, there was an ashtray on the attic of Zuilenstraat 25:
an unheard-of indulgence for Miep. The CPN had used the NVR for twenty
years to form coalitions with other groups but gradually became accepted
as an independent party. In 1966, a first teach-in about the Vietnam
War took place. There, resistance fighter Gerard Maas spoke on behalf
of the CPN's executive board.

Additionally, the World Peace Council took a stance that annoyed the
CPN. At a 1968 meeting of communist parties in Brussels, the Russian
and French parties declared that the Vietnamese should stop the war
because a world war loomed. The CPN refused to sign a communiqué to
that effect. When the Soviet Union and its satellites also crushed
the Prague Spring, the CPN no longer found the World Peace Council
an interesting forum. The Dutch Peace Council was abruptly disbanded,
and the weekly deliberations between Maas and Minnaert-Coelingh ceased.
While the world opened up further for Minnaert and he found himself
short on hands and brains, his wife lost the meaning of her existence.

\section*{Religion unnecessary for morality and conscience}

The criticism of the two superpowers responsible for Vietnam and Prague
was widespread. Minnaert spoke on radio and television and gradually
became a well-known cultural figure. In March 1968, he discussed on
De Vrije Gedachte, the Freethinkers' Radio Broadcasting, how as an
atheist he could still uphold strict moral standards: 'When we are
born, we inherit a series of hereditary traits from our ancestors;
or rather: hereditary response patterns, because that is what is actually
inherited. An ivy plant does not have the characteristic of the well-known
multi-sided leaves. What distinguishes ivy is its reaction to shade
by forming multi-sided leaves and to light by forming round leaves.
It is the same with humans. What becomes of them will be largely determined
through the environment in which he grows up.

The investigations of Freud and many others have revealed how the
impressions from the very earliest years of childhood have a very
significant influence on the developing human being. During that time,
his parents continually instill in him what is good and what is naughty.
Later, he is further educated by the friends with whom he plays, by
the boys and girls at school, by the workers in his factory, by his
wife, by his children... He is influenced by the books he reads, by
the works of art that shape feelings about good and evil.

All these influences can work for better or for worse, but on average,
they will convey the aspirations of current society to the growing
person in more or less perfect ways. This is how his conscience takes
shape, that conscience which made such a tremendous impression on
Kant and whose origin he could not understand. It is completely unnecessary
to invoke the intervention of supernatural powers here, which would
have instilled this conscience in us once and for all, ready-made.

Minnaert followed in the humanistic footsteps of his uncle Gillis.
With his reasoning, he could uncover both the origin and the mutability
of 'conscience' and 'morality': 'The morality that we impart to young
people will later be passed on by them to the next generation: in
a certain sense, one can speak of a kind of heredity, a pseudo-heredity.
But that morality does not remain unchanged; it gradually evolves,
there is an evolution, a continuous adaptation to changing circumstances
in society.'

Forces from the past and present played their role in shaping morality:
'Christianity has also played a major role, partly by helping shape
beautiful ideas that already existed earlier, partly by adding new
norms. While some clung rigidly to the letter of Scripture, others
strove to bring out the noblest thoughts and feelings from it; the
storm of history surged through humanity. Socialism brought an important
new impulse; humanists, animal protectors, scientific workers, and
not least the youth, each contributed in their own way to the formation
of today's ethics, and that evolution continues unabated.'

Morality arose from people living together, as a set of indispensable
rules of conduct aimed at making humanity as a whole happy: 'Kropotkin
demonstrated in his book on Mutual Aid in the Animal Kingdom that
such rules exist and have played an indispensable role in the history
of mankind to overcome the difficulties of existence. We promote the
safety and happiness of others, and thereby gain the advantage that
they, in turn, contribute to our safety and happiness. But what is
encouraging now is that doing good itself gives us such great inner
satisfaction, regardless of whether we receive anything in return
or not.\textquotedbl{}

Minnaert had filled in the gaps in his polemic with Burgers from Sint
Michielsgestel: \textquotedbl We do good for the sake of good itself
and not to be rewarded in any way or to enter heaven. The forces of
conscience are apparently very strong within us. Religious people
add that, by acting virtuously, they fulfill God's will. This gives
virtue, so to speak, a sacred character, and they feel more inclined
to live a good life. It seems like an advantage of religion. But this
advantage is bought at a high price. The believer thinks he knows
what God prescribes or that his church knows it. He can hardly accept
that others think differently about it, or that God's commandments
change over time or with circumstances. Thus, there is the danger
that the natural evolution of morality will be hindered, and morality
will become rigid. There is also the risk that future humanity, which
will likely have forgotten religion, would come to think that concepts
of good and evil are built on shifting sand and can be discarded.
Those who listen to sermons on Sundays often get the opportunity to
reflect on good and evil and strengthen their feelings within themselves.
But the Church is not necessary for that.\textquotedbl{}

It was high time to indicate what a humanistic morality would look
like in the future: \textquotedbl We strengthen ourselves when we
watch Sophocles' Antigone being performed, or when we read Gorter's
Pan, or when Beethoven's Ninth Symphony resonates. The driving force
of our lives is here on earth. We must not seek it in the afterlife
or in the supernatural. It is for and with humanity that we live and
strive, and all incomprehensible things that people add to it only
reduce the value of that all-encompassing principle: humanity is the
highest.\textquotedbl{}

He did not come to write a book on Free Will as he had decidedly intended.
Nevertheless, Minnaert succeeded in making it clear in a series of
articles and lectures how his rejection of free will, as expressed
in the discussion with Burgers, could coexist with ethics and morality.
Kropotkin was central again, just as in his 1916 brochure. It is also
remarkable that he stated that his Flemish upbringing must have determined
his life.

\section*{Anti-vivisection and pro-Esperanto}

Minnaert also committed himself to activities against vivisection.
He condemned the indifference towards the suffering of animals simply
considered as 'material' and expressed his doubts about 'the lack
of pain sensation in animals.' He wanted to break the silence and
'disassociated' himself from the relevant scientists. Annually, 50
million animals were killed: less than 10\% for medical purposes.
He condemned an American researcher who had used 150,000 monkeys.
Many student experiments could just as well be arranged as demonstrations.
He pointed out the numbing effect and the unintended stimulation of
cruel tendencies in young people. He condemned experiments on idiots
and lunatics, on criminals and incurably ill individuals, which were
justified under the guise of science. As an example, he mentioned
the polio experiments on mentally disabled children. People did not
like to talk about such things, just as they avoided discussing slavery
and child labor in the previous century. The public was lulled to
sleep by invoking the 'halo of science.'

He remained an ardent propagandist for an international scientific
language. The KNAW had decided, partly due to his efforts, that their
publications could be summarized in Esperanto as early as 1947. At
the IAU Congress in Hamburg in 1964, he delivered a speech in Esperanto.
There, he compared the language issue with the problem of measures
and weights in 19th-century Europe: 'The longer one waits, the greater
the problem becomes.' Science suffered from the lack of a unifying
language: discoveries remained unnoticed and energy wasted. The languages
of scientists, based on population numbers, would amount to 22\% Chinese,
17\% Hindi, 9\% English, 7\% Russian, 3\% Portuguese, and 42\% other
languages.

Such a state would not tolerate English as the dominant language.
Moreover, it was not a neutral language but the language of one of
the camps in the Cold War. It was the worst possible world language
due to the irregularity of its pronunciation: 'How vague, inexact,
and poor is English compared to the constructed language.' Incidentally,
Minnaert, who grew up with French and German as scientific languages,
wrote English well but spoke it so poorly that some wondered if it
wasn’t an unconscious form of resistance. He found Esperanto sounded
remarkably beautiful:
\begin{quote}
'En la mondon venis nova sento,

Tra la mondo iras forta voko,

Per flugiloj de facila vento,

Nun de loko flugu âi al loko.'
\end{quote}
The pleasure of recognizing these lines, even among laymen, did not
lead Minnaert to consider that the 22\% Chinese, 17\% Hindi, and 7\%
Russians might not feel much for this Western European constructed
language. He believed one should not say the language had no perspective
unless they had first occupied themselves with Esperanto. Nevertheless,
he wondered why the language did not make more progress. The main
obstacle was the conservatism of scientists who had 'no time' for
it, raised half-mystical objections to the 'artificial language,'
or used the opportunistic argument that 'it wasn’t used enough yet.'
All supporters had to fight for their conviction, which was the ideal
expression of the unity of mankind, brotherhood, and love for humanity!
In the late 1960s, he was still working on a pocket dictionary of
astronomical terms in Esperanto. A committee of the Academy, of which
he was a member, made progress, as he informed Esperantist G.F. Makkink
in October 1970.

\section*{The completion of two astronomical works}

Minnaert’s concluded his involvement in four promotions in the second
half of the 1960s. He could also put a finish to the editing and secretary
work of {*}The Principal Works of Simon Stevin{*}. The project with
the American Charlotte Moore-Sitterly also needed to be completed.
Finally, in 1966, the {*}Second Revision of Rowland's Preliminary
Table of Solar Spectrum Wavelengths{*} appeared: a liberation for
the Utrecht-based Minnaert and Houtgast and their American counterparts.
By then, both Minnaert and Sitterly had been working on cataloging
the Fraunhofer lines for nearly half a century. His resistance, in
the name of equivalent widths, against the overly subjective First
Revision of 1928 had been rewarded with prominent participation in
the Second Revision. He had kept up with physical literature and benefited
from the persistent work of Russians like Kolesnikow and Leskow on
transition probability values.

As early as 1960, the Utrecht team had presented an incomplete Catalog.
In January 1966, publication of this Table of all Fraunhofer line
wavelengths between 2935 and 8770 Å could begin. Like the Solar Atlas
of 1940, it was an impressive book. The edition had received support
from UNESCO and appeared under the auspices of the IAU at the National
Bureau of Standards in Washington. Enthusiasts could enjoy 340 pages,
containing 680 columns with approximately 34 lines per column, thus
detailing 23,000 Fraunhofer lines. For each line, the following details
were provided: the wavelength in six or seven digits (1), equivalent
width (2), reduced width (3), behavior in a sunspot (4), identification
of the atom or ion (5), lowest excitation potential (6), and multiplet
number, essential for the growth curve (7). For the titanium(II) ion,
these values were continuously changing: 4316.802; 38; 10.2; unchanged
(u); Ti II; 2.05; and 94.

The German Kirchhoff had created a Catalog of Fraunhofer lines, and
the American Rowland the first preliminary Table. Minnaert's name
would remain linked to this Table, which would remain the standard
work for decades. For Minnaert, it must have been a triumph after
fifty years of painstaking work in the tradition of these illustrious
predecessors.

It wasn’t the only astronomical publication. His colleagues had urged
him to convert into a book the assignments developed over thirty years
for the Astronomical Practicum. That became {*}Practical Work in Elementary
Astronomy{*}, which was published in 1969 by Reidel in Dordrecht.
The IAU recommended the Utrecht approach in its foreword: students
'had to get acquainted with the sky before they would stare at the
blackboard.' The assignments ranged from observing constellations
to measuring astrographic plates, from working with the gnomon (sundial)
to measuring parallax for determining star distances, from setting
up a telescope on an object to determining position 'at sea.' Many
assignments involved determining orbits: of the moon, Mars, meteors,
and satellites. Of course, lunar observations were part of the program,
as well as observations of planets such as determining 'the albedo
of Venus' or 'the rotation period of Saturn.' The sun gave rise to
assignments about its midline, the solar constant, the profile of
the Fraunhofer lines, the equivalent width, the growth curve, sunspots
and solar rotation, the shape of the corona, and the radio bursts
from a solar flare.

Sometimes it involved direct observations, other times work based
on prepared photographs. An assignment about sketching the Milky Way
under a clear sky is of a different order than determining the dynamic
parallax of double stars. The book perfectly matched Minnaert's educational
activities for the IAU.

\section*{The Apollo landings and moon studies}

In the 1960s, Minnaert’s original research work was limited to the
moon. He had the pleasure of seeing that his 1941 article on the reciprocity
principle and the photometry of the moon was frequently consulted.
Also, the clinophotometric work by Van Diggelen and himself, which
had formed part of his contribution to Kuipers' {*}Planets and Satellites{*},
continued to attract interest. After all, attention to the lunar surface
became part of the practical preparation for the Russian and American
moon landings. In 1967, moon expert Z. Kopal wrote in {*}Measure of
the Moon{*}: 'The repeated failures in attempts to approach the reality
of the lunar surface in an acceptable way gradually led to an acceptance
of Minnaert's view, namely 'that the photometric properties of the
Moon are primarily determined by shadow phenomena caused by countless
millions of irregularities on the surface; and that the precise form
of the scattering and reflection laws is relatively unimportant compared
to the effects of the micro-relief of the surface.' Minnaert also
had the satisfaction of seeing his prediction regarding the reciprocity
of measurement points confirmed by Barabashev (1962) and Jones (1968).

In 1959, a Sputnik had flown around the Moon, making the first images
of the far side of the Moon available. Minnaert was a member of the
IAU commission on Venus and other planets. In 1963, he was asked by
the French chairman A. Dollfus to join the committee tasked with proposing
a new nomenclature for the Moon. The first 'soft' landings of the
Luna flights were credited to the Russians, while the Americans countered
with the Surveyor missions. The photographs came from both the Russians
and the Americans. A subcommittee of three---Russian Mikhailov, American
Menzel, and chairman Minnaert---had to make a proposal in the late
1960s. This involved modernizing the existing naming system and also
assigning names to objects on the far side. These names needed to
be approved by both sides of the Iron Curtain.

This required Minnaert to travel to Dollfus' Institut d'Astrophysique
in Meudon, to the Soviet Union, and to the United States: many meetings
in hotel rooms and much consultation. One time in Meudon, Minnaert
unexpectedly visited astronomer Chriet Titulaer, who had eagerly followed
his lectures in the 1960s: 'One afternoon he rang my bell at my flat
in Meudon and seriously wanted to talk to me. He told me that I could
become a good popularizer or a good astronomer, but that I had to
choose. I decided to follow his wise advice and went into popularization.'

Ultimately, the nomenclature proved acceptable to all parties. The
differences of opinion focused on the relative importance of the various
professions. The designers of the rockets---Minnaert obstinately
referred to them as 'fire arrows'---considered themselves just as
important as the astronomers and demanded more recognition for their
work. It was equally challenging to compare the relative contributions
of the many scholars from the past. Minnaert insisted that the IAU
publications on nomenclature should include the correct phonetic pronunciation
of the chosen names and took it upon himself to ensure this task was
carried out.

Just before the first manned moon mission, Minnaert wrote a summary
of the scientific issues at hand: for instance, it was still uncertain
whether the Moon's surface had been formed by volcanism or by the
impact of meteorites and comets. The bizarre photometric behavior
of the porous lunar rock could be the result of countless small meteorites
hitting the surface at enormous speeds, causing microscopic explosions:
'Who has ever seen a remotely illuminated sphere that was uniformly
bright across its entire surface? This is a sight the Moon shows us
repeatedly, except for the dark spots. Equally remarkable is the unexpectedly
strong increase in brightness we observe as we approach full moon,
especially when only a few hours away from the full phase. - Both
effects are characteristic of extremely porous material, and both
have been replicated in laboratory experiments, where illumination
with a very narrow beam is essential.'

The dark color of the lunar rocks also remained a mystery. Minnaert,
however, believed that the 'solar wind,' a continuous tenuous stream
of protons, penetrated the crystal lattice of the rocks, reduced the
iron ions, and caused strong optical absorption: 'We eagerly await
the first on-site observations: the first rock samples, the first
moonquake research, the first observations of potential volcanism,
the first drilling. Understanding how the Moon is built will help
us comprehend how the planetary system originated and how Earth was
formed.'

At the observatory, he continued his measurements of the reflection
of incoming light on lunar surface simulants and grappled with the
mathematical processing of the data. While Minnaert commuted between
Paris and Moscow, preparations for the moon landing were daily news.
He watched the first moon landing at Truus van Cittert-Eymers' home
because he didn't have a TV on Zuilenstraat. In Flanders, Minnaert
provided commentary in July 1969 and April 1970 on the historical
flights of Apollo 12 and 13. Not long after, VRT dedicated an hour-long
biographical documentary to Minnaert in the renowned series {*}Ten
huize van...{*} by program maker J. Florquin.

\section*{The reprint of {*}De Natuurkunde van 't Vrije Veld{*}}

Publisher Thieme urged Minnaert for a revised edition of his magnum
opus, which was no longer available. Starting in 1966, Minnaert worked
on this project. Of course, it wasn’t just a cosmetic operation. He
added a considerable number of new topics and removed outdated ones.
What was the point of discussing the rhythm of towboats in part II
if they were no longer towed? Out of a total of 770 subjects, he added
a hundred new ones; some old ones were updated. Publications from
the past quarter century were occasionally included, and literary
quotes were polished here and there. For example, under the topic
of river mist, a reference was added to Hubert Lampo's {*}De goden
moeten hun getal hebben{*} (1969):

'As far as his gaze reached, the broad river was covered with fog.
The mist on the stream looked like a dense, neatly rolled mass of
cotton, firm as a layer of snow in the high mountains and seemingly
solid enough to walk safely over it...'

Minnaert chose an excerpt from Walt Whitman's {*}Leaves of Grass{*}
as the introduction to his trilogy, recognizing the American romanticist
as a kindred spirit. It was {*}The Song of the Open Road{*} in M.
Wagenvoort’s translation:
\begin{verse}
'On foot and cheerful I take the open road,

Healthy, free, the world before me,

The endless brown path before me leading wherever I choose to go.

From now on I will not ask for happiness, I have happiness within
myself,

From now on I will not complain, nor postpone anything,

From now on I will feel nothing lacking, It is done with sitting in
the house and complaining, with bookish wisdom, with useless judgments
about others,

Strong and content I go out onto the open road.

I think: all heroic deeds, and all free thoughts have found their
inspiration in the open air,

I think: here I could work wonders

I think: whatever I encounter on the way will be welcome to me and
everyone who sees me will love me.

I think: everyone I see must be happy...
\end{verse}
Before he himself could take the open path, he had many obligations
like this revision. He improved several terms: the green 'ray' became
the green 'glimpse,' a lorgnette lens became a pair of glasses, and
'earth light' became ionosphere light. Minnaert included recent photos
in the reissue of part I, especially from W.C. Livingston, which better
highlighted the optical phenomena.

He had raised questions at the time that were later answered. For
instance, there was an 'unexplained contrast phenomenon' in part I.
An observer described how, on a clear night, he saw the moon 20 degrees
above the horizon from his ship and how its light was reflected by
the waves like a triangle, from the ship to the edge of visibility.
The strange thing was that he also saw such a triangle, inverted and
dark, descending from the moon to the horizon. Minnaert had already
deleted that text when two writers sent him identical observations.
It turned out to be a physiological contrast phenomenon that disappeared
when the bright beam of light was blocked by hand.

In the first edition, he had discussed the bluish haze that can be
beautifully observed on warm summer days against the backdrop of dark
forests. Minnaert had cited the scattering effect of air molecules
as an explanation. His colleague F.C. Went demonstrated in 1960 that
coniferous forests and heathlands emit multiple unsaturated organic
molecules, 'terpenes,' which are oxidized by sunlight and ozone into
macromolecules responsible for the bluish scattering. The story about
air molecules remained valid for the blue atmospheric perspective
over vast distances.

His work for the KNAW committee against superstition could be incorporated
into new passages about 'Flying Saucers' with references to the friendly
spectroscopist and lunar scientist Donald H. Menzel. He provided instructions
for UFO observers but warned: 'Let us not be swept away by fear, war
psychosis, or mysticism, but rather remember how many natural phenomena
we have described in this book that are perfectly ordinary to explain
and that many people nevertheless never noticed.'

Part II was completed by Minnaert in 1969. An interesting detail is
that he did not change his remark about the Dutch coast, which runs
from Boonen (Boulogne, thus in French Flanders) to Groningen! An added
paragraph on {*}The temperature in a tent{*} provided an empirical
result that it is colder inside a tent at night than outside, which,
upon closer inspection, seems plausible. Mistakes were inevitable.
For instance, he removed the outdated 'psychrometer' but later used
it to determine the height of low-hanging clouds. Revising compact,
labor-intensive books with the limited tools available in the 1960s
was no easy task.

For the revision of Part III, Minnaert involved physicist Truus van
Cittert-Eymers, who had retired in 1968 as director of the Utrecht
University Museum. In the 1942 second edition, Minnaert had already
added about twenty topics such as diving, swimming, sailing, skiing,
rowing, and bowling, significantly expanding the 'sports and games'
section. At that time, he had thanked his colleagues A.D. Fokker for
'spinning top' and J.M. Burgers for 'aerodynamics.' Part III would
remain the most intact compared to the other parts. He had to leave
its completion to his friend.

\section*{The final travels and honors}

For his many activities, Minnaert traveled the world: from Moscow
to New York, from Meudon to Hamburg, from Dakar to Varna. In 1965,
1966, and 1967, he utilized his knowledge of Russian while organizing
and leading tours for amateur astronomers in the Soviet Union. In
1968, at the age of 75, he traveled through South America for a month
on the occasion of the festivities marking the opening of the European
Southern Observatory in Chile. Astronomers Oort and Blaauw had both
played prominent roles; Minnaert was involved as chairman of the committee
responsible for providing technical instrumentation. He traveled via
Texas, where he discussed moon nomenclature with his friend Kuiper,
then to Guatemala and Rio de Janeiro, where he visited the Museo de
l'Arte Moderno. Via Buenos Aires and Santiago de Chile, he crossed
the Andes to attend the reception on March 21st. On the 22nd, he visited
the Museum of Fine Arts and the Natural History Museum to participate
in the inauguration on the 25th. On the 28th, he took part in a symposium
in Santiago, was in La Paz on the 30th, visited the Valley of the
Moon and Lake Titicaca, experienced the Good Friday procession in
Cuzco, explored Machu Picchu, Pisac, and the Inca ruins, was at Pachacamac
on April 7th, visited the Archaeological Museum in Mexico City, saw
the frescoes in the Museum of Fine Arts, the pyramids, and the Anthropological
Museum in Chapultepec. He enjoyed it all, as always, with great enthusiasm.

The honors piled up. He was already a member or honorary member of
several scientific societies. In addition, in 1965, he received the
title of Commander from the French Société d'Encouragement pour la
Recherche et l'Invention, an honorary medal from the Free University
of Brussels, and membership in the Deutsche Akademie der Naturforscher
Leopoldinia (1965). In 1966, he was awarded the Prix Janssen with
a gold honorary medal from the Société Astronomique de France. In
1969, he received an invitation for membership in the American Philosophical
Society. Additionally, he received honorary doctorates from the University
of Heidelberg (1965), Lomonosov Moscow State University (1967), and
the University of Nice (1970).

This last honor was granted to him by a French former student who
had become the general secretary of the IAU. When Jean-Claude Pecker
presented the honorary doctorate at the French Embassy, he referred
to the rebellious spirit of both the Flemish Tijl Uilenspiegel and
the Dutch Erasmus, and succinctly summarized Minnaert's significance
for solar physics:

'Minnaert knew how to provide solar research with two fundamental
tools at the right time: spectra that were good enough to be quantitatively
useful and a theory that was sufficiently developed to have quantitative
meaning and serve as an interpretative framework for those spectra.
At that time, there were three men who laid the foundation for the
quantitative analysis of the universe, three men who put the sun in
their test tube: Russell in the United States, Unsöld in Germany,
and Minnaert in Utrecht. Minnaert's work is characterized by its practical
orientation, by the ease with which it can be applied, while still
providing sufficient scientific support. Unsöld was perhaps the most
strictly scientific, Russell the most general; but it was probably
Minnaert who, more than the other two, opened the way toward a fundamental
branch of astrophysics.'

Endnotes:

1 In {*}Between Ivory Tower \& Big Business. Utrecht University 1936-1986{*},
final editor H.W. von der Dunk and his team analyzed recent developments.

2 Van den Broek, 1996, uses the term 'cohort' for a defined part of
a generation.

3 Van der Heiden, A., {*}Turbulence and Reorganization; from 1966
to the present day{*}, in: Von der Dunk, 1986.

4 Snijders, K.-J., {*}The Student Movement{*}, in: Von der Dunk, 1986.

5 Minnaert, front page of {*}Trophonios{*}, October 25, 1966. The
consternation split the editorial board down the middle.

6 Minnaert during the Dies celebration, March 31, 1968.

7 Memory of De Jager. The Prague Spring was in full bloom: the protesters
apparently assumed that Minnaert would be an opponent of it.

8 {*}General Trade Journal{*}, February 14, 1966: seven people signed,
including anthropologist Wim Wertheim, writer Annie Verschoor, and
theologian Hannes de Graaf. Also the future chairman of the Medical
Committee Netherlands Vietnam, Prof. Dr. J.H. de Haas. 9 Molenaar,
1994, 244.

10 Minnaert, {*}On the Responsibility of the Engineer{*}, TU Delft,
March 15, 1966. History Archive.

11 Minnaert would today find an Augean stable, given the 2002 campaign
by university institutions for the purchase of the Joint Strike Fighter.
12 Molenaar, 1994, 221.

13 Minnaert, {*}At the Home of...{*}, J. Florquin, VRT TV script,
June 1970, p. 55.

14 De Jager in {*}De Groene{*}, Minnaert, a brave man, November 5,
1970.

15 Gerard Maas, W\&S, October 1965.

16 Minnaert, {*}Morality and Religion{*}, VRO radio address, March
15, 1968.

17 He had made his first communion at the age of twelve and wrote
about it to L. Buning on May 31, 1969: 'On scientific and philosophical
grounds, I soon came to a complete rejection of all religion, which
I then, and still now, consider a relic of ancient magic that is quickly
dying out.'

18 {*}New Utrecht Daily{*} about the anti-vivisection forum, April
3, 1963. Minnaert spoke in Utrecht on the occasion of 65 years of
the Dutch Association for the Prevention of Vivisection. Interview
with mathematician F. van der Blij.

19 Minnaert, lecture at the IAU Congress on Esperanto and the Language
Barrier, Hamburg, 1964. Archive-History.

20 A new feeling has entered the world/ A loud call is going through
the world/ May it go from place to place on the wings of a favorable
wind.

21 Minnaert to G.F. Makkink, October 15, 1970.

22 Minnaert, Moore-Sitterly, Houtgast, (1966).

23 Minnaert, 1969.

24 S.M. de Roode in Nieuw-Vennep made an inventory of his functions
for the IAU and sent the author a copy on August 30, 2001. In the
Paris archive of the IAU are the minutes of the IAU commissions of
which Minnaert was a part with his correspondence. These will also
largely be found in his own preserved scientific archive; Archive-Astronomy.
See under Notes part III the interview with the Groningen astronomer
A. Blaauw.

25 Minnaert, in: Kuiper, 1961, 213. Van Diggelen, 1962, 161.

26 Kopal, Z., 1969, 364. In Kopal, 1986, the lunar scientist looks
back gratefully at the 'faithful' Utrecht scholars Minnaert, Underhill,
and De Jager and condemns the Leiden-Groningen coup against his journal
Astrophysics and Space Science.

27 Minnaert's functions for the IAU are listed by De Jager in the
obituary he wrote for Minnaert for Kopal's journal Astrophysics and
Space Science. In Brighton, England, where the ill Minnaert was still
appointed chairman of commission 17 on The Moon during the General
Assembly of the IAU, the curious situation arose that Minnaert had
to report to De Jager on lunar nomenclature. De Jager had taken over
the general secretariat of the IAU from the Frenchman Pecker.

28 Contribution by Drs. C. Titulaer in Molenaar, 1998.

29 Minnaert, The Physical Research of the Moon, 1968. Archive-Astronomy.

30 Minnaert for the VRT, April 1970.

31 Minnaert, The Physics of the Open Field 1, 2, and 3, the major
revision of 1967-1971.

32 Walt Whitman, Leaves of Grass, 1855, reprinted in London 1961.
Also for the Ghent graphic artist Frans Masereel, Whitman and Kropotkin
were lifelong sources of inspiration: two parallel lives in art and
science, it seems.

33 Interview with the Groningen astronomer A. Blaauw.

34 Minnaert, Travels.

35 Speech by J.-C. Pecker in The Hague, February 1970.\textquotedbl{}

\chapter{Our hero is Prometheus}
\begin{quote}
'The only thing you can do is reason or intuit in which direction
it should go.'
\end{quote}

\section*{The tree of knowledge of good and evil}

At the end of 1967, Minnaert was hospitalized for several months due
to a bowel operation. He handled his mail from the hospital. He wrote
to his daughter-in-law that he had criticized an article about Stokely
Carmichael, the leader of the Black Panthers, which she had sent him.
He opposed Carmichael's statements about the necessity of armed struggle:
'In my opinion, it depends on the circumstances. It may be that the
natural evolution, the healing forces in society, will make injustice
disappear without having to fight for it.' He himself felt more akin
to the Norwegian economist Gunnar Myrdal, from whom he quoted an article
in Trouw: 'Trouw is the only major newspaper that takes an excellent
position on Vietnam.'

Minnaert remained a combative humanist until the end. Els had told
her children, aged six and four, about the biblical creation story
during visiting hours in front of her father-in-law. He had interrupted
her abruptly, 'not at all pedagogical! It slipped out before I knew
it.' He explained to her why he had to intervene: 'It seems to me
that in some cases it makes sense to tell a child stories and immediately
afterward explain the true facts. But here it strikes me as particularly
unsuitable. First of all, I don't know how you would escape from the
concept of a God, a 'creator'... And then the story about Eve being
created from Adam's rib. What a nonsense! And then the serpent, the
talking serpent! - But the only thing you told me that made me overflow
was the story of 'the tree of knowledge': if you know too much, you
are expelled from that wonderful paradise and punished... I now regret
for another reason that I didn't let you finish. I would have loved
to see you wriggle out of it, especially if I could have asked a few
silly questions myself!! - But, putting all jokes aside, when you
view this entire story with modern eyes, it is truly nonsensical and
even pernicious.'

According to Minnaert, knowledge itself could not cause disaster;
only its misuse or the misuse of science could: \textquotedbl Would
it have been better if humanity had refrained from seeking knowledge?
If we were to tell a story about this, then let it be the story of
Prometheus! He is our hero. And as for those gods, we will surely
bring them down.\textquotedbl{} Els, however, did not succeed. She
could not have chosen a more unfortunate story than the one about
the \textquotedbl tree of knowledge\textquotedbl{} from Genesis,
which indeed struck at the heart of Minnaert's objections to the religions
\textquotedbl of the book.\textquotedbl{}

\section*{Still so much to do...}

Minnaert recovered and would make numerous trips and undertake various
activities in the three years he had left. He felt a sense of urgency
because he still had so much to accomplish while sensing the end approaching.
He became a reporter on the moon landings, began work on an Esperanto
booklet for astronomical terms, worked on part III of {*}De Natuurkunde
van 't Vrije Veld{*} (The Physics of the Open Field), and gathered
notes for his book on free will. In the summer of 1970, his health
condition became critical. Astronomer Kuperus encountered him in July
1970 at the opening of the second radio astronomical observatory in
Westerbork. \textquotedbl I greeted him enthusiastically. Minnaert
had come from Paris and said, 'I am very tired.' Such a statement
was unlike him. We knew he had prostate cancer and underwent surgeries
from time to time. But it never occurred to us that he was terminally
ill and could die. It came as a shock to me!\textquotedbl{}

De Jager recounted that in August of that year, at the IAU congress
in Brighton, he succeeded the Frenchman Pecker as general secretary.
Minnaert had been determined to attend but eventually had to give
up due to his illness. The congress appointed him chairman of the
commission responsible for finalizing the lunar nomenclature: \textquotedbl At
the time, this was such a politically charged issue that they wanted
the best man for the job. I attended the deliberations and heard it
firsthand. I was asked if Minnaert would accept. I answered affirmatively.
Indeed, Minnaert accepted and was quite delighted about it. From the
hospital, he sent the letters.

People came by for what would be their farewell visit, such as the
Swiss Edith Müller, a young astronomer who had worked in Utrecht and
with whom he had become friends. Boudewijn came from New Guinea in
September when his father was still well, 'so I could talk a lot with
him while he was fully conscious.' He never complained. He could say
things like 'this isn't going so well' or 'that's costing me a lot
of effort,' but it was the observation of an outsider. He distanced
himself from himself and looked at himself from outside. Kees de Jager
and his wife Doeti came by regularly: 'When he felt that he was going
to die, he asked me to continue the Books for Hanoi campaign. There
were a few hundred regular contributors from all over the Netherlands.
The administration and correspondence: he had handled everything on
his own.' I remember him during our last visit, a few days before
his death. He was sitting at home in a chair. To the left, a tall
stack of corrected proofs of 'De Natuurkunde van 't Vrije Veld,' to
the right, a new stack. He was very open and enjoyed that we came
by. Conversations with him always revolved around important things,
never about the weather. He was working on the phonetic spelling of
the names of the scholars after whom the moon's features were named.
I tried hard to get a Hungarian name phonetically correct. After half
an hour, we left: he was 'very tired indeed.'

\section*{The last weeks}

Miep took care of him at home. Their mutual friend Truus van Cittert-Eymers
was often present and discussed the revision of part III with Minnaert.
His daughter-in-law Els visited him during those last weeks at home
and in the hospital. She wrote down her experiences for her brother-in-law
and sister-in-law in Australian New Guinea and drew strength from
it herself. Several passages from her letter give a good picture of
Minnaert, who had to tear himself away from life.

Shortly after Boudewijn's visit, the bleeding had started, and he
had experienced a lot of pain in his lower abdomen: 'Perhaps you could
say that he bore it all with dignity. He didn’t want to die at all,
he wasn’t tired of life, and he worked until the very end. I once
felt ashamed sitting next to him because he still had such great expectations
for the future, while I, being almost 40 years younger and healthy,
view the future of humanity as rather gloomy. He was so thrilled about
the news that in Chile, the left-wing socialist Allende had received
the most votes for the presidency. I feared another American intervention
or military coup, which they are so prone to in South America, but
he saw a golden future unfolding for the Chileans. I’m now so glad
that Allende did become president after all. Father said, 'I even
shook Frey’s hand (the previous president).'

Around that time, he also became a bit eccentric for his standards.
He put eau de cologne on a handkerchief and wiped it across his face.
He even laughed at himself for doing that. He once told me that when
his own father died---he was about ten years old---his mother thought
it necessary to initiate him into the mysteries of finance, investments,
and bonds. But every time she started talking about it, he would automatically
fall asleep; he found it so boring. Mother was very devoted to Father
throughout his illness. She looked absolutely terrible during the
last days of August when things were critical with Father---pale
and with sunken eyes. Lately, she’s been looking better. But I’ve
never seen them kiss or even shake hands, not even in recent times,
and I totally can’t understand that.'

Minnaert had fully recovered by late September: 'He even got out of
bed and occasionally sat down to work at the little table in his room.
Then there was a period of beautiful autumn weather. A sister had
suggested that he might go into the garden. I wasn't allowed to give
him an arm because he wanted to do everything on his own strength.
It was a terrifying walk. At first, I was afraid he would fall, he
walked so unsteadily. Still, we went down the stairs arm in arm because
otherwise the risk of falling would have been too great. Once in the
garden, he incredibly enjoyed the large old trees in the autumn light.
Also the grass, 'Do you smell that grass?' he said. He stroked the
flowers. He had never enjoyed nature so much in his life. 'I never
had time for it,' he said. He soon sought a bench in the shade, but
sitting there wasn't easy either. After five minutes, he sat forward
or hung diagonally along the armrest (I thought for a moment he wanted
to lie down) because he could only maintain any position for a short
time.

Els had visited on a Saturday morning with the children: 'We did go
back up with the elevator. He held the door open for me: he wanted
to be gallant again, just like before.' He also went to the barber.
That was much needed because his white hair was all around him. He
described how he had sat down in the chair in the room and how then
the white hairs swirled around him like snowflakes. Then he had his
head washed thoroughly with some lotion. After that treatment, he
looked as if he had started a new life.

Minnaert felt reborn and wrote optimistically to acquaintances: 'Chance
has it that there's a real chance of full recovery.' He was allowed
to go home at the end of September. Els had visited on Zuilenstraat,
on a Monday, the week before his death: 'Father still discussed a
gift. He had a whole list of suggestions just like before.' We used
the autumn vacation so that Father could give it himself. The visit
lasted sixty minutes and that was ten too many because he couldn't
take it anymore.\textquotedbl{}

\textquotedbl He saw white in the distance and sat with a gruesomely
unnatural grin, trying to keep himself together. He said: 'So, you’re
having your birthday, congratulations.' Outside, I tried to wave at
him through the window (from the Nieuwe Gracht, I had seen him sitting
in that chair by the window) but then I only saw a small patch of
white hair, from which I gathered that he was sitting with his head
in his hands, extremely tired.

That Thursday, Minnaert was hospitalized again. Els visited him on
Sunday morning: 'I held his hand for a while, hoping he would notice
I was there, but at one point he pulled away and waved his hand back
and forth; that familiar impatient gesture. A little later, he did
realize someone was there, lifted his head, and stared through the
mist in my direction. Then he faintly pressed my hand, and that was
the last time. I stood crying by his bed without him noticing.'

\section*{His body for science}

In the night of Sunday to Monday, October 26, 1970, around two o’clock,
Marcel Minnaert passed away. No family members were present at the
time. He had made his body available to science. Apparently, he, whether
in consultation with his wife or not, had not wanted a farewell gathering.
Miep Coelingh called Els around 7:15 that Monday morning to inform
her of his passing. At that moment, Miep also mentioned that her father-in-law
would not be buried or cremated.

Els wrote to Boudewijn that she still couldn’t understand why he hadn’t
told her about it: 'You might understand that it’s even harder for
children to accept something like this, and besides, I didn’t have
a proper story ready. Only now, after a week, am I starting to come
up with one. Earlier, my youngest had once said: ‘When Grandpa dies,
I want to walk in the procession with the other mournful people.’
Not knowing any better, I said then: ‘I think it’s fine, but we’ll
have to discuss that first with Grandma.’ Now, I would have found
it very comforting if Father had also taken us (and especially the
children) into account in this regard. We (not just me) are left with
the unsatisfying feeling that there was no closure, no gathering of
people who had been close to him.'\textquotedbl{}

From Port Moresby, Boudewijn and Noortje thanked their sister-in-law
for the detailed letter, which they had read with mixed emotions.
Boudewijn wrote: 'Mother also found it terribly sad that Father would
not be buried. I didn't know that was Father's wish until Mother told
me when I was in Holland. But it is so much in line with Father's
entire philosophy and his view of life that I believe you must respect
his wishes in this regard. He was a man of science, and it was so
deeply ingrained in him that everything he believed would advance
science had to be done.

It’s a shame you didn’t know about it; I hope you’ve managed to give
the children some kind of 'explanation' in the meantime.' That Minnaert
would make his body available was completely understandable. It by
no means implied that there wouldn’t be a gathering. His wife took
no initiative whatsoever. Minnaert’s colleagues, none of whom were
in contact with Miep Coelingh, had waited for the family to take the
lead. When it dawned on them that there would be no farewell ceremony,
the right moment for taking their own initiative had already passed.
Minnaert had been so much in control of everything during his life
that upon his death, no one seemed prepared to make decisions.

When Pannekoek had died ten years earlier, Minnaert had delivered
a well-considered eulogy about the scientific work and social involvement
of his 'great teacher and dear friend.' At the end, he had said: 'A
magnificent, long, and fruitful life has passed. Is it truly gone?
Of course not! Pannekoek lives on---in his work, but also in the
inspiration, fighting spirit, and happiness he shared with those around
him, which will be passed on to others. Farewell, Pannekoek, we thank
you.'

How dearly would De Jager have loved to say the same at Minnaert’s
funeral!

During those weeks, he wrote more than a dozen loving obituaries for
national and international professional journals, academies, and magazines,
but that was just on paper. For many people, Minnaert had been a beloved
figure. The fact that they couldn’t participate in a joint ceremony
was painful for many, even traumatic for some. Minnaert wouldn’t have
understood that, even though he had written to youngest son from Michielsgestel
how optimistic and inspiring a funeral of an elderly person should
be. For himself, he had waived every farewell: he had simply left.
It may indicate his modesty and willingness to sacrifice, but even
more painfully clear was that he could not and did not want to put
himself in the position of loved ones, friends, and acquaintances.

The following year, the Dutch Astronomers Conference organized a stylish
memorial service, where De Jager could bid farewell on behalf of the
astronomical community to their leader. On the other side of the ocean,
the New York Academy of Sciences and The American Astronomical Society
organized a 10 Minnaert Memorial Conference that summer of 1971. Minnaert's
Swiss friend Edith Müller, now a prominent astronomer within the IAU,
delivered the memorial speech in New York.

\section*{The death of Minnaert in the media: anima pia}

In Utrecht, on October 28, 1970, the university flag was lowered to
half-staff. Astronomer Jacob Houtgast wrote in NRC about A versatile
astronomer: 'We heard from intimate friends about the death of Prof.
Dr. M.G.J. Minnaert - a man of great style, a great scholar, who during
his life was someone from whom everyone he met (and there were countless)
could derive much joy.' Even someone like Houtgast, a close colleague,
was thus indirectly informed of Minnaert's death. Houtgast emphasized
the discovery of the growth curves, 'still powerful tools in the study
of stellar atmospheres.' According to him, Minnaert had photographically
recorded the facade stones of Utrecht's inner city in the 1930s.

Journalist Gerton van Wageningen chose the headline Prof Minnaert:
fighter for popularizing astronomy in Nieuw Utrechts Dagblad and Het
Parool. Minnaert had been a committee member during his final exams
'and from that day on, I remembered him.' Later, he met him at the
Observatory, always busy yet willing to receive you. This Minnaert
from Bruges, who formulated with the meticulousness of the intelligent
Southern Netherlander was an intensely moved man: 'Minnaert may have
been a rebel in some sense, but it cannot be otherwise---he must
have been driven by an exact scientific pursuit of justice and balance.
Essentially, he was what the Romans called a devout soul: {*}anima
pia.{*}' The {*}Utrechts Nieuwsblad{*} extensively quoted Minnaert
on the responsibility of the scientist, which aligned with 'the spirit
of the times.'

In {*}De Tijd{*}, A.J.M. Wanders wrote 'Prof. Minnaert: a restless,
world-renowned astronomer'. Minnaert had shown heartfelt interest
in everyone's fate: 'Even more uncountable, however, are the crowds
of merely interested individuals whom he managed to inspire with the
wonders of the starry world through his captivating speaking talent
and clear expository skills. \textquotedbl No matter how complicated
a thing is, it can be made clear in five minutes,\textquotedbl{} was
one of his sayings.' Wanders recalled the decline of astronomy as
a subject: 'Apart from its many other merits, this unique field of
study represented the end of the possibility to offer upcoming natural
scientists meaningful counterbalance against an increasingly suffocating
\textquotedbl specialization madness\textquotedbl{} observed everywhere.'
Hopefully, more enlightened minds would undo this blunder one day:
'Although the passionate advocate Minnaert himself would no longer
witness it. This restlessly active man, in whom there was no injustice,
now rests in peace.'

De Jager called him in his article for the Royal Flemish Academy one
of the greatest Dutch astronomers of the century, a man of world renown,
known and beloved by many, a true global citizen. In {*}De Groene{*},
he wrote that Minnaert, through {*}Boeken voor Hanoi{*}, had tried
to give something back to the Vietnamese people: 'Minnaert's pronounced
ideas, due to their straightforwardness, had a great influence on
his students and colleagues. He never polemicized for its own sake
or tried to belittle an opponent but always sought the one truth that,
he was convinced, should apply to every problem. Those who disagreed
with him politically or intellectually he disarmed more through his
charm than provoked, and yet they always respected the honesty of
his convictions and arguments. He himself, who never discriminated
based on political or philosophical beliefs, could be astonished and
was more saddened than indignant when such discrimination affected
him.' The solar physicist Unsöld wrote in Solar Physics: 'There will
be no gravestone, but his personality, so kind and versatile, will
live on in the thoughts of many of his friends around the world; his
scientific work will remain in the annals of scientific research.'

The Flemish magazine Het Pennoen noted that 'Minnaert's indestructible
idealism was accompanied by an ascetic trait. This gave him the strength
to commit his entire personality to the task he had accepted as a
calling.' The magazine called him a 'restless seeker of order in nature,
who also strove for order in human relations. His attitude was not
that of a revolutionary or anarchist. He was a humble worker who,
without longing for personal gain, honor, or fame, conscientiously
contributed to the evolution of humanity.' A few months before his
death, the VRT had given Minnaert an hour-long interview in his armchair
on Zuilenstraat: this would become Flanders' farewell tribute.

There was no discord: nil nisi bene about the dead. Nothing but good
about Minnaert.

\section*{Snapshot 1970: Man of the Cosmos}

After the war, Minnaert frees himself from the constraints of his
upbringing. Gradually, he develops a meaningful, albeit functional
involvement with others. In the scientific field, he compiles works
that help young people orient themselves to the current state of affairs.
He continues to formulate projects that he pursues after many years
of diligent work. His work on the Fraunhofer lines, resulting in the
new Table, will take another twenty-five years; his didactic activities
for physics education permanently occupy him; his advocacy for the
school subject 'astronomy' lasts a quarter of a century; and his Stevin
project implies 'great labor' over twenty years. He identifies developments
in his field, in didactics, and at the university that will be important
in the future. He develops a sense of history, albeit mainly limited
to the history of natural science.

Love and dedication to youth and to science, from elementary school
to university, characterize him throughout his life. Hence his unwavering
commitment to didactics. Odes to youth form a constant in his life.
He remains unwaveringly loyal to ideals and people. He refutes the
statement: 'Who is not red at 20 has no heart; who is still red at
60 is a fool.' For him, it is rather the opposite. In nature, he discovers
solidarity early on through Kropotkin. He struggles with the relationship
between his Flemish nationalism and internationalist socialism. He
resolves this struggle with his fist in Lage Vuursche. From that moment
on, the social aspect begins to play a greater role in his thinking.
He discovers the beauty of socialist ideals. Whereas he was initially
purely an astrophysicist and educator, he gradually wants to give
the social dimension of his profession a full place.

He connects seemingly separate domains. During the war, philosophy
also enters his field of vision. When, as a natural scientist, he
rejects 'free will,' what then can be the basis for conscience, for
societal values and morality? Minnaert accepts, on one hand, the 'truth'
of scientific views on brain functions and, on the other, Freud's
psycho-sociological views on conscience. His 'solution' shows great
consistency, even though he proclaims a curious 'continuity thesis'
due to a misunderstanding of dialectics. He demands a place for the
intellectual core of evolutionary astronomy in pre-university education:
The Unity of the Universe---from the Big Bang to human society.

Minnaert does not uphold double standards. He loves animals, opposes
vivisection, and therefore does not eat meat. He calls on students
to take responsibility and initiates 'Books for Hanoi.' He preaches
tolerance and peace and tries to realize them in his microcosm. He
is against air and water pollution and therefore travels by bike and
train. He supports women's emancipation and holds his wife in honor.Koen's
death shocks him and he accepts responsibility for his grandchildren.
The intensity of his views on the 'green time' seems rooted in his
youth. He is averse to deliberate humiliation and abhors institutional
violence that breeds victims who later become perpetrators. He is
not the Messiah: he exhibits strong egocentric traits, as evident
in his behavior following Koen's death and on the eve of his own passing.

He becomes a more integrated personality. When Minnaert emerges as
a 'bridge builder' in the late 1960s, something new is happening.
He could not have been that in 1914, nor was he in 1936 or 1957. At
those times, he was convinced of his own righteousness, whether it
concerned Flanders or the Soviet Union. Toward the end of his life,
he becomes wiser and able to view a cause from multiple perspectives
while still maintaining his own opinion---a ability his father had
highly praised in his Farewell Letter. Perhaps Koen's death had softened
him. In 1968, he has much life experience to offer his colleagues
and the rebellious students. The rebellious generation of the late
1960s can rely on people like Minnaert to protect them from turning
their absolute beliefs into violence and terrorism.

Acquaintances often say that Minnaert is gentle and vulnerable, kind
and helpful, that he honors his name, even if they also consider him
paternalistic and patriarchal. Those closer to him experience that
he keeps his distance and never reveals himself, that he is fanatic
and can be hard as nails. The socialization of his early youth leaves
Minnaert underdeveloped emotionally. Initially, science and culture
are his anchors; for the middle-aged Minnaert, the social dimension
is added. His rebellious stance persists, as seen in his obstinate
preference for the youth, in his struggle for Flanders, for the Soviet
Union, and for Vietnam, but he gradually learns to channel it into
activities accepted by his surroundings. His love for nature and the
happiness he finds within himself during solitary travels help him
maintain mental stability. His father writes that the Minnaerts are
a 'hot-tempered lineage,' his son has learned to deal with it over
time. Minnaert sees it as his social duty to give many people a lot
of attention. Those who meet him usually experience him as a striking
and amiable figure---nomen est omen. His parents apparently did not
give him the life motto in vain:
\begin{quote}
'I am Minnaer of pure joy,

of beauty and truth, art and virtue.'
\end{quote}
The circle is complete.

\section*{Minnaert's Ideals}

The TV interviewer had finally asked him during the summer of 1970:
'What is your ideal vision of the world?'

His response: 'The only thing you can do is reason or intuit the direction
it should take. It is clear that the great principles must be that
all children have equal rights to education and development. That
men and women must have equal rights: women must play a much greater
role in societal life. Of course, no racial discrimination, no colonial
exploitation of a country. The preservation of the environment is
of great importance. We must resist the pollution of air and water,
now also of the seas. The means of production must be in the hands
of the community. And most urgent: the disappearance of weapons, peace
in the world. Society should not be based on competition but on cooperation,
on mutual human love.'

In brief, Minnaert presented a coherent vision for the future that
can still guide human lives today. This crumb on the robe of the universe
disappeared back into the cosmos. His ideas and actions live on.\\

Endnotes:

1 Minnaert to Els Hondius, December 11, 1967. Hondius Archive.

2 Interview with the astronomer M. Kuperus.

3 Boudewijn Minnaert to Els Minnaert-Hondius, December 5, 1970.

4 De Jager recounted: 'After his death, I met an American named Menzel
who had taken over the work on the nomenclature. He said: 'What nonsense,
there are even people who call me Mentsel.' I told him that Menzel
is a German name and you just pronounce it that way. 'Really?' he
asked. That's how things go.'

5 Els Minnaert-Hondius to Bou Minnaert, October 30, 1970.

6 Minnaert to Buning, September 13, 1970. Something similar he wrote
to Esperantist G.F. Makkink, who on September 17 heard that he was
eager to resume work on the astronomical vocabulary list.

7 Boudewijn Minnaert to Els Minnaert-Hondius, December 5, 1970.

8 Eulogy for Pannekoek, who passed away on April 28, 1960, at the
age of 87. History Archive. 9 De Jager (1970), several obituaries
in the literature list. In professional journals also reflections
by the Irishman E. Öpik (Irish Astronomical Journal), the Frenchman
Pecker (Icarus), his Swedish friend Arne Wallenquist (Astronomisk
Tidskrift), J. Ashbrook (Sky and Telescope), and others.

10 On August 30, 31, and September 1, 1971, in New York.

11 Houtgast, J., A versatile astronomer, NRC, November 2, 1970.

12 Gerton van Wageningen, Fighter for popularizing astronomy, NUD
and Het Parool of October 29, 1970.

13 Wanders, A.J.M., Restless world-renowned astronomer, De Tijd of
November 5, 1970.

14 Unsöld, A., In Memoriam, in: Solar Physics, 1971, 3.

15 Het Pennoen, L. Buning, February 1971.

16 An issue of Hypothese, the magazine of the Dutch Organization for
Scientific Research (NWO), was dedicated to Free Will in 2001. None
of the scientists interviewed could provide a commentary that came
close to the consistency of Minnaert's solution.

17 Molenaar, 1994, dissertation statement.

18 Minnaert, TV script, page 56.

\chapter*{Epilogue}

\section*{The Family}

For Miep, her work in the Peace Council was over. A few years after
Marcel’s death, Miep Minnaert-Coelingh had herself registered as a
member by Barend Schreuders, secretary of district Utrecht of the
CPN. Through the CPN, she became involved in the initiative 'Stop
the Neutron Bomb.' She opened and closed the International Forum against
the neutron bomb on March 18, 1978, with speeches. She proved to be
a communist of the old stamp: 'These days, so much is said about human
rights. And naturally, we are all in favor of human rights, without
question, there can be no doubt about that. It is part of our entire
cultural pattern, our traditions, our ideas of freedom and democracy,
but we must beware that the talk about human rights does not become
a prelude to the start of a new Cold War, and sometimes I do feel
that way.'

After her husband’s death, she reduced contact with her daughter-in-law
and grandchildren. Her grandchild Peter Kruiper organized the celebration
of her eightieth birthday in 1986: she too received a Liber Amicorum.
Friends from the CPN and some peace organizations directly linked
their contributions to the activities of the 1950s and the movement
against neutron bombs and cruise missiles: their efforts had prepared
minds! Her comrade Theun de Vries wrote the poem 'Begroet de mens'
for Miep Minnaert, in which he sang about people who, like Miep (and
Marcel), consistently turned against the barbarism of armament:
\begin{verse}
'Steep is the path and long the patience

For those who want to break the malice,

Banish the servitude of the earth.

But the music has sounded,

The word peace has been spoken,

The kingdom of freedom, so often destroyed,

Is always re-established anew.

Greet de mens. Greet The fighter.

Greet His infallible future.'
\end{verse}
Miep Minnaert-Coelingh continued to live on Zuilenstraat until she
needed care. After a short stay in a nursing home, she was able to
return when a caregiver arranged a solid schedule for a group of volunteers,
feminists, and party comrades. She passed away on 13 July 1990 on
the first floor of Zuilenstraat 25bis, at the age of 84. Marcus Bakker
described her as a proud, upright, and serious woman.

Boudewijn and Noortje regularly visited the Netherlands: in the early
1970s, Miep even visited them in Australian New Guinea. Noortje contracted
a muscle disease from which she died within a short period in 1997.
'Bou' remained actively involved from Sydney in the biography, which
he considers a tribute to his father.

Els Hondius was initially fully occupied with raising the children.
Later, she started the publishing house De Els. She wholeheartedly
contributed to the biography. Paul Minnaert is married and has a daughter.

\section*{His scientific work}

Obviously, the International Astronomical Union's honored her leader
Minnaert with a crater on the back side of the moon and an asteroid
somewhere between Mars and Jupiter. Nevertheless, the memory of Minnaert
faded relatively quickly in the Netherlands. This can be understood:
due to the scaling up in astronomy, the work of the pioneers was overshadowed.
The spectacular rise of radio astronomy and telescopic, X-ray, and
space research, as well as satellites and probes, caused excitement
that made looking back at the pioneers of solar physics seem antiquated.
Some of his articles are still being consulted, such as the one on
the reciprocity principle in lunar photometry.

Outside of astronomy, Minnaert was often referenced, especially in
the English-speaking world. The American W.C. Livingston, who had
been involved by Minnaert in his revision of Licht en kleur in het
landschap, drew inspiration from it for his 1995 book Color and Light
in Nature, which he dedicated to Minnaert. In 1999, the Optical Society
of America celebrated the fact that they had organized conferences
on Licht en Kleur in het Landschap for twenty years with the event
On Minnaert's shoulders. Students from the Utrecht teacher training
college Fontys named their website De Natuurkunde van 't Vrije Veld.

In Utrecht itself, the Astronomical Practicum was abolished in the
1980s as a result of the reduction of the study program to four years
at the time. The Astronomical Institute moved in 1987 to the top floor
of the Buys-Ballot Laboratory: the Sonnenborgh Observatory became
the domain of amateurs, and Minnaert's spectrograph is now part of
the museum setup. However, Minnaert's 1969 book on Practical Astronomy
gave educators from other institutions an idea. The Czech J. Kleczek
published a new edition in 1987, again with Reidel in Dordrecht. Of
Minnaert's 74 exercises, 53 remained, and twenty astronomers, including
Kleczek, added twenty new ones. Minnaert's book, in this form, met
a worldwide need.

The working group on physics education from the Working Community
for Educational Renewal also did not forget him and established the
biennial Minnaert Prize. Chairman Th. Wubbels motivated the prize:
'Professor Minnaert was one of the co-founders of the working group
in the 1950s. We thought it better for the name of the working group
that he himself would not be the chairman. In practice, however, he
was the great motivating force behind the working group. By linking
Minnaert's name to a prize, we want to honor his merits for the working
group and the development of physics education.' For the uninitiated,
the beginning of this motivation must have been cryptic.

In the wake of didacticians and physicists, Utrecht astronomers finally
agreed with the proposal to name the new building for the didactics
of natural sciences and theoretical physics after Minnaert. Since
its opening in 1998, Minnaert lives on in this popular building by
architect Neutelings. Also, the Dutch research school of astronomy
(NOVA), the collaboration organization of the astronomical faculties
of Utrecht, Leiden, Amsterdam, and Groningen, has had a Minnaert Committee
since 2000. And: triumph! The subject of astronomy finally made its
entrance in the late 1990s in secondary education as a mandatory part
of the new subject General Natural Sciences.

Minnaert could and would not complain about the extent to which his
ideas and actions, at least in the first decades after his death,
have endured.

Endnotes:

1 Liber Amicorum for Miep Minnaert-Coelingh.

2 Bulletin Stop the N-bomb, March 1978.

3 Theun de Vries, Liber Amicorum. 1986.

4 The Science Citation Index (SCI) provides clarity on this matter.

5 Lynch and Livingston, 1995.

6 Kleczek, 1987.

7 Wubbels, Woudschoten Conference 1987.

8 Neutelings Riedijk, 1998.

\chapter*{Appendix}

\section*{Anomalous dispersion, Julius' sun theory, and Minnaert's mission.}

W.H. Julius (1860-1925) became a professor of experimental physics
in Utrecht in 1896. He was an experimental physicist specializing
in precise measurements of heat radiation, for which he had designed
the instruments himself. According to Einstein, he was 'converted'
in the 1890s after reading the book by the German teacher Carl August
Schmidt about irregular light bending in the sun. To explain the fact
that the sun appears as a sharply defined disk in the sky, Schmidt
had suggested that outgoing light rays are refracted back toward the
center due to variable gas density. This argument is accompanied by
a suggestive illustration.

\textbackslash figcaption The bending of light rays in different
layers of the sun, according to A. Schmidt.

Also on Earth, refraction in the atmosphere is a known phenomenon:
for example, a star observed in the night sky is actually lower. Since
this already applies on Earth due to small differences in density
between atmospheric layers, it must surely apply even more strongly
to the sun. Schmidt, as an amateur astronomer, had been astonished
that no solar theorist took this into account.

\textbackslash figcaption Due to refraction, the star S is observed
higher (W) than it actually stands in the sky.

Julius therefore began investigating the refractive power of various
gases in his laboratory. Unlike Schmidt, he also assumed that the
density in the sun's atmosphere would not decrease uniformly. In his
research, he soon encountered curious refraction phenomena already
known at the time, referred to as anomalous dispersion.

\section*{Dispersion and Anomalous Dispersion}

\textbackslash figcaption 'Refraction toward the normal'

When light of a specific wavelength falls on a piece of uncolored
glass under a certain angle of incidence \textbackslash (i\textbackslash ),
it will be refracted at a smaller angle \textbackslash (r\textbackslash ).
According to optics, light can be considered as a rayin a straight
line within a medium. The law of the Dutchman Snellius provides the
relationship between the two angles: \textbackslash (\textbackslash sin
i / \textbackslash sin r = n\textbackslash ). This \textbackslash (n\textbackslash )
is called the refractive index. Since \textbackslash (\textbackslash sin
i > \textbackslash sin r\textbackslash ) for this transition, \textbackslash (n
> 1\textbackslash ).

Newton discovered that a beam of sunlight falling on glass is also
decomposed into colors. Each color has a different frequency and wavelength
and a different refractive index in glass. The colors that differ
most in wavelength are red, with the smallest frequency, and violet,
with the highest frequency. The violet ray is refracted more strongly
than the red one: for violet, \textbackslash (\textbackslash sin
i / \textbackslash sin r\textbackslash ) is therefore greater than
for red. This phenomenon, where the refractive index \textbackslash (n\textbackslash )
increases with increasing frequency (and thus decreasing wavelength),
is called dispersion.

\textbackslash figcaption The refractive index \textbackslash (n\textbackslash ),
plotted vertically against the wavelength (from violet to red), shows
a decreasing \textbackslash (n\textbackslash ) as the wavelength
increases (and the frequency decreases). The dispersion curve illustrates
this relationship.

Julius soon became familiar with the work of the Frenchman A.H. Becquerel
on remarkable refraction phenomena. In the spectrum of a beam of sunlight
passing through a colored transparent substance, one or more spectral
bands or lines are always missing. Sometimes even an inversion of
the spectrum occurs, as in experiments with violet crown glass. The
colored substance in the medium 'absorbs' certain colors of the incident
light. Light that has a frequency approaching an absorption line deviates
strongly.

\textbackslash figcaption Deviation in the refractive index near
the sodium absorption lines.

From the side of longer wavelengths, the refractive index takes on
increasingly positive values, and from the side of shorter wavelengths,
increasingly negative values (Kundt's Rule).

This phenomenon is called anomalous dispersion. The term 'anomalous'
refers to the Greek ómalos, meaning smooth or even, and thus signifies
'uneven' or 'irregular.' Forty years later, August Kundt first observed
anomalous dispersion in gases: he examined the spectrum of radiating
sodium vapor and discovered that the refractive index changed strongly
near the prominent double line of sodium. Julius repeated Kundt's
experiments and determined the refractive index of the sodium vapor
up to just before this 'absorption line.' He was able to confirm that,
similar to Becquerel's experiments, the refractive index increased
strongly on the long-wavelength 'red' side and decreased strongly
on the short-wavelength, 'violet' side. From this, Julius concluded
that the width of a Fraunhofer line is almost exclusively caused by
light that is anomalously 'broken away' at the edges of the actual
central line. Only an extremely narrow core formed the 'actual' absorption
line: the widths of the lines observed on photographic plates were
primarily due to this anomalous dispersion.

\section*{Verification of Julius' sun theory}

Julius distinguished two cases of anomalous dispersion near the Fraunhofer
lines. Anomalous refraction occurs when the narrow part of a light
beam around a specific frequency entirely changes direction. Additionally,
light of these frequencies, which have either a much larger or much
smaller refractive index, will experience greater anomalous scattering
according to Rayleigh's law. Julius clung to anomalous refraction
for opportunistic reasons, as the resulting beams of a certain wavelength
seemed to have the greatest explanatory power for the physics of the
sun.

Astronomers, for instance, thought they saw protuberances---gas eruptions
on the sun moving at unimaginable speeds---while according to Julius,
it was anomalously refracted light from parts of the solar disk causing
something like ripples in the 'solar sea': 'If a broad wave rolls
onto the beach and breaks first here, then there, then over there,
no one would speak of the \textquotedbl speed\textquotedbl{} at which
the foam propagates along the coast.' Astronomers believed they saw
emission lines in the chromosphere, while that too could be anomalously
refracted light. They detected fllocks or 'flocculi' on the solar
surface, while it could also involve vortices and density differences
spreading like the foam of the aforementioned waves. Julius actually
had a solution for all solar problem

Julius published the first version of his theory in 1900. It turned
out to be more appealing to physicists than to astronomers. The idea
that theories with great explanatory power were superior was more
valid than ever at the time. Skepticism grew when it became clear
that Julius' suspicions could not be quantified. His ideas about irregular
beam curvature, for example, were based on the intuition that the
photosphere of the sun would certainly exhibit large density differences.
No one knew if that was true. M

oreover, Julius formulated his ideas so ambiguously that no experiment
could refute his claims. The conclusion of his argument read: 'The
above may have been sufficient to clarify that in constructing a solar
theory, it is necessary to consider not only regular refraction but
also anomalous dispersal.' Therefore, Julius could agreed to any percentage
of his truth from 1 to 99\%. If others argued that the chromosphere
also involved glowing, strongly radiating gases, Julius' defense was
that he had not ruled this out. As the experiment turned out, Julius'
theory could be 'adjusted,' was the impression around 1910.

Another disadvantage of Julius' theory was its monomania. It was an
optical hypothesis that hardly connected to contemporary discoveries
in ionization, electromagnetism, the Zeeman effect, quantum hypothesis,
and gravity. The astronomers of the 1910s focused on the order of
the day. Julius therefore sharpened his position again polemically:
'When it comes to sunspots, torches, flocculi, protuberances, one
speaks as if they are 'objects' whose viewers show us geometrically
projected 'images.' It is thus assumed that the curvature of rays
in the sun is too insignificant to significantly alter the phenomena.'
He almost condescendingly expressed himself about astrophysicists
who dared not leave the beaten paths: 'At first glance, it seems as
if one would thereby lose grip on solar phenomena, as if all our notions
would dissolve into vagueness. But the step must now be taken: and
we want to try to show that this does not lead to vagueness, but rather
to sharp notions that clearly reveal the coherence of the main solar
phenomena.'

Such sentences formed the prelude to Minnaert's eulogy for Julius
in 1921.

\section*{The necessity of a Heliofysical Institute}

When Julius visited his American colleague Hale in Pasadena in 1907,
he naturally inspected Hale's phenomenal solar setup on Mount Wilson.
He took data with him for building a scaled-down version in Utrecht.
After all, Julius had been dependent on foreign experiments. When
Julius wrote in 1909 that he doubted whether every aspect of solar
physics could be verified using earthly experiments, most astrophysicists
permanently turned away from him. Julius was director of the Physical
Laboratory and decided to establish a Heliofysical Institute, where
he could conduct research with the help of his own spectrograph.

Julius felt compelled to come up with verifiable predictions. According
to optical formulas, the effect of anomalous refraction on the long-wave
side of the Fraunhofer line was expected to be relatively large due
to the strong increase in refractive index. On the short-wave side
of the line, the refractive index had decreased significantly. Therefore,
there had to be a pronounced difference, an asymmetry, between the
long-wave side and the short-wave side on either side of the center
of the Fraunhofer line profile; in other words, more light should
disappear on the 'red' side compared to the 'violet' side of the unchanged
center. For Julius, this marked the beginning of a quest for line
profiles showing an apparent 'redshift' of the wings relative to an
unchanging center.

Using the spectrograph, he could create photographic plates himself.
The conversion of Fraunhofer lines into line profiles had to be extremely
precise. Measurements of the exact surfaces of the two sides of the
line profile would have to provide definitive results. For this, the
new photometric equipment developed by Julius' colleague W.J.H. Moll
was required. This microphotometer was completed in late 1918.

A redshift of solar lines relative to terrestrial lines, Julius noted
in advance, had already been discovered by the American astronomer
Jewett in 1896; however, this systematic deviation amounted to a few
thousandths of an Ångström. Rowland's wavelength scale, which recorded
the centers of the Fraunhofer lines, was accurate to within 0.01 Ångströms.
Predicting, detecting, and verifying such 'redshifts' therefore occurred
at the limits of measurability. At this point, measurements become
susceptible to the psychology of the researchers---and, to an amplified
extent, that of their assistants. Moreover, did Jewett's results involve
an apparent shift around or an actual shift of the line center? This,
it turned out, made a great deal of difference. In 1911, it suddenly
became clear to Julius that he was not the only one on the lookout
for a form of redshift.

\section*{Julius and Einstein: Two Claims on Redshift}

In 1911, he unexpectedly became engaged in a discussion with Einstein,
whom he had to sound out for the vacant Utrecht chair in theoretical
physics. For six months, every letter between the two concerned Julius'
solar theory. Einstein, to Julius' surprise, also claimed a minimal
redshift. However, as a consequence of his gravity theory, Einstein
predicted an actual redshift of the line centers, not like Julius,
who suggested an apparent asymmetric shift relative to a stable core.
After two months, Einstein pointed out that Julius' alleged asymmetry
of the Fraunhofer lines could never be the result of anomalous scattering
because, due to the square of (n - 1) in the numerator of Rayleigh's
law, it was symmetric with respect to the center of the line. Only
anomalous refraction could cause such asymmetry. Julius had not realized
this (!) and from then on threw himself into studying anomalous refraction
with even greater persistence.

The correspondence ended with Einstein accepting a professorship in
Zurich. Nine years later, he became an extraordinary professor in
Leiden, and their discussion continued orally, sometimes after musical
sessions at Julius' home in Utrecht. Einstein was drawn to Julius'
sun model, which could explain many phenomena. Dutch physicist H.
Lorentz had strong doubts but, as chairman of the wealthy Belgian
Institut Solvay, did not hesitate to approve a contribution for the
Utrecht spectrograph. Fundraising went smoothly: components were purchased,
and the curators approved the extensive expansion of the Physical
Laboratory with a Heliofysical Institute. By the end of 1918, all
threads came together. The instruments for the spectrograph had arrived;
Moll's microphotometer was ready. Julius struggled with his health.
At that moment, Minnaert stopped by to ask if there was any work available:
he could immediately start working with the spectrograph and microphotometer
to demonstrate Julius' hoped-for proof.

1 Maas, 2001, 43-44, 57-61.

2 Heijmans, 1994, Chapter 2, De natuurkunde van de zon, 20.

3 Einstein, 1925.

4 Schmidt, 1891.

5 Minnaert would use his purist term difraction indicator.

6 Summary of what Julius' theory ‘in principal’ was about in Hentschel,
1991, 51.

7 Minnaert, (1921a).

8 Romein, 1967, 604.

9 Julius, (1899-1900), final sentences. Emphasis by the author.

10 Hentschel, 1991, Chapter 8: Dispute about the Anomalous Dispersion,
87.

11 Julius, (1909-1910).

12 Hentschel, 1991, 73.

13 Hentschel, 1991, 111.

14 The chair of C.H. Wind, who passed away on August 7, 1911. See
Herwaarden, 1971.

15 Einstein to Julius, November 15, 1911. The correspondence between
Julius and Einstein is also in The Collected Papers 5, Klein, 1993.

16 Lorentz advised Julius at the time to include the anomalous dispersion
in his theory! The scattering law of Rayleigh in Minnaert, 1937, 228.
Heijmans, 1994, 30.

17 Maas, 2001, 118, points out that the expansion of educational capacity
was rejected. The curators unequivocally chose a research laboratory.
In the 1920s, they responded to Ornstein's demands, enabling the Physical
Laboratory to truly become a world center for photometry after 1925.

\section*{Bibliography}

Not translated thus far. Can be glanced from original
\end{document}
